\chapter{Introduction}
Everyone experiences the process of taking decisions during his life. Drastically, the life of an individual can pretty much be synthesized in its \textit{perception} of the world and its \textit{interaction} with it. The concepts of perception and interaction might seem quite straightforward to understand: for a human being the perception of the world comes from its senses and the interaction comes from its possibility to change its surroundings. On the contrary, these concepts are actually absolutely hard to define and aroused, during the centuries, a strong debate between scientists, biologists, and even philosophers.
For instance, the perception of the world around an individual consists in its interpretation of the information provided by its senses, but the process of information retrieval by senses and the mental processes in understanding them, inevitably introduce a certain level of noise that distorts the original true information. On the other hand, the interaction with the world deals with the will of the individual to perform actions to change the environment around it but, also in this case, this apparently simple operation involves complex biologic mechanisms to coordinate the body according to the will of the individual and the perception of the consequences of the interaction.
It is arguable that the concept of true information and the concept of will lack of theoretical explanation since for many centuries scientists and philosophers argued about the real nature of perception and the concept of free will. However, for the purpose of this thesis we lighten the definition of these concepts to those provided by common sense.
