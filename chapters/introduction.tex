\chapter{Introduction}
\lettrine{E}{veryone} experiences the process of taking decisions during his life. As a matter of fact, drastically the life of an individual can be synthesized in its \textit{perception} of the world and its \textit{interaction} with it. The concepts of perception and interaction might seem quite straightforward to understand: for a human being the perception of the world comes from its senses and the interaction comes from its possibility to change its surroundings. On the contrary, these concepts are actually absolutely hard to define and aroused, during the centuries, a strong debate between scientists, biologists, and even philosophers.
For instance, the perception of the world around an individual consists in its interpretation of the information provided by its senses, but the process of information retrieval by senses and the mental processes to understand them, inevitably introduce a certain level of noise that distorts the original true information. On the other hand, the interaction with the world deals with the will of the individual to perform actions to change the environment around it, but this apparently simple operation involves complex biologic mechanisms to coordinate the body according to the will of the individual and the perception of the consequences of the interaction.
It is arguable that discussing about the concept of true information and the concept of will requires strong theoretical considerations since they are both hardly definable concepts. For many centuries scientists and philosophers debated about these topics, in particular trying to solve complex problems like the real nature of perceivable things and the concept of free will. However, to make the discussion of these concepts suitable throughout all this thesis, we lighten the definition of them to those provided by common sense.

We start from the assumption that, by definition, an individual perceive the environment around it and acts on it in order to achieve goals expressed by its will. In other words, all the actions made by the individual are done to satisfy its will to obtain something from the world it lives in. This task is naturally performed by humans, but it implies some challenging problems that are hard, or unfeasible, to solve even for us. One of them comes from the intrinsic uncertainty of the perception we have of the world around us, as also briefly discussed before. Another comes from the uncertainty about the consequences of our actions on the world and whether they will let us reach our goal or not, and in which way. Moreover, also the concept of goal can be unclear in our mind and we may result in performing actions without being sure of what we want. 
