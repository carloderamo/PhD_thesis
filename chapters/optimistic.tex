\chapter{Exploration Driven by an Optimistic Bellman Equation}\label{C:opt}
The complexity of exploration in \gls{rl} is, among other reasons, explained by the presence a sparse reward function; indeed before seeing the states yielding a sparse reward the agent does not necessarily have much information to base its decisions on. In this particular case, the attempt to address the \textit{exploration/exploitation} trade-off (near)-optimally makes no
sense since the agent has no information to reason about possible
rewards that it has not yet observed. In this setting, classical exploration
approaches such as $\epsilon$-greedy may fail as the probability of
reaching the positive reward can be low. A more effective strategy
should take into account the underlying uncertainty and try to
minimize it, in order to maximize the information gain. Bayesian approaches consider the uncertainty
in a principled way but are often
computational demanding~\cite{vlassis2012bayesian,engel2005reinforcement}. Recently, computationally feasible
algorithms inspired by Bayesian principles have been
introduced such as Bayesian \gls{dqn}~\cite{azizzadenesheli2517efficient} and \gls{bdqn}~\cite{osband2017deep} discussed in Chapter~\ref{C:ts}.
However, to the best of our knowledge, there is no algorithm among these approximate techniques that is particularly suited for very sparse rewards in high-dimensional state space. Our hypothesis is that Bayesian methods are
in general more focused on balancing between exploration and
exploitation while they cannot achieve deep exploration. The broad
category of algorithms based on \gls{im}~\cite{singh2004intrinsically}, have less theoretical guarantees
than Bayesian approaches, yet they have obtained impressive results
for example in the challenging Montezuma's Revenge task~\cite{bellemare2016unifying}. \gls{im}
algorithms define an additional \textit{intrinsic} reward, which acts as an exploration bonus. Often, the additional
reward is defined using heuristics, such as counting state visits and
rewarding less visited states~\cite{ostrovski2017count}, or by
\textit{surprise} which is the error in predicting future
states~\cite{pathak2017curiosity}. However, the drawback of \gls{im} techniques is
their lack of a principled definition of the intrinsic reward for exploration.

The \gls{ofu}-based techniques already discussed in Chapter~\ref{C:ts} provide an optimistic estimation under uncertainty, encouraging in this way exploration of uncertain region. Optimism can be categorized in
\begin{itemize}
 \item \textbf{optimistic initialization} of the action-value function approximator to an optimistic value, a method proposed first by~\cite{sutton1998reinforcement} about which subsequently~\cite{even2002convergence} proved convergence to a near-optimal solution;
 \item \textbf{confidence interval estimation}, such as IEQL+~\cite{meuleau1999exploration} and UCRL~\cite{auer2007logarithmic}, which directly estimates $Q$-value confidence intervals.
\end{itemize}

Considering this premise, we worked on the proposal of a novel \gls{obe}. The \gls{obe} results in an optimistic $Q$-value estimate
from an ensemble of value functions where the optimistic estimate is obtained from a maximum-entropy principle. For the exploration bonus that \gls{obe} implicitly defines, we can prove that the bonus decreases consistently with the number of
state visits. Our proposed algorithm can be seen as a mixture of
different techniques: as an approximated Bayesian method~\cite{engel2005reinforcement,vlassis2012bayesian}, we estimate
the uncertainty with an ensemble~\cite{osband2017deep}; like optimism-based methods~\cite{lai1985asymptotically,kearns2002near,brafman2002r,azizzadenesheli2517efficient}, we select
optimistic estimates, and like \gls{im}, we propagate an implicit
exploration bonus~\cite{singh2004intrinsically,schmidhuber2008driven,white2010interval}.

\section{Learning Value Function Ensembles with Optimistic Estimate Selection}
\label{sec:obe}
Ensemble methods~\cite{opitz1999popular} constitutes a popular set of \gls{ml} technique where multiple models are used to learn the same target function. In addition to being commonly used to improve the generalization of the prediction, ensemble methods offer a simple way to estimate the uncertainty of the prediction. We consider the application of ensemble methods in the \gls{rl} framework with the purpose of approximating the action-value functions while having an estimate of their uncertainty, in order to apply the \gls{ofu} principle in action selection.

\subsection{An Optimistic Bellman Equation for Action-Value Function Ensembles}
The core of our work consists in the proposal of a \gls{be} which incorporates the information about the uncertainty provided by a $Q$-function ensemble.
In more detail, we want to overestimate the action-value functions with the result of encouraging exploration.   
Thus, we propose an \gls{obe} which propagates an optimistic estimate of the action-value function. 
We want to emphasize that when all the $Q$-functions of the ensemble are identical, we assume that there is no uncertainty, and under this condition the \gls{obe} will behave exactly equivalently to the classic \gls{be}. The solution $Q^*$ of \gls{obe} is the same of the classic \gls{be}. In other words, the \gls{obe} differs from the classic \gls{be} when it is not satisfied, and more precisely when approximation is introduced either by limited availability of samples and/or functional approximation. This makes sense, since when the perfect solution is available there is no need of optimism and exploration. 
The optimistic Bellman operator derived from the \gls{obe}, enjoys the properties of the classical one, like contractivity and the existence of a unique fixed point, potentially enabling its usage in value-based or actor-critic reinforcement algorithms. 
The diversity in the $Q$-value ensemble should be ideally consistent with the uncertainty of the estimation; e.g. when the estimate is certain, all the values in the ensemble should agree on the same value, otherwise the ensemble should have discordant values.
Given an ensemble of $Q$-value functions $\{Q_m\}_{m=1}^M$, we want to work out an optimistic estimate from the diverse estimates provided by the ensemble. The simplest and most optimistic solution is to select the highest value $\max_m Q_m(s,a)$.
However selecting the highest estimate makes poor use of the information provided by the ensemble and can be sensible to noise. In order to mitigate this effect, we introduce a notion of \textit{belief} over the estimates where $b_m(s,a)$ is the belief of $Q_m(s,a)$. The main idea is to add an entropic regularization term to the objective (i.e., $\max_{b(s,a)} \sum_m b_m(s,a) Q_m(s,a) + c \sum_{m} p_m \log p_m $); or to bound the information loss (i.e., $\sum_m p_m \log p_m < \psi$). Hard constraint on the information loss is more appealing since the introduced hyper-parameter does not depend on the magnitude of the rewards but has no closed-form solution. In contrast, the penalization weighting constant introduced by the soft-constraint regularization term is sensitive to the magnitude of the rewards, but admits a closed-form solution.
We define two different problems where we use an optimistic estimate of the $Q$-value function.

\subsubsection{Entropy-Regularized Optimistic $Q$ Selection}
We define a \gls{be} over the $Q$-function ensemble by introducing an optimistic estimate penalized by an entropic regularization term.
\begin{probdef}[Regularized version]
\begin{equation}
\arraycolsep=1.4pt\def\arraystretch{2.2}
\begin{array}{rrclcl}
\displaystyle Q_i(s,a) = \max_{b(s,a) \in \mathcal{P}^M} & \multicolumn{3}{l}{R(s,a) + \gamma \sum_m{b}_m(s,a)V_m'(s,a)-\frac{1}{\eta}D_{\mathrm{KL}}\big(p(s,a)\big{\|}u\big) } \\
\textrm{s.t.} & \sum_{m=1}^{M} b_m(s,a) & = & 1 \\
\multicolumn{4}{l}{ \forall s,a,i \in \mathcal{S} \times \mathcal{A} \times \{1,\dots,M\}}
\end{array}\nonumber
\end{equation}\label{PROB:regularized}
\end{probdef}
\noindent with $u_m  =  \nicefrac{1}{M}$, $D_{KL}(b(s,a)\| u)$ is the \gls{kl} divergence between the belief $b(s,a)$ and the uniform distribution $u$, and $V_m'(s,a) \! = \! \sum_{s'} P(s'|s,a)\max_{a'}Q_m(s',a')$.
The choice of using the relative entropy instead of the absolute one has two main advantages: it admits a solution for $\eta \to 0$ and provides a normalization factor.
Since Problem~\ref{PROB:regularized} is a convex constrained problem, it is solvable by dual optimization. Introducing $\lambda$ as Lagrangian multiplier for the constraint, we write the Lagrangian
\begin{eqnarray}
L_i(s,a) \!&=& \!f(s,a; b(s,a)) -\frac{1}{\eta}D_{\mathrm{KL}}\big(b(s,a)\big{\|}u\big)\nonumber \\ && + \lambda\bigg(\sum_m b_m(s,a) - 1\bigg).\label{E:lagrangian}
\end{eqnarray}
Requiring the partial derivatives of $L_i$ w.r.t $p_m$ and $\lambda$ to be zero yields 
\begin{equation}
b_m(s,a) = \frac{e^{\eta  \gamma V_m'(s,a)}}{\sum_{k=1}^{M} e^{\eta \gamma   V_k'(s,a)}}\label{pm}.
\end{equation} 
By substituting $b_m$ in Equation~\ref{E:lagrangian}, we obtain the solution to the problem:
\begin{equation}
Q_i(s,a) = \begin{cases}
\overline{R}(s,a) + \frac{1}{\eta} \log \frac{\sum_{m=1}^Me^{\eta \gamma  V_m'(s,a) }}{M} & \mathrm{if} \eta \neq 0 \label{OBE} \\
\overline{R}(s,a) + \frac{\gamma}M \sum_{m=1}^{M} V_m'(s,a) & \mathrm{otherwise}
\end{cases}.
\end{equation}
Notice that $\eta>0$ leads to a positive (optimistic) biased estimation, while $\eta<0$ will leads to a negative (pessimistic) estimate; in this work we will always assume $\eta>0$ (and therefore we refer to the equation as optimistic). 
However, in general, the choice of $\eta$ is difficult since it depends on the magnitude of the reward function. For this reason we introduce the constrained version of the proposed problem.
 
\subsubsection{Optimistic $Q$ Selection Bounding the Information Loss.}
We bound the information loss between the distribution $b_m$ and the uniform distribution to maintain compatibility with Problem~\ref{PROB:regularized}. The information loss is bounded between $-\log M$ and $0$ where  $-\log M$ stands for complete information loss (i.e., only one model is selected) while $0$ corresponds to no information loss (i.e., uniform belief distribution). Constraining the information loss has succeeded in prior work, for instance in policy search methods such as~\cite{peters2010relative}.
\begin{probdef}[Constrained version]
\begin{equation}
\begin{array}{rrclcl}
\displaystyle Q_i(s,a) = \max_{b(s,a) \in \mathcal{P}^M} & \multicolumn{3}{l}{R(s,a) + \gamma \sum_m{b}_m(s,a)V_m'(s,a)\nonumber } \\
\text{s.t.} & D_{\mathrm{KL}}\big(b(s,a)\big{\|} u \big)& \leq & \iota_{\max} \\
& \sum_{m=1}^{M} b_m(s,a) & = & 1 \\
\multicolumn{4}{l}{ \forall s,a,i \in \mathcal{S} \times \mathcal{A} \times \{1,\dots,M\}}
\end{array} \nonumber
\end{equation}\label{PROB:constrversion}
\end{probdef}
\noindent By letting $\beta$ be the Lagrangian multiplier associated with the KL constraint, we obtain the Lagrangian
\begin{eqnarray}
L_i &\! = \!& f(s,a;b(s,a)) + \beta (D_{\mathrm{KL}}(b(s,a){\|}u) - \iota_{\max}) \nonumber \\
& &   + \lambda(\sum_m b_m(s,a) - 1).\label{E:lagrangian2}
\end{eqnarray}
Substituting $\beta$ with $\nicefrac{-1}{\eta}$ we note that Equation~\ref{E:lagrangian2} becomes identical to~\ref{E:lagrangian} except for a constant factor. Since we can not solve $\eta$ (or $\beta$) analytically, we obtain an approximate solution by iteratively optimizing $\eta$ (or $\beta$) and $b_m$ subsequently.
\gls{obe} takes its name from the fact that when $\eta > 0$, the \textit{logsumexp} acts as a softmax operator. Such operator is also well known as an \textit{entropic mapping}, as it can be derived from a maximum-entropy principle.

The use of the entropic mapping is not new in \gls{rl}: \cite{pmlr-v70-asadi17a} propose an interesting use of the entropic mapping as a soft-max over the action in the \gls{be}; \cite{peters2010relative} instead obtain it from an entropic regularization over the state-action distribution.

\paragraph{Relation to Intrinsic Motivation}
In order to highlight the connection between \gls{obe} and \gls{im}, we reformulate \gls{obe} utilizing the unbiased average of the estimates instead of the logsumexp, and by introducing the resulting exploratory bonus $U$ which includes the positive bias
\begin{eqnarray}
Q_i(s,a)\! &=&\! \overline{R}(s,a) + U(s,a) +  \gamma  \sum_{m=1}^M \frac{V_m'(s,a)}{M} \label{bonusBE}
\end{eqnarray}
with
\begin{equation}
U(s,a) \! =\!  \frac{1}{\eta}\log\sum_{m=1}^M \frac{e^{\eta\gamma V_m'(s,a)}}{M} - \gamma  \sum_{m=1}^M \frac{V_m'(s,a)}{M}. \label{bonusdef}
\end{equation}
Noticing that $\dfrac{\sum_{i=1}^N e^{\eta x_i}}{N}$ is the \textit{sample moment generator} w.r.t. samples $\{x_i\}_{i=1}^N$ we can rephrase the exploration bonus as
\begin{eqnarray}
U(s,a)& = & \lim_{N \to +\infty}\frac{1}{\eta} \log [  1 + \sum_{n=2}^{N} \frac{(\eta\gamma)^n}{n!}\mathcal{M}_n(s,a)] \nonumber \\
& = & \eta \gamma \mathcal{M}_2(s,a)  + O(\eta^2)\label{bonus}
\end{eqnarray}
where $\mathcal{M}_n$ is the $n$\textsuperscript{th} central moment of the random variable $V_m'$
\begin{equation}
\mathcal{M}_n(s,a) = M^{-1} \sum_{m=1}^M [( V_m'(s,a) -\overline{V}(s,a))^n]\nonumber
\end{equation}
with 
\begin{equation}
\overline{V}(s,a) = M^{-1} \sum_{m=1}^M V_m'(s,a).\nonumber
\end{equation}
Equation \eqref{bonusBE} shows that \gls{obe} is equivalent to \gls{be} with an additional bonus defined by Equation~\ref{bonus}. The bonus $U$ (for any positive $\eta$) is always positive, and provides a measure of the uncertainty w.r.t. $Q$. This is why \gls{obe} can be interpreted as a special principled form of \gls{im}.

\paragraph{Explicit Exploration} A general problem affecting \gls{im} algorithms, is that the policy greedy to the obtained $Q$-value function, is not optimized for the original problem. As a solution to this issue we approximate two functions: $\tilde{Q}$, which will be updated using the true reward and $Q_E$ which will be updated using only the intrinsic reward \cite{szita2008many}. In this way we obtain both the \gls{im} policy $\pi_o(s) = \arg \max_{a} \tilde{Q}(s,a) + Q_E(s,a)$ and the classic policy  $\pi_u(s) = \arg \max_{a} \tilde{Q}(s,a)$.
Define 
\begin{eqnarray}
	\tilde{Q}_i(s,a) = R(s,a) + \gamma \sum_{m=1}^M\frac{\tilde{V}_m'(s,a)}{M} \quad \text{with} \\
	\tilde{V}_m'(s,a) =  \sum_{s'} P(s'|s,a)\max_{a'}\tilde{Q}_m(s',a')
\end{eqnarray}
to obtain an unbiased estimate of the $Q$-value function, yielding
\begin{eqnarray}
Q_E(s,a) & = & \sum_{t=0}^T \gamma^t U(s_t,a_t) \quad \text{where} \quad s_0 = s, a_0 = a \nonumber \\
& = & \eta^{-1}\log \frac{\sum_{k=1}^M e^{\eta\gamma\max_{a'}\tilde{Q}_k(s',a') + Q_E(s',a')}}{M}\nonumber \\
& &  -  \frac{\sum_{k=1}^M \gamma\max_{a'}\tilde{Q}_k(s',a')}{M}.
\end{eqnarray}
By a simple equation rearrangement, it is possible to show that $\tilde{Q}_i(s,a) + Q_E(s,a)$ is equivalent to $Q_i(s,a)$ as defined in the \gls{obe}~\ref{OBE}.
 
\subsection{Optimistic Value Function Estimators}
The \gls{obe} offers a theoretical framework in which it is possible to develop optimistic value based algorithms. In fact, \gls{obe} enjoys all the desirable properties of the \gls{be} (e.g. max-norm contractivity), as shown in the supplement. We present briefly two practical applications of the OBE that are optimistic variants of $Q$-learning and \gls{dqn} that we call, respectively, \gls{oql} and \gls{odqn}.
\paragraph{Optimistic $Q$-Learning.}
Motivated by the idea of employing an ensemble of regressors as is done in \gls{bdqn}~\cite{osband2017deep}, we assume to have $M$ randomly initialized $Q$-tables. Inspired by the well known $Q$-learning update rule, we derive an optimistic version which is consistent with the \gls{obe} as follows:
\begin{align*}
      \small  Q_{i, t+1}(s,a) &= (1-\alpha_t)Q_{i,t}(s,a)  \nonumber  \\
  & \small + \alpha_t (r_t + \frac{1}{\eta} \log M^{-1} \sum_{j=1}^M e^{\gamma \max_{a'} Q{j,t}(s', a')}).\nonumber
\end{align*}
\label{def:optimistic_qlearning}
We show that, with the update rule proposed, given infinite visits of each state-action pair, all the tables will converge to the same values, and more precisely, after each update, the $n$-th central moment of the updated cell is scaled exactly by $(1 - \alpha_t)^n$:
\begin{equation}
	\mathcal{M}_{n,t+1}(s,a) = (1-\alpha_t)^n \mathcal{M}_{n,t}(s,a) \label{momentdecreasing}
\end{equation}
where 
\begin{equation}
	\mathcal{M}_{n,t}(s,a) = M^{-1} \sum_{i=1}^M (Q_{i,t}(s,a) - \sum_{k=1}^M \frac{Q_{k,t}(s,a)}{M})^n. \nonumber
\end{equation}
This implies that a cell updated $N$ times, with learning rates $\{\alpha_i\}$, will have the $n$-th central moments scaled by $\Pi_{\alpha_i}(1-\alpha_i)^n$ w.r.t.\ the initial one. This leads us to some interesting considerations: 1) the bonus decrease accordingly to the number of state visits; 2) differing from several count-based approaches, our algorithm takes into account the impact of the learning rate; 3) in the limit of an infinite number of visits, the exploration bonus converges to zero.
Further details, including a proof of convergence, are given in the supplemental material\footnote{We based our convergence proof for \gls{oql} on the works of~\cite{melo2001convergence} and~\cite{jaakkola1994convergence}.}.
All the considerations done so far provide a deeper insight about how the algorithm works and its properties. However, in a more complex settings, (e.g., function approximation) the convergence to zero of the exploratory bonus is not guaranteed in general.

\paragraph{Optimistic Deep $Q$-Network}
\label{sec:proposedalg}

In addition to the novel \gls{oql} algorithm described previously that can be used for limited discrete state spaces, we
propose another algorithm for continuous state spaces based on our \gls{obe}. We take inspiration from the framework provided by \gls{bdqn}~\cite{osband2017deep}
that uses an ensemble of neural networks as estimator for the $Q$ value function.
\gls{bdqn} minimizes the loss
\begin{equation}
	\mathcal{L}_B(s,a)\! = \! \sum_{k=1}^M (r + \gamma \max_{a'}Q_k^T(s',a') - Q_k(s,a))^2, \nonumber
\end{equation}
where $Q_k^T$ is the target network of the $k$\textsuperscript{th} approximator. 
To get an unbiased performance evaluation, we decided to update $M-1$ components of the ensemble with the update rule provided by \gls{bdqn}. We make this choice in order to maintain diversity between the approximations of the ensemble as shown in~\cite{osband2016deep}. We use the remaining single component of the ensemble to approximate $Q_E$. Using the first component to approximate $Q_E$, we get for our new algorithm \gls{odqn} the loss
\begin{eqnarray}
	&\mathcal{L}_O(s,a) = (\eta^{-1}\log \frac{\sum_{k=2}^M
	 e^{\eta\gamma\max_{a'}Q_k^T(s',a')+ Q_1^T(s',a')}}{M}\nonumber \\
&\qquad\qquad\;\;\;\;  -  \frac{\sum_{k=2}^M  \gamma\max_{a'}Q^T_k(s',a')}{M} -Q_1(s,a))^2\nonumber \\
&\;  + \sum_{k=2}^M (r + \gamma\max_{a'}Q^T_k(s',a') -
	 Q_k(s,a))^2.  \label{optimisticloss}
\end{eqnarray}
The exploratory bonus represented by $Q_E = Q_1$ in the proposed \gls{oql} and \gls{odqn} algorithms is needed to
guide exploration during learning. During evaluation, we use majority voting on the remaining $M-1$
components $\{Q_k\}_{k=2}^M$. While we always select an optimistic policy in \gls{oql} during the training phase, in
\gls{odqn} the neural network function approximator may have problems learning to approximate the optimal
policy: if there are not enough unbiased samples the approximator may learn to model only the optimistic
biased samples. Note that in the tabular case, this is not a problem since there is no $Q$-function
approximation. In order to mitigate this problem, we introduce a hyper-parameter $\chi$ which denotes the probability to select an optimistic policy $\pi_o$ in place of the unbiased one $\pi_u$. In this way, we can balance the number of unbiased and optimistic samples.
Algorithm \ref{optimisticdqn} shows the pseudocode of \gls{odqn}. 
\begin{algorithm}[t]
	\caption{Optimistic Deep $Q$-Network}
	\label{optimisticdqn}
	\begin{algorithmic}
		\STATE \textbf{Input:} $\{Q_k\}_{k=1}^K$, $\iota_{\max}$, $\eta_{\mathrm{init}}$, $\chi$, $N$, $C$
		\STATE Let $B$ be a replay buffer storing the experience for training.
		\STATE $\eta = \eta_{\mathrm{init}}$.
		\STATE Let $i \sim \mathrm{Uniform}\{1 \dots M\}$ and $\psi = 1$ w.p. $\chi$ otherwise $\psi = 0$
		\FOR{$N$ epochs}
		\FOR{$C$ steps}
		\STATE Observe $s$
		\STATE Choose $a = \arg \max_a Q_i(s,a) + \psi Q_1(s,a)$
		\STATE Observe reward $r$, next state $s'$, end of episode $t$
		\STATE If $t$ is terminal, $i \sim \mathrm{Uniform}\{2 \dots M\}$ and \\ \ \ \ \ \ \ $\psi = 1$ w.p. $\chi$ otherwise $\psi=0$
		\STATE Store $<s,a,r,s',t>$ in buffer $B$
		\STATE Sample mini-batch $B_{\mathrm{batch}}$
		\STATE Update $\{Q_k\}_{k=1}^K$ using equation \eqref{optimisticloss}
		\STATE $V \leftarrow V + |$ violated constraints \eqref{batchconstr} in $B_{\mathrm{batch}}|$ 
		\ENDFOR
		\STATE Let $\rho = \frac{V}{C * \mathrm{batch\_size}}$
		\STATE Update $\eta$ by \eqref{etaupdate}
		\STATE Update target network
		\ENDFOR
	\end{algorithmic}
\end{algorithm}

\paragraph{Automatic Hyper-parameter Adaptation.}
\label{subsec:adaptive}
Recalling that the regularization coefficient $\eta$ in the \gls{obe} is hard to tune, we want to focus our attention on Problem~\ref{PROB:constrversion}. We propose a way, inspired by~\cite{schulman2017proximal}, to optimize $\eta$.
One of the optimization techniques proposed in this work is to measure the ``degree'' of constraint violation and to update the Lagrangian multiplier accordingly. We have to adapt the technique to multiple constraints, as the problem is defined for each state-action pair: we count the number of times the constraints have been violated and then update $\eta$. In more detail, suppose to have $N$ state-action pairs and for each pair $(s_i, a_i)$
\begin{eqnarray}
\sum_m b_m(s_i,a_i) (\log b_m(s_i,a_i) + \log M) \leq \iota_{\max}, \label{batchconstr}
\end{eqnarray}
where $\iota_{\max}$ is defined in Problem~\ref{constrversion}, while $b_m(s_i,a_i)$ is defined by \eqref{pm}. We define $\rho$ as the ratio of violated constraints. We update $\eta$ according to the following rule
\begin{equation}
\eta_{T+1} = \frac{\eta_{T}}{(0.5 +  10 \rho)} \label{etaupdate}.
\end{equation}
In \gls{odqn}, we decided to count the number of constraints violated every $C$ time-steps (basically every update of the target network), using the samples of all the extracted mini-batches. See Algorithm~\ref{optimisticdqn} for further details.  
  
\paragraph{Ensuring a Prior Distribution.} As already discussed, it is important to maintain diversity in our ensemble, and this diversity should reflect the degree of uncertainty. For this reason, we should introduce a sort of prior distribution, as happens in the Bayesian framework. In the case of \gls{oql}, we observe that it is sufficient to randomly initialize each element of the ensemble, since diversity between estimates is a sufficient condition to obtain positive bonus.
For \gls{odqn}, as is done in \gls{bdqn}, we choose to maintain the diversity between approximation, by a random initialization of each component's parameters and by using the bootstrapping technique, so by adding a \textsl{mask} in the replay memory which is sampled by  and use different data samples per regressor.

\section{Experimental Evaluation}
\label{S:odqn_experiments}
In the experiments, we compare in the tabular $Q$-function case our novel \gls{oql} with the well-known state-of-the-art bootstrapped $Q$-learning method (BQL) \cite{osband2016deep}, classical $Q$-learning~\ref{S:FQI}, and $Q$-learning with optimistic initialization~\cite{sutton1998reinforcement} in the $50$-Chain \cite{osband2016deep}, Taxi (also known as Maze) \cite{dearden1998bayesian} and Frozen Lake \cite{brockman2016openai} environments. In the case of neural networks approximation, we compare our \gls{odqn} method with \gls{bdqn} and classical \gls{dqn} in the Taxi and Acrobot environments already described in the previous chapters.
The environments are chosen to cover different types of dynamics, have sparse rewards, and include both discrete and continuous states. For Acrobot and Frozen Lake, we used the implementation provided by OpenAI Gym~\cite{brockman2016openai}. First, we will discuss the environments in more detail, then we will provide some details on the
initialization of the methods, and finally finish with an analysis of the results.

The \textbf{N-Chain} environment \cite{osband2016deep} requires a long sequence of non-rewarding actions to achieve the optimal reward. The \gls{mdp} consists of a chain with $N$ states $\{s_i\}_{i=1}^N$, and two actions that move the agent to state $s_{i+1}$ or $s_{i-1}$. The agent always starts in state $s_2$. In state $s_1$ the agent observes a small reward $r(s_1)=1/1000$, while in the $N$\textsuperscript{th} state the agent observes $r(s_N) = 1$. The reward function is zero elsewhere. The agent needs to explore until reaching state $s_N$ even if state $s_1$ looks promising.
   
The \textbf{Frozen Lake} environment is a $8\times 8$ grid-world, in which the agent has to reach a goal position without falling into some holes. The stochastic perturbation of the agent's movement makes the environment challenging.

This problem consists of modeling a car which is placed in a valley, and the power of its engine is not enough to reach the top of the mountain \cite{moore1990efficient}. The car needs to build momentum by driving backward. We use the implementation provided by OpenAI Gym \cite{brockman2016openai}.

\begin{figure*}[t]
  \centering
  \begin{tikzpicture}[scale=1, transform shape]
  \node[state]             (1) {1};
  \node[state, right=of 1] (2) {2};
  \node[state, right=of 2] (3) {\dots};
  \node[state, right=of 3] (4) {N};
  \node[above=0.5 of 4] (5) {R};
  \draw[-stealth,
  decoration={snake, 
	  amplitude = .4mm,
	  segment length = 2mm,
	  post length=0.9mm},decorate] (4) -- (5);
  \draw[every loop]
  (1) edge[loop above] node {b, $0.001$} (1)
  (1) edge[bend left, auto=left] node {a} (2)
  (2) edge[loop above] node {a} (2)
  (2) edge[bend left, auto=left] node {a} (3)
  (3) edge[loop above] node {a} (3)
  (3) edge[bend left, auto=left] node {a} (4)
  (1) edge[loop below] node {b} (1)
  (2) edge[bend left, auto=left] node {b} (1)
  (2) edge[loop below] node {b} (2)
  (3) edge[bend left, auto=left] node {b} (2)
  (3) edge[loop below] node {b, $1$} (3);
  (4) edge[bend left, auto=left] node {b} (3)
  (4) edge[loop above] node {a, $1$} (4);
  \end{tikzpicture}\caption{Illustration of the environments, from the left to the right: $N$-Chain,  Taxi, Frozen Lake and Acrobot. }
\end{figure*}

\paragraph{Initialization of the $Q$-functions.} In the tabular setting, for optimistically initialized Q-learning (OIQL), we initialize the $Q$-function to $15$ in Taxi and to $1$ in $N$-Chain and Frozen Lake. For the other algorithms, we initialize $Q(s,a) \sim \mathcal{N}(\mu=0,\sigma=2)$, except $Q_E$ of \gls{oql} is initialized to $0$.
In the taxi environment, for both \gls{odqn} and \gls{bdqn}, we use a shared convolutional layer with multiple heads as described in \cite{osband2016deep}. For the Acrobot, each component of the ensemble corresponds to a one-layer neural network. For \gls{odqn}, in both the environment, we initialize the output layer corresponding to $Q_E$ to small values of the parameters, in order to obtain initially $Q_E \approx 0$. 

\paragraph{Hyper-Parameter Tuning.} For the tabular settings, we did not run any hyper-parameter optimization. With neural networks we performed a small grid search over the number of neurons of the network, and whether to use bootstrapping or not. More specifically, we selected hyper-parameters maximizing the mean return averaged over the whole learning curve, using $20$ different seeds. In the plots; we compare \gls{odqn} and \gls{bdqn} using the best hyper-parameter setup found for \gls{bdqn}. In \gls{oql} we use $\eta=10$ and for \gls{odqn}, we use $\chi=0.25$ and $\iota_{\max}=1$. For further details about the implementation of the algorithms, and the grid search, please see the Supplement.

\setlength\figureheight{4cm}
\setlength\figurewidth{4cm}
\begin{figure*}[t]
	% This file was created by matplotlib2tikz v0.6.18.
\begin{tikzpicture}

\definecolor{color0}{rgb}{0.12156862745098,0.466666666666667,0.705882352941177}
\definecolor{color1}{rgb}{1,0.498039215686275,0.0549019607843137}
\definecolor{color2}{rgb}{0.172549019607843,0.627450980392157,0.172549019607843}
\definecolor{color3}{rgb}{0.83921568627451,0.152941176470588,0.156862745098039}
\definecolor{color4}{rgb}{0.549019607843137,0.337254901960784,0.294117647058824}
\definecolor{color5}{rgb}{0.890196078431372,0.466666666666667,0.76078431372549}

\begin{groupplot}[group style={group size=4 by 2}]
\nextgroupplot[
height=\figureheight,
tick align=outside,
tick pos=left,
title={50-Chain},
title style={font=\small, yshift=-1.5ex},
xlabel style={font=\scriptsize},
yticklabel style={font=\scriptsize},
xticklabel style={font=\scriptsize},
grid=both,
width=\figurewidth,
x grid style={white!69.01960784313725!black},
xmin=-14.95, xmax=313.95,
y grid style={white!69.01960784313725!black},
ymin=-0.618016756356909, ymax=10.6593649530358
]
\path [fill=color0, fill opacity=0.3] (axis cs:0,0.00823092684258694)
--(axis cs:0,0.0191220143338837)
--(axis cs:1,0.0146868972189267)
--(axis cs:2,0.0161087413567854)
--(axis cs:3,0.0176472840019849)
--(axis cs:4,0.0249082448559281)
--(axis cs:5,0.026376326086257)
--(axis cs:6,0.0287891093301067)
--(axis cs:7,0.0284939685639376)
--(axis cs:8,0.0249156509494596)
--(axis cs:9,0.0309062507427155)
--(axis cs:10,0.0255619512105429)
--(axis cs:11,0.022384516846227)
--(axis cs:12,0.0215650269443974)
--(axis cs:13,0.0182076082981445)
--(axis cs:14,0.0181649810611131)
--(axis cs:15,0.019350550033596)
--(axis cs:16,0.017523547360708)
--(axis cs:17,0.0239144890778476)
--(axis cs:18,0.0205246158253211)
--(axis cs:19,0.021783704911025)
--(axis cs:20,0.0162951225630373)
--(axis cs:21,0.0221030107288587)
--(axis cs:22,0.0219878001488333)
--(axis cs:23,0.0165610164967063)
--(axis cs:24,0.0231463356829901)
--(axis cs:25,0.0251815647686646)
--(axis cs:26,0.0174170274179694)
--(axis cs:27,0.0150828793432824)
--(axis cs:28,0.0140076872054926)
--(axis cs:29,0.0150947419541605)
--(axis cs:30,0.0246753172560022)
--(axis cs:31,0.0149404571265054)
--(axis cs:32,0.0143934636518426)
--(axis cs:33,0.0169578737153203)
--(axis cs:34,0.0171669724200997)
--(axis cs:35,0.0192906721050006)
--(axis cs:36,0.018973880896521)
--(axis cs:37,0.0191732556685254)
--(axis cs:38,0.0169468641346492)
--(axis cs:39,0.0210951737020129)
--(axis cs:40,0.0143501374386799)
--(axis cs:41,0.0197373171037569)
--(axis cs:42,0.0121056359941294)
--(axis cs:43,0.0140442183437932)
--(axis cs:44,0.0159555134523136)
--(axis cs:45,0.0173941084893068)
--(axis cs:46,0.0163579338134927)
--(axis cs:47,0.0154442374869428)
--(axis cs:48,0.0119977016129309)
--(axis cs:49,0.0157137570190656)
--(axis cs:50,0.0143805776930276)
--(axis cs:51,0.0114636873661823)
--(axis cs:52,0.0136742173982719)
--(axis cs:53,0.00986221514669076)
--(axis cs:54,0.0114975730770707)
--(axis cs:55,0.0139151843324922)
--(axis cs:56,0.0195267701033303)
--(axis cs:57,0.0144733426766197)
--(axis cs:58,0.0100741660128398)
--(axis cs:59,0.0136020335454657)
--(axis cs:60,0.0130678497649668)
--(axis cs:61,0.0210320259278234)
--(axis cs:62,0.0155212611951926)
--(axis cs:63,0.0112954331480545)
--(axis cs:64,0.00760327546025448)
--(axis cs:65,0.0129712736459636)
--(axis cs:66,0.00879525283831337)
--(axis cs:67,0.0119214768110086)
--(axis cs:68,0.00977351062666433)
--(axis cs:69,0.00994959429595321)
--(axis cs:70,0.0104262800764816)
--(axis cs:71,0.00569823937821569)
--(axis cs:72,0.0112872462653019)
--(axis cs:73,0.00889369660400013)
--(axis cs:74,0.00602202968260877)
--(axis cs:75,0.00961544550070766)
--(axis cs:76,0.00619397825267707)
--(axis cs:77,0.0130960171424155)
--(axis cs:78,0.0114573967608302)
--(axis cs:79,0.00569006291232065)
--(axis cs:80,0.0112362740736754)
--(axis cs:81,0.0067612436429803)
--(axis cs:82,0.00492546296098201)
--(axis cs:83,0.0104728316144554)
--(axis cs:84,0.0115691584505391)
--(axis cs:85,0.0108453502275619)
--(axis cs:86,0.0104830377439265)
--(axis cs:87,0.00702905130837173)
--(axis cs:88,0.00873767903590831)
--(axis cs:89,0.00523658631531032)
--(axis cs:90,0.0110755709068476)
--(axis cs:91,0.0119983250096664)
--(axis cs:92,0.0111311316364887)
--(axis cs:93,0.00732282986191366)
--(axis cs:94,0.00757296283695606)
--(axis cs:95,0.0108368021323785)
--(axis cs:96,0.0056513952107335)
--(axis cs:97,0.00838232850169643)
--(axis cs:98,0.00924798308206145)
--(axis cs:99,0.0135235089894456)
--(axis cs:100,0.00928289445740305)
--(axis cs:101,0.00700876096927406)
--(axis cs:102,0.00764281680414605)
--(axis cs:103,0.00609961908787784)
--(axis cs:104,0.00744142864410255)
--(axis cs:105,0.00938988938827763)
--(axis cs:106,0.00869450553749319)
--(axis cs:107,0.00675649349647292)
--(axis cs:108,0.0054550284522141)
--(axis cs:109,0.00779470745841356)
--(axis cs:110,0.00831526108242772)
--(axis cs:111,0.00713620835375248)
--(axis cs:112,0.00886390580128766)
--(axis cs:113,0.00896175244298851)
--(axis cs:114,0.0108612127008784)
--(axis cs:115,0.00487676701913783)
--(axis cs:116,0.00487837930224114)
--(axis cs:117,0.00660171501368254)
--(axis cs:118,0.00348465572858743)
--(axis cs:119,0.194481042986987)
--(axis cs:120,0.026156473126586)
--(axis cs:121,0.110373385766031)
--(axis cs:122,0.00810021538752208)
--(axis cs:123,0.0838242397773776)
--(axis cs:124,0.16105960989288)
--(axis cs:125,0.398067686401346)
--(axis cs:126,0.638376040021761)
--(axis cs:127,0.823541173464604)
--(axis cs:128,0.923698458151547)
--(axis cs:129,1.11648660457872)
--(axis cs:130,0.941197152698387)
--(axis cs:131,1.13947399042241)
--(axis cs:132,1.24147256203967)
--(axis cs:133,1.80540713019755)
--(axis cs:134,1.67453809991011)
--(axis cs:135,1.61386070885513)
--(axis cs:136,1.69081434948269)
--(axis cs:137,1.78914234257951)
--(axis cs:138,2.58833926090572)
--(axis cs:139,2.9704101262019)
--(axis cs:140,2.7480047931436)
--(axis cs:141,3.43633973333455)
--(axis cs:142,3.66271125112156)
--(axis cs:143,3.37108252021361)
--(axis cs:144,4.42513085590231)
--(axis cs:145,3.97191485171215)
--(axis cs:146,4.15824245047414)
--(axis cs:147,4.67763390633239)
--(axis cs:148,4.70393487587522)
--(axis cs:149,4.7448987669476)
--(axis cs:150,4.98703214462356)
--(axis cs:151,5.14750393385474)
--(axis cs:152,4.89389023304979)
--(axis cs:153,5.52868050064082)
--(axis cs:154,5.60833994543606)
--(axis cs:155,5.71213848488464)
--(axis cs:156,5.64915698974785)
--(axis cs:157,6.13221331976387)
--(axis cs:158,6.35822817888478)
--(axis cs:159,6.50111881253497)
--(axis cs:160,6.53916885317289)
--(axis cs:161,6.64660554746605)
--(axis cs:162,6.86562019769854)
--(axis cs:163,6.98683914379692)
--(axis cs:164,7.32911060491088)
--(axis cs:165,7.21824070900347)
--(axis cs:166,7.03550553437291)
--(axis cs:167,7.27063328360013)
--(axis cs:168,7.65868284418737)
--(axis cs:169,7.7361019916513)
--(axis cs:170,7.88820356848253)
--(axis cs:171,7.6033538396979)
--(axis cs:172,7.68230856533816)
--(axis cs:173,8.05411799884602)
--(axis cs:174,8.22433254930616)
--(axis cs:175,8.19413093469737)
--(axis cs:176,8.32812955042697)
--(axis cs:177,8.38394391420854)
--(axis cs:178,8.32773820770463)
--(axis cs:179,8.43181316054328)
--(axis cs:180,8.44411930510518)
--(axis cs:181,8.35984388155945)
--(axis cs:182,8.69854026549978)
--(axis cs:183,8.69558259175899)
--(axis cs:184,8.77100081233972)
--(axis cs:185,8.93901952109522)
--(axis cs:186,8.88141695760004)
--(axis cs:187,9.09890203500363)
--(axis cs:188,9.18073312560618)
--(axis cs:189,9.16306156114697)
--(axis cs:190,9.15175969793183)
--(axis cs:191,9.1630245588641)
--(axis cs:192,9.43067479351263)
--(axis cs:193,9.43067479351263)
--(axis cs:194,9.3764630820644)
--(axis cs:195,9.54849834184733)
--(axis cs:196,9.49806376316662)
--(axis cs:197,9.49806376316662)
--(axis cs:198,9.49806376316662)
--(axis cs:199,9.64784391856315)
--(axis cs:200,9.71345277436147)
--(axis cs:201,9.7227270225012)
--(axis cs:202,9.71345277436147)
--(axis cs:203,9.7227270225012)
--(axis cs:204,9.77462438651804)
--(axis cs:205,9.77454874960612)
--(axis cs:206,9.77319755268516)
--(axis cs:207,9.78013066950677)
--(axis cs:208,9.78013066950677)
--(axis cs:209,9.77483713079612)
--(axis cs:210,9.83545868492831)
--(axis cs:211,9.93812603040178)
--(axis cs:212,9.98402528744401)
--(axis cs:213,9.98402528744401)
--(axis cs:214,9.98402528744401)
--(axis cs:215,9.98402528744401)
--(axis cs:216,9.98402528744401)
--(axis cs:217,9.98402528744401)
--(axis cs:218,9.98402528744401)
--(axis cs:219,9.98402528744401)
--(axis cs:220,9.98402528744401)
--(axis cs:221,9.98402528744401)
--(axis cs:222,10.0568909795146)
--(axis cs:223,10.0568909795146)
--(axis cs:224,10.0568909795146)
--(axis cs:225,10.073799001065)
--(axis cs:226,10.073799001065)
--(axis cs:227,10.073799001065)
--(axis cs:228,10.073799001065)
--(axis cs:229,10.073799001065)
--(axis cs:230,10.073799001065)
--(axis cs:231,10.073799001065)
--(axis cs:232,10.073799001065)
--(axis cs:233,10.073799001065)
--(axis cs:234,10.073799001065)
--(axis cs:235,10.073799001065)
--(axis cs:236,10.073799001065)
--(axis cs:237,10.073799001065)
--(axis cs:238,10.073799001065)
--(axis cs:239,10.073799001065)
--(axis cs:240,10.073799001065)
--(axis cs:241,10.073799001065)
--(axis cs:242,10.073799001065)
--(axis cs:243,10.073799001065)
--(axis cs:244,10.073799001065)
--(axis cs:245,10.073799001065)
--(axis cs:246,10.073799001065)
--(axis cs:247,10.073799001065)
--(axis cs:248,10.073799001065)
--(axis cs:249,10.073799001065)
--(axis cs:250,10.073799001065)
--(axis cs:251,10.073799001065)
--(axis cs:252,10.073799001065)
--(axis cs:253,10.073799001065)
--(axis cs:254,10.073799001065)
--(axis cs:255,10.073799001065)
--(axis cs:256,10.073799001065)
--(axis cs:257,10.073799001065)
--(axis cs:258,10)
--(axis cs:259,10)
--(axis cs:260,10)
--(axis cs:261,10)
--(axis cs:262,10)
--(axis cs:263,10)
--(axis cs:264,10)
--(axis cs:265,10)
--(axis cs:266,10)
--(axis cs:267,10)
--(axis cs:268,10)
--(axis cs:269,10)
--(axis cs:270,10)
--(axis cs:271,10)
--(axis cs:272,10)
--(axis cs:273,10)
--(axis cs:274,10)
--(axis cs:275,10)
--(axis cs:276,10)
--(axis cs:277,10)
--(axis cs:278,10)
--(axis cs:279,10)
--(axis cs:280,10)
--(axis cs:281,10)
--(axis cs:282,10)
--(axis cs:283,10)
--(axis cs:284,10)
--(axis cs:285,10)
--(axis cs:286,10)
--(axis cs:287,10)
--(axis cs:288,10)
--(axis cs:289,10)
--(axis cs:290,10)
--(axis cs:291,10)
--(axis cs:292,10)
--(axis cs:293,10)
--(axis cs:294,10)
--(axis cs:295,10)
--(axis cs:296,10)
--(axis cs:297,10)
--(axis cs:298,10)
--(axis cs:299,10)
--(axis cs:299,10)
--(axis cs:299,10)
--(axis cs:298,10)
--(axis cs:297,10)
--(axis cs:296,10)
--(axis cs:295,10)
--(axis cs:294,10)
--(axis cs:293,10)
--(axis cs:292,10)
--(axis cs:291,10)
--(axis cs:290,10)
--(axis cs:289,10)
--(axis cs:288,10)
--(axis cs:287,10)
--(axis cs:286,10)
--(axis cs:285,10)
--(axis cs:284,10)
--(axis cs:283,10)
--(axis cs:282,10)
--(axis cs:281,10)
--(axis cs:280,10)
--(axis cs:279,10)
--(axis cs:278,10)
--(axis cs:277,10)
--(axis cs:276,10)
--(axis cs:275,10)
--(axis cs:274,10)
--(axis cs:273,10)
--(axis cs:272,10)
--(axis cs:271,10)
--(axis cs:270,10)
--(axis cs:269,10)
--(axis cs:268,10)
--(axis cs:267,10)
--(axis cs:266,10)
--(axis cs:265,10)
--(axis cs:264,10)
--(axis cs:263,10)
--(axis cs:262,10)
--(axis cs:261,10)
--(axis cs:260,10)
--(axis cs:259,10)
--(axis cs:258,10)
--(axis cs:257,9.76995099893496)
--(axis cs:256,9.76995099893496)
--(axis cs:255,9.76995099893496)
--(axis cs:254,9.76995099893496)
--(axis cs:253,9.76995099893496)
--(axis cs:252,9.76995099893496)
--(axis cs:251,9.76995099893496)
--(axis cs:250,9.76995099893496)
--(axis cs:249,9.76995099893496)
--(axis cs:248,9.76995099893496)
--(axis cs:247,9.76995099893496)
--(axis cs:246,9.76995099893496)
--(axis cs:245,9.76995099893496)
--(axis cs:244,9.76995099893496)
--(axis cs:243,9.76995099893496)
--(axis cs:242,9.76995099893496)
--(axis cs:241,9.76995099893496)
--(axis cs:240,9.76995099893496)
--(axis cs:239,9.76995099893496)
--(axis cs:238,9.76995099893496)
--(axis cs:237,9.76995099893496)
--(axis cs:236,9.76995099893496)
--(axis cs:235,9.76995099893496)
--(axis cs:234,9.76995099893496)
--(axis cs:233,9.76995099893496)
--(axis cs:232,9.76995099893496)
--(axis cs:231,9.76995099893496)
--(axis cs:230,9.76995099893496)
--(axis cs:229,9.76995099893496)
--(axis cs:228,9.76995099893496)
--(axis cs:227,9.76995099893496)
--(axis cs:226,9.76995099893496)
--(axis cs:225,9.76995099893496)
--(axis cs:224,9.63060902048541)
--(axis cs:223,9.63060902048541)
--(axis cs:222,9.63060902048541)
--(axis cs:221,9.39097471255599)
--(axis cs:220,9.39097471255599)
--(axis cs:219,9.39097471255599)
--(axis cs:218,9.39097471255599)
--(axis cs:217,9.39097471255599)
--(axis cs:216,9.39097471255599)
--(axis cs:215,9.39097471255599)
--(axis cs:214,9.39097471255599)
--(axis cs:213,9.39097471255599)
--(axis cs:212,9.39097471255599)
--(axis cs:211,9.28062396959822)
--(axis cs:210,9.07079131507169)
--(axis cs:209,8.93655992802741)
--(axis cs:208,8.96986933049323)
--(axis cs:207,8.96986933049323)
--(axis cs:206,8.91797891790307)
--(axis cs:205,8.82010750039388)
--(axis cs:204,8.81912561348196)
--(axis cs:203,8.8710229774988)
--(axis cs:202,8.72404722563853)
--(axis cs:201,8.8710229774988)
--(axis cs:200,8.72404722563853)
--(axis cs:199,8.62053843437803)
--(axis cs:198,8.31443623683338)
--(axis cs:197,8.31443623683338)
--(axis cs:196,8.31443623683338)
--(axis cs:195,8.4956193052115)
--(axis cs:194,8.07941927087677)
--(axis cs:193,8.22557520648737)
--(axis cs:192,8.22557520648737)
--(axis cs:191,7.7744754411359)
--(axis cs:190,7.72414655206817)
--(axis cs:189,7.83696417414715)
--(axis cs:188,7.8239911390997)
--(axis cs:187,7.67683325911402)
--(axis cs:186,7.36575583651761)
--(axis cs:185,7.42064224361067)
--(axis cs:184,7.16733742295439)
--(axis cs:183,7.08744865824101)
--(axis cs:182,7.09832186685317)
--(axis cs:181,6.55098883902879)
--(axis cs:180,6.71213069489482)
--(axis cs:179,6.69318683945672)
--(axis cs:178,6.49023789523655)
--(axis cs:177,6.59399726226205)
--(axis cs:176,6.54687044957303)
--(axis cs:175,6.4124701682438)
--(axis cs:174,6.43560495069384)
--(axis cs:173,6.18847207468339)
--(axis cs:172,5.67258665525008)
--(axis cs:171,5.56086858677269)
--(axis cs:170,5.92986260798805)
--(axis cs:169,5.79330058187811)
--(axis cs:168,5.68133369993028)
--(axis cs:167,5.14128399581164)
--(axis cs:166,4.90191174503886)
--(axis cs:165,5.10903870276124)
--(axis cs:164,5.21862652744206)
--(axis cs:163,4.82188695914425)
--(axis cs:162,4.68678421406616)
--(axis cs:161,4.43332459959277)
--(axis cs:160,4.31551680859182)
--(axis cs:159,4.29593633452385)
--(axis cs:158,4.14170012993875)
--(axis cs:157,3.91748520964789)
--(axis cs:156,3.4114220543698)
--(axis cs:155,3.54533945629183)
--(axis cs:154,3.41198726044629)
--(axis cs:153,3.28466876406506)
--(axis cs:152,2.65104910518551)
--(axis cs:151,2.94188025732174)
--(axis cs:150,2.75927300243527)
--(axis cs:149,2.55586777717005)
--(axis cs:148,2.48430041824243)
--(axis cs:147,2.47787896131467)
--(axis cs:146,2.03813438776115)
--(axis cs:145,1.93710720711138)
--(axis cs:144,2.30899965880357)
--(axis cs:143,1.41274836213933)
--(axis cs:142,1.71570419005491)
--(axis cs:141,1.4963643107831)
--(axis cs:140,1.02982425097404)
--(axis cs:139,1.17786193262162)
--(axis cs:138,0.87452654791781)
--(axis cs:137,0.44720324565578)
--(axis cs:136,0.365533076987898)
--(axis cs:135,0.352049217615461)
--(axis cs:134,0.322919620678123)
--(axis cs:133,0.438526693331862)
--(axis cs:132,0.124411629136803)
--(axis cs:131,0.0974947595775881)
--(axis cs:130,0.000630053183966139)
--(axis cs:129,0.0704380277742201)
--(axis cs:128,-0.00401831109272405)
--(axis cs:127,-0.0611863940528392)
--(axis cs:126,-0.0614109664923493)
--(axis cs:125,-0.0387607011072279)
--(axis cs:124,-0.0461147569517038)
--(axis cs:123,-0.0233830633067894)
--(axis cs:122,0.00100272578894851)
--(axis cs:121,-0.0325461798836779)
--(axis cs:120,-0.0061656643030566)
--(axis cs:119,-0.0592273665163992)
--(axis cs:118,0.000109094271412572)
--(axis cs:117,0.00105453498631747)
--(axis cs:116,-0.00025154106694702)
--(axis cs:115,0.000114041804391589)
--(axis cs:114,0.00269944906382748)
--(axis cs:113,0.00130846814524679)
--(axis cs:112,0.00180521184577117)
--(axis cs:111,0.00122224752860047)
--(axis cs:110,0.00112591538816052)
--(axis cs:109,0.00122735136511586)
--(axis cs:108,7.98980183741417e-05)
--(axis cs:107,0.000785785915291786)
--(axis cs:106,0.00186983269780093)
--(axis cs:105,0.00158805178819297)
--(axis cs:104,0.00117254194413275)
--(axis cs:103,0.000409572088592756)
--(axis cs:102,0.00072115378408925)
--(axis cs:101,0.00105373903072594)
--(axis cs:100,0.00191563495436166)
--(axis cs:99,0.00430185865761323)
--(axis cs:98,0.00183289927087974)
--(axis cs:97,0.00128127443948005)
--(axis cs:96,0.000769560671619447)
--(axis cs:95,0.00271466845585678)
--(axis cs:94,0.00105019892774983)
--(axis cs:93,0.0012451848439687)
--(axis cs:92,0.00303982424586428)
--(axis cs:91,0.0034483661668042)
--(axis cs:90,0.00256781144609358)
--(axis cs:89,0.000463781331748504)
--(axis cs:88,0.00114467390526817)
--(axis cs:87,0.00118969869162828)
--(axis cs:86,0.00238460931489704)
--(axis cs:85,0.00270060565479109)
--(axis cs:84,0.00294370919651973)
--(axis cs:83,0.00281577132672112)
--(axis cs:82,0.000339242921370939)
--(axis cs:81,0.00082515341584323)
--(axis cs:80,0.00303027004397171)
--(axis cs:79,-0.000208445265261821)
--(axis cs:78,0.00328157382740512)
--(axis cs:77,0.00384883579876103)
--(axis cs:76,0.000410801159087645)
--(axis cs:75,0.00211249567576294)
--(axis cs:74,0.000420985023273591)
--(axis cs:73,0.0012754210430587)
--(axis cs:72,0.00307304785234519)
--(axis cs:71,0.000255804739431374)
--(axis cs:70,0.00211783757057722)
--(axis cs:69,0.00205224393934092)
--(axis cs:68,0.00248200407921804)
--(axis cs:67,0.00333036142428557)
--(axis cs:66,0.00233893833815723)
--(axis cs:65,0.00377504988344814)
--(axis cs:64,0.00141326865739259)
--(axis cs:63,0.00284978744018079)
--(axis cs:62,0.00520300351068978)
--(axis cs:61,0.00924187113100019)
--(axis cs:60,0.0036931796467979)
--(axis cs:59,0.00418473116041664)
--(axis cs:58,0.00249017222245435)
--(axis cs:57,0.00512592202926265)
--(axis cs:56,0.00826550930843441)
--(axis cs:55,0.00460319802044903)
--(axis cs:54,0.00300794162881171)
--(axis cs:53,0.00207160838272102)
--(axis cs:52,0.00422835613113986)
--(axis cs:51,0.00291498910440591)
--(axis cs:50,0.00495765760109007)
--(axis cs:49,0.00567778709858151)
--(axis cs:48,0.00330928368118679)
--(axis cs:47,0.00583885074835129)
--(axis cs:46,0.00617147795121317)
--(axis cs:45,0.00673089151069326)
--(axis cs:44,0.00645625125356879)
--(axis cs:43,0.0048068845973833)
--(axis cs:42,0.00371054047645888)
--(axis cs:41,0.00875716819036073)
--(axis cs:40,0.00498258314955537)
--(axis cs:39,0.0103588704156342)
--(axis cs:38,0.00661747410064495)
--(axis cs:37,0.00801792080206285)
--(axis cs:36,0.00819891322112611)
--(axis cs:35,0.00793359260088177)
--(axis cs:34,0.0068532481681356)
--(axis cs:33,0.00633440569644445)
--(axis cs:32,0.00508631575992213)
--(axis cs:31,0.00534998404996521)
--(axis cs:30,0.0133504180381154)
--(axis cs:29,0.00534091981054545)
--(axis cs:28,0.00445370985333096)
--(axis cs:27,0.0056046206567176)
--(axis cs:26,0.00705172258203062)
--(axis cs:25,0.0132301999372178)
--(axis cs:24,0.0116790319640687)
--(axis cs:23,0.00687832173858783)
--(axis cs:22,0.0108614645570491)
--(axis cs:21,0.0107150039770237)
--(axis cs:20,0.00609090684872741)
--(axis cs:19,0.0107218097948574)
--(axis cs:18,0.00925663417467893)
--(axis cs:17,0.0118649226868583)
--(axis cs:16,0.00767682028635085)
--(axis cs:15,0.00816967055463935)
--(axis cs:14,0.00732031305653399)
--(axis cs:13,0.00821150934891433)
--(axis cs:12,0.0104349730556027)
--(axis cs:11,0.0109224684478907)
--(axis cs:10,0.0144141517306336)
--(axis cs:9,0.0187150727866963)
--(axis cs:8,0.0137148637564228)
--(axis cs:7,0.0157523549654742)
--(axis cs:6,0.0159664053757756)
--(axis cs:5,0.0142155856784489)
--(axis cs:4,0.0132094022028955)
--(axis cs:3,0.00742992188036805)
--(axis cs:2,0.00606956746674405)
--(axis cs:1,0.0050079557222498)
--(axis cs:0,0.00823092684258694)
--cycle;

\path [fill=color1, fill opacity=0.3] (axis cs:0,0.0112454039685986)
--(axis cs:0,0.0211295960314014)
--(axis cs:1,0.0251694244938035)
--(axis cs:2,0.0324910632826005)
--(axis cs:3,0.0428027351192343)
--(axis cs:4,0.0440798609089635)
--(axis cs:5,0.0457919814191554)
--(axis cs:6,0.0449943475862456)
--(axis cs:7,0.0441877859033846)
--(axis cs:8,0.0434766571814508)
--(axis cs:9,0.0445380631819354)
--(axis cs:10,0.0442933307737974)
--(axis cs:11,0.0441877859033846)
--(axis cs:12,0.0445380631819354)
--(axis cs:13,0.0420826838233301)
--(axis cs:14,0.0417178149849583)
--(axis cs:15,0.0421858014668619)
--(axis cs:16,0.0455557539728697)
--(axis cs:17,0.0490037904107241)
--(axis cs:18,0.0482435963929269)
--(axis cs:19,0.047358738337358)
--(axis cs:20,0.0452065751370212)
--(axis cs:21,0.0438339291719946)
--(axis cs:22,0.0460080282934099)
--(axis cs:23,0.0462428090125956)
--(axis cs:24,0.0459012302478389)
--(axis cs:25,0.0447508736588593)
--(axis cs:26,0.0459012302478389)
--(axis cs:27,0.0452065751370212)
--(axis cs:28,0.0447508736588593)
--(axis cs:29,0.043937563042542)
--(axis cs:30,0.0432162731287603)
--(axis cs:31,0.0419187154638738)
--(axis cs:32,0.0425492404857353)
--(axis cs:33,0.0409790321332454)
--(axis cs:34,0.0395551809185559)
--(axis cs:35,0.0387970187757745)
--(axis cs:36,0.0391774427770935)
--(axis cs:37,0.0408785905312075)
--(axis cs:38,0.0405063556483102)
--(axis cs:39,0.0417178149849583)
--(axis cs:40,0.0433726496892972)
--(axis cs:41,0.0433726496892972)
--(axis cs:42,0.0452065751370212)
--(axis cs:43,0.0443966278704)
--(axis cs:44,0.043937563042542)
--(axis cs:45,0.0462428090125956)
--(axis cs:46,0.0465802735367695)
--(axis cs:47,0.0459012302478389)
--(axis cs:48,0.0478021412611309)
--(axis cs:49,0.0479160480969261)
--(axis cs:50,0.0486831076383477)
--(axis cs:51,0.0486831076383477)
--(axis cs:52,0.500737985944274)
--(axis cs:53,0.502895419746398)
--(axis cs:54,0.501611144284097)
--(axis cs:55,0.500750568547276)
--(axis cs:56,0.502471559665277)
--(axis cs:57,0.502035083027994)
--(axis cs:58,0.502458983189965)
--(axis cs:59,0.49987724683906)
--(axis cs:60,0.778853310469433)
--(axis cs:61,0.777990555762607)
--(axis cs:62,1.02497163471156)
--(axis cs:63,1.02286273164399)
--(axis cs:64,1.02076650029136)
--(axis cs:65,1.01951112980063)
--(axis cs:66,1.25104630845363)
--(axis cs:67,1.25272640856136)
--(axis cs:68,1.2527130784654)
--(axis cs:69,1.25104630845363)
--(axis cs:70,1.25064956338965)
--(axis cs:71,1.2506628975248)
--(axis cs:72,1.24772507520745)
--(axis cs:73,1.25104630845363)
--(axis cs:74,1.47186616994734)
--(axis cs:75,1.47145968785766)
--(axis cs:76,1.24937919399966)
--(axis cs:77,1.25063622922083)
--(axis cs:78,1.2498026728081)
--(axis cs:79,1.47022679119547)
--(axis cs:80,1.68448509708268)
--(axis cs:81,1.89020744782897)
--(axis cs:82,1.88899387138679)
--(axis cs:83,1.88776686401897)
--(axis cs:84,1.88655297329709)
--(axis cs:85,1.88612617933027)
--(axis cs:86,1.97284491147816)
--(axis cs:87,2.56429426501875)
--(axis cs:88,2.7525428032875)
--(axis cs:89,2.75292958760653)
--(axis cs:90,2.75174323563269)
--(axis cs:91,2.75214302766424)
--(axis cs:92,2.75175621234007)
--(axis cs:93,2.75174323563269)
--(axis cs:94,2.75174323563269)
--(axis cs:95,2.75134342719571)
--(axis cs:96,2.75214302766424)
--(axis cs:97,3.12275351825036)
--(axis cs:98,3.12431818181486)
--(axis cs:99,3.12394950574549)
--(axis cs:100,3.4626223173602)
--(axis cs:101,3.54156304278527)
--(axis cs:102,3.7949774265887)
--(axis cs:103,3.79533334944726)
--(axis cs:104,3.97176900326762)
--(axis cs:105,3.97065961840212)
--(axis cs:106,4.21746199627407)
--(axis cs:107,4.21894264558356)
--(axis cs:108,4.21820843929073)
--(axis cs:109,4.39099564345827)
--(axis cs:110,4.22077178402261)
--(axis cs:111,4.22039869144462)
--(axis cs:112,4.39172426491755)
--(axis cs:113,4.73013204227168)
--(axis cs:114,4.8966171656343)
--(axis cs:115,4.96851175316565)
--(axis cs:116,5.29651029141322)
--(axis cs:117,5.43684525029014)
--(axis cs:118,5.43615991119813)
--(axis cs:119,5.43685659142004)
--(axis cs:120,5.75677787833814)
--(axis cs:121,5.75609448847784)
--(axis cs:122,5.75643619438881)
--(axis cs:123,6.07215717418572)
--(axis cs:124,6.07018210463897)
--(axis cs:125,6.13864691622711)
--(axis cs:126,6.27101511999637)
--(axis cs:127,6.42576245464245)
--(axis cs:128,6.57820854866435)
--(axis cs:129,6.57820854866435)
--(axis cs:130,6.81555631278474)
--(axis cs:131,6.81494310373317)
--(axis cs:132,6.8152547542216)
--(axis cs:133,6.8152547542216)
--(axis cs:134,7.04959674366744)
--(axis cs:135,7.13384677494609)
--(axis cs:136,7.13415154046937)
--(axis cs:137,7.36391304206683)
--(axis cs:138,7.50777562577749)
--(axis cs:139,7.59124470549227)
--(axis cs:140,7.59095305194753)
--(axis cs:141,7.65132628309946)
--(axis cs:142,7.65103915938496)
--(axis cs:143,7.79249125314277)
--(axis cs:144,7.79221033296631)
--(axis cs:145,7.79249125314277)
--(axis cs:146,7.93153841746538)
--(axis cs:147,8.0694888581511)
--(axis cs:148,8.47298390429668)
--(axis cs:149,8.60328917028524)
--(axis cs:150,8.60328917028524)
--(axis cs:151,8.60328917028524)
--(axis cs:152,8.60305747233087)
--(axis cs:153,8.60374518322314)
--(axis cs:154,8.60351361351906)
--(axis cs:155,8.73196805644752)
--(axis cs:156,8.73175460983763)
--(axis cs:157,8.80738440942258)
--(axis cs:158,8.88283846370356)
--(axis cs:159,8.88368427838133)
--(axis cs:160,9.00741798287797)
--(axis cs:161,9.09991141314772)
--(axis cs:162,9.4309603042469)
--(axis cs:163,9.4309603042469)
--(axis cs:164,9.4309603042469)
--(axis cs:165,9.4309603042469)
--(axis cs:166,9.4309603042469)
--(axis cs:167,9.4309603042469)
--(axis cs:168,9.4309603042469)
--(axis cs:169,9.43110116255807)
--(axis cs:170,9.4309603042469)
--(axis cs:171,9.46930136534358)
--(axis cs:172,9.50461608837118)
--(axis cs:173,9.64096127302909)
--(axis cs:174,9.64096127302909)
--(axis cs:175,9.64096127302909)
--(axis cs:176,9.64096127302909)
--(axis cs:177,9.64096127302909)
--(axis cs:178,9.64096127302909)
--(axis cs:179,9.64096127302909)
--(axis cs:180,9.80066440072178)
--(axis cs:181,9.80066440072178)
--(axis cs:182,9.88266768451562)
--(axis cs:183,9.80066440072178)
--(axis cs:184,9.80070731179372)
--(axis cs:185,9.88266768451562)
--(axis cs:186,9.88266768451562)
--(axis cs:187,9.88266768451562)
--(axis cs:188,9.88266768451562)
--(axis cs:189,9.88266768451562)
--(axis cs:190,9.88266768451562)
--(axis cs:191,9.88266768451562)
--(axis cs:192,9.88266768451562)
--(axis cs:193,9.88266768451562)
--(axis cs:194,9.94271006887192)
--(axis cs:195,9.94271006887192)
--(axis cs:196,9.94271006887192)
--(axis cs:197,9.94271006887192)
--(axis cs:198,9.94271006887192)
--(axis cs:199,10.0012740499813)
--(axis cs:200,10.0581655862693)
--(axis cs:201,10.0581655862693)
--(axis cs:202,10.0581655862693)
--(axis cs:203,10.0581655862693)
--(axis cs:204,10.0581655862693)
--(axis cs:205,10.0581655862693)
--(axis cs:206,10.0581655862693)
--(axis cs:207,10.1025184820461)
--(axis cs:208,10.1025184820461)
--(axis cs:209,10.1025184820461)
--(axis cs:210,10.1025184820461)
--(axis cs:211,10.1467566935179)
--(axis cs:212,10.1467566935179)
--(axis cs:213,10.1467566935179)
--(axis cs:214,10.1467566935179)
--(axis cs:215,10.1467566935179)
--(axis cs:216,10.1467566935179)
--(axis cs:217,10.1467566935179)
--(axis cs:218,10.1467566935179)
--(axis cs:219,10.1467566935179)
--(axis cs:220,10.1467566935179)
--(axis cs:221,10.1467566935179)
--(axis cs:222,10.1467566935179)
--(axis cs:223,10.1467566935179)
--(axis cs:224,10.1467566935179)
--(axis cs:225,10.1467566935179)
--(axis cs:226,10.073799001065)
--(axis cs:227,10.073799001065)
--(axis cs:228,10.073799001065)
--(axis cs:229,10.073799001065)
--(axis cs:230,10.073799001065)
--(axis cs:231,10.073799001065)
--(axis cs:232,10.073799001065)
--(axis cs:233,10.073799001065)
--(axis cs:234,10.073799001065)
--(axis cs:235,10.073799001065)
--(axis cs:236,10.073799001065)
--(axis cs:237,10.073799001065)
--(axis cs:238,10.073799001065)
--(axis cs:239,10.073799001065)
--(axis cs:240,10.073799001065)
--(axis cs:241,10.073799001065)
--(axis cs:242,10.073799001065)
--(axis cs:243,10.073799001065)
--(axis cs:244,10.073799001065)
--(axis cs:245,10.073799001065)
--(axis cs:246,10.073799001065)
--(axis cs:247,10.073799001065)
--(axis cs:248,10.073799001065)
--(axis cs:249,10.073799001065)
--(axis cs:250,10.073799001065)
--(axis cs:251,10.073799001065)
--(axis cs:252,10.073799001065)
--(axis cs:253,10.073799001065)
--(axis cs:254,10.073799001065)
--(axis cs:255,10.073799001065)
--(axis cs:256,10.073799001065)
--(axis cs:257,10.073799001065)
--(axis cs:258,10.073799001065)
--(axis cs:259,10.073799001065)
--(axis cs:260,10.073799001065)
--(axis cs:261,10.073799001065)
--(axis cs:262,10.073799001065)
--(axis cs:263,10.073799001065)
--(axis cs:264,10.073799001065)
--(axis cs:265,10.073799001065)
--(axis cs:266,10)
--(axis cs:267,10)
--(axis cs:268,10)
--(axis cs:269,10)
--(axis cs:270,10)
--(axis cs:271,10)
--(axis cs:272,10)
--(axis cs:273,10)
--(axis cs:274,10)
--(axis cs:275,10)
--(axis cs:276,10)
--(axis cs:277,10)
--(axis cs:278,10)
--(axis cs:279,10)
--(axis cs:280,10)
--(axis cs:281,10)
--(axis cs:282,10)
--(axis cs:283,10)
--(axis cs:284,10)
--(axis cs:285,10)
--(axis cs:286,10)
--(axis cs:287,10)
--(axis cs:288,10)
--(axis cs:289,10)
--(axis cs:290,10)
--(axis cs:291,10)
--(axis cs:292,10)
--(axis cs:293,10)
--(axis cs:294,10)
--(axis cs:295,10)
--(axis cs:296,10)
--(axis cs:297,10)
--(axis cs:298,10)
--(axis cs:299,10)
--(axis cs:299,10)
--(axis cs:299,10)
--(axis cs:298,10)
--(axis cs:297,10)
--(axis cs:296,10)
--(axis cs:295,10)
--(axis cs:294,10)
--(axis cs:293,10)
--(axis cs:292,10)
--(axis cs:291,10)
--(axis cs:290,10)
--(axis cs:289,10)
--(axis cs:288,10)
--(axis cs:287,10)
--(axis cs:286,10)
--(axis cs:285,10)
--(axis cs:284,10)
--(axis cs:283,10)
--(axis cs:282,10)
--(axis cs:281,10)
--(axis cs:280,10)
--(axis cs:279,10)
--(axis cs:278,10)
--(axis cs:277,10)
--(axis cs:276,10)
--(axis cs:275,10)
--(axis cs:274,10)
--(axis cs:273,10)
--(axis cs:272,10)
--(axis cs:271,10)
--(axis cs:270,10)
--(axis cs:269,10)
--(axis cs:268,10)
--(axis cs:267,10)
--(axis cs:266,10)
--(axis cs:265,9.76995099893496)
--(axis cs:264,9.76995099893496)
--(axis cs:263,9.76995099893496)
--(axis cs:262,9.76995099893496)
--(axis cs:261,9.76995099893496)
--(axis cs:260,9.76995099893496)
--(axis cs:259,9.76995099893496)
--(axis cs:258,9.76995099893496)
--(axis cs:257,9.76995099893496)
--(axis cs:256,9.76995099893496)
--(axis cs:255,9.76995099893496)
--(axis cs:254,9.76995099893496)
--(axis cs:253,9.76995099893496)
--(axis cs:252,9.76995099893496)
--(axis cs:251,9.76995099893496)
--(axis cs:250,9.76995099893496)
--(axis cs:249,9.76995099893496)
--(axis cs:248,9.76995099893496)
--(axis cs:247,9.76995099893496)
--(axis cs:246,9.76995099893496)
--(axis cs:245,9.76995099893496)
--(axis cs:244,9.76995099893496)
--(axis cs:243,9.76995099893496)
--(axis cs:242,9.76995099893496)
--(axis cs:241,9.76995099893496)
--(axis cs:240,9.76995099893496)
--(axis cs:239,9.76995099893496)
--(axis cs:238,9.76995099893496)
--(axis cs:237,9.76995099893496)
--(axis cs:236,9.76995099893496)
--(axis cs:235,9.76995099893496)
--(axis cs:234,9.76995099893496)
--(axis cs:233,9.76995099893496)
--(axis cs:232,9.76995099893496)
--(axis cs:231,9.76995099893496)
--(axis cs:230,9.76995099893496)
--(axis cs:229,9.76995099893496)
--(axis cs:228,9.76995099893496)
--(axis cs:227,9.76995099893496)
--(axis cs:226,9.76995099893496)
--(axis cs:225,9.54252455648207)
--(axis cs:224,9.54252455648207)
--(axis cs:223,9.54252455648207)
--(axis cs:222,9.54252455648207)
--(axis cs:221,9.54252455648207)
--(axis cs:220,9.54252455648207)
--(axis cs:219,9.54252455648207)
--(axis cs:218,9.54252455648207)
--(axis cs:217,9.54252455648207)
--(axis cs:216,9.54252455648207)
--(axis cs:215,9.54252455648207)
--(axis cs:214,9.54252455648207)
--(axis cs:213,9.54252455648207)
--(axis cs:212,9.54252455648207)
--(axis cs:211,9.54252455648207)
--(axis cs:210,9.43051276795388)
--(axis cs:209,9.43051276795388)
--(axis cs:208,9.43051276795388)
--(axis cs:207,9.43051276795388)
--(axis cs:206,9.16414691373073)
--(axis cs:205,9.16414691373073)
--(axis cs:204,9.16414691373073)
--(axis cs:203,9.16414691373073)
--(axis cs:202,9.16414691373073)
--(axis cs:201,9.16414691373073)
--(axis cs:200,9.16414691373073)
--(axis cs:199,9.06478845001868)
--(axis cs:198,8.96710243112808)
--(axis cs:197,8.96710243112808)
--(axis cs:196,8.96710243112808)
--(axis cs:195,8.96710243112808)
--(axis cs:194,8.96710243112808)
--(axis cs:193,8.87089481548438)
--(axis cs:192,8.87089481548438)
--(axis cs:191,8.87089481548438)
--(axis cs:190,8.87089481548438)
--(axis cs:189,8.87089481548438)
--(axis cs:188,8.87089481548438)
--(axis cs:187,8.87089481548438)
--(axis cs:186,8.87089481548438)
--(axis cs:185,8.87089481548438)
--(axis cs:184,8.64213643820628)
--(axis cs:183,8.64039809927822)
--(axis cs:182,8.87089481548438)
--(axis cs:181,8.64039809927822)
--(axis cs:180,8.64039809927822)
--(axis cs:179,8.33313247697091)
--(axis cs:178,8.33313247697091)
--(axis cs:177,8.33313247697091)
--(axis cs:176,8.33313247697091)
--(axis cs:175,8.33313247697091)
--(axis cs:174,8.33313247697091)
--(axis cs:173,8.33313247697091)
--(axis cs:172,8.15697766162883)
--(axis cs:171,8.03872988465642)
--(axis cs:170,7.9216959457531)
--(axis cs:169,7.92246133744193)
--(axis cs:168,7.9216959457531)
--(axis cs:167,7.9216959457531)
--(axis cs:166,7.9216959457531)
--(axis cs:165,7.9216959457531)
--(axis cs:164,7.9216959457531)
--(axis cs:163,7.9216959457531)
--(axis cs:162,7.9216959457531)
--(axis cs:161,7.47327608685228)
--(axis cs:160,7.25505076712203)
--(axis cs:159,7.06806572161867)
--(axis cs:158,7.06534903629645)
--(axis cs:157,6.98633434057742)
--(axis cs:156,6.90840164016237)
--(axis cs:155,6.90906319355248)
--(axis cs:154,6.72501763648094)
--(axis cs:153,6.72569231677686)
--(axis cs:152,6.72369252766913)
--(axis cs:151,6.72436707971476)
--(axis cs:150,6.72436707971476)
--(axis cs:149,6.72436707971476)
--(axis cs:148,6.54395359570332)
--(axis cs:147,6.01351114184889)
--(axis cs:146,5.84074283253462)
--(axis cs:145,5.67085249685723)
--(axis cs:144,5.67022716703369)
--(axis cs:143,5.67085249685723)
--(axis cs:142,5.50067959061504)
--(axis cs:141,5.50129871690054)
--(axis cs:140,5.40629694805247)
--(axis cs:139,5.40691154450773)
--(axis cs:138,5.33144312422251)
--(axis cs:137,5.16549320793317)
--(axis cs:136,4.92828595953063)
--(axis cs:135,4.9276844750539)
--(axis cs:134,4.85659075633256)
--(axis cs:133,4.6221827457784)
--(axis cs:132,4.6221827457784)
--(axis cs:131,4.62158814626683)
--(axis cs:130,4.62275618721526)
--(axis cs:129,4.39135395133564)
--(axis cs:128,4.39135395133564)
--(axis cs:127,4.23220629535755)
--(axis cs:126,4.07267238000363)
--(axis cs:125,3.89791558377289)
--(axis cs:124,3.80831789536103)
--(axis cs:123,3.81168657581428)
--(axis cs:122,3.50331380561119)
--(axis cs:121,3.50274926152216)
--(axis cs:120,3.50387837166186)
--(axis cs:119,3.20236215857996)
--(axis cs:118,3.20124633880187)
--(axis cs:117,3.20234224970986)
--(axis cs:116,3.03373970858678)
--(axis cs:115,2.73939449683435)
--(axis cs:114,2.6568203343657)
--(axis cs:113,2.51258670772832)
--(axis cs:112,2.22690073508245)
--(axis cs:111,2.08660130855538)
--(axis cs:110,2.08713446597738)
--(axis cs:109,2.22584810654172)
--(axis cs:108,2.08347906070927)
--(axis cs:107,2.08452610441644)
--(axis cs:106,2.08241300372593)
--(axis cs:105,1.86674663159788)
--(axis cs:104,1.86829349673238)
--(axis cs:103,1.73041665055274)
--(axis cs:102,1.7299288234113)
--(axis cs:101,1.51906195721473)
--(axis cs:100,1.4399401826398)
--(axis cs:99,1.15633174425452)
--(axis cs:98,1.15680681818514)
--(axis cs:97,1.15477773174964)
--(axis cs:96,0.903106972335763)
--(axis cs:95,0.90209407280429)
--(axis cs:94,0.902600514367307)
--(axis cs:93,0.902600514367307)
--(axis cs:92,0.902618787659926)
--(axis cs:91,0.903106972335763)
--(axis cs:90,0.902600514367307)
--(axis cs:89,0.90410166239347)
--(axis cs:88,0.903613446712499)
--(axis cs:87,0.782924484981253)
--(axis cs:86,0.432373838521838)
--(axis cs:85,0.361936320669728)
--(axis cs:84,0.362447026702909)
--(axis cs:83,0.363889385981032)
--(axis cs:82,0.365349878613211)
--(axis cs:81,0.366792552171029)
--(axis cs:80,0.261796152917317)
--(axis cs:79,0.159991958804529)
--(axis cs:78,0.0679160771919024)
--(axis cs:77,0.0688637707791694)
--(axis cs:76,0.067433306000341)
--(axis cs:75,0.161415312142337)
--(axis cs:74,0.161883830052661)
--(axis cs:73,0.0693286915463678)
--(axis cs:72,0.0655561747925488)
--(axis cs:71,0.068899602475205)
--(axis cs:70,0.0688816866103458)
--(axis cs:69,0.0693286915463678)
--(axis cs:68,0.0712244215345991)
--(axis cs:67,0.0712423414386454)
--(axis cs:66,0.0693286915463678)
--(axis cs:65,-0.0125111298006322)
--(axis cs:64,-0.0111102502913567)
--(axis cs:63,-0.00876898164399154)
--(axis cs:62,-0.0064091347115629)
--(axis cs:61,-0.0710530557626066)
--(axis cs:60,-0.0701033104694329)
--(axis cs:59,-0.10540849683906)
--(axis cs:58,-0.102646483189964)
--(axis cs:57,-0.103097583027993)
--(axis cs:56,-0.102627809665276)
--(axis cs:55,-0.104469318547275)
--(axis cs:54,-0.103548644284097)
--(axis cs:53,-0.102176669746398)
--(axis cs:52,-0.104487985944274)
--(axis cs:51,0.0368481423616523)
--(axis cs:50,0.0368481423616523)
--(axis cs:49,0.0358339519030738)
--(axis cs:48,0.0359791087388691)
--(axis cs:47,0.0334112697521611)
--(axis cs:46,0.0345447264632304)
--(axis cs:45,0.0339759409874043)
--(axis cs:44,0.030906186957458)
--(axis cs:43,0.0313221221296)
--(axis cs:42,0.0322934248629788)
--(axis cs:41,0.0306273503107028)
--(axis cs:40,0.0306273503107028)
--(axis cs:39,0.0287196850150417)
--(axis cs:38,0.0272436443516898)
--(axis cs:37,0.0277776594687925)
--(axis cs:36,0.0259163072229065)
--(axis cs:35,0.0253904812242255)
--(axis cs:34,0.026444819081444)
--(axis cs:33,0.0276459678667546)
--(axis cs:32,0.0296695095142646)
--(axis cs:31,0.0284562845361262)
--(axis cs:30,0.0298149768712397)
--(axis cs:29,0.030906186957458)
--(axis cs:28,0.0318741263411407)
--(axis cs:27,0.0322934248629788)
--(axis cs:26,0.0334112697521611)
--(axis cs:25,0.0318741263411407)
--(axis cs:24,0.0334112697521611)
--(axis cs:23,0.0339759409874043)
--(axis cs:22,0.03327322170659)
--(axis cs:21,0.0310410708280054)
--(axis cs:20,0.0322934248629788)
--(axis cs:19,0.035547511662642)
--(axis cs:18,0.036412653607073)
--(axis cs:17,0.0374337095892759)
--(axis cs:16,0.0328504960271303)
--(axis cs:15,0.0291266985331381)
--(axis cs:14,0.0287196850150417)
--(axis cs:13,0.0292610661766699)
--(axis cs:12,0.0321494368180646)
--(axis cs:11,0.0315934640966154)
--(axis cs:10,0.0314566692262026)
--(axis cs:9,0.0321494368180646)
--(axis cs:8,0.0304920928185492)
--(axis cs:7,0.0315934640966154)
--(axis cs:6,0.0325681524137544)
--(axis cs:5,0.0335517685808446)
--(axis cs:4,0.0317326390910365)
--(axis cs:3,0.0303535148807656)
--(axis cs:2,0.0192589367173995)
--(axis cs:1,0.0132993255061966)
--(axis cs:0,0.0112454039685986)
--cycle;

\path [fill=color2, fill opacity=0.3] (axis cs:0,0.0125467930421031)
--(axis cs:0,0.0249532069578969)
--(axis cs:1,0.0238655349497666)
--(axis cs:2,0.0215408590814754)
--(axis cs:3,0.0305518614405651)
--(axis cs:4,0.0275145167325904)
--(axis cs:5,0.0278389350291508)
--(axis cs:6,0.0326854351461938)
--(axis cs:7,0.0293046085886192)
--(axis cs:8,0.0334972372818436)
--(axis cs:9,0.0311646516872443)
--(axis cs:10,0.0312623051582674)
--(axis cs:11,0.0220977038772393)
--(axis cs:12,0.022205078920869)
--(axis cs:13,0.022432580025455)
--(axis cs:14,0.0260266506355295)
--(axis cs:15,0.0274141257629471)
--(axis cs:16,0.0265584480083621)
--(axis cs:17,0.0307516075964109)
--(axis cs:18,0.028784581320649)
--(axis cs:19,0.0312623051582674)
--(axis cs:20,0.0292038336377071)
--(axis cs:21,0.0268853240479503)
--(axis cs:22,0.0269872885493093)
--(axis cs:23,0.0253866726916123)
--(axis cs:24,0.0224325800254549)
--(axis cs:25,0.0285779076853572)
--(axis cs:26,0.0239740866472531)
--(axis cs:27,0.0268853240479503)
--(axis cs:28,0.0225427607237878)
--(axis cs:29,0.0225427607237878)
--(axis cs:30,0.019620308364558)
--(axis cs:31,0.0246218038419338)
--(axis cs:32,0.0235348808574202)
--(axis cs:33,0.0254916211021913)
--(axis cs:34,0.0236406191166965)
--(axis cs:35,0.0200755240774804)
--(axis cs:36,0.015283629548521)
--(axis cs:37,0.0190479863457371)
--(axis cs:38,0.0231990768952273)
--(axis cs:39,0.0190479863457371)
--(axis cs:40,0.0220977038772393)
--(axis cs:41,0.0192659569463014)
--(axis cs:42,0.0202954082089194)
--(axis cs:43,0.0191629629606948)
--(axis cs:44,0.0175442850918984)
--(axis cs:45,0.0220977038772393)
--(axis cs:46,0.0181266923183063)
--(axis cs:47,0.0217577825629026)
--(axis cs:48,0.0231990768952273)
--(axis cs:49,0.0187034832020576)
--(axis cs:50,0.0236406191166965)
--(axis cs:51,0.0180084151531997)
--(axis cs:52,0.0160154246969362)
--(axis cs:53,0.0200755240774804)
--(axis cs:54,0.0200755240774804)
--(axis cs:55,0.0212014448388492)
--(axis cs:56,0.0220977038772393)
--(axis cs:57,0.0234266877868589)
--(axis cs:58,0.0209795733061309)
--(axis cs:59,0.0235348808574202)
--(axis cs:60,0.025057713312614)
--(axis cs:61,0.0264569978159527)
--(axis cs:62,0.0235348808574202)
--(axis cs:63,0.0275145167325904)
--(axis cs:64,0.0256948443884866)
--(axis cs:65,0.0259235088433022)
--(axis cs:66,0.0276128894944298)
--(axis cs:67,0.029821878083784)
--(axis cs:68,0.0274141257629471)
--(axis cs:69,0.0236406191166965)
--(axis cs:70,0.019620308364558)
--(axis cs:71,0.0201876230193151)
--(axis cs:72,0.0190479863457371)
--(axis cs:73,0.0262265507563942)
--(axis cs:74,0.0190479863457371)
--(axis cs:75,0.0169558048738729)
--(axis cs:76,0.0181266923183063)
--(axis cs:77,0.0226503662238521)
--(axis cs:78,0.0230936367788346)
--(axis cs:79,0.0181266923183063)
--(axis cs:80,0.0200755240774804)
--(axis cs:81,0.0137059364145749)
--(axis cs:82,0.0205286124048076)
--(axis cs:83,0.0229857588323091)
--(axis cs:84,0.0246218038419338)
--(axis cs:85,0.02064078390082)
--(axis cs:86,0.0266578229912873)
--(axis cs:87,0.0216505965411074)
--(axis cs:88,0.0215408590814754)
--(axis cs:89,0.02064078390082)
--(axis cs:90,0.0170778861214893)
--(axis cs:91,0.022205078920869)
--(axis cs:92,0.0187034832020576)
--(axis cs:93,0.00451044865568763)
--(axis cs:94,0.0141894007119645)
--(axis cs:95,0.0191629629606948)
--(axis cs:96,0.0143263152845355)
--(axis cs:97,0.0187034832020576)
--(axis cs:98,0.0200755240774804)
--(axis cs:99,0.0144586659753149)
--(axis cs:100,0.0176627484662686)
--(axis cs:101,0.0170778861214893)
--(axis cs:102,0.0170778861214893)
--(axis cs:103,0.019732380625578)
--(axis cs:104,0.0161382123154771)
--(axis cs:105,0.0192750544845079)
--(axis cs:106,0.0144586659753149)
--(axis cs:107,0.0175442850918984)
--(axis cs:108,0.0181266923183063)
--(axis cs:109,0.0192750544845079)
--(axis cs:110,0.0161382123154771)
--(axis cs:111,0.0125854606724321)
--(axis cs:112,0.0149379348043453)
--(axis cs:113,0.016609201719911)
--(axis cs:114,0.0158889300824498)
--(axis cs:115,0.0184702898827551)
--(axis cs:116,0.0164867982553063)
--(axis cs:117,0.0187034832020576)
--(axis cs:118,0.237034456580951)
--(axis cs:119,0.0139768324152925)
--(axis cs:120,0.0154146779690684)
--(axis cs:121,0.238785326597845)
--(axis cs:122,0.0127305599533959)
--(axis cs:123,0.0156235143569605)
--(axis cs:124,0.0115956662208029)
--(axis cs:125,0.0185884349250672)
--(axis cs:126,0.687318888890836)
--(axis cs:127,0.950102766874603)
--(axis cs:128,0.946745109147614)
--(axis cs:129,0.692834856425988)
--(axis cs:130,0.946731999339815)
--(axis cs:131,0.686914044513496)
--(axis cs:132,0.947584683939339)
--(axis cs:133,1.62620389252764)
--(axis cs:134,1.40744013751938)
--(axis cs:135,1.41074050990015)
--(axis cs:136,1.67985645590826)
--(axis cs:137,1.18419588920078)
--(axis cs:138,1.37007341387491)
--(axis cs:139,1.37006032604052)
--(axis cs:140,1.80170600793187)
--(axis cs:141,2.00699049607611)
--(axis cs:142,2.3727049156766)
--(axis cs:143,2.56824427507301)
--(axis cs:144,2.92440405920692)
--(axis cs:145,2.92403849564212)
--(axis cs:146,3.00168623079282)
--(axis cs:147,3.00324096962757)
--(axis cs:148,3.37204082282826)
--(axis cs:149,3.98382965779232)
--(axis cs:150,3.62767931331468)
--(axis cs:151,4.30453000242361)
--(axis cs:152,3.90625721520745)
--(axis cs:153,4.40096924705012)
--(axis cs:154,4.64707616734629)
--(axis cs:155,4.81370447289667)
--(axis cs:156,4.98022894740379)
--(axis cs:157,4.98129129466364)
--(axis cs:158,4.64565905006664)
--(axis cs:159,4.88747990500018)
--(axis cs:160,5.05267920445871)
--(axis cs:161,5.54110700231696)
--(axis cs:162,5.44991195226834)
--(axis cs:163,5.47057645130738)
--(axis cs:164,6.48216098684476)
--(axis cs:165,6.46226546881469)
--(axis cs:166,6.52524075300174)
--(axis cs:167,6.54726050550313)
--(axis cs:168,6.76378779473227)
--(axis cs:169,7.41100876861969)
--(axis cs:170,6.91245104271804)
--(axis cs:171,7.05996847849361)
--(axis cs:172,7.20830574690894)
--(axis cs:173,7.63508615573165)
--(axis cs:174,7.83228065857291)
--(axis cs:175,7.69326828801725)
--(axis cs:176,7.96865449487907)
--(axis cs:177,8.10353427821207)
--(axis cs:178,8.23579348388878)
--(axis cs:179,8.23579348388878)
--(axis cs:180,8.15755342268881)
--(axis cs:181,8.15664113875478)
--(axis cs:182,8.15664113875478)
--(axis cs:183,8.44411930510518)
--(axis cs:184,8.31498203592431)
--(axis cs:185,8.39330764319254)
--(axis cs:186,8.47122267369772)
--(axis cs:187,8.64868109781094)
--(axis cs:188,8.72404296552476)
--(axis cs:189,8.890625)
--(axis cs:190,8.84598176680645)
--(axis cs:191,8.84722150843472)
--(axis cs:192,8.92118285152573)
--(axis cs:193,8.96498828285809)
--(axis cs:194,9.00685822150695)
--(axis cs:195,9.00657614902603)
--(axis cs:196,9.03852024261064)
--(axis cs:197,9.15188827278369)
--(axis cs:198,9.29476834009959)
--(axis cs:199,9.29464868487917)
--(axis cs:200,9.29453191964333)
--(axis cs:201,9.36491365821981)
--(axis cs:202,9.3651371932862)
--(axis cs:203,9.43500521057438)
--(axis cs:204,9.50403573091208)
--(axis cs:205,9.50403573091208)
--(axis cs:206,9.50403573091208)
--(axis cs:207,9.56462329153905)
--(axis cs:208,9.57260770632104)
--(axis cs:209,9.60377453510564)
--(axis cs:210,9.60383252155562)
--(axis cs:211,9.65100178880679)
--(axis cs:212,9.63030269666826)
--(axis cs:213,9.63032222886937)
--(axis cs:214,9.69505395796027)
--(axis cs:215,9.75877348672159)
--(axis cs:216,9.834375)
--(axis cs:217,9.834375)
--(axis cs:218,9.83417658334213)
--(axis cs:219,9.88831372937606)
--(axis cs:220,9.88831372937606)
--(axis cs:221,9.88831372937606)
--(axis cs:222,9.88831372937606)
--(axis cs:223,9.88831372937606)
--(axis cs:224,9.88831372937606)
--(axis cs:225,9.88831372937606)
--(axis cs:226,9.88831372937606)
--(axis cs:227,9.93812603040178)
--(axis cs:228,9.98402528744401)
--(axis cs:229,9.98402528744401)
--(axis cs:230,9.98402528744401)
--(axis cs:231,9.98402528744401)
--(axis cs:232,9.98402528744401)
--(axis cs:233,9.98402528744401)
--(axis cs:234,10.0245545756504)
--(axis cs:235,10.0245545756504)
--(axis cs:236,10.0245545756504)
--(axis cs:237,10.0245545756504)
--(axis cs:238,10.0245545756504)
--(axis cs:239,10.0245545756504)
--(axis cs:240,10.0245545756504)
--(axis cs:241,10.0245545756504)
--(axis cs:242,10.0245545756504)
--(axis cs:243,10.0245545756504)
--(axis cs:244,10.0245545756504)
--(axis cs:245,10.0245545756504)
--(axis cs:246,10.0245545756504)
--(axis cs:247,10.0568909795146)
--(axis cs:248,10.0568909795146)
--(axis cs:249,10.0568909795146)
--(axis cs:250,10.0568909795146)
--(axis cs:251,10.0568909795146)
--(axis cs:252,10.073799001065)
--(axis cs:253,10.073799001065)
--(axis cs:254,10.073799001065)
--(axis cs:255,10.073799001065)
--(axis cs:256,10.073799001065)
--(axis cs:257,10.073799001065)
--(axis cs:258,10.073799001065)
--(axis cs:259,10.073799001065)
--(axis cs:260,10.073799001065)
--(axis cs:261,10.073799001065)
--(axis cs:262,10.073799001065)
--(axis cs:263,10.073799001065)
--(axis cs:264,10.073799001065)
--(axis cs:265,10.073799001065)
--(axis cs:266,10)
--(axis cs:267,10)
--(axis cs:268,10)
--(axis cs:269,10)
--(axis cs:270,10)
--(axis cs:271,10)
--(axis cs:272,10)
--(axis cs:273,10)
--(axis cs:274,10)
--(axis cs:275,10)
--(axis cs:276,10)
--(axis cs:277,10)
--(axis cs:278,10)
--(axis cs:279,10)
--(axis cs:280,10)
--(axis cs:281,10)
--(axis cs:282,10)
--(axis cs:283,10)
--(axis cs:284,10)
--(axis cs:285,10)
--(axis cs:286,10)
--(axis cs:287,10)
--(axis cs:288,10)
--(axis cs:289,10)
--(axis cs:290,10)
--(axis cs:291,10)
--(axis cs:292,10)
--(axis cs:293,10)
--(axis cs:294,10)
--(axis cs:295,10)
--(axis cs:296,10)
--(axis cs:297,10)
--(axis cs:298,10)
--(axis cs:299,10)
--(axis cs:299,10)
--(axis cs:299,10)
--(axis cs:298,10)
--(axis cs:297,10)
--(axis cs:296,10)
--(axis cs:295,10)
--(axis cs:294,10)
--(axis cs:293,10)
--(axis cs:292,10)
--(axis cs:291,10)
--(axis cs:290,10)
--(axis cs:289,10)
--(axis cs:288,10)
--(axis cs:287,10)
--(axis cs:286,10)
--(axis cs:285,10)
--(axis cs:284,10)
--(axis cs:283,10)
--(axis cs:282,10)
--(axis cs:281,10)
--(axis cs:280,10)
--(axis cs:279,10)
--(axis cs:278,10)
--(axis cs:277,10)
--(axis cs:276,10)
--(axis cs:275,10)
--(axis cs:274,10)
--(axis cs:273,10)
--(axis cs:272,10)
--(axis cs:271,10)
--(axis cs:270,10)
--(axis cs:269,10)
--(axis cs:268,10)
--(axis cs:267,10)
--(axis cs:266,10)
--(axis cs:265,9.76995099893496)
--(axis cs:264,9.76995099893496)
--(axis cs:263,9.76995099893496)
--(axis cs:262,9.76995099893496)
--(axis cs:261,9.76995099893496)
--(axis cs:260,9.76995099893496)
--(axis cs:259,9.76995099893496)
--(axis cs:258,9.76995099893496)
--(axis cs:257,9.76995099893496)
--(axis cs:256,9.76995099893496)
--(axis cs:255,9.76995099893496)
--(axis cs:254,9.76995099893496)
--(axis cs:253,9.76995099893496)
--(axis cs:252,9.76995099893496)
--(axis cs:251,9.63060902048541)
--(axis cs:250,9.63060902048541)
--(axis cs:249,9.63060902048541)
--(axis cs:248,9.63060902048541)
--(axis cs:247,9.63060902048541)
--(axis cs:246,9.50669542434962)
--(axis cs:245,9.50669542434962)
--(axis cs:244,9.50669542434962)
--(axis cs:243,9.50669542434962)
--(axis cs:242,9.50669542434962)
--(axis cs:241,9.50669542434962)
--(axis cs:240,9.50669542434962)
--(axis cs:239,9.50669542434962)
--(axis cs:238,9.50669542434962)
--(axis cs:237,9.50669542434962)
--(axis cs:236,9.50669542434962)
--(axis cs:235,9.50669542434962)
--(axis cs:234,9.50669542434962)
--(axis cs:233,9.39097471255599)
--(axis cs:232,9.39097471255599)
--(axis cs:231,9.39097471255599)
--(axis cs:230,9.39097471255599)
--(axis cs:229,9.39097471255599)
--(axis cs:228,9.39097471255599)
--(axis cs:227,9.28062396959822)
--(axis cs:226,9.17418627062394)
--(axis cs:225,9.17418627062394)
--(axis cs:224,9.17418627062394)
--(axis cs:223,9.17418627062394)
--(axis cs:222,9.17418627062394)
--(axis cs:221,9.17418627062394)
--(axis cs:220,9.17418627062394)
--(axis cs:219,9.17418627062394)
--(axis cs:218,8.91760466665787)
--(axis cs:217,8.915625)
--(axis cs:216,8.915625)
--(axis cs:215,8.67872651327841)
--(axis cs:214,8.58797729203973)
--(axis cs:213,8.49558402113063)
--(axis cs:212,8.49469730333174)
--(axis cs:211,8.63024821119321)
--(axis cs:210,8.36669872844438)
--(axis cs:209,8.36588171489436)
--(axis cs:208,8.23989229367896)
--(axis cs:207,8.40412670846095)
--(axis cs:206,8.15221426908792)
--(axis cs:205,8.15221426908792)
--(axis cs:204,8.15221426908792)
--(axis cs:203,8.06677603942562)
--(axis cs:202,7.9803940567138)
--(axis cs:201,7.97883634178019)
--(axis cs:200,7.89387433035667)
--(axis cs:199,7.89463256512083)
--(axis cs:198,7.89541915990041)
--(axis cs:197,7.72489297721631)
--(axis cs:196,7.52576100738937)
--(axis cs:195,7.55592385097397)
--(axis cs:194,7.55742302849305)
--(axis cs:193,7.44304296714191)
--(axis cs:192,7.33150464847427)
--(axis cs:191,7.25277849156528)
--(axis cs:190,7.24776823319355)
--(axis cs:189,7.359375)
--(axis cs:188,7.05811328447524)
--(axis cs:187,6.98078765218906)
--(axis cs:186,6.68680857630228)
--(axis cs:185,6.60847360680746)
--(axis cs:184,6.53054921407569)
--(axis cs:183,6.71213069489482)
--(axis cs:182,6.37460886124522)
--(axis cs:181,6.37460886124522)
--(axis cs:180,6.3772590773112)
--(axis cs:179,6.45170651611122)
--(axis cs:178,6.45170651611122)
--(axis cs:177,6.27327822178793)
--(axis cs:176,6.09562675512093)
--(axis cs:175,5.74691921198275)
--(axis cs:174,5.92128184142709)
--(axis cs:173,5.64885134426835)
--(axis cs:172,5.14703800309106)
--(axis cs:171,4.97662527150639)
--(axis cs:170,4.81254895728197)
--(axis cs:169,5.40595998138031)
--(axis cs:168,4.65049345526773)
--(axis cs:167,4.39558324449687)
--(axis cs:166,4.41760299699826)
--(axis cs:165,4.33057828118531)
--(axis cs:164,4.30533901315524)
--(axis cs:163,3.28479854869262)
--(axis cs:162,3.30811929773165)
--(axis cs:161,3.37226799768304)
--(axis cs:160,2.92497704554129)
--(axis cs:159,2.77945759499982)
--(axis cs:158,2.54896594993336)
--(axis cs:157,2.84105245533636)
--(axis cs:156,2.83939605259621)
--(axis cs:155,2.69345177710333)
--(axis cs:154,2.55111133265371)
--(axis cs:153,2.32228075294988)
--(axis cs:152,1.88568028479255)
--(axis cs:151,2.26421999757639)
--(axis cs:150,1.69375818668532)
--(axis cs:149,1.96792034220768)
--(axis cs:148,1.48689667717174)
--(axis cs:147,1.22982153037243)
--(axis cs:146,1.22775126920718)
--(axis cs:145,1.15546150435788)
--(axis cs:144,1.15593969079308)
--(axis cs:143,0.879943224926995)
--(axis cs:142,0.760357584323401)
--(axis cs:141,0.50816575392389)
--(axis cs:140,0.401856492068127)
--(axis cs:139,0.207595923959478)
--(axis cs:138,0.207614086125091)
--(axis cs:137,0.0809916107992223)
--(axis cs:136,0.369268544091736)
--(axis cs:135,0.170478240099847)
--(axis cs:134,0.166653612480623)
--(axis cs:133,0.26663985747236)
--(axis cs:132,0.00779031606066094)
--(axis cs:131,-0.0467265445134956)
--(axis cs:130,0.00683050066018476)
--(axis cs:129,-0.0402098564259875)
--(axis cs:128,0.0068486408523859)
--(axis cs:127,0.0106159831253967)
--(axis cs:126,-0.0462876388908361)
--(axis cs:125,0.00728656507493285)
--(axis cs:124,0.00268558377919713)
--(axis cs:123,0.00587648564303952)
--(axis cs:122,0.00333194004660409)
--(axis cs:121,-0.0646603265978453)
--(axis cs:120,0.00511657203093166)
--(axis cs:119,0.00383566758470751)
--(axis cs:118,-0.0665032065809508)
--(axis cs:117,0.00714026679794241)
--(axis cs:116,0.00582570174469376)
--(axis cs:115,0.00743596011724493)
--(axis cs:114,0.00554856991755017)
--(axis cs:113,0.00567204828008899)
--(axis cs:112,0.00468706519565474)
--(axis cs:111,0.00350828932756791)
--(axis cs:110,0.0052367876845229)
--(axis cs:109,0.0074436955154921)
--(axis cs:108,0.00684205768169373)
--(axis cs:107,0.00654946490810165)
--(axis cs:106,0.00426008402468513)
--(axis cs:105,0.0074436955154921)
--(axis cs:104,0.0052367876845229)
--(axis cs:103,0.00789261937442201)
--(axis cs:102,0.00610961387851073)
--(axis cs:101,0.00610961387851073)
--(axis cs:100,0.00639975153373137)
--(axis cs:99,0.00426008402468513)
--(axis cs:98,0.00848697592251964)
--(axis cs:97,0.00714026679794241)
--(axis cs:96,0.00442368471546449)
--(axis cs:95,0.0075870370393052)
--(axis cs:94,0.00459184928803552)
--(axis cs:93,-1.04486556876267e-05)
--(axis cs:92,0.00714026679794241)
--(axis cs:91,0.00988867107913105)
--(axis cs:90,0.00610961387851073)
--(axis cs:89,0.00879671609918004)
--(axis cs:88,0.0097091409185246)
--(axis cs:87,0.0095681534588926)
--(axis cs:86,0.0134359270087127)
--(axis cs:85,0.00879671609918004)
--(axis cs:84,0.0119406961580662)
--(axis cs:83,0.0109517411676909)
--(axis cs:82,0.00894013759519241)
--(axis cs:81,0.00416906358542512)
--(axis cs:80,0.00848697592251964)
--(axis cs:79,0.00684205768169373)
--(axis cs:78,0.0108126132211655)
--(axis cs:77,0.010349633776148)
--(axis cs:76,0.00684205768169373)
--(axis cs:75,0.00626294512612714)
--(axis cs:74,0.00773326365426291)
--(axis cs:73,0.0129609492436058)
--(axis cs:72,0.00773326365426291)
--(axis cs:71,0.00834362698068495)
--(axis cs:70,0.00803594163544204)
--(axis cs:69,0.0111406308833036)
--(axis cs:68,0.0145233742370529)
--(axis cs:67,0.016553121916216)
--(axis cs:66,0.0142621105055702)
--(axis cs:65,0.0133577411566978)
--(axis cs:64,0.0126176556115134)
--(axis cs:63,0.0143917332674096)
--(axis cs:62,0.0112776191425798)
--(axis cs:61,0.0136992521840474)
--(axis cs:60,0.012411036687386)
--(axis cs:59,0.0112776191425798)
--(axis cs:58,0.00939542669386914)
--(axis cs:57,0.0114170622131411)
--(axis cs:56,0.0100272961227608)
--(axis cs:55,0.00911105516115085)
--(axis cs:54,0.00848697592251964)
--(axis cs:53,0.00848697592251964)
--(axis cs:52,0.00539082530306382)
--(axis cs:51,0.00699158484680031)
--(axis cs:50,0.0111406308833036)
--(axis cs:49,0.00714026679794241)
--(axis cs:48,0.0106759231047728)
--(axis cs:47,0.00942971743709742)
--(axis cs:46,0.00684205768169373)
--(axis cs:45,0.0100272961227608)
--(axis cs:44,0.00654946490810165)
--(axis cs:43,0.0075870370393052)
--(axis cs:42,0.00923584179108065)
--(axis cs:41,0.00848404305369864)
--(axis cs:40,0.0100272961227607)
--(axis cs:39,0.00773326365426291)
--(axis cs:38,0.0106759231047728)
--(axis cs:37,0.00773326365426291)
--(axis cs:36,0.00527887045147903)
--(axis cs:35,0.00848697592251964)
--(axis cs:34,0.0111406308833036)
--(axis cs:33,0.0128833788978087)
--(axis cs:32,0.0112776191425798)
--(axis cs:31,0.0119406961580662)
--(axis cs:30,0.00803594163544203)
--(axis cs:29,0.0104884892762122)
--(axis cs:28,0.0104884892762122)
--(axis cs:27,0.0141771759520497)
--(axis cs:26,0.0117446633527469)
--(axis cs:25,0.0161095923146428)
--(axis cs:24,0.0106299199745451)
--(axis cs:23,0.0130195773083877)
--(axis cs:22,0.0140439614506907)
--(axis cs:21,0.0141771759520497)
--(axis cs:20,0.0163274163622929)
--(axis cs:19,0.0177689448417326)
--(axis cs:18,0.015840418679351)
--(axis cs:17,0.0174046424035891)
--(axis cs:16,0.0135665519916379)
--(axis cs:15,0.0145233742370529)
--(axis cs:14,0.0132233493644705)
--(axis cs:13,0.0106299199745451)
--(axis cs:12,0.00988867107913104)
--(axis cs:11,0.0100272961227608)
--(axis cs:10,0.0177689448417326)
--(axis cs:9,0.0178978483127557)
--(axis cs:8,0.0200027627181564)
--(axis cs:7,0.0161953914113808)
--(axis cs:6,0.0190020648538062)
--(axis cs:5,0.0150048149708492)
--(axis cs:4,0.0143917332674096)
--(axis cs:3,0.0176668885594349)
--(axis cs:2,0.0097091409185246)
--(axis cs:1,0.0118844650502334)
--(axis cs:0,0.0125467930421031)
--cycle;

\path [fill=color3, fill opacity=0.3] (axis cs:0,0.00894013759519241)
--(axis cs:0,0.0205286124048076)
--(axis cs:1,0.0210918633079414)
--(axis cs:2,0.0269872885493093)
--(axis cs:3,0.0260266506355295)
--(axis cs:4,0.0288846932993362)
--(axis cs:5,0.0316740122739969)
--(axis cs:6,0.0270871520306579)
--(axis cs:7,0.0240801639346782)
--(axis cs:8,0.0254916211021913)
--(axis cs:9,0.0260266506355295)
--(axis cs:10,0.0215408590814754)
--(axis cs:11,0.025057713312614)
--(axis cs:12,0.02064078390082)
--(axis cs:13,0.0190479863457371)
--(axis cs:14,0.0231990768952273)
--(axis cs:15,0.0249532069578969)
--(axis cs:16,0.0259235088433022)
--(axis cs:17,0.0246218038419338)
--(axis cs:18,0.0201876230193151)
--(axis cs:19,0.023426687786859)
--(axis cs:20,0.0236406191166965)
--(axis cs:21,0.0265584480083621)
--(axis cs:22,0.0273116055690234)
--(axis cs:23,0.0249532069578969)
--(axis cs:24,0.0245176982680706)
--(axis cs:25,0.0239740866472531)
--(axis cs:26,0.021428404395064)
--(axis cs:27,0.0229857588323091)
--(axis cs:28,0.0241839089556708)
--(axis cs:29,0.023426687786859)
--(axis cs:30,0.0181266923183063)
--(axis cs:31,0.0215408590814754)
--(axis cs:32,0.0231990768952273)
--(axis cs:33,0.0229857588323091)
--(axis cs:34,0.0170778861214893)
--(axis cs:35,0.0236406191166965)
--(axis cs:36,0.020750252800251)
--(axis cs:37,0.0191629629606948)
--(axis cs:38,0.0220977038772393)
--(axis cs:39,0.0160154246969362)
--(axis cs:40,0.019620308364558)
--(axis cs:41,0.0158889300824498)
--(axis cs:42,0.0144586659753149)
--(axis cs:43,0.0127305599533959)
--(axis cs:44,0.0144586659753149)
--(axis cs:45,0.0138437050041346)
--(axis cs:46,0.0141894007119645)
--(axis cs:47,0.0180084151531997)
--(axis cs:48,0.0180084151531997)
--(axis cs:49,0.0132196758782926)
--(axis cs:50,0.022205078920869)
--(axis cs:51,0.0175442850918984)
--(axis cs:52,0.0143263152845355)
--(axis cs:53,0.0229857588323091)
--(axis cs:54,0.0191629629606948)
--(axis cs:55,0.02064078390082)
--(axis cs:56,0.0154146779690684)
--(axis cs:57,0.0216505965411074)
--(axis cs:58,0.0115956662208029)
--(axis cs:59,0.0170778861214893)
--(axis cs:60,0.0202970223819249)
--(axis cs:61,0.02064078390082)
--(axis cs:62,0.0178867981465723)
--(axis cs:63,0.0216505965411074)
--(axis cs:64,0.0175442850918984)
--(axis cs:65,0.0237440428968468)
--(axis cs:66,0.0268853240479503)
--(axis cs:67,0.0270871520306579)
--(axis cs:68,0.019620308364558)
--(axis cs:69,0.0229857588323091)
--(axis cs:70,0.0240801639346782)
--(axis cs:71,0.0190479863457371)
--(axis cs:72,0.0237440428968468)
--(axis cs:73,0.0246218038419338)
--(axis cs:74,0.0220977038772393)
--(axis cs:75,0.0181266923183063)
--(axis cs:76,0.0169558048738729)
--(axis cs:77,0.0164867982553063)
--(axis cs:78,0.0190479863457371)
--(axis cs:79,0.0188085338710187)
--(axis cs:80,0.0193873134237302)
--(axis cs:81,0.0176627484662686)
--(axis cs:82,0.02064078390082)
--(axis cs:83,0.0209795733061309)
--(axis cs:84,0.019620308364558)
--(axis cs:85,0.02064078390082)
--(axis cs:86,0.0155416597112815)
--(axis cs:87,0.0161382123154771)
--(axis cs:88,0.016609201719911)
--(axis cs:89,0.0132196758782926)
--(axis cs:90,0.0122385224862907)
--(axis cs:91,0.0202970223819249)
--(axis cs:92,0.0132196758782926)
--(axis cs:93,0.017422465592459)
--(axis cs:94,0.0154146779690684)
--(axis cs:95,0.0155416597112815)
--(axis cs:96,0.0148062759183795)
--(axis cs:97,0.0182418620404714)
--(axis cs:98,0.237904927993672)
--(axis cs:99,0.0189299199982008)
--(axis cs:100,0.241393872468044)
--(axis cs:101,0.381400153459367)
--(axis cs:102,0.507718687226403)
--(axis cs:103,0.68905314007891)
--(axis cs:104,1.05177565558584)
--(axis cs:105,0.796095719601825)
--(axis cs:106,0.896787111030499)
--(axis cs:107,1.14197120081889)
--(axis cs:108,1.23753011625344)
--(axis cs:109,1.23506766619077)
--(axis cs:110,1.76898544076845)
--(axis cs:111,1.3314864358046)
--(axis cs:112,1.42516870670469)
--(axis cs:113,1.76978965108186)
--(axis cs:114,1.73594561181497)
--(axis cs:115,1.55979931272247)
--(axis cs:116,1.60985922126005)
--(axis cs:117,1.82806767879592)
--(axis cs:118,2.40335724736316)
--(axis cs:119,2.67947727389583)
--(axis cs:120,2.84203911443028)
--(axis cs:121,3.40438436388809)
--(axis cs:122,3.11096851362372)
--(axis cs:123,3.47890884113644)
--(axis cs:124,3.48406626238793)
--(axis cs:125,3.83842525589033)
--(axis cs:126,4.01613878385497)
--(axis cs:127,4.01467235636931)
--(axis cs:128,4.36086217384914)
--(axis cs:129,4.43609375052784)
--(axis cs:130,4.18758783742281)
--(axis cs:131,4.50811198755928)
--(axis cs:132,4.67551681620081)
--(axis cs:133,4.48325875067257)
--(axis cs:134,4.74788173137203)
--(axis cs:135,5.22225912183845)
--(axis cs:136,5.61084793136891)
--(axis cs:137,5.61246610932806)
--(axis cs:138,6.06210575717781)
--(axis cs:139,5.99547593689025)
--(axis cs:140,6.36880782459956)
--(axis cs:141,6.52167438664231)
--(axis cs:142,6.60664907264819)
--(axis cs:143,7.16981527114407)
--(axis cs:144,7.50879299225892)
--(axis cs:145,7.3128813299197)
--(axis cs:146,7.31365510605156)
--(axis cs:147,7.83396995172987)
--(axis cs:148,8.04469748983135)
--(axis cs:149,7.70049628782144)
--(axis cs:150,7.96512420843385)
--(axis cs:151,8.04508475944068)
--(axis cs:152,8.04547757211635)
--(axis cs:153,8.12476408401354)
--(axis cs:154,8.25087168912876)
--(axis cs:155,8.52662745875211)
--(axis cs:156,8.75276964094379)
--(axis cs:157,8.75305187659252)
--(axis cs:158,8.82698038853077)
--(axis cs:159,8.82670690329011)
--(axis cs:160,9.01141286778833)
--(axis cs:161,8.90064344780613)
--(axis cs:162,9.08282047447383)
--(axis cs:163,9.1537929505575)
--(axis cs:164,9.32301750991767)
--(axis cs:165,9.29372401818837)
--(axis cs:166,9.36275675046298)
--(axis cs:167,9.45704947794045)
--(axis cs:168,9.58735571203905)
--(axis cs:169,9.65100178880679)
--(axis cs:170,9.71345277436147)
--(axis cs:171,9.77462438651804)
--(axis cs:172,9.77447670982529)
--(axis cs:173,9.77447670982529)
--(axis cs:174,9.834375)
--(axis cs:175,9.89226880956464)
--(axis cs:176,9.89226880956464)
--(axis cs:177,9.98402528744401)
--(axis cs:178,10.0245545756504)
--(axis cs:179,10.0245545756504)
--(axis cs:180,10.0245545756504)
--(axis cs:181,10.0245545756504)
--(axis cs:182,10.0568909795146)
--(axis cs:183,10.0568909795146)
--(axis cs:184,10.0568909795146)
--(axis cs:185,10.0568909795146)
--(axis cs:186,10.0568909795146)
--(axis cs:187,10.073799001065)
--(axis cs:188,10)
--(axis cs:189,10)
--(axis cs:190,10)
--(axis cs:191,10)
--(axis cs:192,10)
--(axis cs:193,10)
--(axis cs:194,10)
--(axis cs:195,10)
--(axis cs:196,10)
--(axis cs:197,10)
--(axis cs:198,10)
--(axis cs:199,10)
--(axis cs:200,10)
--(axis cs:201,10)
--(axis cs:202,10)
--(axis cs:203,10)
--(axis cs:204,10)
--(axis cs:205,10)
--(axis cs:206,10)
--(axis cs:207,10)
--(axis cs:208,10)
--(axis cs:209,10)
--(axis cs:210,10)
--(axis cs:211,10)
--(axis cs:212,10)
--(axis cs:213,10)
--(axis cs:214,10)
--(axis cs:215,10)
--(axis cs:216,10)
--(axis cs:217,10)
--(axis cs:218,10)
--(axis cs:219,10)
--(axis cs:220,10)
--(axis cs:221,10)
--(axis cs:222,10)
--(axis cs:223,10)
--(axis cs:224,10)
--(axis cs:225,10)
--(axis cs:226,10)
--(axis cs:227,10)
--(axis cs:228,10)
--(axis cs:229,10)
--(axis cs:230,10)
--(axis cs:231,10)
--(axis cs:232,10)
--(axis cs:233,10)
--(axis cs:234,10)
--(axis cs:235,10)
--(axis cs:236,10)
--(axis cs:237,10)
--(axis cs:238,10)
--(axis cs:239,10)
--(axis cs:240,10)
--(axis cs:241,10)
--(axis cs:242,10)
--(axis cs:243,10)
--(axis cs:244,10)
--(axis cs:245,10)
--(axis cs:246,10)
--(axis cs:247,10)
--(axis cs:248,10)
--(axis cs:249,10)
--(axis cs:250,10)
--(axis cs:251,10)
--(axis cs:252,10)
--(axis cs:253,10)
--(axis cs:254,10)
--(axis cs:255,10)
--(axis cs:256,10)
--(axis cs:257,10)
--(axis cs:258,10)
--(axis cs:259,10)
--(axis cs:260,10)
--(axis cs:261,10)
--(axis cs:262,10)
--(axis cs:263,10)
--(axis cs:264,10)
--(axis cs:265,10)
--(axis cs:266,10)
--(axis cs:267,10)
--(axis cs:268,10)
--(axis cs:269,10)
--(axis cs:270,10)
--(axis cs:271,10)
--(axis cs:272,10)
--(axis cs:273,10)
--(axis cs:274,10)
--(axis cs:275,10)
--(axis cs:276,10)
--(axis cs:277,10)
--(axis cs:278,10)
--(axis cs:279,10)
--(axis cs:280,10)
--(axis cs:281,10)
--(axis cs:282,10)
--(axis cs:283,10)
--(axis cs:284,10)
--(axis cs:285,10)
--(axis cs:286,10)
--(axis cs:287,10)
--(axis cs:288,10)
--(axis cs:289,10)
--(axis cs:290,10)
--(axis cs:291,10)
--(axis cs:292,10)
--(axis cs:293,10)
--(axis cs:294,10)
--(axis cs:295,10)
--(axis cs:296,10)
--(axis cs:297,10)
--(axis cs:298,10)
--(axis cs:299,10)
--(axis cs:299,10)
--(axis cs:299,10)
--(axis cs:298,10)
--(axis cs:297,10)
--(axis cs:296,10)
--(axis cs:295,10)
--(axis cs:294,10)
--(axis cs:293,10)
--(axis cs:292,10)
--(axis cs:291,10)
--(axis cs:290,10)
--(axis cs:289,10)
--(axis cs:288,10)
--(axis cs:287,10)
--(axis cs:286,10)
--(axis cs:285,10)
--(axis cs:284,10)
--(axis cs:283,10)
--(axis cs:282,10)
--(axis cs:281,10)
--(axis cs:280,10)
--(axis cs:279,10)
--(axis cs:278,10)
--(axis cs:277,10)
--(axis cs:276,10)
--(axis cs:275,10)
--(axis cs:274,10)
--(axis cs:273,10)
--(axis cs:272,10)
--(axis cs:271,10)
--(axis cs:270,10)
--(axis cs:269,10)
--(axis cs:268,10)
--(axis cs:267,10)
--(axis cs:266,10)
--(axis cs:265,10)
--(axis cs:264,10)
--(axis cs:263,10)
--(axis cs:262,10)
--(axis cs:261,10)
--(axis cs:260,10)
--(axis cs:259,10)
--(axis cs:258,10)
--(axis cs:257,10)
--(axis cs:256,10)
--(axis cs:255,10)
--(axis cs:254,10)
--(axis cs:253,10)
--(axis cs:252,10)
--(axis cs:251,10)
--(axis cs:250,10)
--(axis cs:249,10)
--(axis cs:248,10)
--(axis cs:247,10)
--(axis cs:246,10)
--(axis cs:245,10)
--(axis cs:244,10)
--(axis cs:243,10)
--(axis cs:242,10)
--(axis cs:241,10)
--(axis cs:240,10)
--(axis cs:239,10)
--(axis cs:238,10)
--(axis cs:237,10)
--(axis cs:236,10)
--(axis cs:235,10)
--(axis cs:234,10)
--(axis cs:233,10)
--(axis cs:232,10)
--(axis cs:231,10)
--(axis cs:230,10)
--(axis cs:229,10)
--(axis cs:228,10)
--(axis cs:227,10)
--(axis cs:226,10)
--(axis cs:225,10)
--(axis cs:224,10)
--(axis cs:223,10)
--(axis cs:222,10)
--(axis cs:221,10)
--(axis cs:220,10)
--(axis cs:219,10)
--(axis cs:218,10)
--(axis cs:217,10)
--(axis cs:216,10)
--(axis cs:215,10)
--(axis cs:214,10)
--(axis cs:213,10)
--(axis cs:212,10)
--(axis cs:211,10)
--(axis cs:210,10)
--(axis cs:209,10)
--(axis cs:208,10)
--(axis cs:207,10)
--(axis cs:206,10)
--(axis cs:205,10)
--(axis cs:204,10)
--(axis cs:203,10)
--(axis cs:202,10)
--(axis cs:201,10)
--(axis cs:200,10)
--(axis cs:199,10)
--(axis cs:198,10)
--(axis cs:197,10)
--(axis cs:196,10)
--(axis cs:195,10)
--(axis cs:194,10)
--(axis cs:193,10)
--(axis cs:192,10)
--(axis cs:191,10)
--(axis cs:190,10)
--(axis cs:189,10)
--(axis cs:188,10)
--(axis cs:187,9.76995099893496)
--(axis cs:186,9.63060902048541)
--(axis cs:185,9.63060902048541)
--(axis cs:184,9.63060902048541)
--(axis cs:183,9.63060902048541)
--(axis cs:182,9.63060902048541)
--(axis cs:181,9.50669542434962)
--(axis cs:180,9.50669542434962)
--(axis cs:179,9.50669542434962)
--(axis cs:178,9.50669542434962)
--(axis cs:177,9.39097471255599)
--(axis cs:176,9.01576244043536)
--(axis cs:175,9.01576244043536)
--(axis cs:174,8.915625)
--(axis cs:173,8.82105454017471)
--(axis cs:172,8.82105454017471)
--(axis cs:171,8.81912561348196)
--(axis cs:170,8.72404722563853)
--(axis cs:169,8.63024821119321)
--(axis cs:168,8.53942553796096)
--(axis cs:167,8.35723177205955)
--(axis cs:166,8.14080574953702)
--(axis cs:165,8.05180723181163)
--(axis cs:164,8.17876374008233)
--(axis cs:163,7.8792382994425)
--(axis cs:162,7.79396077552617)
--(axis cs:161,7.50738780219387)
--(axis cs:160,7.71089963221167)
--(axis cs:159,7.42329309670989)
--(axis cs:158,7.42480086146923)
--(axis cs:157,7.34426062340748)
--(axis cs:156,7.34276160905621)
--(axis cs:155,7.09837254124789)
--(axis cs:154,6.75269081087124)
--(axis cs:153,6.56807966598646)
--(axis cs:152,6.49111617788365)
--(axis cs:151,6.48972774055932)
--(axis cs:150,6.41256329156615)
--(axis cs:149,6.05306621217856)
--(axis cs:148,6.48836501016865)
--(axis cs:147,6.23121754827013)
--(axis cs:146,5.50687614394844)
--(axis cs:145,5.5049624200803)
--(axis cs:144,5.77423825774108)
--(axis cs:143,5.33377847885593)
--(axis cs:142,4.64691342735181)
--(axis cs:141,4.58276311335769)
--(axis cs:140,4.41956717540044)
--(axis cs:139,4.01343031310975)
--(axis cs:138,4.10483174282219)
--(axis cs:137,3.61609639067194)
--(axis cs:136,3.61324581863109)
--(axis cs:135,3.23127212816155)
--(axis cs:134,2.76636826862797)
--(axis cs:133,2.56583499932743)
--(axis cs:132,2.68251443379918)
--(axis cs:131,2.54276301244071)
--(axis cs:130,2.23647466257719)
--(axis cs:129,2.46209374947215)
--(axis cs:128,2.37751282615085)
--(axis cs:127,2.10045264363069)
--(axis cs:126,2.10261121614503)
--(axis cs:125,1.96423099410967)
--(axis cs:124,1.70427748761207)
--(axis cs:123,1.69696615886356)
--(axis cs:122,1.44081273637628)
--(axis cs:121,1.62327188611191)
--(axis cs:120,1.24096088556972)
--(axis cs:119,1.09102272610417)
--(axis cs:118,0.902861502636844)
--(axis cs:117,0.551369821204081)
--(axis cs:116,0.452578278739949)
--(axis cs:115,0.347325687277527)
--(axis cs:114,0.478304388185026)
--(axis cs:113,0.439147848918138)
--(axis cs:112,0.318612543295313)
--(axis cs:111,0.253326064195398)
--(axis cs:110,0.438170809231554)
--(axis cs:109,0.187244833809233)
--(axis cs:108,0.190126133746562)
--(axis cs:107,0.127653799181106)
--(axis cs:106,0.0630253889695002)
--(axis cs:105,0.0101855303981747)
--(axis cs:104,0.0741305944141609)
--(axis cs:103,-0.0443656400789099)
--(axis cs:102,-0.0086249372264032)
--(axis cs:101,-0.0439314034593665)
--(axis cs:100,-0.0619251224680435)
--(axis cs:99,0.00788258000179922)
--(axis cs:98,-0.0655924279936721)
--(axis cs:97,0.0066956379595286)
--(axis cs:96,0.00484997408162056)
--(axis cs:95,0.00495834028871849)
--(axis cs:94,0.00511657203093166)
--(axis cs:93,0.00670253440754106)
--(axis cs:92,0.00374907412170737)
--(axis cs:91,0.00820297761807516)
--(axis cs:90,0.00291772751370931)
--(axis cs:89,0.00374907412170737)
--(axis cs:88,0.00567204828008899)
--(axis cs:87,0.0052367876845229)
--(axis cs:86,0.00495834028871849)
--(axis cs:85,0.00879671609918004)
--(axis cs:84,0.00803594163544203)
--(axis cs:83,0.00939542669386914)
--(axis cs:82,0.00879671609918004)
--(axis cs:81,0.00639975153373137)
--(axis cs:80,0.00833143657626981)
--(axis cs:79,0.00803521612898129)
--(axis cs:78,0.00773326365426291)
--(axis cs:77,0.00582570174469376)
--(axis cs:76,0.00626294512612714)
--(axis cs:75,0.00684205768169372)
--(axis cs:74,0.0100272961227607)
--(axis cs:73,0.0119406961580662)
--(axis cs:72,0.0110059571031532)
--(axis cs:71,0.00773326365426291)
--(axis cs:70,0.0116073360653218)
--(axis cs:69,0.0109517411676909)
--(axis cs:68,0.00803594163544204)
--(axis cs:67,0.0139128479693421)
--(axis cs:66,0.0141771759520497)
--(axis cs:65,0.0110059571031532)
--(axis cs:64,0.00654946490810165)
--(axis cs:63,0.0095681534588926)
--(axis cs:62,0.00714445185342775)
--(axis cs:61,0.00879671609918004)
--(axis cs:60,0.00820297761807516)
--(axis cs:59,0.00610961387851073)
--(axis cs:58,0.00268558377919713)
--(axis cs:57,0.0095681534588926)
--(axis cs:56,0.00511657203093166)
--(axis cs:55,0.00879671609918004)
--(axis cs:54,0.0075870370393052)
--(axis cs:53,0.0109517411676909)
--(axis cs:52,0.00442368471546449)
--(axis cs:51,0.00654946490810165)
--(axis cs:50,0.00988867107913105)
--(axis cs:49,0.00374907412170737)
--(axis cs:48,0.00699158484680031)
--(axis cs:47,0.00699158484680031)
--(axis cs:46,0.00459184928803552)
--(axis cs:45,0.0040000449958654)
--(axis cs:44,0.00426008402468513)
--(axis cs:43,0.00333194004660409)
--(axis cs:42,0.00426008402468513)
--(axis cs:41,0.00554856991755017)
--(axis cs:40,0.00803594163544203)
--(axis cs:39,0.00539082530306381)
--(axis cs:38,0.0100272961227607)
--(axis cs:37,0.0075870370393052)
--(axis cs:36,0.00865599719974902)
--(axis cs:35,0.0111406308833036)
--(axis cs:34,0.00610961387851073)
--(axis cs:33,0.0109517411676909)
--(axis cs:32,0.0106759231047728)
--(axis cs:31,0.0097091409185246)
--(axis cs:30,0.00684205768169373)
--(axis cs:29,0.0114170622131411)
--(axis cs:28,0.0114723410443292)
--(axis cs:27,0.0109517411676909)
--(axis cs:26,0.00985284560493598)
--(axis cs:25,0.0117446633527469)
--(axis cs:24,0.0120760517319294)
--(axis cs:23,0.0125467930421031)
--(axis cs:22,0.0146571444309766)
--(axis cs:21,0.0135665519916379)
--(axis cs:20,0.0111406308833036)
--(axis cs:19,0.0114170622131411)
--(axis cs:18,0.00834362698068495)
--(axis cs:17,0.0119406961580662)
--(axis cs:16,0.0133577411566978)
--(axis cs:15,0.0125467930421031)
--(axis cs:14,0.0106759231047728)
--(axis cs:13,0.00773326365426291)
--(axis cs:12,0.00879671609918004)
--(axis cs:11,0.012411036687386)
--(axis cs:10,0.0097091409185246)
--(axis cs:9,0.0132233493644705)
--(axis cs:8,0.0128833788978087)
--(axis cs:7,0.0116073360653218)
--(axis cs:6,0.0139128479693421)
--(axis cs:5,0.0182634877260031)
--(axis cs:4,0.0157090567006638)
--(axis cs:3,0.0132233493644705)
--(axis cs:2,0.0140439614506907)
--(axis cs:1,0.00925188669205867)
--(axis cs:0,0.00894013759519241)
--cycle;

\addplot [semithick, color0, dash pattern=on 1pt off 3pt on 3pt off 3pt, forget plot]
table [row sep=\\]{%
0	0.0136764705882353 \\
1	0.00984742647058824 \\
2	0.0110891544117647 \\
3	0.0125386029411765 \\
4	0.0190588235294118 \\
5	0.0202959558823529 \\
6	0.0223777573529412 \\
7	0.0221231617647059 \\
8	0.0193152573529412 \\
9	0.0248106617647059 \\
10	0.0199880514705882 \\
11	0.0166534926470588 \\
12	0.016 \\
13	0.0132095588235294 \\
14	0.0127426470588235 \\
15	0.0137601102941177 \\
16	0.0126001838235294 \\
17	0.0178897058823529 \\
18	0.014890625 \\
19	0.0162527573529412 \\
20	0.0111930147058824 \\
21	0.0164090073529412 \\
22	0.0164246323529412 \\
23	0.0117196691176471 \\
24	0.0174126838235294 \\
25	0.0192058823529412 \\
26	0.012234375 \\
27	0.01034375 \\
28	0.00923069852941177 \\
29	0.010217830882353 \\
30	0.0190128676470588 \\
31	0.0101452205882353 \\
32	0.00973988970588236 \\
33	0.0116461397058824 \\
34	0.0120101102941177 \\
35	0.0136121323529412 \\
36	0.0135863970588235 \\
37	0.0135955882352941 \\
38	0.0117821691176471 \\
39	0.0157270220588235 \\
40	0.00966636029411765 \\
41	0.0142472426470588 \\
42	0.00790808823529413 \\
43	0.00942555147058824 \\
44	0.0112058823529412 \\
45	0.0120625 \\
46	0.011264705882353 \\
47	0.0106415441176471 \\
48	0.00765349264705883 \\
49	0.0106957720588235 \\
50	0.00966911764705883 \\
51	0.00718933823529412 \\
52	0.00895128676470589 \\
53	0.00596691176470589 \\
54	0.00725275735294118 \\
55	0.0092591911764706 \\
56	0.0138961397058824 \\
57	0.00979963235294119 \\
58	0.00628216911764706 \\
59	0.00889338235294118 \\
60	0.00838051470588236 \\
61	0.0151369485294118 \\
62	0.0103621323529412 \\
63	0.00707261029411765 \\
64	0.00450827205882353 \\
65	0.00837316176470589 \\
66	0.0055670955882353 \\
67	0.00762591911764706 \\
68	0.00612775735294118 \\
69	0.00600091911764706 \\
70	0.00627205882352942 \\
71	0.00297702205882353 \\
72	0.00718014705882353 \\
73	0.00508455882352942 \\
74	0.00322150735294118 \\
75	0.0058639705882353 \\
76	0.00330238970588236 \\
77	0.00847242647058824 \\
78	0.00736948529411765 \\
79	0.00274080882352941 \\
80	0.00713327205882353 \\
81	0.00379319852941177 \\
82	0.00263235294117647 \\
83	0.00664430147058824 \\
84	0.00725643382352942 \\
85	0.00677297794117648 \\
86	0.00643382352941177 \\
87	0.004109375 \\
88	0.00494117647058824 \\
89	0.00285018382352941 \\
90	0.00682169117647059 \\
91	0.0077233455882353 \\
92	0.00708547794117648 \\
93	0.00428400735294118 \\
94	0.00431158088235294 \\
95	0.00677573529411765 \\
96	0.00321047794117647 \\
97	0.00483180147058824 \\
98	0.00554044117647059 \\
99	0.00891268382352942 \\
100	0.00559926470588236 \\
101	0.00403125 \\
102	0.00418198529411765 \\
103	0.0032545955882353 \\
104	0.00430698529411765 \\
105	0.0054889705882353 \\
106	0.00528216911764706 \\
107	0.00377113970588236 \\
108	0.00276746323529412 \\
109	0.00451102941176471 \\
110	0.00472058823529412 \\
111	0.00417922794117647 \\
112	0.00533455882352942 \\
113	0.00513511029411765 \\
114	0.00678033088235295 \\
115	0.00249540441176471 \\
116	0.00231341911764706 \\
117	0.003828125 \\
118	0.001796875 \\
119	0.0676268382352941 \\
120	0.00999540441176471 \\
121	0.0389136029411765 \\
122	0.0045514705882353 \\
123	0.0302205882352941 \\
124	0.0574724264705882 \\
125	0.179653492647059 \\
126	0.288482536764706 \\
127	0.381177389705882 \\
128	0.459840073529412 \\
129	0.593462316176471 \\
130	0.470913602941176 \\
131	0.618484375 \\
132	0.682942095588235 \\
133	1.12196691176471 \\
134	0.998728860294118 \\
135	0.982954963235294 \\
136	1.02817371323529 \\
137	1.11817279411765 \\
138	1.73143290441176 \\
139	2.07413602941176 \\
140	1.88891452205882 \\
141	2.46635202205882 \\
142	2.68920772058824 \\
143	2.39191544117647 \\
144	3.36706525735294 \\
145	2.95451102941176 \\
146	3.09818841911765 \\
147	3.57775643382353 \\
148	3.59411764705882 \\
149	3.65038327205882 \\
150	3.87315257352941 \\
151	4.04469209558824 \\
152	3.77246966911765 \\
153	4.40667463235294 \\
154	4.51016360294118 \\
155	4.62873897058824 \\
156	4.53028952205882 \\
157	5.02484926470588 \\
158	5.24996415441177 \\
159	5.39852757352941 \\
160	5.42734283088235 \\
161	5.53996507352941 \\
162	5.77620220588235 \\
163	5.90436305147059 \\
164	6.27386856617647 \\
165	6.16363970588235 \\
166	5.96870863970588 \\
167	6.20595863970588 \\
168	6.67000827205882 \\
169	6.76470128676471 \\
170	6.90903308823529 \\
171	6.58211121323529 \\
172	6.67744761029412 \\
173	7.12129503676471 \\
174	7.32996875 \\
175	7.30330055147059 \\
176	7.4375 \\
177	7.48897058823529 \\
178	7.40898805147059 \\
179	7.5625 \\
180	7.578125 \\
181	7.45541636029412 \\
182	7.89843106617647 \\
183	7.891515625 \\
184	7.96916911764706 \\
185	8.17983088235294 \\
186	8.12358639705882 \\
187	8.38786764705882 \\
188	8.50236213235294 \\
189	8.50001286764706 \\
190	8.437953125 \\
191	8.46875 \\
192	8.828125 \\
193	8.828125 \\
194	8.72794117647059 \\
195	9.02205882352941 \\
196	8.90625 \\
197	8.90625 \\
198	8.90625 \\
199	9.13419117647059 \\
200	9.21875 \\
201	9.296875 \\
202	9.21875 \\
203	9.296875 \\
204	9.296875 \\
205	9.297328125 \\
206	9.34558823529412 \\
207	9.375 \\
208	9.375 \\
209	9.35569852941176 \\
210	9.453125 \\
211	9.609375 \\
212	9.6875 \\
213	9.6875 \\
214	9.6875 \\
215	9.6875 \\
216	9.6875 \\
217	9.6875 \\
218	9.6875 \\
219	9.6875 \\
220	9.6875 \\
221	9.6875 \\
222	9.84375 \\
223	9.84375 \\
224	9.84375 \\
225	9.921875 \\
226	9.921875 \\
227	9.921875 \\
228	9.921875 \\
229	9.921875 \\
230	9.921875 \\
231	9.921875 \\
232	9.921875 \\
233	9.921875 \\
234	9.921875 \\
235	9.921875 \\
236	9.921875 \\
237	9.921875 \\
238	9.921875 \\
239	9.921875 \\
240	9.921875 \\
241	9.921875 \\
242	9.921875 \\
243	9.921875 \\
244	9.921875 \\
245	9.921875 \\
246	9.921875 \\
247	9.921875 \\
248	9.921875 \\
249	9.921875 \\
250	9.921875 \\
251	9.921875 \\
252	9.921875 \\
253	9.921875 \\
254	9.921875 \\
255	9.921875 \\
256	9.921875 \\
257	9.921875 \\
258	10 \\
259	10 \\
260	10 \\
261	10 \\
262	10 \\
263	10 \\
264	10 \\
265	10 \\
266	10 \\
267	10 \\
268	10 \\
269	10 \\
270	10 \\
271	10 \\
272	10 \\
273	10 \\
274	10 \\
275	10 \\
276	10 \\
277	10 \\
278	10 \\
279	10 \\
280	10 \\
281	10 \\
282	10 \\
283	10 \\
284	10 \\
285	10 \\
286	10 \\
287	10 \\
288	10 \\
289	10 \\
290	10 \\
291	10 \\
292	10 \\
293	10 \\
294	10 \\
295	10 \\
296	10 \\
297	10 \\
298	10 \\
299	10 \\
};
\addplot [semithick, color1, forget plot]
table [row sep=\\]{%
0	0.0161875 \\
1	0.019234375 \\
2	0.025875 \\
3	0.036578125 \\
4	0.03790625 \\
5	0.039671875 \\
6	0.03878125 \\
7	0.037890625 \\
8	0.036984375 \\
9	0.03834375 \\
10	0.037875 \\
11	0.037890625 \\
12	0.03834375 \\
13	0.035671875 \\
14	0.03521875 \\
15	0.03565625 \\
16	0.039203125 \\
17	0.04321875 \\
18	0.042328125 \\
19	0.041453125 \\
20	0.03875 \\
21	0.0374375 \\
22	0.039640625 \\
23	0.040109375 \\
24	0.03965625 \\
25	0.0383125 \\
26	0.03965625 \\
27	0.03875 \\
28	0.0383125 \\
29	0.037421875 \\
30	0.036515625 \\
31	0.0351875 \\
32	0.036109375 \\
33	0.0343125 \\
34	0.033 \\
35	0.03209375 \\
36	0.032546875 \\
37	0.034328125 \\
38	0.033875 \\
39	0.03521875 \\
40	0.037 \\
41	0.037 \\
42	0.03875 \\
43	0.037859375 \\
44	0.037421875 \\
45	0.040109375 \\
46	0.0405625 \\
47	0.03965625 \\
48	0.041890625 \\
49	0.041875 \\
50	0.042765625 \\
51	0.042765625 \\
52	0.198125 \\
53	0.200359375 \\
54	0.19903125 \\
55	0.198140625 \\
56	0.199921875 \\
57	0.19946875 \\
58	0.19990625 \\
59	0.197234375 \\
60	0.354375 \\
61	0.35346875 \\
62	0.50928125 \\
63	0.507046875 \\
64	0.504828125 \\
65	0.5035 \\
66	0.6601875 \\
67	0.661984375 \\
68	0.66196875 \\
69	0.6601875 \\
70	0.659765625 \\
71	0.65978125 \\
72	0.656640625 \\
73	0.6601875 \\
74	0.816875 \\
75	0.8164375 \\
76	0.65840625 \\
77	0.65975 \\
78	0.658859375 \\
79	0.815109375 \\
80	0.973140625 \\
81	1.1285 \\
82	1.127171875 \\
83	1.125828125 \\
84	1.1245 \\
85	1.12403125 \\
86	1.202609375 \\
87	1.673609375 \\
88	1.828078125 \\
89	1.828515625 \\
90	1.827171875 \\
91	1.827625 \\
92	1.8271875 \\
93	1.827171875 \\
94	1.827171875 \\
95	1.82671875 \\
96	1.827625 \\
97	2.138765625 \\
98	2.1405625 \\
99	2.140140625 \\
100	2.45128125 \\
101	2.5303125 \\
102	2.762453125 \\
103	2.762875 \\
104	2.92003125 \\
105	2.918703125 \\
106	3.1499375 \\
107	3.151734375 \\
108	3.15084375 \\
109	3.308421875 \\
110	3.153953125 \\
111	3.1535 \\
112	3.3093125 \\
113	3.621359375 \\
114	3.77671875 \\
115	3.853953125 \\
116	4.165125 \\
117	4.31959375 \\
118	4.318703125 \\
119	4.319609375 \\
120	4.630328125 \\
121	4.629421875 \\
122	4.629875 \\
123	4.941921875 \\
124	4.93925 \\
125	5.01828125 \\
126	5.17184375 \\
127	5.328984375 \\
128	5.48478125 \\
129	5.48478125 \\
130	5.71915625 \\
131	5.718265625 \\
132	5.71871875 \\
133	5.71871875 \\
134	5.95309375 \\
135	6.030765625 \\
136	6.03121875 \\
137	6.264703125 \\
138	6.419609375 \\
139	6.499078125 \\
140	6.498625 \\
141	6.5763125 \\
142	6.575859375 \\
143	6.731671875 \\
144	6.73121875 \\
145	6.731671875 \\
146	6.886140625 \\
147	7.0415 \\
148	7.50846875 \\
149	7.663828125 \\
150	7.663828125 \\
151	7.663828125 \\
152	7.663375 \\
153	7.66471875 \\
154	7.664265625 \\
155	7.820515625 \\
156	7.820078125 \\
157	7.896859375 \\
158	7.97409375 \\
159	7.975875 \\
160	8.131234375 \\
161	8.28659375 \\
162	8.676328125 \\
163	8.676328125 \\
164	8.676328125 \\
165	8.676328125 \\
166	8.676328125 \\
167	8.676328125 \\
168	8.676328125 \\
169	8.67678125 \\
170	8.676328125 \\
171	8.754015625 \\
172	8.830796875 \\
173	8.987046875 \\
174	8.987046875 \\
175	8.987046875 \\
176	8.987046875 \\
177	8.987046875 \\
178	8.987046875 \\
179	8.987046875 \\
180	9.22053125 \\
181	9.22053125 \\
182	9.37678125 \\
183	9.22053125 \\
184	9.221421875 \\
185	9.37678125 \\
186	9.37678125 \\
187	9.37678125 \\
188	9.37678125 \\
189	9.37678125 \\
190	9.37678125 \\
191	9.37678125 \\
192	9.37678125 \\
193	9.37678125 \\
194	9.45490625 \\
195	9.45490625 \\
196	9.45490625 \\
197	9.45490625 \\
198	9.45490625 \\
199	9.53303125 \\
200	9.61115625 \\
201	9.61115625 \\
202	9.61115625 \\
203	9.61115625 \\
204	9.61115625 \\
205	9.61115625 \\
206	9.61115625 \\
207	9.766515625 \\
208	9.766515625 \\
209	9.766515625 \\
210	9.766515625 \\
211	9.844640625 \\
212	9.844640625 \\
213	9.844640625 \\
214	9.844640625 \\
215	9.844640625 \\
216	9.844640625 \\
217	9.844640625 \\
218	9.844640625 \\
219	9.844640625 \\
220	9.844640625 \\
221	9.844640625 \\
222	9.844640625 \\
223	9.844640625 \\
224	9.844640625 \\
225	9.844640625 \\
226	9.921875 \\
227	9.921875 \\
228	9.921875 \\
229	9.921875 \\
230	9.921875 \\
231	9.921875 \\
232	9.921875 \\
233	9.921875 \\
234	9.921875 \\
235	9.921875 \\
236	9.921875 \\
237	9.921875 \\
238	9.921875 \\
239	9.921875 \\
240	9.921875 \\
241	9.921875 \\
242	9.921875 \\
243	9.921875 \\
244	9.921875 \\
245	9.921875 \\
246	9.921875 \\
247	9.921875 \\
248	9.921875 \\
249	9.921875 \\
250	9.921875 \\
251	9.921875 \\
252	9.921875 \\
253	9.921875 \\
254	9.921875 \\
255	9.921875 \\
256	9.921875 \\
257	9.921875 \\
258	9.921875 \\
259	9.921875 \\
260	9.921875 \\
261	9.921875 \\
262	9.921875 \\
263	9.921875 \\
264	9.921875 \\
265	9.921875 \\
266	10 \\
267	10 \\
268	10 \\
269	10 \\
270	10 \\
271	10 \\
272	10 \\
273	10 \\
274	10 \\
275	10 \\
276	10 \\
277	10 \\
278	10 \\
279	10 \\
280	10 \\
281	10 \\
282	10 \\
283	10 \\
284	10 \\
285	10 \\
286	10 \\
287	10 \\
288	10 \\
289	10 \\
290	10 \\
291	10 \\
292	10 \\
293	10 \\
294	10 \\
295	10 \\
296	10 \\
297	10 \\
298	10 \\
299	10 \\
};
\addplot [semithick, color2, dashed, forget plot]
table [row sep=\\]{%
0	0.01875 \\
1	0.017875 \\
2	0.015625 \\
3	0.024109375 \\
4	0.020953125 \\
5	0.021421875 \\
6	0.02584375 \\
7	0.02275 \\
8	0.02675 \\
9	0.02453125 \\
10	0.024515625 \\
11	0.0160625 \\
12	0.016046875 \\
13	0.01653125 \\
14	0.019625 \\
15	0.02096875 \\
16	0.0200625 \\
17	0.024078125 \\
18	0.0223125 \\
19	0.024515625 \\
20	0.022765625 \\
21	0.02053125 \\
22	0.020515625 \\
23	0.019203125 \\
24	0.01653125 \\
25	0.02234375 \\
26	0.017859375 \\
27	0.02053125 \\
28	0.016515625 \\
29	0.016515625 \\
30	0.013828125 \\
31	0.01828125 \\
32	0.01740625 \\
33	0.0191875 \\
34	0.017390625 \\
35	0.01428125 \\
36	0.01028125 \\
37	0.013390625 \\
38	0.0169375 \\
39	0.013390625 \\
40	0.0160625 \\
41	0.013875 \\
42	0.014765625 \\
43	0.013375 \\
44	0.012046875 \\
45	0.0160625 \\
46	0.012484375 \\
47	0.01559375 \\
48	0.0169375 \\
49	0.012921875 \\
50	0.017390625 \\
51	0.0125 \\
52	0.010703125 \\
53	0.01428125 \\
54	0.01428125 \\
55	0.01515625 \\
56	0.0160625 \\
57	0.017421875 \\
58	0.0151875 \\
59	0.01740625 \\
60	0.018734375 \\
61	0.020078125 \\
62	0.01740625 \\
63	0.020953125 \\
64	0.01915625 \\
65	0.019640625 \\
66	0.0209375 \\
67	0.0231875 \\
68	0.02096875 \\
69	0.017390625 \\
70	0.013828125 \\
71	0.014265625 \\
72	0.013390625 \\
73	0.01959375 \\
74	0.013390625 \\
75	0.011609375 \\
76	0.012484375 \\
77	0.0165 \\
78	0.016953125 \\
79	0.012484375 \\
80	0.01428125 \\
81	0.00893750000000001 \\
82	0.014734375 \\
83	0.01696875 \\
84	0.01828125 \\
85	0.01471875 \\
86	0.020046875 \\
87	0.015609375 \\
88	0.015625 \\
89	0.01471875 \\
90	0.01159375 \\
91	0.016046875 \\
92	0.012921875 \\
93	0.00225 \\
94	0.00939062500000001 \\
95	0.013375 \\
96	0.00937500000000001 \\
97	0.012921875 \\
98	0.01428125 \\
99	0.00935937500000001 \\
100	0.01203125 \\
101	0.01159375 \\
102	0.01159375 \\
103	0.0138125 \\
104	0.0106875 \\
105	0.013359375 \\
106	0.00935937500000001 \\
107	0.012046875 \\
108	0.012484375 \\
109	0.013359375 \\
110	0.0106875 \\
111	0.00804687500000001 \\
112	0.00981250000000001 \\
113	0.011140625 \\
114	0.01071875 \\
115	0.012953125 \\
116	0.01115625 \\
117	0.012921875 \\
118	0.085265625 \\
119	0.00890625000000001 \\
120	0.010265625 \\
121	0.0870625 \\
122	0.00803125000000001 \\
123	0.01075 \\
124	0.00714062500000001 \\
125	0.0129375 \\
126	0.320515625 \\
127	0.480359375 \\
128	0.476796875 \\
129	0.3263125 \\
130	0.47678125 \\
131	0.32009375 \\
132	0.4776875 \\
133	0.946421875 \\
134	0.787046875 \\
135	0.790609375 \\
136	1.0245625 \\
137	0.63259375 \\
138	0.78884375 \\
139	0.788828125 \\
140	1.10178125 \\
141	1.257578125 \\
142	1.56653125 \\
143	1.72409375 \\
144	2.040171875 \\
145	2.03975 \\
146	2.11471875 \\
147	2.11653125 \\
148	2.42946875 \\
149	2.975875 \\
150	2.66071875 \\
151	3.284375 \\
152	2.89596875 \\
153	3.361625 \\
154	3.59909375 \\
155	3.753578125 \\
156	3.9098125 \\
157	3.911171875 \\
158	3.5973125 \\
159	3.83346875 \\
160	3.988828125 \\
161	4.4566875 \\
162	4.379015625 \\
163	4.3776875 \\
164	5.39375 \\
165	5.396421875 \\
166	5.471421875 \\
167	5.471421875 \\
168	5.707140625 \\
169	6.408484375 \\
170	5.8625 \\
171	6.018296875 \\
172	6.177671875 \\
173	6.64196875 \\
174	6.87678125 \\
175	6.72009375 \\
176	7.032140625 \\
177	7.18840625 \\
178	7.34375 \\
179	7.34375 \\
180	7.26740625 \\
181	7.265625 \\
182	7.265625 \\
183	7.578125 \\
184	7.422765625 \\
185	7.500890625 \\
186	7.579015625 \\
187	7.814734375 \\
188	7.891078125 \\
189	8.125 \\
190	8.046875 \\
191	8.05 \\
192	8.12634375 \\
193	8.204015625 \\
194	8.282140625 \\
195	8.28125 \\
196	8.282140625 \\
197	8.438390625 \\
198	8.59509375 \\
199	8.594640625 \\
200	8.594203125 \\
201	8.671875 \\
202	8.672765625 \\
203	8.750890625 \\
204	8.828125 \\
205	8.828125 \\
206	8.828125 \\
207	8.984375 \\
208	8.90625 \\
209	8.984828125 \\
210	8.985265625 \\
211	9.140625 \\
212	9.0625 \\
213	9.062953125 \\
214	9.141515625 \\
215	9.21875 \\
216	9.375 \\
217	9.375 \\
218	9.375890625 \\
219	9.53125 \\
220	9.53125 \\
221	9.53125 \\
222	9.53125 \\
223	9.53125 \\
224	9.53125 \\
225	9.53125 \\
226	9.53125 \\
227	9.609375 \\
228	9.6875 \\
229	9.6875 \\
230	9.6875 \\
231	9.6875 \\
232	9.6875 \\
233	9.6875 \\
234	9.765625 \\
235	9.765625 \\
236	9.765625 \\
237	9.765625 \\
238	9.765625 \\
239	9.765625 \\
240	9.765625 \\
241	9.765625 \\
242	9.765625 \\
243	9.765625 \\
244	9.765625 \\
245	9.765625 \\
246	9.765625 \\
247	9.84375 \\
248	9.84375 \\
249	9.84375 \\
250	9.84375 \\
251	9.84375 \\
252	9.921875 \\
253	9.921875 \\
254	9.921875 \\
255	9.921875 \\
256	9.921875 \\
257	9.921875 \\
258	9.921875 \\
259	9.921875 \\
260	9.921875 \\
261	9.921875 \\
262	9.921875 \\
263	9.921875 \\
264	9.921875 \\
265	9.921875 \\
266	10 \\
267	10 \\
268	10 \\
269	10 \\
270	10 \\
271	10 \\
272	10 \\
273	10 \\
274	10 \\
275	10 \\
276	10 \\
277	10 \\
278	10 \\
279	10 \\
280	10 \\
281	10 \\
282	10 \\
283	10 \\
284	10 \\
285	10 \\
286	10 \\
287	10 \\
288	10 \\
289	10 \\
290	10 \\
291	10 \\
292	10 \\
293	10 \\
294	10 \\
295	10 \\
296	10 \\
297	10 \\
298	10 \\
299	10 \\
};
\addplot [semithick, color3, dotted, forget plot]
table [row sep=\\]{%
0	0.014734375 \\
1	0.015171875 \\
2	0.020515625 \\
3	0.019625 \\
4	0.022296875 \\
5	0.02496875 \\
6	0.0205 \\
7	0.01784375 \\
8	0.0191875 \\
9	0.019625 \\
10	0.015625 \\
11	0.018734375 \\
12	0.01471875 \\
13	0.013390625 \\
14	0.0169375 \\
15	0.01875 \\
16	0.019640625 \\
17	0.01828125 \\
18	0.014265625 \\
19	0.017421875 \\
20	0.017390625 \\
21	0.0200625 \\
22	0.020984375 \\
23	0.01875 \\
24	0.018296875 \\
25	0.017859375 \\
26	0.015640625 \\
27	0.01696875 \\
28	0.017828125 \\
29	0.017421875 \\
30	0.012484375 \\
31	0.015625 \\
32	0.0169375 \\
33	0.01696875 \\
34	0.01159375 \\
35	0.017390625 \\
36	0.014703125 \\
37	0.013375 \\
38	0.0160625 \\
39	0.010703125 \\
40	0.013828125 \\
41	0.01071875 \\
42	0.00935937500000001 \\
43	0.00803125000000001 \\
44	0.00935937500000001 \\
45	0.00892187500000001 \\
46	0.00939062500000001 \\
47	0.0125 \\
48	0.0125 \\
49	0.00848437500000001 \\
50	0.016046875 \\
51	0.012046875 \\
52	0.00937500000000001 \\
53	0.01696875 \\
54	0.013375 \\
55	0.01471875 \\
56	0.010265625 \\
57	0.015609375 \\
58	0.00714062500000001 \\
59	0.01159375 \\
60	0.01425 \\
61	0.01471875 \\
62	0.012515625 \\
63	0.015609375 \\
64	0.012046875 \\
65	0.017375 \\
66	0.02053125 \\
67	0.0205 \\
68	0.013828125 \\
69	0.01696875 \\
70	0.01784375 \\
71	0.013390625 \\
72	0.017375 \\
73	0.01828125 \\
74	0.0160625 \\
75	0.012484375 \\
76	0.011609375 \\
77	0.01115625 \\
78	0.013390625 \\
79	0.013421875 \\
80	0.013859375 \\
81	0.01203125 \\
82	0.01471875 \\
83	0.0151875 \\
84	0.013828125 \\
85	0.01471875 \\
86	0.01025 \\
87	0.0106875 \\
88	0.011140625 \\
89	0.00848437500000001 \\
90	0.00757812500000001 \\
91	0.01425 \\
92	0.00848437500000001 \\
93	0.0120625 \\
94	0.010265625 \\
95	0.01025 \\
96	0.00982812500000001 \\
97	0.01246875 \\
98	0.08615625 \\
99	0.01340625 \\
100	0.0897343750000001 \\
101	0.168734375 \\
102	0.249546875 \\
103	0.32234375 \\
104	0.562953125 \\
105	0.403140625 \\
106	0.47990625 \\
107	0.6348125 \\
108	0.713828125 \\
109	0.71115625 \\
110	1.103578125 \\
111	0.79240625 \\
112	0.871890625 \\
113	1.10446875 \\
114	1.107125 \\
115	0.9535625 \\
116	1.03121875 \\
117	1.18971875 \\
118	1.653109375 \\
119	1.88525 \\
120	2.0415 \\
121	2.513828125 \\
122	2.275890625 \\
123	2.5879375 \\
124	2.594171875 \\
125	2.901328125 \\
126	3.059375 \\
127	3.0575625 \\
128	3.3691875 \\
129	3.44909375 \\
130	3.21203125 \\
131	3.5254375 \\
132	3.679015625 \\
133	3.524546875 \\
134	3.757125 \\
135	4.226765625 \\
136	4.612046875 \\
137	4.61428125 \\
138	5.08346875 \\
139	5.004453125 \\
140	5.3941875 \\
141	5.55221875 \\
142	5.62678125 \\
143	6.251796875 \\
144	6.641515625 \\
145	6.408921875 \\
146	6.410265625 \\
147	7.03259375 \\
148	7.26653125 \\
149	6.87678125 \\
150	7.18884375 \\
151	7.26740625 \\
152	7.268296875 \\
153	7.346421875 \\
154	7.50178125 \\
155	7.8125 \\
156	8.047765625 \\
157	8.04865625 \\
158	8.125890625 \\
159	8.125 \\
160	8.36115625 \\
161	8.204015625 \\
162	8.438390625 \\
163	8.516515625 \\
164	8.750890625 \\
165	8.672765625 \\
166	8.75178125 \\
167	8.907140625 \\
168	9.063390625 \\
169	9.140625 \\
170	9.21875 \\
171	9.296875 \\
172	9.297765625 \\
173	9.297765625 \\
174	9.375 \\
175	9.454015625 \\
176	9.454015625 \\
177	9.6875 \\
178	9.765625 \\
179	9.765625 \\
180	9.765625 \\
181	9.765625 \\
182	9.84375 \\
183	9.84375 \\
184	9.84375 \\
185	9.84375 \\
186	9.84375 \\
187	9.921875 \\
188	10 \\
189	10 \\
190	10 \\
191	10 \\
192	10 \\
193	10 \\
194	10 \\
195	10 \\
196	10 \\
197	10 \\
198	10 \\
199	10 \\
200	10 \\
201	10 \\
202	10 \\
203	10 \\
204	10 \\
205	10 \\
206	10 \\
207	10 \\
208	10 \\
209	10 \\
210	10 \\
211	10 \\
212	10 \\
213	10 \\
214	10 \\
215	10 \\
216	10 \\
217	10 \\
218	10 \\
219	10 \\
220	10 \\
221	10 \\
222	10 \\
223	10 \\
224	10 \\
225	10 \\
226	10 \\
227	10 \\
228	10 \\
229	10 \\
230	10 \\
231	10 \\
232	10 \\
233	10 \\
234	10 \\
235	10 \\
236	10 \\
237	10 \\
238	10 \\
239	10 \\
240	10 \\
241	10 \\
242	10 \\
243	10 \\
244	10 \\
245	10 \\
246	10 \\
247	10 \\
248	10 \\
249	10 \\
250	10 \\
251	10 \\
252	10 \\
253	10 \\
254	10 \\
255	10 \\
256	10 \\
257	10 \\
258	10 \\
259	10 \\
260	10 \\
261	10 \\
262	10 \\
263	10 \\
264	10 \\
265	10 \\
266	10 \\
267	10 \\
268	10 \\
269	10 \\
270	10 \\
271	10 \\
272	10 \\
273	10 \\
274	10 \\
275	10 \\
276	10 \\
277	10 \\
278	10 \\
279	10 \\
280	10 \\
281	10 \\
282	10 \\
283	10 \\
284	10 \\
285	10 \\
286	10 \\
287	10 \\
288	10 \\
289	10 \\
290	10 \\
291	10 \\
292	10 \\
293	10 \\
294	10 \\
295	10 \\
296	10 \\
297	10 \\
298	10 \\
299	10 \\
};
\nextgroupplot[
height=\figureheight,
tick align=outside,
tick pos=left,
title={Taxi},
title style={font=\small, yshift=-1.5ex},
xlabel style={font=\scriptsize},
yticklabel style={font=\scriptsize},
xticklabel style={font=\scriptsize},
grid=both,
width=\figurewidth,
x grid style={white!69.01960784313725!black},
xmin=-29.95, xmax=628.95,
y grid style={white!69.01960784313725!black},
ymin=-0.991500232103536, ymax=15.9507708038818
]
\path [fill=color0, fill opacity=0.3] (axis cs:0,0)
--(axis cs:0,0)
--(axis cs:1,0)
--(axis cs:2,0)
--(axis cs:3,0)
--(axis cs:4,0)
--(axis cs:5,0)
--(axis cs:6,0.00920196004260147)
--(axis cs:7,0.00766830003550123)
--(axis cs:8,0)
--(axis cs:9,0.00920196004260148)
--(axis cs:10,0.0230049001065037)
--(axis cs:11,0.00920196004260147)
--(axis cs:12,0)
--(axis cs:13,0)
--(axis cs:14,0)
--(axis cs:15,0)
--(axis cs:16,0.00920196004260147)
--(axis cs:17,0.0272104086567935)
--(axis cs:18,0.0184039200852029)
--(axis cs:19,0)
--(axis cs:20,0)
--(axis cs:21,0)
--(axis cs:22,0)
--(axis cs:23,0)
--(axis cs:24,0)
--(axis cs:25,0.0345073501597556)
--(axis cs:26,0.0482766926874117)
--(axis cs:27,0.00920196004260148)
--(axis cs:28,0)
--(axis cs:29,0)
--(axis cs:30,0)
--(axis cs:31,0)
--(axis cs:32,0)
--(axis cs:33,0)
--(axis cs:34,0)
--(axis cs:35,0)
--(axis cs:36,0)
--(axis cs:37,0.018403920085203)
--(axis cs:38,0.0591554574167238)
--(axis cs:39,0.00920196004260148)
--(axis cs:40,0)
--(axis cs:41,0.138029400639022)
--(axis cs:42,0.0276058801278044)
--(axis cs:43,0)
--(axis cs:44,0.0587444348289932)
--(axis cs:45,0.0276058801278044)
--(axis cs:46,0)
--(axis cs:47,0.144894666180385)
--(axis cs:48,0)
--(axis cs:49,0)
--(axis cs:50,0.394369716111492)
--(axis cs:51,0.0276058801278044)
--(axis cs:52,0)
--(axis cs:53,0)
--(axis cs:54,0)
--(axis cs:55,0)
--(axis cs:56,0)
--(axis cs:57,0.0276058801278044)
--(axis cs:58,0)
--(axis cs:59,0)
--(axis cs:60,0)
--(axis cs:61,0)
--(axis cs:62,0.00920196004260147)
--(axis cs:63,0)
--(axis cs:64,0)
--(axis cs:65,0)
--(axis cs:66,0)
--(axis cs:67,0)
--(axis cs:68,0)
--(axis cs:69,0)
--(axis cs:70,0)
--(axis cs:71,0)
--(axis cs:72,0)
--(axis cs:73,0)
--(axis cs:74,0)
--(axis cs:75,0)
--(axis cs:76,0.0788739432222983)
--(axis cs:77,0)
--(axis cs:78,0)
--(axis cs:79,0)
--(axis cs:80,0)
--(axis cs:81,0.00920196004260148)
--(axis cs:82,0)
--(axis cs:83,0.0276058801278044)
--(axis cs:84,0.0153366000710025)
--(axis cs:85,0.018403920085203)
--(axis cs:86,0.0306732001420049)
--(axis cs:87,0.0373746800892421)
--(axis cs:88,0.0327376626482705)
--(axis cs:89,0.018403920085203)
--(axis cs:90,0.0920196004260148)
--(axis cs:91,0.0887975673044251)
--(axis cs:92,0.0230049001065037)
--(axis cs:93,0.018403920085203)
--(axis cs:94,0)
--(axis cs:95,0.0921423120437851)
--(axis cs:96,0.12735465200943)
--(axis cs:97,0.032864143009291)
--(axis cs:98,0.0986733895885294)
--(axis cs:99,0)
--(axis cs:100,0)
--(axis cs:101,0)
--(axis cs:102,0.128170157736235)
--(axis cs:103,0.460098002130074)
--(axis cs:104,0.0460098002130074)
--(axis cs:105,0.1821947950791)
--(axis cs:106,0)
--(axis cs:107,0)
--(axis cs:108,0.158126531850475)
--(axis cs:109,0.138029400639022)
--(axis cs:110,0.690147003195111)
--(axis cs:111,0.633190795284025)
--(axis cs:112,0.690147003195111)
--(axis cs:113,0.699597235538878)
--(axis cs:114,0)
--(axis cs:115,0)
--(axis cs:116,0.552117602556089)
--(axis cs:117,0.492962145139365)
--(axis cs:118,0.744627098010829)
--(axis cs:119,0)
--(axis cs:120,0.690147003195111)
--(axis cs:121,0.690147003195111)
--(axis cs:122,0.517610252396333)
--(axis cs:123,0.483102902236578)
--(axis cs:124,0)
--(axis cs:125,0.00920196004260148)
--(axis cs:126,0)
--(axis cs:127,0.221634587708752)
--(axis cs:128,0.0276058801278044)
--(axis cs:129,0)
--(axis cs:130,0)
--(axis cs:131,0)
--(axis cs:132,0.0276058801278044)
--(axis cs:133,0)
--(axis cs:134,0.329708635661931)
--(axis cs:135,0.638661479057535)
--(axis cs:136,0.115397104293692)
--(axis cs:137,0.0828176403834133)
--(axis cs:138,0.0276058801278044)
--(axis cs:139,0.754019175476164)
--(axis cs:140,0.690147003195111)
--(axis cs:141,0.744627098010829)
--(axis cs:142,0.699597235538878)
--(axis cs:143,0.699597235538878)
--(axis cs:144,0.690147003195111)
--(axis cs:145,0.754019175476164)
--(axis cs:146,0.774949919443944)
--(axis cs:147,0.690147003195111)
--(axis cs:148,0.690147003195111)
--(axis cs:149,0.690147003195111)
--(axis cs:150,0.690147003195111)
--(axis cs:151,0.754019175476164)
--(axis cs:152,0.690147003195111)
--(axis cs:153,0.699597235538878)
--(axis cs:154,0.754019175476164)
--(axis cs:155,0.690147003195111)
--(axis cs:156,0.705808656400907)
--(axis cs:157,0.696366135051176)
--(axis cs:158,0.690147003195111)
--(axis cs:159,0.779240443271581)
--(axis cs:160,0.726400254551604)
--(axis cs:161,0.690147003195111)
--(axis cs:162,0.709035883979379)
--(axis cs:163,0.819831818313745)
--(axis cs:164,0.73980289292847)
--(axis cs:165,0.699597235538878)
--(axis cs:166,0.727232142857142)
--(axis cs:167,0.852396503836172)
--(axis cs:168,0.746239534208812)
--(axis cs:169,0.771619610763632)
--(axis cs:170,0.690147003195111)
--(axis cs:171,0.690147003195111)
--(axis cs:172,0.93645189116944)
--(axis cs:173,1.11094169966697)
--(axis cs:174,1.08763023259188)
--(axis cs:175,0.690147003195111)
--(axis cs:176,0.925113981565625)
--(axis cs:177,0.76412791205658)
--(axis cs:178,0.709035883979379)
--(axis cs:179,1.21967112237347)
--(axis cs:180,0.87751436496717)
--(axis cs:181,0.793061431256344)
--(axis cs:182,1.07621190730183)
--(axis cs:183,1.12082333674756)
--(axis cs:184,0.887906909037854)
--(axis cs:185,0.757720076304956)
--(axis cs:186,1.34100918796613)
--(axis cs:187,0.79437931020754)
--(axis cs:188,0.841571548770548)
--(axis cs:189,1.12155770560577)
--(axis cs:190,1.14407901597481)
--(axis cs:191,0.835694054618704)
--(axis cs:192,1.27711229418277)
--(axis cs:193,1.24273695385867)
--(axis cs:194,1.31949639537332)
--(axis cs:195,1.18749014219164)
--(axis cs:196,1.26567450443733)
--(axis cs:197,1.22994549083833)
--(axis cs:198,1.02148898483625)
--(axis cs:199,1.26863352878571)
--(axis cs:200,0.803659380855277)
--(axis cs:201,1.19097208013668)
--(axis cs:202,1.413326687566)
--(axis cs:203,1.30408902122077)
--(axis cs:204,1.27010267112958)
--(axis cs:205,1.36210562492892)
--(axis cs:206,1.58676769795962)
--(axis cs:207,1.47443069367897)
--(axis cs:208,1.45938683357169)
--(axis cs:209,1.2994656705114)
--(axis cs:210,1.51527650353622)
--(axis cs:211,1.24040116960016)
--(axis cs:212,1.25258439621309)
--(axis cs:213,1.15736628487186)
--(axis cs:214,1.30881699628012)
--(axis cs:215,1.32888380769426)
--(axis cs:216,1.46530536866698)
--(axis cs:217,1.69761266404241)
--(axis cs:218,1.1432075854853)
--(axis cs:219,1.43090642618427)
--(axis cs:220,1.56844238208817)
--(axis cs:221,1.60912884293854)
--(axis cs:222,1.42629100680793)
--(axis cs:223,1.43288937966497)
--(axis cs:224,1.38153564793775)
--(axis cs:225,1.55807977572902)
--(axis cs:226,1.63406228452036)
--(axis cs:227,1.76548210758276)
--(axis cs:228,1.37207776932551)
--(axis cs:229,1.420274149415)
--(axis cs:230,1.54658928554892)
--(axis cs:231,1.537388781996)
--(axis cs:232,1.64658854275435)
--(axis cs:233,1.83601462605015)
--(axis cs:234,2.2083770513147)
--(axis cs:235,1.77467322308499)
--(axis cs:236,2.69828478258581)
--(axis cs:237,2.66362829162445)
--(axis cs:238,2.85813004996317)
--(axis cs:239,2.48318508866113)
--(axis cs:240,2.3495469415201)
--(axis cs:241,2.5556637207669)
--(axis cs:242,2.81036467932361)
--(axis cs:243,2.41954774745048)
--(axis cs:244,2.3735377627733)
--(axis cs:245,1.55635907442162)
--(axis cs:246,2.57667396617205)
--(axis cs:247,2.35936116219349)
--(axis cs:248,2.18765624288675)
--(axis cs:249,2.00863510882263)
--(axis cs:250,2.73114043962671)
--(axis cs:251,2.65544550754798)
--(axis cs:252,2.73277526130596)
--(axis cs:253,2.80049810231132)
--(axis cs:254,2.759903449063)
--(axis cs:255,3.16336550655679)
--(axis cs:256,2.81598197620261)
--(axis cs:257,2.47468217227089)
--(axis cs:258,2.49942439057734)
--(axis cs:259,2.52240773799698)
--(axis cs:260,2.80635106319115)
--(axis cs:261,2.45437975926182)
--(axis cs:262,2.61799624580605)
--(axis cs:263,2.64064393694192)
--(axis cs:264,2.72642426892291)
--(axis cs:265,2.49449303861664)
--(axis cs:266,2.90940701730179)
--(axis cs:267,2.68810364936867)
--(axis cs:268,2.74287885218161)
--(axis cs:269,3.03829635743915)
--(axis cs:270,2.99542645387868)
--(axis cs:271,2.85887732532177)
--(axis cs:272,2.81952828880897)
--(axis cs:273,3.20832961344203)
--(axis cs:274,3.48563071160971)
--(axis cs:275,3.55684322593105)
--(axis cs:276,3.22049967161745)
--(axis cs:277,2.95136978376839)
--(axis cs:278,3.27266990367358)
--(axis cs:279,3.40714412009987)
--(axis cs:280,3.59608834619469)
--(axis cs:281,3.39463580705534)
--(axis cs:282,3.56940100326139)
--(axis cs:283,3.57723412021956)
--(axis cs:284,3.34381580227413)
--(axis cs:285,3.56090671824437)
--(axis cs:286,4.02264633982618)
--(axis cs:287,3.74681759039526)
--(axis cs:288,3.65135857900445)
--(axis cs:289,3.63331161359288)
--(axis cs:290,3.68801315168939)
--(axis cs:291,3.36489820426079)
--(axis cs:292,3.87162421703693)
--(axis cs:293,3.48927316960865)
--(axis cs:294,4.13689162292257)
--(axis cs:295,4.22479647164943)
--(axis cs:296,3.99560103642673)
--(axis cs:297,4.36637662191009)
--(axis cs:298,4.70516795350979)
--(axis cs:299,4.46235056105008)
--(axis cs:300,4.17295135429722)
--(axis cs:301,4.28231404704931)
--(axis cs:302,4.53778266253011)
--(axis cs:303,4.21903163822315)
--(axis cs:304,4.39348012736159)
--(axis cs:305,4.56167849214903)
--(axis cs:306,4.36430696405878)
--(axis cs:307,4.70268641762501)
--(axis cs:308,4.35360340360011)
--(axis cs:309,5.25899526660798)
--(axis cs:310,4.96471384311397)
--(axis cs:311,5.05891616221389)
--(axis cs:312,5.12481072554697)
--(axis cs:313,5.3200814719455)
--(axis cs:314,5.20401848706709)
--(axis cs:315,5.64477318908699)
--(axis cs:316,5.41964283292533)
--(axis cs:317,5.69563017101325)
--(axis cs:318,5.80891052346957)
--(axis cs:319,5.90210971998634)
--(axis cs:320,5.81116498704168)
--(axis cs:321,5.40877949441924)
--(axis cs:322,5.50402155830159)
--(axis cs:323,6.18807652021301)
--(axis cs:324,5.8850362122739)
--(axis cs:325,6.26491974459912)
--(axis cs:326,6.09139541478938)
--(axis cs:327,6.0080356948381)
--(axis cs:328,6.02063745777006)
--(axis cs:329,5.73238441802236)
--(axis cs:330,6.12795259044141)
--(axis cs:331,5.55832134685217)
--(axis cs:332,6.37427781327002)
--(axis cs:333,5.75906557450987)
--(axis cs:334,5.94323822977353)
--(axis cs:335,5.81127790663862)
--(axis cs:336,5.9836564296015)
--(axis cs:337,6.12736338808281)
--(axis cs:338,6.04566028521385)
--(axis cs:339,6.00797167521486)
--(axis cs:340,6.2033174314014)
--(axis cs:341,6.2805942835475)
--(axis cs:342,6.4788488370282)
--(axis cs:343,6.27629339679509)
--(axis cs:344,5.94310195475139)
--(axis cs:345,6.40354962890327)
--(axis cs:346,6.63157802907963)
--(axis cs:347,6.39680255582471)
--(axis cs:348,7.1502866445891)
--(axis cs:349,6.52620993276783)
--(axis cs:350,7.26921649548936)
--(axis cs:351,7.33863078617962)
--(axis cs:352,7.20504139038953)
--(axis cs:353,7.46134548728625)
--(axis cs:354,7.20512658503492)
--(axis cs:355,7.08115856210243)
--(axis cs:356,7.44714852237698)
--(axis cs:357,7.44675041409383)
--(axis cs:358,7.49870594621114)
--(axis cs:359,7.59864634081189)
--(axis cs:360,7.52253503425219)
--(axis cs:361,7.11213674290418)
--(axis cs:362,7.51232176998794)
--(axis cs:363,7.21984171275075)
--(axis cs:364,7.48405852253873)
--(axis cs:365,7.55144111989972)
--(axis cs:366,7.72163432295742)
--(axis cs:367,8.0767095468654)
--(axis cs:368,7.94092530411285)
--(axis cs:369,7.70307056606656)
--(axis cs:370,8.00841674790379)
--(axis cs:371,7.98244068767668)
--(axis cs:372,7.81869386397731)
--(axis cs:373,7.58430881928049)
--(axis cs:374,7.82961525746305)
--(axis cs:375,8.00243004101178)
--(axis cs:376,8.06007304472143)
--(axis cs:377,7.97337065075167)
--(axis cs:378,8.22485081824108)
--(axis cs:379,7.8558854263115)
--(axis cs:380,8.14774090155562)
--(axis cs:381,8.46145661440567)
--(axis cs:382,8.5382570022677)
--(axis cs:383,8.64512634655252)
--(axis cs:384,8.65585839508803)
--(axis cs:385,8.57957714369156)
--(axis cs:386,8.5453154910743)
--(axis cs:387,8.36736615206522)
--(axis cs:388,8.90471135304776)
--(axis cs:389,9.1276430341662)
--(axis cs:390,9.34827802976696)
--(axis cs:391,9.36093446260867)
--(axis cs:392,9.2978614879371)
--(axis cs:393,9.05764297724247)
--(axis cs:394,9.23336965093244)
--(axis cs:395,9.80415759854861)
--(axis cs:396,9.41900389848375)
--(axis cs:397,10.0347759939634)
--(axis cs:398,9.87554217320577)
--(axis cs:399,9.57358601316094)
--(axis cs:400,9.67134550024694)
--(axis cs:401,9.90761613930274)
--(axis cs:402,10.0440008736162)
--(axis cs:403,10.4048137189031)
--(axis cs:404,10.3205599921988)
--(axis cs:405,10.5374059187346)
--(axis cs:406,10.6581187039708)
--(axis cs:407,10.645634087688)
--(axis cs:408,10.4528860094603)
--(axis cs:409,10.2928086284427)
--(axis cs:410,10.7595580631987)
--(axis cs:411,10.8120217860746)
--(axis cs:412,10.6569117144577)
--(axis cs:413,10.5280958404865)
--(axis cs:414,11.1332348675034)
--(axis cs:415,10.8035605286617)
--(axis cs:416,10.8573052748583)
--(axis cs:417,11.376835129764)
--(axis cs:418,11.2474695537922)
--(axis cs:419,11.2628001850767)
--(axis cs:420,11.2727821630728)
--(axis cs:421,11.4189807663744)
--(axis cs:422,11.4734878395063)
--(axis cs:423,11.349934040755)
--(axis cs:424,11.4126928708313)
--(axis cs:425,11.2440082779448)
--(axis cs:426,11.2654390914634)
--(axis cs:427,11.442019802276)
--(axis cs:428,11.3677805048003)
--(axis cs:429,11.4165232407281)
--(axis cs:430,11.5678732238927)
--(axis cs:431,11.7177698604965)
--(axis cs:432,11.5802701564513)
--(axis cs:433,11.7065276725738)
--(axis cs:434,11.9468861293128)
--(axis cs:435,11.7743295928217)
--(axis cs:436,12.2152551277229)
--(axis cs:437,12.0942857341041)
--(axis cs:438,11.9862640381834)
--(axis cs:439,12.1078249789317)
--(axis cs:440,12.186650062282)
--(axis cs:441,12.2827271138332)
--(axis cs:442,12.4249769700591)
--(axis cs:443,12.3614089990608)
--(axis cs:444,12.326768328199)
--(axis cs:445,12.4836220197623)
--(axis cs:446,12.6477426188166)
--(axis cs:447,12.5560705090958)
--(axis cs:448,12.614787643904)
--(axis cs:449,12.7565700189891)
--(axis cs:450,12.6717018789638)
--(axis cs:451,12.665897522739)
--(axis cs:452,12.7876177598153)
--(axis cs:453,12.7904743665931)
--(axis cs:454,12.7786576442225)
--(axis cs:455,12.942321614459)
--(axis cs:456,13.0663148618752)
--(axis cs:457,13.1670929756209)
--(axis cs:458,13.1851872419156)
--(axis cs:459,13.2737849689325)
--(axis cs:460,13.189383517029)
--(axis cs:461,13.0437445368646)
--(axis cs:462,13.2022102823234)
--(axis cs:463,13.2213639107379)
--(axis cs:464,13.1668014351821)
--(axis cs:465,13.1740402322175)
--(axis cs:466,13.2247042674183)
--(axis cs:467,13.1670990939408)
--(axis cs:468,13.1981173757808)
--(axis cs:469,13.2554898130459)
--(axis cs:470,13.2207844442465)
--(axis cs:471,13.247211506355)
--(axis cs:472,13.2491304855231)
--(axis cs:473,13.2523866356525)
--(axis cs:474,13.1998037785268)
--(axis cs:475,13.2561655190592)
--(axis cs:476,13.2238363204298)
--(axis cs:477,13.383658460211)
--(axis cs:478,13.2495896351128)
--(axis cs:479,13.3255019559246)
--(axis cs:480,13.5308401749439)
--(axis cs:481,13.5160088251206)
--(axis cs:482,13.4739110218701)
--(axis cs:483,13.5395848317766)
--(axis cs:484,13.4991901788529)
--(axis cs:485,13.5683107418327)
--(axis cs:486,13.502676966676)
--(axis cs:487,13.3303759154472)
--(axis cs:488,13.6972394789705)
--(axis cs:489,13.7064242690122)
--(axis cs:490,13.6901883729838)
--(axis cs:491,13.6867479661455)
--(axis cs:492,13.7225509977731)
--(axis cs:493,13.7224818286593)
--(axis cs:494,13.8189099415337)
--(axis cs:495,13.857208153577)
--(axis cs:496,13.8415178094274)
--(axis cs:497,13.8211237732684)
--(axis cs:498,13.8845074632794)
--(axis cs:499,13.8952551893997)
--(axis cs:500,13.9214666550514)
--(axis cs:501,13.8933925717431)
--(axis cs:502,13.9677321071453)
--(axis cs:503,14.0650110003761)
--(axis cs:504,14.1119125446524)
--(axis cs:505,14.0256230640263)
--(axis cs:506,14.0526771077287)
--(axis cs:507,14.0309224250322)
--(axis cs:508,14.2203681482247)
--(axis cs:509,14.2617624153443)
--(axis cs:510,14.2914801906557)
--(axis cs:511,14.2996533627174)
--(axis cs:512,14.3220525819533)
--(axis cs:513,14.3196253751287)
--(axis cs:514,14.3196253751287)
--(axis cs:515,14.3196253751287)
--(axis cs:516,14.3196253751287)
--(axis cs:517,14.3196253751287)
--(axis cs:518,14.3111617250048)
--(axis cs:519,14.3196253751287)
--(axis cs:520,14.3167431063772)
--(axis cs:521,14.3196253751287)
--(axis cs:522,14.3008875878608)
--(axis cs:523,14.3116248977758)
--(axis cs:524,14.3036435967232)
--(axis cs:525,14.3036435967232)
--(axis cs:526,14.3158738936049)
--(axis cs:527,14.3082101218397)
--(axis cs:528,14.3300180597084)
--(axis cs:529,14.355703779096)
--(axis cs:530,14.3061646707919)
--(axis cs:531,14.4594996738322)
--(axis cs:532,14.3986750648113)
--(axis cs:533,14.340018415779)
--(axis cs:534,14.3036435967232)
--(axis cs:535,14.4376622244083)
--(axis cs:536,14.4454431661262)
--(axis cs:537,14.4612366498257)
--(axis cs:538,14.5065063120334)
--(axis cs:539,14.5521942980968)
--(axis cs:540,14.5545210033773)
--(axis cs:541,14.6673455130772)
--(axis cs:542,14.6668327027081)
--(axis cs:543,14.709484901889)
--(axis cs:544,14.7106266763779)
--(axis cs:545,14.7183824264706)
--(axis cs:546,14.7119403137423)
--(axis cs:547,14.7116502283341)
--(axis cs:548,14.717847409708)
--(axis cs:549,14.7230670121287)
--(axis cs:550,14.7109516951201)
--(axis cs:551,14.7262801954599)
--(axis cs:552,14.7145619654924)
--(axis cs:553,14.7247156799717)
--(axis cs:554,14.7247156799717)
--(axis cs:555,14.7247156799717)
--(axis cs:556,14.7247156799717)
--(axis cs:557,14.7247156799717)
--(axis cs:558,14.8411634513853)
--(axis cs:559,14.7247156799717)
--(axis cs:560,14.7319529505025)
--(axis cs:561,14.7335642703955)
--(axis cs:562,14.8322575195573)
--(axis cs:563,14.7417470165077)
--(axis cs:564,14.7401337940092)
--(axis cs:565,14.744521914493)
--(axis cs:566,14.8408152986463)
--(axis cs:567,14.8408152986463)
--(axis cs:568,14.8423832543871)
--(axis cs:569,14.8420397268159)
--(axis cs:570,14.8515024729643)
--(axis cs:571,14.8389230530324)
--(axis cs:572,14.8481431783805)
--(axis cs:573,14.8481431783805)
--(axis cs:574,14.8483910568781)
--(axis cs:575,14.8515024729643)
--(axis cs:576,14.8515024729643)
--(axis cs:577,14.8515024729643)
--(axis cs:578,14.8515024729643)
--(axis cs:579,14.7247156799717)
--(axis cs:580,14.8421478997896)
--(axis cs:581,14.8515024729643)
--(axis cs:582,14.8515024729643)
--(axis cs:583,14.8515024729643)
--(axis cs:584,14.8515024729643)
--(axis cs:585,14.8420397268159)
--(axis cs:586,14.8507300243796)
--(axis cs:587,14.8515024729643)
--(axis cs:588,14.8485418633619)
--(axis cs:589,14.849725418956)
--(axis cs:590,14.8481917968341)
--(axis cs:591,14.8515024729643)
--(axis cs:592,14.8427027227991)
--(axis cs:593,14.8515024729643)
--(axis cs:594,14.8515024729643)
--(axis cs:595,14.8515024729643)
--(axis cs:596,14.8420397268159)
--(axis cs:597,14.8515024729643)
--(axis cs:598,14.8515024729643)
--(axis cs:599,14.8515024729643)
--(axis cs:599,13.2734975270357)
--(axis cs:599,13.2734975270357)
--(axis cs:598,13.2734975270357)
--(axis cs:597,13.2734975270357)
--(axis cs:596,13.2649314270302)
--(axis cs:595,13.2734975270357)
--(axis cs:594,13.2734975270357)
--(axis cs:593,13.2734975270357)
--(axis cs:592,13.2655562057724)
--(axis cs:591,13.2734975270357)
--(axis cs:590,13.2018082031659)
--(axis cs:589,13.2484888667583)
--(axis cs:588,13.2202081366381)
--(axis cs:587,13.2734975270357)
--(axis cs:586,13.2638533089537)
--(axis cs:585,13.2649314270302)
--(axis cs:584,13.2734975270357)
--(axis cs:583,13.2734975270357)
--(axis cs:582,13.2734975270357)
--(axis cs:581,13.2734975270357)
--(axis cs:580,13.2624667955509)
--(axis cs:579,12.9315343200283)
--(axis cs:578,13.2734975270357)
--(axis cs:577,13.2734975270357)
--(axis cs:576,13.2734975270357)
--(axis cs:575,13.2734975270357)
--(axis cs:574,13.2141089431219)
--(axis cs:573,13.1831068216195)
--(axis cs:572,13.1831068216195)
--(axis cs:571,13.2055481008137)
--(axis cs:570,13.2734975270357)
--(axis cs:569,13.2649314270302)
--(axis cs:568,13.2652556345017)
--(axis cs:567,13.1765458124648)
--(axis cs:566,13.1765458124648)
--(axis cs:565,13.0715063113135)
--(axis cs:564,13.0026546675293)
--(axis cs:563,13.0082529834923)
--(axis cs:562,13.2552424804427)
--(axis cs:561,13.0605978500979)
--(axis cs:560,13.0180470494975)
--(axis cs:559,12.9315343200283)
--(axis cs:558,13.2640997065095)
--(axis cs:557,12.9315343200283)
--(axis cs:556,12.9315343200283)
--(axis cs:555,12.9315343200283)
--(axis cs:554,12.9315343200283)
--(axis cs:553,12.9315343200283)
--(axis cs:552,12.9061645302341)
--(axis cs:551,12.9612198045401)
--(axis cs:550,12.8773037993854)
--(axis cs:549,12.9206829878713)
--(axis cs:548,12.875902590292)
--(axis cs:547,12.8959886605548)
--(axis cs:546,12.9007126582857)
--(axis cs:545,12.8816175735294)
--(axis cs:544,12.8589045736221)
--(axis cs:543,12.8578762092221)
--(axis cs:542,12.7984450750697)
--(axis cs:541,12.8014044869228)
--(axis cs:540,12.5789115363053)
--(axis cs:539,12.5774798472023)
--(axis cs:538,12.5707786342031)
--(axis cs:537,12.4308241609852)
--(axis cs:536,12.3670568338738)
--(axis cs:535,12.3604146986686)
--(axis cs:534,12.1338564032768)
--(axis cs:533,12.2492672985068)
--(axis cs:532,12.4020117483755)
--(axis cs:531,12.4467503261678)
--(axis cs:530,12.1426989655717)
--(axis cs:529,12.2603676494754)
--(axis cs:528,12.2711244134099)
--(axis cs:527,12.1605398781603)
--(axis cs:526,12.1966261063951)
--(axis cs:525,12.1338564032768)
--(axis cs:524,12.1338564032768)
--(axis cs:523,12.2012657272242)
--(axis cs:522,12.1606271904643)
--(axis cs:521,12.2116246248713)
--(axis cs:520,12.2082568936228)
--(axis cs:519,12.2116246248713)
--(axis cs:518,12.1994361010821)
--(axis cs:517,12.2116246248713)
--(axis cs:516,12.2116246248713)
--(axis cs:515,12.2116246248713)
--(axis cs:514,12.2116246248713)
--(axis cs:513,12.2116246248713)
--(axis cs:512,12.2279474180467)
--(axis cs:511,12.1940966372826)
--(axis cs:510,12.1861983807729)
--(axis cs:509,12.1531082743109)
--(axis cs:508,12.0977742128864)
--(axis cs:507,11.7607442416345)
--(axis cs:506,11.8399699510948)
--(axis cs:505,11.7614748891901)
--(axis cs:504,11.9238829098931)
--(axis cs:503,11.9180548249276)
--(axis cs:502,11.7480779842905)
--(axis cs:501,11.5886523843973)
--(axis cs:500,11.6329976306629)
--(axis cs:499,11.5953698106003)
--(axis cs:498,11.5596889652921)
--(axis cs:497,11.4982371435925)
--(axis cs:496,11.4954258602306)
--(axis cs:495,11.5358284469193)
--(axis cs:494,11.4764626546202)
--(axis cs:493,11.3164664535044)
--(axis cs:492,11.3014874637654)
--(axis cs:491,11.2265808924958)
--(axis cs:490,11.2360616270162)
--(axis cs:489,11.2477727164348)
--(axis cs:488,11.2351289420821)
--(axis cs:487,10.659365246169)
--(axis cs:486,11.0415488321962)
--(axis cs:485,11.117989299017)
--(axis cs:484,11.0357205354328)
--(axis cs:483,11.084731641224)
--(axis cs:482,11.0007011714921)
--(axis cs:481,11.0560047216774)
--(axis cs:480,11.0526727560906)
--(axis cs:479,10.7186271801048)
--(axis cs:478,10.6722036655408)
--(axis cs:477,10.8587346318943)
--(axis cs:476,10.5857339920702)
--(axis cs:475,10.6734548605611)
--(axis cs:474,10.5189462214732)
--(axis cs:473,10.6690075951167)
--(axis cs:472,10.6571195144769)
--(axis cs:471,10.6524575207386)
--(axis cs:470,10.5783226986107)
--(axis cs:469,10.6530007508669)
--(axis cs:468,10.553271513108)
--(axis cs:467,10.4907258397462)
--(axis cs:466,10.5877957325817)
--(axis cs:465,10.5578313684041)
--(axis cs:464,10.5187754878948)
--(axis cs:463,10.5798036716797)
--(axis cs:462,10.5419429434831)
--(axis cs:461,10.3984632553432)
--(axis cs:460,10.5554210735417)
--(axis cs:459,10.7448271634204)
--(axis cs:458,10.5547321129232)
--(axis cs:457,10.5210270640617)
--(axis cs:456,10.4467196775985)
--(axis cs:455,10.2835884437847)
--(axis cs:454,9.99358614459739)
--(axis cs:453,9.95837502762022)
--(axis cs:452,9.97950957975685)
--(axis cs:451,9.88344964704655)
--(axis cs:450,9.89605794857427)
--(axis cs:449,10.0271868491427)
--(axis cs:448,9.79757880908743)
--(axis cs:447,9.72450597859273)
--(axis cs:446,9.85053271695875)
--(axis cs:445,9.64791952717255)
--(axis cs:444,9.43967361108775)
--(axis cs:443,9.52109391469208)
--(axis cs:442,9.5331480299409)
--(axis cs:441,9.4295013904212)
--(axis cs:440,9.30473806348115)
--(axis cs:439,9.10751643368793)
--(axis cs:438,8.98120755272568)
--(axis cs:437,9.26326132463781)
--(axis cs:436,9.41503105087477)
--(axis cs:435,8.79351363653768)
--(axis cs:434,9.02412582597862)
--(axis cs:433,8.79499559582451)
--(axis cs:432,8.54362193053705)
--(axis cs:431,8.65401583813606)
--(axis cs:430,8.50678433362328)
--(axis cs:429,8.34838441252631)
--(axis cs:428,8.23790650510139)
--(axis cs:427,8.30007252568588)
--(axis cs:426,8.10498923021489)
--(axis cs:425,8.16933214507305)
--(axis cs:424,8.35132917952111)
--(axis cs:423,8.26969900848745)
--(axis cs:422,8.39008446441524)
--(axis cs:421,8.37695337341051)
--(axis cs:420,8.1321438262796)
--(axis cs:419,8.18787891390401)
--(axis cs:418,8.1436642787731)
--(axis cs:417,8.34312108589534)
--(axis cs:416,7.69712103713379)
--(axis cs:415,7.67863540040919)
--(axis cs:414,8.03897004452005)
--(axis cs:413,7.28046645295479)
--(axis cs:412,7.39921663941886)
--(axis cs:411,7.65390083297304)
--(axis cs:410,7.63112403978601)
--(axis cs:409,6.94949432722235)
--(axis cs:408,7.1758466294286)
--(axis cs:407,7.40660950205559)
--(axis cs:406,7.42529448062691)
--(axis cs:405,7.24144452769394)
--(axis cs:404,7.01335399006076)
--(axis cs:403,7.18739096979052)
--(axis cs:402,6.78066317761032)
--(axis cs:401,6.5987798952183)
--(axis cs:400,6.40963904220434)
--(axis cs:399,6.23996370308441)
--(axis cs:398,6.56947504915521)
--(axis cs:397,6.77282659501016)
--(axis cs:396,6.15458014939077)
--(axis cs:395,6.50344205873855)
--(axis cs:394,5.87510110101076)
--(axis cs:393,5.82986123729299)
--(axis cs:392,5.98977974730225)
--(axis cs:391,6.01783019721068)
--(axis cs:390,6.05076535561604)
--(axis cs:389,5.91599116491477)
--(axis cs:388,5.64739819783458)
--(axis cs:387,5.14115662756248)
--(axis cs:386,5.28183759709711)
--(axis cs:385,5.29376852106036)
--(axis cs:384,5.40664780530879)
--(axis cs:383,5.46591525028266)
--(axis cs:382,5.36262418454425)
--(axis cs:381,5.22032513201001)
--(axis cs:380,4.92423407562332)
--(axis cs:379,4.61525792686311)
--(axis cs:378,5.10820857028706)
--(axis cs:377,4.75306485068384)
--(axis cs:376,4.81704543901671)
--(axis cs:375,4.78842090874313)
--(axis cs:374,4.59201507516199)
--(axis cs:373,4.35982607808021)
--(axis cs:372,4.6930795096143)
--(axis cs:371,4.79609973891063)
--(axis cs:370,4.82964586137139)
--(axis cs:369,4.5006761812119)
--(axis cs:368,4.73653303804318)
--(axis cs:367,4.89232450792738)
--(axis cs:366,4.54304952387526)
--(axis cs:365,4.35194484491037)
--(axis cs:364,4.27929217190572)
--(axis cs:363,4.14138573167982)
--(axis cs:362,4.25947398429862)
--(axis cs:361,3.92972405255037)
--(axis cs:360,4.44508086580148)
--(axis cs:359,4.54847536971442)
--(axis cs:358,4.35074505865317)
--(axis cs:357,4.22933944998946)
--(axis cs:356,4.23440226525287)
--(axis cs:355,3.89412678851187)
--(axis cs:354,4.01000016015187)
--(axis cs:353,4.22445986973196)
--(axis cs:352,4.05759349550462)
--(axis cs:351,4.06767562442366)
--(axis cs:350,4.06941469791015)
--(axis cs:349,3.43681924343354)
--(axis cs:348,3.99629691857712)
--(axis cs:347,3.3202354678888)
--(axis cs:346,3.62202792576565)
--(axis cs:345,3.37923768645567)
--(axis cs:344,2.93076143675906)
--(axis cs:343,3.1659403061619)
--(axis cs:342,3.40513768426301)
--(axis cs:341,3.16870828888007)
--(axis cs:340,3.09753718273446)
--(axis cs:339,2.89498717659509)
--(axis cs:338,2.92602385039692)
--(axis cs:337,3.0283285762029)
--(axis cs:336,2.8389998203985)
--(axis cs:335,2.75341861007996)
--(axis cs:334,2.90186737277451)
--(axis cs:333,2.74730305685876)
--(axis cs:332,3.28891321206133)
--(axis cs:331,2.60735928792442)
--(axis cs:330,3.09364768428386)
--(axis cs:329,2.64290444752386)
--(axis cs:328,2.94191345075125)
--(axis cs:327,2.93513942525617)
--(axis cs:326,3.02550603991179)
--(axis cs:325,3.17878618488806)
--(axis cs:324,2.90657608168215)
--(axis cs:323,3.16049357886894)
--(axis cs:322,2.58900159530018)
--(axis cs:321,2.41221955861106)
--(axis cs:320,2.74336494993311)
--(axis cs:319,2.8217088225822)
--(axis cs:318,2.77067092449423)
--(axis cs:317,2.7212997740417)
--(axis cs:316,2.47337360866448)
--(axis cs:315,2.66216202490245)
--(axis cs:314,2.42629807137447)
--(axis cs:313,2.45075026984213)
--(axis cs:312,2.34798288709039)
--(axis cs:311,2.34192370971973)
--(axis cs:310,2.19970482011934)
--(axis cs:309,2.41365710571438)
--(axis cs:308,1.76735966984437)
--(axis cs:307,1.95314454828408)
--(axis cs:306,1.64100217572616)
--(axis cs:305,1.88018722099641)
--(axis cs:304,1.69348153230228)
--(axis cs:303,1.61716357281582)
--(axis cs:302,1.87191591292985)
--(axis cs:301,1.67654857470393)
--(axis cs:300,1.55685016597305)
--(axis cs:299,1.75131015323563)
--(axis cs:298,2.00210631855435)
--(axis cs:297,1.66521521279798)
--(axis cs:296,1.42648859269415)
--(axis cs:295,1.59976094830131)
--(axis cs:294,1.52330182945838)
--(axis cs:293,1.08985629467707)
--(axis cs:292,1.37392866757845)
--(axis cs:291,1.0434833916208)
--(axis cs:290,1.17801128440083)
--(axis cs:289,1.14944079025328)
--(axis cs:288,1.25801642099555)
--(axis cs:287,1.29431522210474)
--(axis cs:286,1.51885210737879)
--(axis cs:285,1.17856932295581)
--(axis cs:284,0.930600406517083)
--(axis cs:283,1.20576502590459)
--(axis cs:282,1.28413322292909)
--(axis cs:281,1.13361510869557)
--(axis cs:280,1.23153652012261)
--(axis cs:279,1.05439096761943)
--(axis cs:278,1.05737696534229)
--(axis cs:277,0.843511994193626)
--(axis cs:276,1.05176595338255)
--(axis cs:275,1.33292604026664)
--(axis cs:274,1.23278732868333)
--(axis cs:273,1.06598353763922)
--(axis cs:272,0.791185996905312)
--(axis cs:271,0.822567987178226)
--(axis cs:270,0.8737970309698)
--(axis cs:269,0.896804088989426)
--(axis cs:268,0.765987932075799)
--(axis cs:267,0.69932766931265)
--(axis cs:266,0.780176316031544)
--(axis cs:265,0.57142958043098)
--(axis cs:264,0.768178003804359)
--(axis cs:263,0.678785553604296)
--(axis cs:262,0.653946394886587)
--(axis cs:261,0.48684620227664)
--(axis cs:260,0.778395960618376)
--(axis cs:259,0.606758928669686)
--(axis cs:258,0.629535236751849)
--(axis cs:257,0.518747314908599)
--(axis cs:256,0.72015779128796)
--(axis cs:255,0.903030597339309)
--(axis cs:254,0.626635012475457)
--(axis cs:253,0.667616075177857)
--(axis cs:252,0.735193488694035)
--(axis cs:251,0.665198431845963)
--(axis cs:250,0.645976714052946)
--(axis cs:249,0.311975010224992)
--(axis cs:248,0.368593757113253)
--(axis cs:247,0.576100147330322)
--(axis cs:246,0.702258685517661)
--(axis cs:245,0.22991324700695)
--(axis cs:244,0.457712237226696)
--(axis cs:243,0.547974573978092)
--(axis cs:242,0.781711213533533)
--(axis cs:241,0.611143001922171)
--(axis cs:240,0.47798282038466)
--(axis cs:239,0.644669348568303)
--(axis cs:238,0.681605941416136)
--(axis cs:237,0.657205041708884)
--(axis cs:236,0.636828853777831)
--(axis cs:235,0.200747449991932)
--(axis cs:234,0.421087234399589)
--(axis cs:233,0.296128231092712)
--(axis cs:232,0.114348957245646)
--(axis cs:231,0.171837408480186)
--(axis cs:230,0.134177083498697)
--(axis cs:229,0.104378628362777)
--(axis cs:228,0.0560472306744872)
--(axis cs:227,0.183178606702955)
--(axis cs:226,0.173080572622499)
--(axis cs:225,0.16212111712812)
--(axis cs:224,0.0684282438689771)
--(axis cs:223,0.112423120335029)
--(axis cs:222,0.10511924960233)
--(axis cs:221,0.0930846987281296)
--(axis cs:220,0.185575475054686)
--(axis cs:219,0.108420264636986)
--(axis cs:218,0.119292414514701)
--(axis cs:217,0.18181441929092)
--(axis cs:216,0.0925071313330169)
--(axis cs:215,0.0293937220676471)
--(axis cs:214,0.010974670386543)
--(axis cs:213,-0.126116284871859)
--(axis cs:212,-0.0432093962130918)
--(axis cs:211,-0.0569877080616943)
--(axis cs:210,0.0757949250352106)
--(axis cs:209,-0.00779900384473309)
--(axis cs:208,-0.0187618335716869)
--(axis cs:207,0.101462163463892)
--(axis cs:206,0.114296862479939)
--(axis cs:205,0.0470855515416678)
--(axis cs:204,-0.0234508854152955)
--(axis cs:203,0.00439312163637529)
--(axis cs:202,0.0038608124339985)
--(axis cs:201,-0.0736425346821362)
--(axis cs:200,-0.128659380855277)
--(axis cs:199,-0.0108307337546539)
--(axis cs:198,-0.0204473181695881)
--(axis cs:197,-0.0596887944097583)
--(axis cs:196,-0.0328620044373262)
--(axis cs:195,-0.0686090233105229)
--(axis cs:194,0.0427738252149111)
--(axis cs:193,0.0627673075049678)
--(axis cs:192,0.0374103405290065)
--(axis cs:191,-0.102881554618704)
--(axis cs:190,-0.0682977659748107)
--(axis cs:189,-0.0899728841771968)
--(axis cs:188,-0.135321548770548)
--(axis cs:187,-0.13812931020754)
--(axis cs:186,-0.0686877593947003)
--(axis cs:185,-0.166242803577683)
--(axis cs:184,-0.0941569090378541)
--(axis cs:183,-0.0288590510332736)
--(axis cs:182,-0.10433690730183)
--(axis cs:181,-0.145334158529072)
--(axis cs:180,-0.0775143649671702)
--(axis cs:179,-0.0759211223734747)
--(axis cs:178,-0.202785883979379)
--(axis cs:177,-0.16412791205658)
--(axis cs:176,-0.0985514815656248)
--(axis cs:175,-0.221397003195111)
--(axis cs:174,-0.0761718992585485)
--(axis cs:173,-0.100785449666975)
--(axis cs:172,-0.12082689116944)
--(axis cs:171,-0.221397003195111)
--(axis cs:170,-0.221397003195111)
--(axis cs:169,-0.151195503620775)
--(axis cs:168,-0.176708284208812)
--(axis cs:167,-0.102396503836172)
--(axis cs:166,-0.191517857142857)
--(axis cs:165,-0.212097235538878)
--(axis cs:164,-0.184514431390008)
--(axis cs:163,-0.119831818313745)
--(axis cs:162,-0.202785883979379)
--(axis cs:161,-0.221397003195111)
--(axis cs:160,-0.192025254551604)
--(axis cs:159,-0.148883300414438)
--(axis cs:158,-0.221397003195111)
--(axis cs:157,-0.215116135051176)
--(axis cs:156,-0.205808656400907)
--(axis cs:155,-0.221397003195111)
--(axis cs:154,-0.172769175476164)
--(axis cs:153,-0.212097235538878)
--(axis cs:152,-0.221397003195111)
--(axis cs:151,-0.172769175476164)
--(axis cs:150,-0.221397003195111)
--(axis cs:149,-0.221397003195111)
--(axis cs:148,-0.221397003195111)
--(axis cs:147,-0.221397003195111)
--(axis cs:146,-0.156199919443944)
--(axis cs:145,-0.172769175476164)
--(axis cs:144,-0.221397003195111)
--(axis cs:143,-0.212097235538878)
--(axis cs:142,-0.212097235538878)
--(axis cs:141,-0.182127098010829)
--(axis cs:140,-0.221397003195111)
--(axis cs:139,-0.172769175476164)
--(axis cs:138,-0.00885588012780441)
--(axis cs:137,-0.0265676403834133)
--(axis cs:136,-0.0153971042936919)
--(axis cs:135,-0.162880229057535)
--(axis cs:134,-0.0564943499476449)
--(axis cs:133,0)
--(axis cs:132,-0.00885588012780441)
--(axis cs:131,0)
--(axis cs:130,0)
--(axis cs:129,0)
--(axis cs:128,-0.00885588012780442)
--(axis cs:127,-0.0341345877087521)
--(axis cs:126,0)
--(axis cs:125,-0.00295196004260148)
--(axis cs:124,0)
--(axis cs:123,-0.154977902236578)
--(axis cs:122,-0.166047752396333)
--(axis cs:121,-0.221397003195111)
--(axis cs:120,-0.221397003195111)
--(axis cs:119,0)
--(axis cs:118,-0.182127098010829)
--(axis cs:117,-0.158140716567936)
--(axis cs:116,-0.177117602556089)
--(axis cs:115,0)
--(axis cs:114,0)
--(axis cs:113,-0.212097235538878)
--(axis cs:112,-0.221397003195111)
--(axis cs:111,-0.166784545284025)
--(axis cs:110,-0.221397003195111)
--(axis cs:109,-0.0442794006390222)
--(axis cs:108,-0.0331265318504755)
--(axis cs:107,0)
--(axis cs:106,0)
--(axis cs:105,-0.0228197950790997)
--(axis cs:104,-0.0147598002130074)
--(axis cs:103,-0.147598002130074)
--(axis cs:102,-0.0411165863076634)
--(axis cs:101,0)
--(axis cs:100,0)
--(axis cs:99,0)
--(axis cs:98,-0.0236733895885294)
--(axis cs:97,-0.0105427144378624)
--(axis cs:96,-0.02735465200943)
--(axis cs:95,-0.0192256453771185)
--(axis cs:94,0)
--(axis cs:93,-0.00590392008520296)
--(axis cs:92,-0.0073799001065037)
--(axis cs:91,-0.0221309006377584)
--(axis cs:90,-0.0295196004260148)
--(axis cs:89,-0.00590392008520296)
--(axis cs:88,-0.00695641264827046)
--(axis cs:87,-0.00612468008924212)
--(axis cs:86,-0.0098398668086716)
--(axis cs:85,-0.00590392008520296)
--(axis cs:84,-0.0049199334043358)
--(axis cs:83,-0.00885588012780444)
--(axis cs:82,0)
--(axis cs:81,-0.00295196004260148)
--(axis cs:80,0)
--(axis cs:79,0)
--(axis cs:78,0)
--(axis cs:77,0)
--(axis cs:76,-0.0253025146508698)
--(axis cs:75,0)
--(axis cs:74,0)
--(axis cs:73,0)
--(axis cs:72,0)
--(axis cs:71,0)
--(axis cs:70,0)
--(axis cs:69,0)
--(axis cs:68,0)
--(axis cs:67,0)
--(axis cs:66,0)
--(axis cs:65,0)
--(axis cs:64,0)
--(axis cs:63,0)
--(axis cs:62,-0.00295196004260147)
--(axis cs:61,0)
--(axis cs:60,0)
--(axis cs:59,0)
--(axis cs:58,0)
--(axis cs:57,-0.00885588012780441)
--(axis cs:56,0)
--(axis cs:55,0)
--(axis cs:54,0)
--(axis cs:53,0)
--(axis cs:52,0)
--(axis cs:51,-0.00885588012780443)
--(axis cs:50,-0.126512573254349)
--(axis cs:49,0)
--(axis cs:48,0)
--(axis cs:47,-0.0386446661803853)
--(axis cs:46,0)
--(axis cs:45,-0.00885588012780442)
--(axis cs:44,-0.0149944348289932)
--(axis cs:43,0)
--(axis cs:42,-0.00885588012780443)
--(axis cs:41,-0.0442794006390222)
--(axis cs:40,0)
--(axis cs:39,-0.00295196004260148)
--(axis cs:38,-0.0189768859881524)
--(axis cs:37,-0.00590392008520296)
--(axis cs:36,0)
--(axis cs:35,0)
--(axis cs:34,0)
--(axis cs:33,0)
--(axis cs:32,0)
--(axis cs:31,0)
--(axis cs:30,0)
--(axis cs:29,0)
--(axis cs:28,0)
--(axis cs:27,-0.00295196004260148)
--(axis cs:26,-0.00799891490963396)
--(axis cs:25,-0.0110698501597556)
--(axis cs:24,0)
--(axis cs:23,0)
--(axis cs:22,0)
--(axis cs:21,0)
--(axis cs:20,0)
--(axis cs:19,0)
--(axis cs:18,-0.00590392008520295)
--(axis cs:17,-0.00533540865679347)
--(axis cs:16,-0.00295196004260147)
--(axis cs:15,0)
--(axis cs:14,0)
--(axis cs:13,0)
--(axis cs:12,0)
--(axis cs:11,-0.00295196004260147)
--(axis cs:10,-0.0073799001065037)
--(axis cs:9,-0.00295196004260148)
--(axis cs:8,0)
--(axis cs:7,-0.0024599667021679)
--(axis cs:6,-0.00295196004260147)
--(axis cs:5,0)
--(axis cs:4,0)
--(axis cs:3,0)
--(axis cs:2,0)
--(axis cs:1,0)
--(axis cs:0,0)
--cycle;

\path [fill=color1, fill opacity=0.3] (axis cs:0,0)
--(axis cs:0,0)
--(axis cs:1,0.00766830003550123)
--(axis cs:2,0)
--(axis cs:3,0)
--(axis cs:4,0)
--(axis cs:5,0)
--(axis cs:6,0)
--(axis cs:7,0.00920196004260148)
--(axis cs:8,0)
--(axis cs:9,0)
--(axis cs:10,0)
--(axis cs:11,0)
--(axis cs:12,0.0197184858055746)
--(axis cs:13,0)
--(axis cs:14,0)
--(axis cs:15,0)
--(axis cs:16,0)
--(axis cs:17,0.0460098002130074)
--(axis cs:18,0.0460098002130074)
--(axis cs:19,0.0460098002130074)
--(axis cs:20,0.0394369716111492)
--(axis cs:21,0)
--(axis cs:22,0)
--(axis cs:23,0)
--(axis cs:24,0)
--(axis cs:25,0.00920196004260148)
--(axis cs:26,0)
--(axis cs:27,0)
--(axis cs:28,0)
--(axis cs:29,0)
--(axis cs:30,0)
--(axis cs:31,0)
--(axis cs:32,0)
--(axis cs:33,0)
--(axis cs:34,0)
--(axis cs:35,0)
--(axis cs:36,0)
--(axis cs:37,0)
--(axis cs:38,0)
--(axis cs:39,0)
--(axis cs:40,0)
--(axis cs:41,0)
--(axis cs:42,0)
--(axis cs:43,0)
--(axis cs:44,0)
--(axis cs:45,0)
--(axis cs:46,0)
--(axis cs:47,0)
--(axis cs:48,0)
--(axis cs:49,0)
--(axis cs:50,0)
--(axis cs:51,0)
--(axis cs:52,0)
--(axis cs:53,0)
--(axis cs:54,0)
--(axis cs:55,0)
--(axis cs:56,0)
--(axis cs:57,0)
--(axis cs:58,0)
--(axis cs:59,0)
--(axis cs:60,0)
--(axis cs:61,0)
--(axis cs:62,0)
--(axis cs:63,0)
--(axis cs:64,0)
--(axis cs:65,0)
--(axis cs:66,0)
--(axis cs:67,0)
--(axis cs:68,0)
--(axis cs:69,0.00920196004260148)
--(axis cs:70,0.00920196004260148)
--(axis cs:71,0.00920196004260148)
--(axis cs:72,0.00920196004260148)
--(axis cs:73,0)
--(axis cs:74,0)
--(axis cs:75,0)
--(axis cs:76,0)
--(axis cs:77,0)
--(axis cs:78,0)
--(axis cs:79,0)
--(axis cs:80,0)
--(axis cs:81,0)
--(axis cs:82,0)
--(axis cs:83,0)
--(axis cs:84,0)
--(axis cs:85,0)
--(axis cs:86,0.138029400639022)
--(axis cs:87,0)
--(axis cs:88,0.138029400639022)
--(axis cs:89,1.04655226203464)
--(axis cs:90,1.47991372695113)
--(axis cs:91,1.47991372695113)
--(axis cs:92,1.4566316555709)
--(axis cs:93,1.48899323634772)
--(axis cs:94,1.76091226127284)
--(axis cs:95,1.80894886181183)
--(axis cs:96,1.82707586233202)
--(axis cs:97,1.84538689660166)
--(axis cs:98,2.15053078331908)
--(axis cs:99,2.16700281699366)
--(axis cs:100,2.16700281699366)
--(axis cs:101,2.14100109981734)
--(axis cs:102,2.15812809120535)
--(axis cs:103,2.15812809120535)
--(axis cs:104,2.45964500439139)
--(axis cs:105,2.46038022857978)
--(axis cs:106,2.46038022857978)
--(axis cs:107,2.47744118812817)
--(axis cs:108,2.47744118812817)
--(axis cs:109,2.47744118812817)
--(axis cs:110,2.71994253128851)
--(axis cs:111,2.78762605478493)
--(axis cs:112,2.77242670647522)
--(axis cs:113,2.77242670647522)
--(axis cs:114,2.77242670647522)
--(axis cs:115,2.78762605478493)
--(axis cs:116,2.78762605478493)
--(axis cs:117,2.79732701992038)
--(axis cs:118,2.78762605478493)
--(axis cs:119,2.80064013666482)
--(axis cs:120,2.77242670647522)
--(axis cs:121,2.80231160486618)
--(axis cs:122,2.77182363882651)
--(axis cs:123,2.77242670647522)
--(axis cs:124,2.78762605478493)
--(axis cs:125,3.02192689120142)
--(axis cs:126,3.0903920085203)
--(axis cs:127,3.09903538276507)
--(axis cs:128,3.13634426251742)
--(axis cs:129,3.09392168719799)
--(axis cs:130,3.09903538276507)
--(axis cs:131,3.14496534732652)
--(axis cs:132,3.11415811672739)
--(axis cs:133,3.11251302516239)
--(axis cs:134,3.13634426251742)
--(axis cs:135,3.28155103358666)
--(axis cs:136,3.46901605815327)
--(axis cs:137,3.60467013770738)
--(axis cs:138,3.67838289746128)
--(axis cs:139,3.9136578547237)
--(axis cs:140,3.88002811011053)
--(axis cs:141,4.230067872793)
--(axis cs:142,4.22603968643857)
--(axis cs:143,4.43268883721261)
--(axis cs:144,4.54338400695997)
--(axis cs:145,4.35108723293945)
--(axis cs:146,4.80455318943983)
--(axis cs:147,4.53938801914354)
--(axis cs:148,4.61341082838558)
--(axis cs:149,4.54326965778364)
--(axis cs:150,4.76325860589769)
--(axis cs:151,4.97606684059348)
--(axis cs:152,5.0996179398239)
--(axis cs:153,5.05714299381427)
--(axis cs:154,5.56052122368706)
--(axis cs:155,5.5883048113405)
--(axis cs:156,5.78073632018295)
--(axis cs:157,5.83455789413618)
--(axis cs:158,5.87106382951634)
--(axis cs:159,6.05755120350177)
--(axis cs:160,6.3599814544092)
--(axis cs:161,6.59976093304158)
--(axis cs:162,6.59525664965736)
--(axis cs:163,6.39131924829887)
--(axis cs:164,6.52900689408761)
--(axis cs:165,6.94740486370902)
--(axis cs:166,6.89215183116217)
--(axis cs:167,7.15888812194212)
--(axis cs:168,7.09021027889699)
--(axis cs:169,7.35333752552354)
--(axis cs:170,7.40415172466403)
--(axis cs:171,7.62759558628601)
--(axis cs:172,7.72279924258721)
--(axis cs:173,7.77471428549859)
--(axis cs:174,8.09853864428269)
--(axis cs:175,8.31378366477699)
--(axis cs:176,8.26761710170526)
--(axis cs:177,8.32986507796443)
--(axis cs:178,8.453646209605)
--(axis cs:179,8.36466385338851)
--(axis cs:180,8.57806926402257)
--(axis cs:181,8.71766866605476)
--(axis cs:182,8.43809886107285)
--(axis cs:183,8.5812731013175)
--(axis cs:184,8.77608452286909)
--(axis cs:185,8.68096276252802)
--(axis cs:186,8.92213409231985)
--(axis cs:187,8.98812436138107)
--(axis cs:188,8.94941893065911)
--(axis cs:189,9.1872890710362)
--(axis cs:190,9.19542895338468)
--(axis cs:191,9.20298423474543)
--(axis cs:192,9.28080437288749)
--(axis cs:193,9.74593523870227)
--(axis cs:194,10.0692935434619)
--(axis cs:195,9.9714437255706)
--(axis cs:196,10.1156533055051)
--(axis cs:197,10.4056421709876)
--(axis cs:198,10.3229129068851)
--(axis cs:199,10.4502570870757)
--(axis cs:200,10.7750658979958)
--(axis cs:201,11.243130337803)
--(axis cs:202,11.4755869532563)
--(axis cs:203,11.647587879988)
--(axis cs:204,11.6226626134273)
--(axis cs:205,11.708564317062)
--(axis cs:206,11.8819832762066)
--(axis cs:207,11.5168658916873)
--(axis cs:208,12.0631982976042)
--(axis cs:209,12.1321879562793)
--(axis cs:210,12.4853133412763)
--(axis cs:211,12.7275312488877)
--(axis cs:212,12.6401397542036)
--(axis cs:213,12.7780803355453)
--(axis cs:214,12.7917507115403)
--(axis cs:215,12.9174035983643)
--(axis cs:216,12.8323019325252)
--(axis cs:217,13.059902127283)
--(axis cs:218,13.1190512496518)
--(axis cs:219,13.2208852473614)
--(axis cs:220,13.4759884235511)
--(axis cs:221,13.4981743453906)
--(axis cs:222,13.4918510448886)
--(axis cs:223,13.5026667544164)
--(axis cs:224,13.5153284269792)
--(axis cs:225,13.6588594166566)
--(axis cs:226,13.7571936837462)
--(axis cs:227,14.0085501970018)
--(axis cs:228,14.0035914444199)
--(axis cs:229,13.8358652635216)
--(axis cs:230,14.176180974812)
--(axis cs:231,14.2000108972013)
--(axis cs:232,14.211418364808)
--(axis cs:233,14.2566283232075)
--(axis cs:234,14.3945499807102)
--(axis cs:235,14.3513399088673)
--(axis cs:236,14.3426236653748)
--(axis cs:237,14.3589891991886)
--(axis cs:238,14.367882016904)
--(axis cs:239,14.3537983380451)
--(axis cs:240,14.5084154099071)
--(axis cs:241,14.5237215875688)
--(axis cs:242,14.6652164918873)
--(axis cs:243,14.6617141777885)
--(axis cs:244,14.6816216602999)
--(axis cs:245,14.5095856359855)
--(axis cs:246,14.5343332507066)
--(axis cs:247,14.6834477432276)
--(axis cs:248,14.6740469705891)
--(axis cs:249,14.757745100416)
--(axis cs:250,14.7903603916005)
--(axis cs:251,14.7955425397723)
--(axis cs:252,14.679870077598)
--(axis cs:253,14.8218825930482)
--(axis cs:254,14.8317810984873)
--(axis cs:255,14.8481431783805)
--(axis cs:256,14.821648782519)
--(axis cs:257,14.7444580806577)
--(axis cs:258,14.90869356208)
--(axis cs:259,14.8998306759008)
--(axis cs:260,14.9053246840307)
--(axis cs:261,14.9080332614569)
--(axis cs:262,14.9186565940955)
--(axis cs:263,14.9423896970925)
--(axis cs:264,14.9168581469395)
--(axis cs:265,14.7001740092209)
--(axis cs:266,14.7328819408722)
--(axis cs:267,14.9033915803714)
--(axis cs:268,14.8986397621198)
--(axis cs:269,14.773268093598)
--(axis cs:270,14.8377158216261)
--(axis cs:271,14.8374544481386)
--(axis cs:272,14.8902526597792)
--(axis cs:273,14.9321258602047)
--(axis cs:274,14.9182947920878)
--(axis cs:275,14.9505526215048)
--(axis cs:276,14.9621225951715)
--(axis cs:277,14.9621225951715)
--(axis cs:278,14.9555714004668)
--(axis cs:279,14.8582213141291)
--(axis cs:280,14.9311082600029)
--(axis cs:281,14.9621225951715)
--(axis cs:282,14.9262960376948)
--(axis cs:283,14.9621225951715)
--(axis cs:284,15.0575577731831)
--(axis cs:285,15.0587340291172)
--(axis cs:286,15.0587340291172)
--(axis cs:287,15.0680084466093)
--(axis cs:288,15.052791779541)
--(axis cs:289,15.0590784846081)
--(axis cs:290,15.1121788396872)
--(axis cs:291,15.1219957675552)
--(axis cs:292,15.1237572077792)
--(axis cs:293,15.1392271210845)
--(axis cs:294,15.1491850743075)
--(axis cs:295,15.1106084091223)
--(axis cs:296,15.0692485584987)
--(axis cs:297,15.1392271210845)
--(axis cs:298,15.1335174575548)
--(axis cs:299,15.1704470104844)
--(axis cs:300,15.1771176025561)
--(axis cs:301,15.1771176025561)
--(axis cs:302,15.1685598337931)
--(axis cs:303,15.1771176025561)
--(axis cs:304,15.1771176025561)
--(axis cs:305,15.1771176025561)
--(axis cs:306,15.1667238097674)
--(axis cs:307,15.1771176025561)
--(axis cs:308,15.1693804538976)
--(axis cs:309,15.1688520162368)
--(axis cs:310,15.1667238097674)
--(axis cs:311,15.1771176025561)
--(axis cs:312,15.1771176025561)
--(axis cs:313,15.1688520162368)
--(axis cs:314,15.1771176025561)
--(axis cs:315,15.1771176025561)
--(axis cs:316,15.1771176025561)
--(axis cs:317,15.1771176025561)
--(axis cs:318,15.1771176025561)
--(axis cs:319,15.1771176025561)
--(axis cs:320,15.1771176025561)
--(axis cs:321,15.1771176025561)
--(axis cs:322,15.1682462891455)
--(axis cs:323,15.1771176025561)
--(axis cs:324,15.1771176025561)
--(axis cs:325,15.1771176025561)
--(axis cs:326,15.1572609773185)
--(axis cs:327,15.1771176025561)
--(axis cs:328,15.1771176025561)
--(axis cs:329,15.1771176025561)
--(axis cs:330,15.1771176025561)
--(axis cs:331,15.1771176025561)
--(axis cs:332,15.1708772483968)
--(axis cs:333,15.1771176025561)
--(axis cs:334,15.1771176025561)
--(axis cs:335,15.1442243286783)
--(axis cs:336,15.1771176025561)
--(axis cs:337,15.1771176025561)
--(axis cs:338,15.1771176025561)
--(axis cs:339,15.1771176025561)
--(axis cs:340,15.1771176025561)
--(axis cs:341,15.1696201608299)
--(axis cs:342,15.1771176025561)
--(axis cs:343,15.1771176025561)
--(axis cs:344,15.1685598337931)
--(axis cs:345,15.1771176025561)
--(axis cs:346,15.1771176025561)
--(axis cs:347,15.1710722266753)
--(axis cs:348,15.1693804538976)
--(axis cs:349,15.1771176025561)
--(axis cs:350,15.1771176025561)
--(axis cs:351,15.1771176025561)
--(axis cs:352,15.1771176025561)
--(axis cs:353,15.1771176025561)
--(axis cs:354,15.1400578963866)
--(axis cs:355,15.1771176025561)
--(axis cs:356,15.1771176025561)
--(axis cs:357,15.1771176025561)
--(axis cs:358,15.1771176025561)
--(axis cs:359,15.1771176025561)
--(axis cs:360,15.1693804538976)
--(axis cs:361,15.1771176025561)
--(axis cs:362,15.1771176025561)
--(axis cs:363,15.1685598337931)
--(axis cs:364,15.1771176025561)
--(axis cs:365,15.1771176025561)
--(axis cs:366,15.1771176025561)
--(axis cs:367,15.1557990727687)
--(axis cs:368,15.1771176025561)
--(axis cs:369,15.1771176025561)
--(axis cs:370,15.1771176025561)
--(axis cs:371,15.1771176025561)
--(axis cs:372,15.1685598337931)
--(axis cs:373,15.1572609773185)
--(axis cs:374,15.1771176025561)
--(axis cs:375,15.1771176025561)
--(axis cs:376,15.1771176025561)
--(axis cs:377,15.1771176025561)
--(axis cs:378,15.1771176025561)
--(axis cs:379,15.1771176025561)
--(axis cs:380,15.1771176025561)
--(axis cs:381,15.1771176025561)
--(axis cs:382,15.1771176025561)
--(axis cs:383,15.1706692893708)
--(axis cs:384,15.1771176025561)
--(axis cs:385,15.1572609773185)
--(axis cs:386,15.1771176025561)
--(axis cs:387,15.1771176025561)
--(axis cs:388,15.1771176025561)
--(axis cs:389,15.1771176025561)
--(axis cs:390,15.1771176025561)
--(axis cs:391,15.1771176025561)
--(axis cs:392,15.1771176025561)
--(axis cs:393,15.1771176025561)
--(axis cs:394,15.1771176025561)
--(axis cs:395,15.1771176025561)
--(axis cs:396,15.1771176025561)
--(axis cs:397,15.1771176025561)
--(axis cs:398,15.1771176025561)
--(axis cs:399,15.1771176025561)
--(axis cs:400,15.1771176025561)
--(axis cs:401,15.1771176025561)
--(axis cs:402,15.1771176025561)
--(axis cs:403,15.1682462891455)
--(axis cs:404,15.1572609773185)
--(axis cs:405,15.1572609773185)
--(axis cs:406,15.1771176025561)
--(axis cs:407,15.1771176025561)
--(axis cs:408,15.1771176025561)
--(axis cs:409,15.1771176025561)
--(axis cs:410,15.1771176025561)
--(axis cs:411,15.1771176025561)
--(axis cs:412,15.1657491042731)
--(axis cs:413,15.1771176025561)
--(axis cs:414,15.1771176025561)
--(axis cs:415,15.1696201608299)
--(axis cs:416,15.1771176025561)
--(axis cs:417,15.1771176025561)
--(axis cs:418,15.1771176025561)
--(axis cs:419,15.1771176025561)
--(axis cs:420,15.1771176025561)
--(axis cs:421,15.1771176025561)
--(axis cs:422,15.1771176025561)
--(axis cs:423,15.1771176025561)
--(axis cs:424,15.1771176025561)
--(axis cs:425,15.1771176025561)
--(axis cs:426,15.1771176025561)
--(axis cs:427,15.1771176025561)
--(axis cs:428,15.1771176025561)
--(axis cs:429,15.1771176025561)
--(axis cs:430,15.1771176025561)
--(axis cs:431,15.1771176025561)
--(axis cs:432,15.1771176025561)
--(axis cs:433,15.1771176025561)
--(axis cs:434,15.1771176025561)
--(axis cs:435,15.1771176025561)
--(axis cs:436,15.1771176025561)
--(axis cs:437,15.1771176025561)
--(axis cs:438,15.1771176025561)
--(axis cs:439,15.1685598337931)
--(axis cs:440,15.1771176025561)
--(axis cs:441,15.1771176025561)
--(axis cs:442,15.1667238097674)
--(axis cs:443,15.1771176025561)
--(axis cs:444,15.1771176025561)
--(axis cs:445,15.1771176025561)
--(axis cs:446,15.1771176025561)
--(axis cs:447,15.1771176025561)
--(axis cs:448,15.1771176025561)
--(axis cs:449,15.1771176025561)
--(axis cs:450,15.1771176025561)
--(axis cs:451,15.1771176025561)
--(axis cs:452,15.1771176025561)
--(axis cs:453,15.1771176025561)
--(axis cs:454,15.1771176025561)
--(axis cs:455,15.1685598337931)
--(axis cs:456,15.1771176025561)
--(axis cs:457,15.1771176025561)
--(axis cs:458,15.1671512569825)
--(axis cs:459,15.1771176025561)
--(axis cs:460,15.1771176025561)
--(axis cs:461,15.1771176025561)
--(axis cs:462,15.1771176025561)
--(axis cs:463,15.1771176025561)
--(axis cs:464,15.1771176025561)
--(axis cs:465,15.1771176025561)
--(axis cs:466,15.1771176025561)
--(axis cs:467,15.1693804538976)
--(axis cs:468,15.1771176025561)
--(axis cs:469,15.1771176025561)
--(axis cs:470,15.1771176025561)
--(axis cs:471,15.1771176025561)
--(axis cs:472,15.1771176025561)
--(axis cs:473,15.1696201608299)
--(axis cs:474,15.1771176025561)
--(axis cs:475,15.1771176025561)
--(axis cs:476,15.1771176025561)
--(axis cs:477,15.1771176025561)
--(axis cs:478,15.1771176025561)
--(axis cs:479,15.1771176025561)
--(axis cs:480,15.1688520162368)
--(axis cs:481,15.1702088821702)
--(axis cs:482,15.1771176025561)
--(axis cs:483,15.1771176025561)
--(axis cs:484,15.1771176025561)
--(axis cs:485,15.1771176025561)
--(axis cs:486,15.1771176025561)
--(axis cs:487,15.1682462891455)
--(axis cs:488,15.1771176025561)
--(axis cs:489,15.1771176025561)
--(axis cs:490,15.1771176025561)
--(axis cs:491,15.1771176025561)
--(axis cs:492,15.1771176025561)
--(axis cs:493,15.1771176025561)
--(axis cs:494,15.1771176025561)
--(axis cs:495,15.1708772483968)
--(axis cs:496,15.1771176025561)
--(axis cs:497,15.1572609773185)
--(axis cs:498,15.1771176025561)
--(axis cs:499,15.1771176025561)
--(axis cs:500,15.1696201608299)
--(axis cs:501,15.1771176025561)
--(axis cs:502,15.1771176025561)
--(axis cs:503,15.1771176025561)
--(axis cs:504,15.1771176025561)
--(axis cs:505,15.1771176025561)
--(axis cs:506,15.1699531493612)
--(axis cs:507,15.1372877474402)
--(axis cs:508,15.1771176025561)
--(axis cs:509,15.1771176025561)
--(axis cs:510,15.1771176025561)
--(axis cs:511,15.1608115013503)
--(axis cs:512,15.1771176025561)
--(axis cs:513,15.1771176025561)
--(axis cs:514,15.1608115013503)
--(axis cs:515,15.1771176025561)
--(axis cs:516,15.1771176025561)
--(axis cs:517,15.1771176025561)
--(axis cs:518,15.1771176025561)
--(axis cs:519,15.1771176025561)
--(axis cs:520,15.1771176025561)
--(axis cs:521,15.1771176025561)
--(axis cs:522,15.1771176025561)
--(axis cs:523,15.1771176025561)
--(axis cs:524,15.1771176025561)
--(axis cs:525,15.1771176025561)
--(axis cs:526,15.1771176025561)
--(axis cs:527,15.1771176025561)
--(axis cs:528,15.1771176025561)
--(axis cs:529,15.1771176025561)
--(axis cs:530,15.1771176025561)
--(axis cs:531,15.1771176025561)
--(axis cs:532,15.1771176025561)
--(axis cs:533,15.1771176025561)
--(axis cs:534,15.1771176025561)
--(axis cs:535,15.1771176025561)
--(axis cs:536,15.1771176025561)
--(axis cs:537,15.1771176025561)
--(axis cs:538,15.1771176025561)
--(axis cs:539,15.1771176025561)
--(axis cs:540,15.1771176025561)
--(axis cs:541,15.1771176025561)
--(axis cs:542,15.1771176025561)
--(axis cs:543,15.1696201608299)
--(axis cs:544,15.1771176025561)
--(axis cs:545,15.1771176025561)
--(axis cs:546,15.1771176025561)
--(axis cs:547,15.1771176025561)
--(axis cs:548,15.1771176025561)
--(axis cs:549,15.1771176025561)
--(axis cs:550,15.1771176025561)
--(axis cs:551,15.1771176025561)
--(axis cs:552,15.1771176025561)
--(axis cs:553,15.1771176025561)
--(axis cs:554,15.1771176025561)
--(axis cs:555,15.1771176025561)
--(axis cs:556,15.1771176025561)
--(axis cs:557,15.1771176025561)
--(axis cs:558,15.1771176025561)
--(axis cs:559,15.1771176025561)
--(axis cs:560,15.1771176025561)
--(axis cs:561,15.1771176025561)
--(axis cs:562,15.1771176025561)
--(axis cs:563,15.1771176025561)
--(axis cs:564,15.1771176025561)
--(axis cs:565,15.1771176025561)
--(axis cs:566,15.1771176025561)
--(axis cs:567,15.1771176025561)
--(axis cs:568,15.1771176025561)
--(axis cs:569,15.1771176025561)
--(axis cs:570,15.1771176025561)
--(axis cs:571,15.1771176025561)
--(axis cs:572,15.1771176025561)
--(axis cs:573,15.1771176025561)
--(axis cs:574,15.1771176025561)
--(axis cs:575,15.1771176025561)
--(axis cs:576,15.1771176025561)
--(axis cs:577,15.1771176025561)
--(axis cs:578,15.1771176025561)
--(axis cs:579,15.1771176025561)
--(axis cs:580,15.1771176025561)
--(axis cs:581,15.1771176025561)
--(axis cs:582,15.1771176025561)
--(axis cs:583,15.1771176025561)
--(axis cs:584,15.1771176025561)
--(axis cs:585,15.1693804538976)
--(axis cs:586,15.1771176025561)
--(axis cs:587,15.1771176025561)
--(axis cs:588,15.1667238097674)
--(axis cs:589,15.1771176025561)
--(axis cs:590,15.1667238097674)
--(axis cs:591,15.1771176025561)
--(axis cs:592,15.1771176025561)
--(axis cs:593,15.1771176025561)
--(axis cs:594,15.1771176025561)
--(axis cs:595,15.1771176025561)
--(axis cs:596,15.1771176025561)
--(axis cs:597,15.1771176025561)
--(axis cs:598,15.1771176025561)
--(axis cs:599,15.1771176025561)
--(axis cs:599,14.4478823974439)
--(axis cs:599,14.4478823974439)
--(axis cs:598,14.4478823974439)
--(axis cs:597,14.4478823974439)
--(axis cs:596,14.4478823974439)
--(axis cs:595,14.4478823974439)
--(axis cs:594,14.4478823974439)
--(axis cs:593,14.4478823974439)
--(axis cs:592,14.4478823974439)
--(axis cs:591,14.4478823974439)
--(axis cs:590,14.4369693720508)
--(axis cs:589,14.4478823974439)
--(axis cs:588,14.4369693720508)
--(axis cs:587,14.4478823974439)
--(axis cs:586,14.4478823974439)
--(axis cs:585,14.4399945461024)
--(axis cs:584,14.4478823974439)
--(axis cs:583,14.4478823974439)
--(axis cs:582,14.4478823974439)
--(axis cs:581,14.4478823974439)
--(axis cs:580,14.4478823974439)
--(axis cs:579,14.4478823974439)
--(axis cs:578,14.4478823974439)
--(axis cs:577,14.4478823974439)
--(axis cs:576,14.4478823974439)
--(axis cs:575,14.4478823974439)
--(axis cs:574,14.4478823974439)
--(axis cs:573,14.4478823974439)
--(axis cs:572,14.4478823974439)
--(axis cs:571,14.4478823974439)
--(axis cs:570,14.4478823974439)
--(axis cs:569,14.4478823974439)
--(axis cs:568,14.4478823974439)
--(axis cs:567,14.4478823974439)
--(axis cs:566,14.4478823974439)
--(axis cs:565,14.4478823974439)
--(axis cs:564,14.4478823974439)
--(axis cs:563,14.4478823974439)
--(axis cs:562,14.4478823974439)
--(axis cs:561,14.4478823974439)
--(axis cs:560,14.4478823974439)
--(axis cs:559,14.4478823974439)
--(axis cs:558,14.4478823974439)
--(axis cs:557,14.4478823974439)
--(axis cs:556,14.4478823974439)
--(axis cs:555,14.4478823974439)
--(axis cs:554,14.4478823974439)
--(axis cs:553,14.4478823974439)
--(axis cs:552,14.4478823974439)
--(axis cs:551,14.4478823974439)
--(axis cs:550,14.4478823974439)
--(axis cs:549,14.4478823974439)
--(axis cs:548,14.4478823974439)
--(axis cs:547,14.4478823974439)
--(axis cs:546,14.4478823974439)
--(axis cs:545,14.4478823974439)
--(axis cs:544,14.4478823974439)
--(axis cs:543,14.4402588714282)
--(axis cs:542,14.4478823974439)
--(axis cs:541,14.4478823974439)
--(axis cs:540,14.4478823974439)
--(axis cs:539,14.4478823974439)
--(axis cs:538,14.4478823974439)
--(axis cs:537,14.4478823974439)
--(axis cs:536,14.4478823974439)
--(axis cs:535,14.4478823974439)
--(axis cs:534,14.4478823974439)
--(axis cs:533,14.4478823974439)
--(axis cs:532,14.4478823974439)
--(axis cs:531,14.4478823974439)
--(axis cs:530,14.4478823974439)
--(axis cs:529,14.4478823974439)
--(axis cs:528,14.4478823974439)
--(axis cs:527,14.4478823974439)
--(axis cs:526,14.4478823974439)
--(axis cs:525,14.4478823974439)
--(axis cs:524,14.4478823974439)
--(axis cs:523,14.4478823974439)
--(axis cs:522,14.4478823974439)
--(axis cs:521,14.4478823974439)
--(axis cs:520,14.4478823974439)
--(axis cs:519,14.4478823974439)
--(axis cs:518,14.4478823974439)
--(axis cs:517,14.4478823974439)
--(axis cs:516,14.4478823974439)
--(axis cs:515,14.4478823974439)
--(axis cs:514,14.4312023875385)
--(axis cs:513,14.4478823974439)
--(axis cs:512,14.4478823974439)
--(axis cs:511,14.4312023875385)
--(axis cs:510,14.4478823974439)
--(axis cs:509,14.4478823974439)
--(axis cs:508,14.4478823974439)
--(axis cs:507,14.3835455858931)
--(axis cs:506,14.4406237737157)
--(axis cs:505,14.4478823974439)
--(axis cs:504,14.4478823974439)
--(axis cs:503,14.4478823974439)
--(axis cs:502,14.4478823974439)
--(axis cs:501,14.4478823974439)
--(axis cs:500,14.4402588714282)
--(axis cs:499,14.4478823974439)
--(axis cs:498,14.4478823974439)
--(axis cs:497,13.9989890226815)
--(axis cs:496,14.4478823974439)
--(axis cs:495,14.4416227516032)
--(axis cs:494,14.4478823974439)
--(axis cs:493,14.4478823974439)
--(axis cs:492,14.4478823974439)
--(axis cs:491,14.4478823974439)
--(axis cs:490,14.4478823974439)
--(axis cs:489,14.4478823974439)
--(axis cs:488,14.4478823974439)
--(axis cs:487,14.4387248647006)
--(axis cs:486,14.4478823974439)
--(axis cs:485,14.4478823974439)
--(axis cs:484,14.4478823974439)
--(axis cs:483,14.4478823974439)
--(axis cs:482,14.4478823974439)
--(axis cs:481,14.4409022289409)
--(axis cs:480,14.4394069123346)
--(axis cs:479,14.4478823974439)
--(axis cs:478,14.4478823974439)
--(axis cs:477,14.4478823974439)
--(axis cs:476,14.4478823974439)
--(axis cs:475,14.4478823974439)
--(axis cs:474,14.4478823974439)
--(axis cs:473,14.4402588714282)
--(axis cs:472,14.4478823974439)
--(axis cs:471,14.4478823974439)
--(axis cs:470,14.4478823974439)
--(axis cs:469,14.4478823974439)
--(axis cs:468,14.4478823974439)
--(axis cs:467,14.4399945461024)
--(axis cs:466,14.4478823974439)
--(axis cs:465,14.4478823974439)
--(axis cs:464,14.4478823974439)
--(axis cs:463,14.4478823974439)
--(axis cs:462,14.4478823974439)
--(axis cs:461,14.4478823974439)
--(axis cs:460,14.4478823974439)
--(axis cs:459,14.4478823974439)
--(axis cs:458,14.4374683082349)
--(axis cs:457,14.4478823974439)
--(axis cs:456,14.4478823974439)
--(axis cs:455,14.4390790550958)
--(axis cs:454,14.4478823974439)
--(axis cs:453,14.4478823974439)
--(axis cs:452,14.4478823974439)
--(axis cs:451,14.4478823974439)
--(axis cs:450,14.4478823974439)
--(axis cs:449,14.4478823974439)
--(axis cs:448,14.4478823974439)
--(axis cs:447,14.4478823974439)
--(axis cs:446,14.4478823974439)
--(axis cs:445,14.4478823974439)
--(axis cs:444,14.4478823974439)
--(axis cs:443,14.4478823974439)
--(axis cs:442,14.4369693720508)
--(axis cs:441,14.4478823974439)
--(axis cs:440,14.4478823974439)
--(axis cs:439,14.4390790550958)
--(axis cs:438,14.4478823974439)
--(axis cs:437,14.4478823974439)
--(axis cs:436,14.4478823974439)
--(axis cs:435,14.4478823974439)
--(axis cs:434,14.4478823974439)
--(axis cs:433,14.4478823974439)
--(axis cs:432,14.4478823974439)
--(axis cs:431,14.4478823974439)
--(axis cs:430,14.4478823974439)
--(axis cs:429,14.4478823974439)
--(axis cs:428,14.4478823974439)
--(axis cs:427,14.4478823974439)
--(axis cs:426,14.4478823974439)
--(axis cs:425,14.4478823974439)
--(axis cs:424,14.4478823974439)
--(axis cs:423,14.4478823974439)
--(axis cs:422,14.4478823974439)
--(axis cs:421,14.4478823974439)
--(axis cs:420,14.4478823974439)
--(axis cs:419,14.4478823974439)
--(axis cs:418,14.4478823974439)
--(axis cs:417,14.4478823974439)
--(axis cs:416,14.4478823974439)
--(axis cs:415,14.4402588714282)
--(axis cs:414,14.4478823974439)
--(axis cs:413,14.4478823974439)
--(axis cs:412,14.4358133957269)
--(axis cs:411,14.4478823974439)
--(axis cs:410,14.4478823974439)
--(axis cs:409,14.4478823974439)
--(axis cs:408,14.4478823974439)
--(axis cs:407,14.4478823974439)
--(axis cs:406,14.4478823974439)
--(axis cs:405,13.9989890226815)
--(axis cs:404,13.9989890226815)
--(axis cs:403,14.4387248647006)
--(axis cs:402,14.4478823974439)
--(axis cs:401,14.4478823974439)
--(axis cs:400,14.4478823974439)
--(axis cs:399,14.4478823974439)
--(axis cs:398,14.4478823974439)
--(axis cs:397,14.4478823974439)
--(axis cs:396,14.4478823974439)
--(axis cs:395,14.4478823974439)
--(axis cs:394,14.4478823974439)
--(axis cs:393,14.4478823974439)
--(axis cs:392,14.4478823974439)
--(axis cs:391,14.4478823974439)
--(axis cs:390,14.4478823974439)
--(axis cs:389,14.4478823974439)
--(axis cs:388,14.4478823974439)
--(axis cs:387,14.4478823974439)
--(axis cs:386,14.4478823974439)
--(axis cs:385,13.9989890226815)
--(axis cs:384,14.4478823974439)
--(axis cs:383,14.4413996761465)
--(axis cs:382,14.4478823974439)
--(axis cs:381,14.4478823974439)
--(axis cs:380,14.4478823974439)
--(axis cs:379,14.4478823974439)
--(axis cs:378,14.4478823974439)
--(axis cs:377,14.4478823974439)
--(axis cs:376,14.4478823974439)
--(axis cs:375,14.4478823974439)
--(axis cs:374,14.4478823974439)
--(axis cs:373,13.9989890226815)
--(axis cs:372,14.4390790550958)
--(axis cs:371,14.4478823974439)
--(axis cs:370,14.4478823974439)
--(axis cs:369,14.4478823974439)
--(axis cs:368,14.4478823974439)
--(axis cs:367,14.4223259272313)
--(axis cs:366,14.4478823974439)
--(axis cs:365,14.4478823974439)
--(axis cs:364,14.4478823974439)
--(axis cs:363,14.4390790550958)
--(axis cs:362,14.4478823974439)
--(axis cs:361,14.4478823974439)
--(axis cs:360,14.4399945461024)
--(axis cs:359,14.4478823974439)
--(axis cs:358,14.4478823974439)
--(axis cs:357,14.4478823974439)
--(axis cs:356,14.4478823974439)
--(axis cs:355,14.4478823974439)
--(axis cs:354,14.3911921036134)
--(axis cs:353,14.4478823974439)
--(axis cs:352,14.4478823974439)
--(axis cs:351,14.4478823974439)
--(axis cs:350,14.4478823974439)
--(axis cs:349,14.4478823974439)
--(axis cs:348,14.4399945461024)
--(axis cs:347,14.4418309991312)
--(axis cs:346,14.4478823974439)
--(axis cs:345,14.4478823974439)
--(axis cs:344,14.4390790550958)
--(axis cs:343,14.4478823974439)
--(axis cs:342,14.4478823974439)
--(axis cs:341,14.4402588714282)
--(axis cs:340,14.4478823974439)
--(axis cs:339,14.4478823974439)
--(axis cs:338,14.4478823974439)
--(axis cs:337,14.4478823974439)
--(axis cs:336,14.4478823974439)
--(axis cs:335,13.9859840046551)
--(axis cs:334,14.4478823974439)
--(axis cs:333,14.4478823974439)
--(axis cs:332,14.4416227516032)
--(axis cs:331,14.4478823974439)
--(axis cs:330,14.4478823974439)
--(axis cs:329,14.4478823974439)
--(axis cs:328,14.4478823974439)
--(axis cs:327,14.4478823974439)
--(axis cs:326,13.9989890226815)
--(axis cs:325,14.4478823974439)
--(axis cs:324,14.4478823974439)
--(axis cs:323,14.4478823974439)
--(axis cs:322,14.4387248647006)
--(axis cs:321,14.4478823974439)
--(axis cs:320,14.4478823974439)
--(axis cs:319,14.4478823974439)
--(axis cs:318,14.4478823974439)
--(axis cs:317,14.4478823974439)
--(axis cs:316,14.4478823974439)
--(axis cs:315,14.4478823974439)
--(axis cs:314,14.4478823974439)
--(axis cs:313,14.4394069123346)
--(axis cs:312,14.4478823974439)
--(axis cs:311,14.4478823974439)
--(axis cs:310,14.4369693720508)
--(axis cs:309,14.4394069123346)
--(axis cs:308,14.4399945461024)
--(axis cs:307,14.4478823974439)
--(axis cs:306,14.4369693720508)
--(axis cs:305,14.4478823974439)
--(axis cs:304,14.4478823974439)
--(axis cs:303,14.4478823974439)
--(axis cs:302,14.4390790550958)
--(axis cs:301,14.4478823974439)
--(axis cs:300,14.4478823974439)
--(axis cs:299,14.4411601323727)
--(axis cs:298,14.3872414710166)
--(axis cs:297,14.3889986853671)
--(axis cs:296,14.2727786421867)
--(axis cs:295,14.1460386266512)
--(axis cs:294,14.4111597532787)
--(axis cs:293,14.3889986853671)
--(axis cs:292,14.1296425174955)
--(axis cs:291,14.356697414263)
--(axis cs:290,14.0889930353128)
--(axis cs:289,13.7152270709474)
--(axis cs:288,13.6990003351543)
--(axis cs:287,13.7132415533907)
--(axis cs:286,13.704487124729)
--(axis cs:285,13.704487124729)
--(axis cs:284,13.7033117920343)
--(axis cs:283,13.3503774048285)
--(axis cs:282,13.3167426776221)
--(axis cs:281,13.3503774048285)
--(axis cs:280,13.3216962271766)
--(axis cs:279,13.1730286858709)
--(axis cs:278,13.3444285995332)
--(axis cs:277,13.3503774048285)
--(axis cs:276,13.3503774048285)
--(axis cs:275,13.3396259499238)
--(axis cs:274,13.3092633038643)
--(axis cs:273,13.3198902688276)
--(axis cs:272,13.2761119562083)
--(axis cs:271,13.1303719688654)
--(axis cs:270,13.1126483701032)
--(axis cs:269,13.0175972910174)
--(axis cs:268,13.2841048030976)
--(axis cs:267,13.2908950357902)
--(axis cs:266,12.9560148886497)
--(axis cs:265,12.7726878328844)
--(axis cs:264,13.3043112078992)
--(axis cs:263,13.3313171994592)
--(axis cs:262,13.3029343149954)
--(axis cs:261,13.2926878923893)
--(axis cs:260,13.2837431095887)
--(axis cs:259,13.2681535138225)
--(axis cs:258,13.2933153664915)
--(axis cs:257,12.9667026336281)
--(axis cs:256,13.0689656955954)
--(axis cs:255,13.1831068216195)
--(axis cs:254,13.0853384667301)
--(axis cs:253,12.9849428545221)
--(axis cs:252,12.6639546831675)
--(axis cs:251,12.9730380644062)
--(axis cs:250,12.9505662675247)
--(axis cs:249,12.9222559447345)
--(axis cs:248,12.6654619579823)
--(axis cs:247,12.7417052288004)
--(axis cs:246,12.4240167793475)
--(axis cs:245,12.332142305191)
--(axis cs:244,12.6183783397001)
--(axis cs:243,12.5743841677997)
--(axis cs:242,12.6035335081127)
--(axis cs:241,12.2887784124312)
--(axis cs:240,12.2752384362467)
--(axis cs:239,12.0549431163126)
--(axis cs:238,12.0492375483134)
--(axis cs:237,12.054560902026)
--(axis cs:236,12.0274225438984)
--(axis cs:235,12.0463163411327)
--(axis cs:234,12.1262833526231)
--(axis cs:233,11.9489423289665)
--(axis cs:232,11.8525805332994)
--(axis cs:231,11.761447436132)
--(axis cs:230,11.7283329140768)
--(axis cs:229,11.1531120092056)
--(axis cs:228,11.4365270308548)
--(axis cs:227,11.4407355172839)
--(axis cs:226,11.0748375662538)
--(axis cs:225,10.9244739166767)
--(axis cs:224,10.7466689732655)
--(axis cs:223,10.7127779571221)
--(axis cs:222,10.7101149578587)
--(axis cs:221,10.7334305080893)
--(axis cs:220,10.6695856714715)
--(axis cs:219,10.3709162232269)
--(axis cs:218,10.2596957662)
--(axis cs:217,10.079477329045)
--(axis cs:216,9.82271248930521)
--(axis cs:215,9.82322140163566)
--(axis cs:214,9.74470460945452)
--(axis cs:213,9.74848567237152)
--(axis cs:212,9.51876557931387)
--(axis cs:211,9.6774621308943)
--(axis cs:210,9.32935733677756)
--(axis cs:209,8.86906204372074)
--(axis cs:208,8.7975921435723)
--(axis cs:207,8.11969546362401)
--(axis cs:206,8.55661757606612)
--(axis cs:205,8.38942675436661)
--(axis cs:204,8.31664697404403)
--(axis cs:203,8.33511649870025)
--(axis cs:202,8.11320966679035)
--(axis cs:201,7.83513670765157)
--(axis cs:200,7.28360980130493)
--(axis cs:199,6.89531921162561)
--(axis cs:198,6.75688321610952)
--(axis cs:197,6.86232966695705)
--(axis cs:196,6.56044963567142)
--(axis cs:195,6.41015124196186)
--(axis cs:194,6.55726895653807)
--(axis cs:193,6.20188758707036)
--(axis cs:192,5.70747202462804)
--(axis cs:191,5.61883496168315)
--(axis cs:190,5.62019604661532)
--(axis cs:189,5.63798374642412)
--(axis cs:188,5.34453531770691)
--(axis cs:187,5.39916565838177)
--(axis cs:186,5.3170501609269)
--(axis cs:185,5.1036125579848)
--(axis cs:184,5.22497743600537)
--(axis cs:183,4.96865896389989)
--(axis cs:182,4.83342356181676)
--(axis cs:181,5.13930136998337)
--(axis cs:180,4.97138868203464)
--(axis cs:179,4.82809182842967)
--(axis cs:178,4.86238671580293)
--(axis cs:177,4.77382810385376)
--(axis cs:176,4.65649004115188)
--(axis cs:175,4.74327822247791)
--(axis cs:174,4.53911016524112)
--(axis cs:173,4.23181043977614)
--(axis cs:172,4.19545668626654)
--(axis cs:171,4.08415497932485)
--(axis cs:170,3.84584827533597)
--(axis cs:169,3.82935478216876)
--(axis cs:168,3.62475184231514)
--(axis cs:167,3.66304815256768)
--(axis cs:166,3.43005799026641)
--(axis cs:165,3.47115574235158)
--(axis cs:164,3.14049860041789)
--(axis cs:163,2.99781765646304)
--(axis cs:162,3.18343653216082)
--(axis cs:161,3.18806801432684)
--(axis cs:160,2.96925961701937)
--(axis cs:159,2.73802171316489)
--(axis cs:158,2.56643617048366)
--(axis cs:157,2.55718317729239)
--(axis cs:156,2.52349162099352)
--(axis cs:155,2.35287165924774)
--(axis cs:154,2.34094408731773)
--(axis cs:153,1.95280018800391)
--(axis cs:152,2.00306063160467)
--(axis cs:151,1.91787255334591)
--(axis cs:150,1.78237558527878)
--(axis cs:149,1.62033328339283)
--(axis cs:148,1.68769211279089)
--(axis cs:147,1.59559461974535)
--(axis cs:146,1.79075931056017)
--(axis cs:145,1.49545294563198)
--(axis cs:144,1.60197584152488)
--(axis cs:143,1.56024271040643)
--(axis cs:142,1.40242013499)
--(axis cs:141,1.41428308874546)
--(axis cs:140,1.19328919758178)
--(axis cs:139,1.21647763478679)
--(axis cs:138,1.03590281682444)
--(axis cs:137,0.994031160993915)
--(axis cs:136,0.944141836583566)
--(axis cs:135,0.806395394984769)
--(axis cs:134,0.707405737482579)
--(axis cs:133,0.684361974837607)
--(axis cs:132,0.699063037118765)
--(axis cs:131,0.717534652673479)
--(axis cs:130,0.66971461723493)
--(axis cs:129,0.677506884230583)
--(axis cs:128,0.707405737482579)
--(axis cs:127,0.66971461723493)
--(axis cs:126,0.659607991479704)
--(axis cs:125,0.643079976930445)
--(axis cs:124,0.493623945215073)
--(axis cs:123,0.490794447370931)
--(axis cs:122,0.490676361173494)
--(axis cs:121,0.510188395133823)
--(axis cs:120,0.490794447370931)
--(axis cs:119,0.50838764111296)
--(axis cs:118,0.493623945215073)
--(axis cs:117,0.504756313412956)
--(axis cs:116,0.493623945215073)
--(axis cs:115,0.493623945215073)
--(axis cs:114,0.490794447370931)
--(axis cs:113,0.490794447370931)
--(axis cs:112,0.490794447370931)
--(axis cs:111,0.493623945215073)
--(axis cs:110,0.486307468711492)
--(axis cs:109,0.335058811871831)
--(axis cs:108,0.335058811871831)
--(axis cs:107,0.335058811871831)
--(axis cs:106,0.332588521420219)
--(axis cs:105,0.332588521420219)
--(axis cs:104,0.332474560826)
--(axis cs:103,0.18562190879465)
--(axis cs:102,0.18562190879465)
--(axis cs:101,0.183998900182658)
--(axis cs:100,0.19549718300634)
--(axis cs:99,0.19549718300634)
--(axis cs:98,0.193940370527076)
--(axis cs:97,0.0671131033983369)
--(axis cs:96,0.047924137667984)
--(axis cs:95,0.0473011381881676)
--(axis cs:94,0.045432435696856)
--(axis cs:93,-0.0639932363477157)
--(axis cs:92,-0.0727030841423326)
--(axis cs:91,-0.0736637269511314)
--(axis cs:90,-0.0736637269511314)
--(axis cs:89,-0.104725338957721)
--(axis cs:88,-0.0442794006390222)
--(axis cs:87,0)
--(axis cs:86,-0.0442794006390222)
--(axis cs:85,0)
--(axis cs:84,0)
--(axis cs:83,0)
--(axis cs:82,0)
--(axis cs:81,0)
--(axis cs:80,0)
--(axis cs:79,0)
--(axis cs:78,0)
--(axis cs:77,0)
--(axis cs:76,0)
--(axis cs:75,0)
--(axis cs:74,0)
--(axis cs:73,0)
--(axis cs:72,-0.00295196004260148)
--(axis cs:71,-0.00295196004260148)
--(axis cs:70,-0.00295196004260148)
--(axis cs:69,-0.00295196004260148)
--(axis cs:68,0)
--(axis cs:67,0)
--(axis cs:66,0)
--(axis cs:65,0)
--(axis cs:64,0)
--(axis cs:63,0)
--(axis cs:62,0)
--(axis cs:61,0)
--(axis cs:60,0)
--(axis cs:59,0)
--(axis cs:58,0)
--(axis cs:57,0)
--(axis cs:56,0)
--(axis cs:55,0)
--(axis cs:54,0)
--(axis cs:53,0)
--(axis cs:52,0)
--(axis cs:51,0)
--(axis cs:50,0)
--(axis cs:49,0)
--(axis cs:48,0)
--(axis cs:47,0)
--(axis cs:46,0)
--(axis cs:45,0)
--(axis cs:44,0)
--(axis cs:43,0)
--(axis cs:42,0)
--(axis cs:41,0)
--(axis cs:40,0)
--(axis cs:39,0)
--(axis cs:38,0)
--(axis cs:37,0)
--(axis cs:36,0)
--(axis cs:35,0)
--(axis cs:34,0)
--(axis cs:33,0)
--(axis cs:32,0)
--(axis cs:31,0)
--(axis cs:30,0)
--(axis cs:29,0)
--(axis cs:28,0)
--(axis cs:27,0)
--(axis cs:26,0)
--(axis cs:25,-0.00295196004260148)
--(axis cs:24,0)
--(axis cs:23,0)
--(axis cs:22,0)
--(axis cs:21,0)
--(axis cs:20,-0.0126512573254349)
--(axis cs:19,-0.0147598002130074)
--(axis cs:18,-0.0147598002130074)
--(axis cs:17,-0.0147598002130074)
--(axis cs:16,0)
--(axis cs:15,0)
--(axis cs:14,0)
--(axis cs:13,0)
--(axis cs:12,-0.00632562866271745)
--(axis cs:11,0)
--(axis cs:10,0)
--(axis cs:9,0)
--(axis cs:8,0)
--(axis cs:7,-0.00295196004260148)
--(axis cs:6,0)
--(axis cs:5,0)
--(axis cs:4,0)
--(axis cs:3,0)
--(axis cs:2,0)
--(axis cs:1,-0.0024599667021679)
--(axis cs:0,0)
--cycle;

\path [fill=color2, fill opacity=0.3] (axis cs:0,0)
--(axis cs:0,0)
--(axis cs:1,0)
--(axis cs:2,0.0496418065340553)
--(axis cs:3,0.0460098002130074)
--(axis cs:4,0.0421756501952568)
--(axis cs:5,0.0455182940511035)
--(axis cs:6,0.0496418065340553)
--(axis cs:7,0.0424705848120068)
--(axis cs:8,0.0230049001065037)
--(axis cs:9,0.0446035504670646)
--(axis cs:10,0.0230049001065037)
--(axis cs:11,0)
--(axis cs:12,0)
--(axis cs:13,0)
--(axis cs:14,0)
--(axis cs:15,0.00920196004260148)
--(axis cs:16,0)
--(axis cs:17,0)
--(axis cs:18,0)
--(axis cs:19,0)
--(axis cs:20,0)
--(axis cs:21,0.018403920085203)
--(axis cs:22,0.0197184858055746)
--(axis cs:23,0.00766830003550123)
--(axis cs:24,0.0230049001065037)
--(axis cs:25,0)
--(axis cs:26,0.00920196004260148)
--(axis cs:27,0)
--(axis cs:28,0)
--(axis cs:29,0)
--(axis cs:30,0)
--(axis cs:31,0)
--(axis cs:32,0.00920196004260148)
--(axis cs:33,0.00920196004260148)
--(axis cs:34,0.00920196004260148)
--(axis cs:35,0)
--(axis cs:36,0.0276058801278044)
--(axis cs:37,0.0153366000710025)
--(axis cs:38,0.0368078401704059)
--(axis cs:39,0.00920196004260148)
--(axis cs:40,0.0536781002485087)
--(axis cs:41,0.0536781002485087)
--(axis cs:42,0.0153366000710025)
--(axis cs:43,0.00920196004260148)
--(axis cs:44,0)
--(axis cs:45,0)
--(axis cs:46,0)
--(axis cs:47,0)
--(axis cs:48,0)
--(axis cs:49,0.0228769439427949)
--(axis cs:50,0)
--(axis cs:51,0.0316253063700951)
--(axis cs:52,0)
--(axis cs:53,0)
--(axis cs:54,0)
--(axis cs:55,0)
--(axis cs:56,0.00920196004260147)
--(axis cs:57,0)
--(axis cs:58,0)
--(axis cs:59,0)
--(axis cs:60,0)
--(axis cs:61,0)
--(axis cs:62,0)
--(axis cs:63,0.00920196004260147)
--(axis cs:64,0)
--(axis cs:65,0)
--(axis cs:66,0.0153366000710025)
--(axis cs:67,0)
--(axis cs:68,0)
--(axis cs:69,0)
--(axis cs:70,0)
--(axis cs:71,0.00920196004260148)
--(axis cs:72,0)
--(axis cs:73,0)
--(axis cs:74,0.00920196004260148)
--(axis cs:75,0.0552117602556089)
--(axis cs:76,0)
--(axis cs:77,0.0276058801278044)
--(axis cs:78,0.00920196004260148)
--(axis cs:79,0.0147756391805835)
--(axis cs:80,0)
--(axis cs:81,0.027155201904571)
--(axis cs:82,0.0790204431457749)
--(axis cs:83,0.018403920085203)
--(axis cs:84,0)
--(axis cs:85,0.0306732001420049)
--(axis cs:86,0.0262913144074328)
--(axis cs:87,0.0276058801278044)
--(axis cs:88,0.00920196004260148)
--(axis cs:89,0)
--(axis cs:90,0)
--(axis cs:91,0.032864143009291)
--(axis cs:92,0.00920196004260148)
--(axis cs:93,0.0147756391805835)
--(axis cs:94,0.0262913144074328)
--(axis cs:95,0)
--(axis cs:96,0)
--(axis cs:97,0.0153366000710025)
--(axis cs:98,0.0345073501597556)
--(axis cs:99,0.00920196004260148)
--(axis cs:100,0)
--(axis cs:101,0.00920196004260148)
--(axis cs:102,0.00920196004260148)
--(axis cs:103,0.0272104086567935)
--(axis cs:104,0.018403920085203)
--(axis cs:105,0)
--(axis cs:106,0.0405871930924423)
--(axis cs:107,0.00920196004260148)
--(axis cs:108,0)
--(axis cs:109,0.018403920085203)
--(axis cs:110,0.0402585751863815)
--(axis cs:111,0.0460098002130074)
--(axis cs:112,0.0434537002011737)
--(axis cs:113,0.0460098002130074)
--(axis cs:114,0.0460098002130074)
--(axis cs:115,0.0460098002130074)
--(axis cs:116,0.0460098002130074)
--(axis cs:117,0)
--(axis cs:118,0.0816312259703804)
--(axis cs:119,0)
--(axis cs:120,0)
--(axis cs:121,0)
--(axis cs:122,0)
--(axis cs:123,0)
--(axis cs:124,0.00920196004260148)
--(axis cs:125,0.00920196004260147)
--(axis cs:126,0)
--(axis cs:127,0)
--(axis cs:128,0)
--(axis cs:129,0)
--(axis cs:130,0.0601854462100455)
--(axis cs:131,0.0544208173135869)
--(axis cs:132,0.0601854462100455)
--(axis cs:133,0.0496418065340553)
--(axis cs:134,0.0707014124178141)
--(axis cs:135,0.00920196004260148)
--(axis cs:136,0.00920196004260148)
--(axis cs:137,0.0153366000710025)
--(axis cs:138,0.0580053697527031)
--(axis cs:139,0.0738781959029174)
--(axis cs:140,0.0460098002130074)
--(axis cs:141,0.0460098002130074)
--(axis cs:142,0.0460098002130074)
--(axis cs:143,0.0601854462100454)
--(axis cs:144,0.0552117602556089)
--(axis cs:145,0)
--(axis cs:146,0.128732539297037)
--(axis cs:147,0.110423520511218)
--(axis cs:148,0.110423520511218)
--(axis cs:149,0.12422646057512)
--(axis cs:150,0.0953516360705834)
--(axis cs:151,0.152580785841439)
--(axis cs:152,0.137414491116718)
--(axis cs:153,0.0276058801278044)
--(axis cs:154,0.103792542517771)
--(axis cs:155,0.133810651401194)
--(axis cs:156,0.0816312259703804)
--(axis cs:157,0.0276058801278044)
--(axis cs:158,0)
--(axis cs:159,0)
--(axis cs:160,0.0443269175417504)
--(axis cs:161,0.0276058801278044)
--(axis cs:162,0)
--(axis cs:163,0)
--(axis cs:164,0)
--(axis cs:165,0)
--(axis cs:166,0)
--(axis cs:167,0.0197184858055746)
--(axis cs:168,0.107356200497017)
--(axis cs:169,0.0276058801278044)
--(axis cs:170,0)
--(axis cs:171,0.0276058801278044)
--(axis cs:172,0.141260444607727)
--(axis cs:173,0.00920196004260148)
--(axis cs:174,0)
--(axis cs:175,0.230049001065037)
--(axis cs:176,0.0920196004260148)
--(axis cs:177,0.138029400639022)
--(axis cs:178,0.750120792801084)
--(axis cs:179,0.690147003195111)
--(axis cs:180,0.72932774852201)
--(axis cs:181,0.690147003195111)
--(axis cs:182,0.699597235538878)
--(axis cs:183,0.71846295336765)
--(axis cs:184,0.690147003195111)
--(axis cs:185,0.690147003195111)
--(axis cs:186,0.735750155130532)
--(axis cs:187,0.690147003195111)
--(axis cs:188,0.690147003195111)
--(axis cs:189,0.690147003195111)
--(axis cs:190,0.690147003195111)
--(axis cs:191,0.690147003195111)
--(axis cs:192,0.690147003195111)
--(axis cs:193,0.690147003195111)
--(axis cs:194,0.720679068262344)
--(axis cs:195,0.727232142857142)
--(axis cs:196,0.699597235538878)
--(axis cs:197,0.693216053144664)
--(axis cs:198,0.690147003195111)
--(axis cs:199,0.690147003195111)
--(axis cs:200,0.699597235538878)
--(axis cs:201,0.724473330901927)
--(axis cs:202,0.690147003195111)
--(axis cs:203,0.690147003195111)
--(axis cs:204,0.690147003195111)
--(axis cs:205,0.719077344668211)
--(axis cs:206,0.690147003195111)
--(axis cs:207,0.706302223038635)
--(axis cs:208,0.712837322862163)
--(axis cs:209,0.709775868148588)
--(axis cs:210,0.690147003195111)
--(axis cs:211,0.693216053144664)
--(axis cs:212,0.690147003195111)
--(axis cs:213,0.690147003195111)
--(axis cs:214,0.699597235538878)
--(axis cs:215,0.690147003195111)
--(axis cs:216,0.690147003195111)
--(axis cs:217,0.726400254551604)
--(axis cs:218,0.673067701741302)
--(axis cs:219,0.699597235538878)
--(axis cs:220,0.690147003195111)
--(axis cs:221,0.690147003195111)
--(axis cs:222,0.719408190272534)
--(axis cs:223,0.744627098010829)
--(axis cs:224,0.690147003195111)
--(axis cs:225,0.690147003195111)
--(axis cs:226,0.699597235538878)
--(axis cs:227,0.690147003195111)
--(axis cs:228,0.690147003195111)
--(axis cs:229,0.664062724474179)
--(axis cs:230,0.706302223038635)
--(axis cs:231,0.704749968239873)
--(axis cs:232,0.706302223038635)
--(axis cs:233,0.752905667051019)
--(axis cs:234,0.753755870491715)
--(axis cs:235,0.743314922574656)
--(axis cs:236,0.760678846004277)
--(axis cs:237,0.760678846004277)
--(axis cs:238,0.706302223038635)
--(axis cs:239,0.705233255341374)
--(axis cs:240,0.760678846004277)
--(axis cs:241,0.699597235538878)
--(axis cs:242,0.699597235538878)
--(axis cs:243,0.690147003195111)
--(axis cs:244,0.709775868148588)
--(axis cs:245,0.699597235538878)
--(axis cs:246,1.14659634107133)
--(axis cs:247,1.11739528387812)
--(axis cs:248,1.3221514580689)
--(axis cs:249,1.30577123990034)
--(axis cs:250,1.19281301350414)
--(axis cs:251,1.18339120752233)
--(axis cs:252,1.17165046125783)
--(axis cs:253,1.26072824147115)
--(axis cs:254,1.50319144872287)
--(axis cs:255,1.22880583057443)
--(axis cs:256,1.24809330925415)
--(axis cs:257,1.19894797224031)
--(axis cs:258,1.18583960921771)
--(axis cs:259,1.22496485694846)
--(axis cs:260,1.40800568311816)
--(axis cs:261,1.34724587434483)
--(axis cs:262,1.56645430119244)
--(axis cs:263,1.20123714785524)
--(axis cs:264,1.24757212436145)
--(axis cs:265,1.3493793878164)
--(axis cs:266,1.22248344701805)
--(axis cs:267,1.20634647032018)
--(axis cs:268,1.36593094035055)
--(axis cs:269,1.25359415112608)
--(axis cs:270,1.29123202102528)
--(axis cs:271,1.19035706939394)
--(axis cs:272,1.19338744645043)
--(axis cs:273,1.21967112237347)
--(axis cs:274,1.16864978588903)
--(axis cs:275,1.2105281840659)
--(axis cs:276,1.2653198603655)
--(axis cs:277,1.19035706939394)
--(axis cs:278,0.792928742763663)
--(axis cs:279,1.25699776332038)
--(axis cs:280,1.28402859217257)
--(axis cs:281,1.43330766750939)
--(axis cs:282,1.35859260622087)
--(axis cs:283,1.76650862796258)
--(axis cs:284,1.76415407298777)
--(axis cs:285,1.96205746932793)
--(axis cs:286,1.99205391234784)
--(axis cs:287,1.97519855563928)
--(axis cs:288,1.6574670504399)
--(axis cs:289,2.27385712560256)
--(axis cs:290,1.92243279805618)
--(axis cs:291,2.20587194850025)
--(axis cs:292,1.88000032090766)
--(axis cs:293,2.2624312810903)
--(axis cs:294,2.24952784854653)
--(axis cs:295,2.21268200759849)
--(axis cs:296,2.20469075042346)
--(axis cs:297,2.31904718116579)
--(axis cs:298,2.29451593124495)
--(axis cs:299,2.31280231191524)
--(axis cs:300,2.08467506168638)
--(axis cs:301,2.37683813503883)
--(axis cs:302,2.24842695259564)
--(axis cs:303,2.11156843780266)
--(axis cs:304,2.18638944564758)
--(axis cs:305,2.33773817477066)
--(axis cs:306,2.31469853051107)
--(axis cs:307,2.3886606793237)
--(axis cs:308,2.32253540347226)
--(axis cs:309,2.24291271163194)
--(axis cs:310,2.22964552357156)
--(axis cs:311,2.32458988121333)
--(axis cs:312,2.24435670028235)
--(axis cs:313,2.2419171831295)
--(axis cs:314,2.58017950450362)
--(axis cs:315,2.67864359672317)
--(axis cs:316,3.08312331338604)
--(axis cs:317,3.30787402209527)
--(axis cs:318,3.19667032492458)
--(axis cs:319,3.18218712778452)
--(axis cs:320,3.22792032492458)
--(axis cs:321,2.96640863505283)
--(axis cs:322,3.26984962861422)
--(axis cs:323,3.3250217144736)
--(axis cs:324,3.61113390624953)
--(axis cs:325,3.39742782422909)
--(axis cs:326,3.30917955277541)
--(axis cs:327,3.47580561025383)
--(axis cs:328,3.25630776847554)
--(axis cs:329,3.36591991626191)
--(axis cs:330,3.03569495889639)
--(axis cs:331,3.27818967398443)
--(axis cs:332,3.25426387676797)
--(axis cs:333,3.44877231235948)
--(axis cs:334,3.59374493886037)
--(axis cs:335,3.58134901415022)
--(axis cs:336,4.14999710383425)
--(axis cs:337,4.4807228253703)
--(axis cs:338,4.227758846619)
--(axis cs:339,4.36227749442805)
--(axis cs:340,4.53522041212974)
--(axis cs:341,4.75475916080651)
--(axis cs:342,4.76237547029463)
--(axis cs:343,4.85107632650453)
--(axis cs:344,4.85760793961504)
--(axis cs:345,4.81444949411884)
--(axis cs:346,5.46642445279424)
--(axis cs:347,5.45470050498171)
--(axis cs:348,5.56481801119114)
--(axis cs:349,5.77284326292925)
--(axis cs:350,5.78368443741351)
--(axis cs:351,5.73324534468541)
--(axis cs:352,5.78583000504837)
--(axis cs:353,6.02001225212857)
--(axis cs:354,5.8377759651193)
--(axis cs:355,5.79842848445871)
--(axis cs:356,5.89837868517623)
--(axis cs:357,5.76335927277394)
--(axis cs:358,5.84321900402076)
--(axis cs:359,5.85523007181874)
--(axis cs:360,5.91267514289763)
--(axis cs:361,5.77701927035769)
--(axis cs:362,5.77701927035769)
--(axis cs:363,6.25798086654519)
--(axis cs:364,6.05541241427591)
--(axis cs:365,6.03929420455279)
--(axis cs:366,5.82986308896599)
--(axis cs:367,5.8351478694768)
--(axis cs:368,6.10947698116293)
--(axis cs:369,5.90451197339017)
--(axis cs:370,6.04029472637164)
--(axis cs:371,6.19717679853065)
--(axis cs:372,5.81916265888237)
--(axis cs:373,5.96910463584735)
--(axis cs:374,5.8351478694768)
--(axis cs:375,5.853841094313)
--(axis cs:376,5.95535638817093)
--(axis cs:377,6.36960244021292)
--(axis cs:378,5.94410390208869)
--(axis cs:379,6.11393400133392)
--(axis cs:380,5.98131660504176)
--(axis cs:381,6.23081963201996)
--(axis cs:382,6.19435358532637)
--(axis cs:383,6.44856980162202)
--(axis cs:384,6.67857627667461)
--(axis cs:385,6.23611487447051)
--(axis cs:386,6.70839747928059)
--(axis cs:387,7.00988636586145)
--(axis cs:388,6.66640958671712)
--(axis cs:389,7.3933504836775)
--(axis cs:390,7.23294281419246)
--(axis cs:391,7.17618912458792)
--(axis cs:392,7.00469583783545)
--(axis cs:393,7.20562008951191)
--(axis cs:394,6.9889700723319)
--(axis cs:395,6.97437573020006)
--(axis cs:396,7.16282562638099)
--(axis cs:397,7.00459657914769)
--(axis cs:398,7.38486192329016)
--(axis cs:399,7.31935003203339)
--(axis cs:400,7.25973718929334)
--(axis cs:401,7.26787950386002)
--(axis cs:402,7.27256495598434)
--(axis cs:403,7.18834650273363)
--(axis cs:404,7.28228148404219)
--(axis cs:405,7.46427172511441)
--(axis cs:406,7.19333568539085)
--(axis cs:407,7.35509839580939)
--(axis cs:408,7.61166659247703)
--(axis cs:409,7.58724130035283)
--(axis cs:410,7.57846324757191)
--(axis cs:411,7.53515309995838)
--(axis cs:412,7.51099312381783)
--(axis cs:413,7.55172438212141)
--(axis cs:414,7.51454415146669)
--(axis cs:415,7.6396704087994)
--(axis cs:416,7.57968118338688)
--(axis cs:417,7.63478710030339)
--(axis cs:418,7.84213196881955)
--(axis cs:419,7.62270908681701)
--(axis cs:420,7.74558923685323)
--(axis cs:421,7.86378782730305)
--(axis cs:422,8.14440432383988)
--(axis cs:423,8.1794666024764)
--(axis cs:424,8.3677934707568)
--(axis cs:425,8.18438671493503)
--(axis cs:426,8.21140023869876)
--(axis cs:427,8.5259231917177)
--(axis cs:428,8.74175658377956)
--(axis cs:429,8.66668631859013)
--(axis cs:430,8.9019348081388)
--(axis cs:431,8.6224745103202)
--(axis cs:432,8.88174902425074)
--(axis cs:433,8.91143437762891)
--(axis cs:434,8.70559694695694)
--(axis cs:435,8.89480151466647)
--(axis cs:436,8.68889690792082)
--(axis cs:437,8.94558851962617)
--(axis cs:438,9.03554094441432)
--(axis cs:439,8.91898680629902)
--(axis cs:440,8.94340410448563)
--(axis cs:441,8.96242753517599)
--(axis cs:442,8.98106602806929)
--(axis cs:443,8.92360260095593)
--(axis cs:444,8.95164686708624)
--(axis cs:445,8.87774602954747)
--(axis cs:446,8.89911414070199)
--(axis cs:447,8.9739567322015)
--(axis cs:448,9.18562937767255)
--(axis cs:449,9.18339985542739)
--(axis cs:450,8.99769304312165)
--(axis cs:451,9.0517384821256)
--(axis cs:452,9.65334830895301)
--(axis cs:453,9.49358460844759)
--(axis cs:454,9.53515565966115)
--(axis cs:455,9.50265514665089)
--(axis cs:456,9.35864140184382)
--(axis cs:457,9.55886307477604)
--(axis cs:458,9.47191654485859)
--(axis cs:459,9.69346290245006)
--(axis cs:460,9.90125623144484)
--(axis cs:461,9.77736439043022)
--(axis cs:462,9.90405064003582)
--(axis cs:463,10.0487431540046)
--(axis cs:464,9.87174308076601)
--(axis cs:465,9.9143077643456)
--(axis cs:466,10.0341822668493)
--(axis cs:467,10.0064122049053)
--(axis cs:468,10.0967505755283)
--(axis cs:469,10.1062463945914)
--(axis cs:470,9.98966681647483)
--(axis cs:471,10.0217382362775)
--(axis cs:472,9.96047849413324)
--(axis cs:473,10.194632494029)
--(axis cs:474,10.140499851701)
--(axis cs:475,10.1528952081409)
--(axis cs:476,10.1390580031663)
--(axis cs:477,10.1203688555278)
--(axis cs:478,10.6109128988606)
--(axis cs:479,10.5128293691125)
--(axis cs:480,10.3567082099239)
--(axis cs:481,10.320511790977)
--(axis cs:482,10.7735350825079)
--(axis cs:483,11.1277692717227)
--(axis cs:484,10.983765481082)
--(axis cs:485,10.9258398716688)
--(axis cs:486,11.0710799719334)
--(axis cs:487,10.9920842662902)
--(axis cs:488,11.1702952108018)
--(axis cs:489,11.1783104741571)
--(axis cs:490,11.3593361668922)
--(axis cs:491,11.3997840767269)
--(axis cs:492,11.464218855385)
--(axis cs:493,11.4227356880608)
--(axis cs:494,11.3430407077764)
--(axis cs:495,11.3468958313102)
--(axis cs:496,11.4767425294584)
--(axis cs:497,11.491319672768)
--(axis cs:498,11.5629474780731)
--(axis cs:499,11.5595710893042)
--(axis cs:500,11.5108787064961)
--(axis cs:501,11.5066946830199)
--(axis cs:502,11.5246229947839)
--(axis cs:503,11.5769334615967)
--(axis cs:504,11.5042080516588)
--(axis cs:505,11.5446217232773)
--(axis cs:506,11.7136474671392)
--(axis cs:507,11.7293200030982)
--(axis cs:508,11.6944602170864)
--(axis cs:509,11.8223151614766)
--(axis cs:510,11.7277701469953)
--(axis cs:511,11.7448961340928)
--(axis cs:512,11.8139474378454)
--(axis cs:513,11.8450467291371)
--(axis cs:514,11.9771230108637)
--(axis cs:515,12.1160964233711)
--(axis cs:516,12.145487994127)
--(axis cs:517,12.2897040834114)
--(axis cs:518,12.5170299868633)
--(axis cs:519,12.2577150375045)
--(axis cs:520,12.3173970851149)
--(axis cs:521,12.2946182706056)
--(axis cs:522,12.3461204756347)
--(axis cs:523,12.3460254155731)
--(axis cs:524,12.3610696096179)
--(axis cs:525,12.479828492559)
--(axis cs:526,12.3413924605447)
--(axis cs:527,12.3525245260678)
--(axis cs:528,12.3823116790528)
--(axis cs:529,12.3818498335525)
--(axis cs:530,12.3591089993131)
--(axis cs:531,12.4717779569565)
--(axis cs:532,12.5114947115179)
--(axis cs:533,12.7176279082034)
--(axis cs:534,12.7139397667903)
--(axis cs:535,12.7193910406396)
--(axis cs:536,12.8646267060239)
--(axis cs:537,12.8705280366026)
--(axis cs:538,12.879352069384)
--(axis cs:539,12.892649101967)
--(axis cs:540,12.8687792170051)
--(axis cs:541,12.8671140688132)
--(axis cs:542,12.9009594919486)
--(axis cs:543,12.8833872612245)
--(axis cs:544,12.8829761405819)
--(axis cs:545,12.8863822181584)
--(axis cs:546,12.9143379695919)
--(axis cs:547,12.8782456240422)
--(axis cs:548,12.8659745110299)
--(axis cs:549,12.8782456240422)
--(axis cs:550,12.8909529397708)
--(axis cs:551,12.9002210100273)
--(axis cs:552,13.0613607252635)
--(axis cs:553,13.1182514132225)
--(axis cs:554,13.0949347479252)
--(axis cs:555,13.100321429245)
--(axis cs:556,13.1107972602784)
--(axis cs:557,13.1250993913117)
--(axis cs:558,13.3071343767515)
--(axis cs:559,13.4600765036439)
--(axis cs:560,13.4396070466115)
--(axis cs:561,13.4498606166654)
--(axis cs:562,13.3377522992164)
--(axis cs:563,13.4866920607672)
--(axis cs:564,13.6300068483023)
--(axis cs:565,13.6143437710924)
--(axis cs:566,13.4763164892704)
--(axis cs:567,13.5971484034709)
--(axis cs:568,13.6132037642506)
--(axis cs:569,13.6523638520362)
--(axis cs:570,13.6084336274688)
--(axis cs:571,13.527414302698)
--(axis cs:572,13.566027197075)
--(axis cs:573,13.5252808469981)
--(axis cs:574,13.4849266843298)
--(axis cs:575,13.510127564716)
--(axis cs:576,13.5843371442995)
--(axis cs:577,13.5425851301386)
--(axis cs:578,13.6148386019577)
--(axis cs:579,13.6475588022631)
--(axis cs:580,13.56799485324)
--(axis cs:581,13.5534156353342)
--(axis cs:582,13.689974081711)
--(axis cs:583,13.66980089124)
--(axis cs:584,13.6869056953697)
--(axis cs:585,13.6849814293622)
--(axis cs:586,13.5263241624043)
--(axis cs:587,13.5108991752688)
--(axis cs:588,13.6645622618965)
--(axis cs:589,13.494948673287)
--(axis cs:590,13.5226229599815)
--(axis cs:591,13.6813499611701)
--(axis cs:592,13.6839817743002)
--(axis cs:593,13.6650505024548)
--(axis cs:594,13.6571932506076)
--(axis cs:595,13.6729836963253)
--(axis cs:596,13.681836353181)
--(axis cs:597,13.6620301710395)
--(axis cs:598,13.6680325668229)
--(axis cs:599,13.6720672389672)
--(axis cs:599,11.1544456578582)
--(axis cs:599,11.1544456578582)
--(axis cs:598,11.1486340998438)
--(axis cs:597,11.1067198289605)
--(axis cs:596,11.1702469801523)
--(axis cs:595,11.1594381786747)
--(axis cs:594,11.0900493964512)
--(axis cs:593,11.1354825583153)
--(axis cs:592,11.1798575114141)
--(axis cs:591,11.1689887087806)
--(axis cs:590,10.9461270400185)
--(axis cs:589,10.8661624378241)
--(axis cs:588,11.1481419864695)
--(axis cs:587,10.961971007287)
--(axis cs:586,10.976688953512)
--(axis cs:585,11.1837685706378)
--(axis cs:584,11.1892144345004)
--(axis cs:583,11.1529263814872)
--(axis cs:582,11.2021674624066)
--(axis cs:581,11.0201018398376)
--(axis cs:580,11.0619732805774)
--(axis cs:579,11.1105184027899)
--(axis cs:578,11.0998672803952)
--(axis cs:577,10.9623427544768)
--(axis cs:576,11.0495093630534)
--(axis cs:575,10.913933075165)
--(axis cs:574,10.8408037723265)
--(axis cs:573,10.9600625090943)
--(axis cs:572,11.0561318938341)
--(axis cs:571,10.9681438408493)
--(axis cs:570,11.0021132475312)
--(axis cs:569,11.1301117973145)
--(axis cs:568,11.0490129253098)
--(axis cs:567,11.1135701055906)
--(axis cs:566,10.8314960107296)
--(axis cs:565,11.0404238571127)
--(axis cs:564,11.119227341421)
--(axis cs:563,10.8613144047501)
--(axis cs:562,10.6381405579264)
--(axis cs:561,10.7869015361123)
--(axis cs:560,10.7663453343409)
--(axis cs:559,10.825894670925)
--(axis cs:558,10.6322071095247)
--(axis cs:557,10.3213385786131)
--(axis cs:556,10.291584954363)
--(axis cs:555,10.2731227758524)
--(axis cs:554,10.2457910585264)
--(axis cs:553,10.3426667009214)
--(axis cs:552,10.2120849382683)
--(axis cs:551,9.97231188470954)
--(axis cs:550,9.96346174276893)
--(axis cs:549,9.92175437595777)
--(axis cs:548,9.90277548897012)
--(axis cs:547,9.92175437595777)
--(axis cs:546,10.0077774150235)
--(axis cs:545,9.9417427818416)
--(axis cs:544,9.93099010493242)
--(axis cs:543,9.93536273877548)
--(axis cs:542,9.99015892910399)
--(axis cs:541,9.90634851669568)
--(axis cs:540,9.92926228097062)
--(axis cs:539,9.97002450914415)
--(axis cs:538,9.92564793061602)
--(axis cs:537,9.90290946339738)
--(axis cs:536,9.94064401387934)
--(axis cs:535,9.74310895936039)
--(axis cs:534,9.77043523320975)
--(axis cs:533,9.80683547563496)
--(axis cs:532,9.49110070164735)
--(axis cs:531,9.45332888065031)
--(axis cs:530,9.30599516735358)
--(axis cs:529,9.35160604880048)
--(axis cs:528,9.36296075684468)
--(axis cs:527,9.28818802781151)
--(axis cs:526,9.27865451657946)
--(axis cs:525,9.48578795516951)
--(axis cs:524,9.30723396181066)
--(axis cs:523,9.27537965961486)
--(axis cs:522,9.28840175994219)
--(axis cs:521,9.18122994368012)
--(axis cs:520,9.22371402599623)
--(axis cs:519,9.11165996249547)
--(axis cs:518,9.521449121394)
--(axis cs:517,9.21093980395124)
--(axis cs:516,9.02951200587304)
--(axis cs:515,9.03458041994686)
--(axis cs:514,8.91919395342202)
--(axis cs:513,8.64473692470901)
--(axis cs:512,8.6881194097807)
--(axis cs:511,8.55519127849461)
--(axis cs:510,8.50649709438398)
--(axis cs:509,8.63925531300504)
--(axis cs:508,8.47417149842032)
--(axis cs:507,8.53583996713988)
--(axis cs:506,8.514327223241)
--(axis cs:505,8.30073000681958)
--(axis cs:504,8.22723726084115)
--(axis cs:503,8.36106257014932)
--(axis cs:502,8.26433174659543)
--(axis cs:501,8.23006737156154)
--(axis cs:500,8.2357879601706)
--(axis cs:499,8.3321861570726)
--(axis cs:498,8.34676711289151)
--(axis cs:497,8.2679460614977)
--(axis cs:496,8.20088531145072)
--(axis cs:495,8.0516792047648)
--(axis cs:494,8.07611906550795)
--(axis cs:493,8.15120877025865)
--(axis cs:492,8.20380316166048)
--(axis cs:491,8.10733683014917)
--(axis cs:490,8.16392974971264)
--(axis cs:489,7.85406353770065)
--(axis cs:488,7.84999361495576)
--(axis cs:487,7.6495343973043)
--(axis cs:486,7.79650424576426)
--(axis cs:485,7.63433494794067)
--(axis cs:484,7.74943764391798)
--(axis cs:483,7.90409710163822)
--(axis cs:482,7.50079527463493)
--(axis cs:481,6.8602917804516)
--(axis cs:480,6.90826319530487)
--(axis cs:479,7.14657234028924)
--(axis cs:478,7.26758412494888)
--(axis cs:477,6.63900614447225)
--(axis cs:476,6.66778723492893)
--(axis cs:475,6.72810677598604)
--(axis cs:474,6.67335248748027)
--(axis cs:473,6.73371572025674)
--(axis cs:472,6.48139963869884)
--(axis cs:471,6.60460450772422)
--(axis cs:470,6.54415999549996)
--(axis cs:469,6.64736605198976)
--(axis cs:468,6.6783152769621)
--(axis cs:467,6.58346874747565)
--(axis cs:466,6.55534291103797)
--(axis cs:465,6.4325672356544)
--(axis cs:464,6.37761589359297)
--(axis cs:463,6.56745358785008)
--(axis cs:462,6.46488537186894)
--(axis cs:461,6.31956212055879)
--(axis cs:460,6.42968401264763)
--(axis cs:459,6.22166831210152)
--(axis cs:458,5.99225738892742)
--(axis cs:457,6.13314763209102)
--(axis cs:456,5.93833890771731)
--(axis cs:455,6.03816635436331)
--(axis cs:454,6.09083105176424)
--(axis cs:453,6.05412868369505)
--(axis cs:452,6.28951018671799)
--(axis cs:451,5.5888865178744)
--(axis cs:450,5.51215386099326)
--(axis cs:449,5.68743347790594)
--(axis cs:448,5.71727240804174)
--(axis cs:447,5.48017351255374)
--(axis cs:446,5.36494835929801)
--(axis cs:445,5.34239285934142)
--(axis cs:444,5.44460313291376)
--(axis cs:443,5.40139739904407)
--(axis cs:442,5.4848430628398)
--(axis cs:441,5.46014945757763)
--(axis cs:440,5.44042572708127)
--(axis cs:439,5.39413483348593)
--(axis cs:438,5.53533343982214)
--(axis cs:437,5.43446829855564)
--(axis cs:436,5.17471815954545)
--(axis cs:435,5.35740436768647)
--(axis cs:434,5.19797448161449)
--(axis cs:433,5.38272869708373)
--(axis cs:432,5.35508324070652)
--(axis cs:431,5.07726507301313)
--(axis cs:430,5.39836567263043)
--(axis cs:429,5.1501406044868)
--(axis cs:428,5.26621418186784)
--(axis cs:427,5.0365768082823)
--(axis cs:426,4.72695203402851)
--(axis cs:425,4.71301582864208)
--(axis cs:424,4.9015562620558)
--(axis cs:423,4.71004822765913)
--(axis cs:422,4.793309157675)
--(axis cs:421,4.48711122612137)
--(axis cs:420,4.40387615615515)
--(axis cs:419,4.25842930604013)
--(axis cs:418,4.47700281892707)
--(axis cs:417,4.27661974446822)
--(axis cs:416,4.20711511940259)
--(axis cs:415,4.27075037372876)
--(axis cs:414,4.1102807212532)
--(axis cs:413,4.16193108006346)
--(axis cs:412,4.11810823459387)
--(axis cs:411,4.14054598649438)
--(axis cs:410,4.19817816656951)
--(axis cs:409,4.21102976794291)
--(axis cs:408,4.25763673948794)
--(axis cs:407,3.99953330061918)
--(axis cs:406,3.79475473980895)
--(axis cs:405,4.11623422726654)
--(axis cs:404,3.91745191475211)
--(axis cs:403,3.78702500076287)
--(axis cs:402,3.90565580094191)
--(axis cs:401,3.89567772064922)
--(axis cs:400,3.88658634011842)
--(axis cs:399,3.9606512533816)
--(axis cs:398,4.06076984164407)
--(axis cs:397,3.66446198177668)
--(axis cs:396,3.8744328277253)
--(axis cs:395,3.62054614479994)
--(axis cs:394,3.65417623448628)
--(axis cs:393,3.85706106990838)
--(axis cs:392,3.65960704677994)
--(axis cs:391,3.7831351448926)
--(axis cs:390,3.88014904186815)
--(axis cs:389,4.03295159965583)
--(axis cs:388,3.3603498670644)
--(axis cs:387,3.66933441335933)
--(axis cs:386,3.45320537099414)
--(axis cs:385,3.06277581817018)
--(axis cs:384,3.40332476499205)
--(axis cs:383,3.22107305552084)
--(axis cs:382,3.02730106210953)
--(axis cs:381,3.02793265477993)
--(axis cs:380,2.82149819275236)
--(axis cs:379,2.93574424834433)
--(axis cs:378,2.77464609791131)
--(axis cs:377,3.11064309550136)
--(axis cs:376,2.78917486182908)
--(axis cs:375,2.661783905687)
--(axis cs:374,2.6398521305232)
--(axis cs:373,2.79792277299607)
--(axis cs:372,2.61833734111763)
--(axis cs:371,2.97148128970464)
--(axis cs:370,2.85000132625994)
--(axis cs:369,2.72673802660983)
--(axis cs:368,2.88392961224366)
--(axis cs:367,2.6398521305232)
--(axis cs:366,2.63524107770068)
--(axis cs:365,2.77320579544722)
--(axis cs:364,2.79458758572409)
--(axis cs:363,2.96250524456592)
--(axis cs:362,2.56673072964231)
--(axis cs:361,2.56673072964231)
--(axis cs:360,2.73732485710237)
--(axis cs:359,2.65985613507781)
--(axis cs:358,2.65886432931257)
--(axis cs:357,2.56789072722606)
--(axis cs:356,2.70787131482376)
--(axis cs:355,2.60937431057234)
--(axis cs:354,2.65275667034868)
--(axis cs:353,2.7936038193)
--(axis cs:352,2.59096100285676)
--(axis cs:351,2.5247903696003)
--(axis cs:350,2.57569056258649)
--(axis cs:349,2.57459991888894)
--(axis cs:348,2.43071770309458)
--(axis cs:347,2.35230498952379)
--(axis cs:346,2.36343448137503)
--(axis cs:345,1.89805050588116)
--(axis cs:344,1.93703491752781)
--(axis cs:343,1.95137903063832)
--(axis cs:342,1.84407424775048)
--(axis cs:341,1.86343499382917)
--(axis cs:340,1.71742981754255)
--(axis cs:339,1.55530063057195)
--(axis cs:338,1.487866153381)
--(axis cs:337,1.66797468817459)
--(axis cs:336,1.43803556662029)
--(axis cs:335,1.04521348584978)
--(axis cs:334,1.05525059685391)
--(axis cs:333,1.02843356999346)
--(axis cs:332,0.842555319660601)
--(axis cs:331,0.868685326015568)
--(axis cs:330,0.742578850627421)
--(axis cs:329,0.949166290634639)
--(axis cs:328,0.849859371675976)
--(axis cs:327,1.0027658183176)
--(axis cs:326,0.930579646358791)
--(axis cs:325,0.977572175770906)
--(axis cs:324,1.07636609375047)
--(axis cs:323,0.907544075000082)
--(axis cs:322,0.861896403131813)
--(axis cs:321,0.680466364947166)
--(axis cs:320,0.803329675075424)
--(axis cs:319,0.755312872215483)
--(axis cs:318,0.772079675075424)
--(axis cs:317,0.900544040105685)
--(axis cs:316,0.755939186613964)
--(axis cs:315,0.508856403276829)
--(axis cs:314,0.438570495496384)
--(axis cs:313,0.270582816870502)
--(axis cs:312,0.283620131614199)
--(axis cs:311,0.350410118786668)
--(axis cs:310,0.257854476428441)
--(axis cs:309,0.272265859796628)
--(axis cs:308,0.352464596527744)
--(axis cs:307,0.383214320676301)
--(axis cs:306,0.342741945679408)
--(axis cs:305,0.371183831531861)
--(axis cs:304,0.238610554352422)
--(axis cs:303,0.197566177581959)
--(axis cs:302,0.315858761690076)
--(axis cs:301,0.368530168532602)
--(axis cs:300,0.247914224027907)
--(axis cs:299,0.38500274760857)
--(axis cs:298,0.305484068755054)
--(axis cs:297,0.348661152167542)
--(axis cs:296,0.235152999576545)
--(axis cs:295,0.266293088186948)
--(axis cs:294,0.29210239620872)
--(axis cs:293,0.289800861766844)
--(axis cs:292,0.116644745981968)
--(axis cs:291,0.231628051499746)
--(axis cs:290,0.190067201943819)
--(axis cs:289,0.319446445826015)
--(axis cs:288,0.0621758067029596)
--(axis cs:287,0.189639731074006)
--(axis cs:286,0.245338329031469)
--(axis cs:285,0.177786280672075)
--(axis cs:284,0.19930746547377)
--(axis cs:283,0.0959913720374159)
--(axis cs:282,-0.0273426062208711)
--(axis cs:281,-0.0301826675093861)
--(axis cs:280,-0.0175513194452934)
--(axis cs:279,-0.045459301781922)
--(axis cs:278,-0.136678742763663)
--(axis cs:277,-0.0827181805050515)
--(axis cs:276,-0.0363425876382288)
--(axis cs:275,-0.0855281840659016)
--(axis cs:274,-0.118649785889034)
--(axis cs:273,-0.0759211223734747)
--(axis cs:272,-0.0933874464504264)
--(axis cs:271,-0.0827181805050515)
--(axis cs:270,0.00063230535043568)
--(axis cs:269,-0.0247479972799238)
--(axis cs:268,0.0278190596494498)
--(axis cs:267,-0.0813464703201766)
--(axis cs:266,-0.0653405898751912)
--(axis cs:265,0.0394598978978901)
--(axis cs:264,-0.0475721243614494)
--(axis cs:263,-0.0887371478552402)
--(axis cs:262,0.061559091664701)
--(axis cs:261,0.0441002795013267)
--(axis cs:260,0.0669943168818401)
--(axis cs:259,-0.068714856948464)
--(axis cs:258,-0.0876793269596443)
--(axis cs:257,-0.089572972240311)
--(axis cs:256,-0.048093309254154)
--(axis cs:255,-0.066305830574428)
--(axis cs:254,0.0069648012771274)
--(axis cs:253,-0.0544782414711464)
--(axis cs:252,-0.115400461257835)
--(axis cs:251,-0.105266207522335)
--(axis cs:250,-0.0911084680495936)
--(axis cs:249,-0.0448337399003417)
--(axis cs:248,-0.00741931521175421)
--(axis cs:247,-0.161145283878118)
--(axis cs:246,-0.134096341071328)
--(axis cs:245,-0.212097235538878)
--(axis cs:244,-0.203525868148588)
--(axis cs:243,-0.221397003195111)
--(axis cs:242,-0.212097235538878)
--(axis cs:241,-0.212097235538878)
--(axis cs:240,-0.166928846004277)
--(axis cs:239,-0.207186380341374)
--(axis cs:238,-0.206302223038635)
--(axis cs:237,-0.166928846004277)
--(axis cs:236,-0.166928846004277)
--(axis cs:235,-0.17635063686037)
--(axis cs:234,-0.170422537158382)
--(axis cs:233,-0.170874417051019)
--(axis cs:232,-0.206302223038635)
--(axis cs:231,-0.207590877330783)
--(axis cs:230,-0.206302223038635)
--(axis cs:229,-0.212394600931056)
--(axis cs:228,-0.221397003195111)
--(axis cs:227,-0.221397003195111)
--(axis cs:226,-0.212097235538878)
--(axis cs:225,-0.221397003195111)
--(axis cs:224,-0.221397003195111)
--(axis cs:223,-0.182127098010829)
--(axis cs:222,-0.174269301383645)
--(axis cs:221,-0.221397003195111)
--(axis cs:220,-0.221397003195111)
--(axis cs:219,-0.212097235538878)
--(axis cs:218,-0.203596547895149)
--(axis cs:217,-0.192025254551604)
--(axis cs:216,-0.221397003195111)
--(axis cs:215,-0.221397003195111)
--(axis cs:214,-0.212097235538878)
--(axis cs:213,-0.221397003195111)
--(axis cs:212,-0.221397003195111)
--(axis cs:211,-0.218216053144664)
--(axis cs:210,-0.221397003195111)
--(axis cs:209,-0.203525868148588)
--(axis cs:208,-0.200337322862163)
--(axis cs:207,-0.206302223038635)
--(axis cs:206,-0.221397003195111)
--(axis cs:205,-0.196755916096782)
--(axis cs:204,-0.221397003195111)
--(axis cs:203,-0.221397003195111)
--(axis cs:202,-0.221397003195111)
--(axis cs:201,-0.193223330901927)
--(axis cs:200,-0.212097235538878)
--(axis cs:199,-0.221397003195111)
--(axis cs:198,-0.221397003195111)
--(axis cs:197,-0.218216053144664)
--(axis cs:196,-0.212097235538878)
--(axis cs:195,-0.191517857142856)
--(axis cs:194,-0.195679068262344)
--(axis cs:193,-0.221397003195111)
--(axis cs:192,-0.221397003195111)
--(axis cs:191,-0.221397003195111)
--(axis cs:190,-0.221397003195111)
--(axis cs:189,-0.221397003195111)
--(axis cs:188,-0.221397003195111)
--(axis cs:187,-0.221397003195111)
--(axis cs:186,-0.186643012273389)
--(axis cs:185,-0.221397003195111)
--(axis cs:184,-0.221397003195111)
--(axis cs:183,-0.19346295336765)
--(axis cs:182,-0.212097235538878)
--(axis cs:181,-0.221397003195111)
--(axis cs:180,-0.19026524852201)
--(axis cs:179,-0.221397003195111)
--(axis cs:178,-0.172888649943941)
--(axis cs:177,-0.0442794006390222)
--(axis cs:176,-0.0295196004260148)
--(axis cs:175,-0.073799001065037)
--(axis cs:174,0)
--(axis cs:173,-0.00295196004260148)
--(axis cs:172,-0.0412604446077268)
--(axis cs:171,-0.00885588012780444)
--(axis cs:170,0)
--(axis cs:169,-0.00885588012780443)
--(axis cs:168,-0.0344395338303506)
--(axis cs:167,-0.00632562866271746)
--(axis cs:166,0)
--(axis cs:165,0)
--(axis cs:164,0)
--(axis cs:163,0)
--(axis cs:162,0)
--(axis cs:161,-0.00885588012780441)
--(axis cs:160,-0.00682691754175042)
--(axis cs:159,0)
--(axis cs:158,0)
--(axis cs:157,-0.00885588012780444)
--(axis cs:156,-0.0160062259703804)
--(axis cs:155,-0.0317273180678604)
--(axis cs:154,-0.0225425425177707)
--(axis cs:153,-0.00885588012780443)
--(axis cs:152,-0.0217894911167183)
--(axis cs:151,-0.0280272144128673)
--(axis cs:150,-0.0266016360705834)
--(axis cs:149,-0.0398514605751201)
--(axis cs:148,-0.0354235205112177)
--(axis cs:147,-0.0354235205112177)
--(axis cs:146,-0.0372552665697639)
--(axis cs:145,0)
--(axis cs:144,-0.0177117602556089)
--(axis cs:143,-0.0101854462100454)
--(axis cs:142,-0.0147598002130074)
--(axis cs:141,-0.0147598002130074)
--(axis cs:140,-0.0147598002130074)
--(axis cs:139,-0.0113781959029174)
--(axis cs:138,-0.0096501065948084)
--(axis cs:137,-0.0049199334043358)
--(axis cs:136,-0.00295196004260148)
--(axis cs:135,-0.00295196004260148)
--(axis cs:134,-0.00435525857166026)
--(axis cs:133,-0.0121418065340553)
--(axis cs:132,-0.0101854462100455)
--(axis cs:131,-0.0106708173135869)
--(axis cs:130,-0.0101854462100455)
--(axis cs:129,0)
--(axis cs:128,0)
--(axis cs:127,0)
--(axis cs:126,0)
--(axis cs:125,-0.00295196004260147)
--(axis cs:124,-0.00295196004260148)
--(axis cs:123,0)
--(axis cs:122,0)
--(axis cs:121,0)
--(axis cs:120,0)
--(axis cs:119,0)
--(axis cs:118,-0.0160062259703804)
--(axis cs:117,0)
--(axis cs:116,-0.0147598002130074)
--(axis cs:115,-0.0147598002130074)
--(axis cs:114,-0.0147598002130074)
--(axis cs:113,-0.0147598002130074)
--(axis cs:112,-0.0139398113122848)
--(axis cs:111,-0.0147598002130074)
--(axis cs:110,-0.0129148251863815)
--(axis cs:109,-0.00590392008520296)
--(axis cs:108,0)
--(axis cs:107,-0.00295196004260148)
--(axis cs:106,-0.00933719309244232)
--(axis cs:105,0)
--(axis cs:104,-0.00590392008520296)
--(axis cs:103,-0.00533540865679347)
--(axis cs:102,-0.00295196004260148)
--(axis cs:101,-0.00295196004260148)
--(axis cs:100,0)
--(axis cs:99,-0.00295196004260148)
--(axis cs:98,-0.0110698501597556)
--(axis cs:97,-0.00491993340433579)
--(axis cs:96,0)
--(axis cs:95,0)
--(axis cs:94,-0.00843417155028994)
--(axis cs:93,-0.00227563918058348)
--(axis cs:92,-0.00295196004260148)
--(axis cs:91,-0.0105427144378624)
--(axis cs:90,0)
--(axis cs:89,0)
--(axis cs:88,-0.00295196004260148)
--(axis cs:87,-0.00885588012780441)
--(axis cs:86,-0.00843417155028994)
--(axis cs:85,-0.0098398668086716)
--(axis cs:84,0)
--(axis cs:83,-0.00590392008520296)
--(axis cs:82,-0.00535972886006057)
--(axis cs:81,-0.00423853523790432)
--(axis cs:80,0)
--(axis cs:79,-0.00227563918058348)
--(axis cs:78,-0.00295196004260148)
--(axis cs:77,-0.00885588012780443)
--(axis cs:76,0)
--(axis cs:75,-0.0177117602556089)
--(axis cs:74,-0.00295196004260148)
--(axis cs:73,0)
--(axis cs:72,0)
--(axis cs:71,-0.00295196004260148)
--(axis cs:70,0)
--(axis cs:69,0)
--(axis cs:68,0)
--(axis cs:67,0)
--(axis cs:66,-0.00491993340433579)
--(axis cs:65,0)
--(axis cs:64,0)
--(axis cs:63,-0.00295196004260147)
--(axis cs:62,0)
--(axis cs:61,0)
--(axis cs:60,0)
--(axis cs:59,0)
--(axis cs:58,0)
--(axis cs:57,0)
--(axis cs:56,-0.00295196004260147)
--(axis cs:55,0)
--(axis cs:54,0)
--(axis cs:53,0)
--(axis cs:52,0)
--(axis cs:51,-0.0066253063700951)
--(axis cs:50,0)
--(axis cs:49,-0.00412694394279487)
--(axis cs:48,0)
--(axis cs:47,0)
--(axis cs:46,0)
--(axis cs:45,0)
--(axis cs:44,0)
--(axis cs:43,-0.00295196004260148)
--(axis cs:42,-0.00491993340433579)
--(axis cs:41,-0.0172197669151753)
--(axis cs:40,-0.0172197669151753)
--(axis cs:39,-0.00295196004260148)
--(axis cs:38,-0.0118078401704059)
--(axis cs:37,-0.00491993340433579)
--(axis cs:36,-0.00885588012780441)
--(axis cs:35,0)
--(axis cs:34,-0.00295196004260148)
--(axis cs:33,-0.00295196004260148)
--(axis cs:32,-0.00295196004260148)
--(axis cs:31,0)
--(axis cs:30,0)
--(axis cs:29,0)
--(axis cs:28,0)
--(axis cs:27,0)
--(axis cs:26,-0.00295196004260148)
--(axis cs:25,0)
--(axis cs:24,-0.0073799001065037)
--(axis cs:23,-0.0024599667021679)
--(axis cs:22,-0.00632562866271745)
--(axis cs:21,-0.00590392008520295)
--(axis cs:20,0)
--(axis cs:19,0)
--(axis cs:18,0)
--(axis cs:17,0)
--(axis cs:16,0)
--(axis cs:15,-0.00295196004260148)
--(axis cs:14,0)
--(axis cs:13,0)
--(axis cs:12,0)
--(axis cs:11,0)
--(axis cs:10,-0.0073799001065037)
--(axis cs:9,-0.0105757726892868)
--(axis cs:8,-0.0073799001065037)
--(axis cs:7,-0.013624430965853)
--(axis cs:6,-0.0121418065340553)
--(axis cs:5,-0.0108592031420126)
--(axis cs:4,-0.0135298168619235)
--(axis cs:3,-0.0147598002130074)
--(axis cs:2,-0.0121418065340553)
--(axis cs:1,0)
--(axis cs:0,0)
--cycle;

\path [fill=color3, fill opacity=0.3] (axis cs:0,0)
--(axis cs:0,0)
--(axis cs:1,0)
--(axis cs:2,0.040123630806697)
--(axis cs:3,0.0512515898323513)
--(axis cs:4,0)
--(axis cs:5,0.0153366000710025)
--(axis cs:6,0.0276058801278044)
--(axis cs:7,0.0758308724656337)
--(axis cs:8,0)
--(axis cs:9,0.00920196004260148)
--(axis cs:10,0.0153366000710025)
--(axis cs:11,0.00920196004260148)
--(axis cs:12,0)
--(axis cs:13,0.00920196004260148)
--(axis cs:14,0.0197321830260151)
--(axis cs:15,0.0601854462100454)
--(axis cs:16,0.0460098002130074)
--(axis cs:17,0.0460098002130074)
--(axis cs:18,0.0460098002130074)
--(axis cs:19,0.04586148519771)
--(axis cs:20,0)
--(axis cs:21,0.0153366000710025)
--(axis cs:22,0)
--(axis cs:23,0)
--(axis cs:24,0)
--(axis cs:25,0)
--(axis cs:26,0.0153366000710025)
--(axis cs:27,0.0334616728821872)
--(axis cs:28,0.147995945507375)
--(axis cs:29,0.0255610001183374)
--(axis cs:30,0.0575989731775016)
--(axis cs:31,0.032864143009291)
--(axis cs:32,0.119239824131235)
--(axis cs:33,0.0686308318283846)
--(axis cs:34,0.0690147003195111)
--(axis cs:35,0.0276058801278044)
--(axis cs:36,0.0276058801278044)
--(axis cs:37,0.0276058801278044)
--(axis cs:38,0.0230049001065037)
--(axis cs:39,0)
--(axis cs:40,0)
--(axis cs:41,0)
--(axis cs:42,0)
--(axis cs:43,0)
--(axis cs:44,0)
--(axis cs:45,0)
--(axis cs:46,0)
--(axis cs:47,0)
--(axis cs:48,0)
--(axis cs:49,0)
--(axis cs:50,0.00920196004260148)
--(axis cs:51,0)
--(axis cs:52,0)
--(axis cs:53,0)
--(axis cs:54,0)
--(axis cs:55,0.00920196004260148)
--(axis cs:56,0)
--(axis cs:57,0)
--(axis cs:58,0.138029400639022)
--(axis cs:59,0.474379595551426)
--(axis cs:60,0)
--(axis cs:61,0)
--(axis cs:62,0)
--(axis cs:63,0.018403920085203)
--(axis cs:64,0)
--(axis cs:65,0)
--(axis cs:66,0.0690147003195111)
--(axis cs:67,0)
--(axis cs:68,0)
--(axis cs:69,0.00920196004260148)
--(axis cs:70,0)
--(axis cs:71,0.0276058801278044)
--(axis cs:72,0)
--(axis cs:73,0)
--(axis cs:74,0)
--(axis cs:75,0.0460098002130074)
--(axis cs:76,0.138029400639022)
--(axis cs:77,0.0460098002130074)
--(axis cs:78,0.0460098002130074)
--(axis cs:79,0.0601854462100454)
--(axis cs:80,0)
--(axis cs:81,0)
--(axis cs:82,0)
--(axis cs:83,0)
--(axis cs:84,0)
--(axis cs:85,0)
--(axis cs:86,0)
--(axis cs:87,0)
--(axis cs:88,0)
--(axis cs:89,0)
--(axis cs:90,0)
--(axis cs:91,0)
--(axis cs:92,0)
--(axis cs:93,0)
--(axis cs:94,0)
--(axis cs:95,0)
--(axis cs:96,0)
--(axis cs:97,0)
--(axis cs:98,0)
--(axis cs:99,0)
--(axis cs:100,0)
--(axis cs:101,0)
--(axis cs:102,0)
--(axis cs:103,0)
--(axis cs:104,0.0276058801278044)
--(axis cs:105,0.0276058801278044)
--(axis cs:106,0)
--(axis cs:107,0)
--(axis cs:108,0)
--(axis cs:109,0)
--(axis cs:110,0)
--(axis cs:111,0)
--(axis cs:112,0.138029400639022)
--(axis cs:113,0)
--(axis cs:114,0.115024500532519)
--(axis cs:115,0)
--(axis cs:116,0)
--(axis cs:117,0)
--(axis cs:118,0)
--(axis cs:119,0)
--(axis cs:120,0)
--(axis cs:121,0)
--(axis cs:122,0)
--(axis cs:123,0)
--(axis cs:124,0)
--(axis cs:125,0)
--(axis cs:126,0)
--(axis cs:127,0)
--(axis cs:128,0)
--(axis cs:129,0)
--(axis cs:130,0)
--(axis cs:131,0)
--(axis cs:132,0)
--(axis cs:133,0)
--(axis cs:134,0)
--(axis cs:135,0.138029400639022)
--(axis cs:136,0)
--(axis cs:137,0)
--(axis cs:138,0)
--(axis cs:139,0)
--(axis cs:140,0)
--(axis cs:141,0)
--(axis cs:142,0)
--(axis cs:143,0)
--(axis cs:144,0)
--(axis cs:145,0)
--(axis cs:146,0)
--(axis cs:147,0)
--(axis cs:148,0)
--(axis cs:149,0.345073501597556)
--(axis cs:150,0)
--(axis cs:151,0.0276058801278044)
--(axis cs:152,0)
--(axis cs:153,0)
--(axis cs:154,0)
--(axis cs:155,0)
--(axis cs:156,0.138029400639022)
--(axis cs:157,0)
--(axis cs:158,0)
--(axis cs:159,0)
--(axis cs:160,0)
--(axis cs:161,0)
--(axis cs:162,0.276058801278044)
--(axis cs:163,0)
--(axis cs:164,0)
--(axis cs:165,0.138029400639022)
--(axis cs:166,0)
--(axis cs:167,0)
--(axis cs:168,0)
--(axis cs:169,0)
--(axis cs:170,0)
--(axis cs:171,0)
--(axis cs:172,0)
--(axis cs:173,0)
--(axis cs:174,0)
--(axis cs:175,0)
--(axis cs:176,0)
--(axis cs:177,0)
--(axis cs:178,0.138029400639022)
--(axis cs:179,0.230049001065037)
--(axis cs:180,0)
--(axis cs:181,0)
--(axis cs:182,0)
--(axis cs:183,0)
--(axis cs:184,0)
--(axis cs:185,0)
--(axis cs:186,0)
--(axis cs:187,0)
--(axis cs:188,0)
--(axis cs:189,0)
--(axis cs:190,0)
--(axis cs:191,0)
--(axis cs:192,0.138029400639022)
--(axis cs:193,0)
--(axis cs:194,0)
--(axis cs:195,0)
--(axis cs:196,0.138029400639022)
--(axis cs:197,0)
--(axis cs:198,0)
--(axis cs:199,0)
--(axis cs:200,0)
--(axis cs:201,0.345073501597556)
--(axis cs:202,0.575122502662593)
--(axis cs:203,0.345073501597556)
--(axis cs:204,0)
--(axis cs:205,0)
--(axis cs:206,0)
--(axis cs:207,0)
--(axis cs:208,0)
--(axis cs:209,0.0460098002130074)
--(axis cs:210,0)
--(axis cs:211,0)
--(axis cs:212,0.230049001065037)
--(axis cs:213,0.591554574167238)
--(axis cs:214,0.138029400639022)
--(axis cs:215,0)
--(axis cs:216,0)
--(axis cs:217,0)
--(axis cs:218,0.221634587708752)
--(axis cs:219,0)
--(axis cs:220,0.138029400639022)
--(axis cs:221,0.343154159141923)
--(axis cs:222,0.138029400639022)
--(axis cs:223,0)
--(axis cs:224,0)
--(axis cs:225,0)
--(axis cs:226,0.221634587708752)
--(axis cs:227,0.414088201917067)
--(axis cs:228,0.591554574167238)
--(axis cs:229,1.07666908212045)
--(axis cs:230,1.09403301331997)
--(axis cs:231,0)
--(axis cs:232,0.744627098010829)
--(axis cs:233,0)
--(axis cs:234,0.138029400639022)
--(axis cs:235,0)
--(axis cs:236,0.343154159141923)
--(axis cs:237,0.276058801278044)
--(axis cs:238,0.138029400639022)
--(axis cs:239,0.408156129851902)
--(axis cs:240,0.681779465378314)
--(axis cs:241,0.343154159141923)
--(axis cs:242,0.138029400639022)
--(axis cs:243,0.591554574167238)
--(axis cs:244,0.138029400639022)
--(axis cs:245,0.731642787990331)
--(axis cs:246,0.816312259703804)
--(axis cs:247,0.138029400639022)
--(axis cs:248,0)
--(axis cs:249,0.889637819481929)
--(axis cs:250,0)
--(axis cs:251,0.738781959029174)
--(axis cs:252,0.757884434462632)
--(axis cs:253,0.744627098010829)
--(axis cs:254,0.276058801278044)
--(axis cs:255,0.295982745390226)
--(axis cs:256,0.500505488659231)
--(axis cs:257,0.276058801278044)
--(axis cs:258,0.138029400639022)
--(axis cs:259,0.138029400639022)
--(axis cs:260,0)
--(axis cs:261,0.221634587708752)
--(axis cs:262,0.470573724904442)
--(axis cs:263,0.518962712588854)
--(axis cs:264,0.550961791918207)
--(axis cs:265,0.469718466854505)
--(axis cs:266,0.690147003195111)
--(axis cs:267,0.744627098010829)
--(axis cs:268,0.176488496926818)
--(axis cs:269,0.138029400639022)
--(axis cs:270,0.591554574167238)
--(axis cs:271,0.221634587708752)
--(axis cs:272,0.343154159141923)
--(axis cs:273,0.529465490780453)
--(axis cs:274,0.503952203962954)
--(axis cs:275,0.669053257005968)
--(axis cs:276,0.470573724904442)
--(axis cs:277,0.608807896386635)
--(axis cs:278,0.414088201917067)
--(axis cs:279,0.345073501597556)
--(axis cs:280,0.654375637857743)
--(axis cs:281,0.500505488659231)
--(axis cs:282,0.825663361392468)
--(axis cs:283,0.560620201338632)
--(axis cs:284,0.393155847885586)
--(axis cs:285,0.775473840437799)
--(axis cs:286,0.392773410172509)
--(axis cs:287,1.06872185925201)
--(axis cs:288,1.10040429667512)
--(axis cs:289,0.601854462100454)
--(axis cs:290,0.407328028568565)
--(axis cs:291,0.636678595164235)
--(axis cs:292,0.804842358458307)
--(axis cs:293,1.25285259980365)
--(axis cs:294,0.518962712588854)
--(axis cs:295,1.70988178856701)
--(axis cs:296,0.720447118019249)
--(axis cs:297,1.61010560326387)
--(axis cs:298,1.01045574326696)
--(axis cs:299,1.53734531919916)
--(axis cs:300,1.05528976630214)
--(axis cs:301,1.04574274324804)
--(axis cs:302,0.957288441388332)
--(axis cs:303,1.08711966926146)
--(axis cs:304,1.43716067625546)
--(axis cs:305,1.65218593056769)
--(axis cs:306,1.8041594034849)
--(axis cs:307,1.85899620675298)
--(axis cs:308,0)
--(axis cs:309,1.54601995313677)
--(axis cs:310,1.70851994142968)
--(axis cs:311,1.5161332225351)
--(axis cs:312,1.98003242780698)
--(axis cs:313,1.29581817160671)
--(axis cs:314,2.49134692313408)
--(axis cs:315,2.13398849864595)
--(axis cs:316,2.31521810860499)
--(axis cs:317,1.79873880647709)
--(axis cs:318,2.91420912689562)
--(axis cs:319,1.81203038285053)
--(axis cs:320,2.36585026649711)
--(axis cs:321,1.97388675738051)
--(axis cs:322,2.79418905601822)
--(axis cs:323,1.63568955330465)
--(axis cs:324,1.96853100367358)
--(axis cs:325,2.21790280226381)
--(axis cs:326,2.7510750722075)
--(axis cs:327,2.20365366035498)
--(axis cs:328,2.73399377320796)
--(axis cs:329,1.92420117687736)
--(axis cs:330,2.80243978184704)
--(axis cs:331,3.78510738947126)
--(axis cs:332,2.74586927233432)
--(axis cs:333,2.86007350116211)
--(axis cs:334,3.73029303178736)
--(axis cs:335,4.38250773205432)
--(axis cs:336,3.48641557846449)
--(axis cs:337,3.45858959267746)
--(axis cs:338,3.32136590568974)
--(axis cs:339,5.01827200581615)
--(axis cs:340,4.06474596337226)
--(axis cs:341,4.81064226742209)
--(axis cs:342,3.62204184670662)
--(axis cs:343,4.03018862859967)
--(axis cs:344,5.03652510977837)
--(axis cs:345,5.22312756940811)
--(axis cs:346,4.80940756876753)
--(axis cs:347,5.25363977833344)
--(axis cs:348,6.00654194554634)
--(axis cs:349,5.46785809072807)
--(axis cs:350,6.06369064757806)
--(axis cs:351,7.08186202178136)
--(axis cs:352,6.06880632156096)
--(axis cs:353,6.43938168708512)
--(axis cs:354,6.7379449652274)
--(axis cs:355,6.58859736177273)
--(axis cs:356,6.53422624046763)
--(axis cs:357,6.68879576549619)
--(axis cs:358,7.26156556335862)
--(axis cs:359,6.91107316078229)
--(axis cs:360,7.79694114498525)
--(axis cs:361,7.49601162591545)
--(axis cs:362,7.97111028989279)
--(axis cs:363,7.8647721581739)
--(axis cs:364,8.22008256958656)
--(axis cs:365,8.41037370380591)
--(axis cs:366,7.9551422362316)
--(axis cs:367,8.45110584349713)
--(axis cs:368,7.86196196544096)
--(axis cs:369,8.31558273299473)
--(axis cs:370,8.81486308885701)
--(axis cs:371,8.77020470713855)
--(axis cs:372,8.54607345588062)
--(axis cs:373,9.85908480926033)
--(axis cs:374,9.34981039419528)
--(axis cs:375,9.36118623384553)
--(axis cs:376,9.08235676258356)
--(axis cs:377,9.85377097692068)
--(axis cs:378,9.61660088394919)
--(axis cs:379,10.1849628032781)
--(axis cs:380,10.3976116058502)
--(axis cs:381,10.1878251641562)
--(axis cs:382,10.7322465289946)
--(axis cs:383,10.7157225319502)
--(axis cs:384,10.5758729450559)
--(axis cs:385,10.1252037393584)
--(axis cs:386,11.5786130677763)
--(axis cs:387,11.0327414030373)
--(axis cs:388,11.4323491297881)
--(axis cs:389,11.4302150344136)
--(axis cs:390,12.1192728320963)
--(axis cs:391,11.8226716997228)
--(axis cs:392,11.4531250162332)
--(axis cs:393,12.4242045564256)
--(axis cs:394,12.1393755452304)
--(axis cs:395,13.1476590942377)
--(axis cs:396,13.2197498487699)
--(axis cs:397,12.6999769612231)
--(axis cs:398,13.3943491675037)
--(axis cs:399,13.0699811198462)
--(axis cs:400,13.2827613848018)
--(axis cs:401,13.252294803798)
--(axis cs:402,13.4565707249155)
--(axis cs:403,13.2810152782986)
--(axis cs:404,13.6795717660049)
--(axis cs:405,13.6666534044373)
--(axis cs:406,13.9962697255027)
--(axis cs:407,14.0157680811754)
--(axis cs:408,13.9261107628044)
--(axis cs:409,14.038480631645)
--(axis cs:410,13.673339585571)
--(axis cs:411,13.8128469853372)
--(axis cs:412,13.7817573761964)
--(axis cs:413,14.4575051745732)
--(axis cs:414,14.0432824524637)
--(axis cs:415,14.2454009694311)
--(axis cs:416,14.450493628511)
--(axis cs:417,14.3915486556144)
--(axis cs:418,14.2676240866346)
--(axis cs:419,14.2756029781823)
--(axis cs:420,14.1078411154163)
--(axis cs:421,14.3793985155642)
--(axis cs:422,14.455578835918)
--(axis cs:423,14.2916932139326)
--(axis cs:424,14.3524097800674)
--(axis cs:425,14.4661451065443)
--(axis cs:426,14.2536629260735)
--(axis cs:427,14.6496229661484)
--(axis cs:428,14.8921082608814)
--(axis cs:429,14.7056949663164)
--(axis cs:430,14.6707174026209)
--(axis cs:431,14.6787506748994)
--(axis cs:432,14.5974660660213)
--(axis cs:433,14.8491863398316)
--(axis cs:434,14.6208332089243)
--(axis cs:435,14.9018269258447)
--(axis cs:436,14.4591017509869)
--(axis cs:437,14.8391581662883)
--(axis cs:438,14.7416593237394)
--(axis cs:439,14.5243961349103)
--(axis cs:440,14.660044739983)
--(axis cs:441,14.7451475211362)
--(axis cs:442,14.6973782811824)
--(axis cs:443,14.7642553056163)
--(axis cs:444,14.6506330317495)
--(axis cs:445,14.8288862709268)
--(axis cs:446,14.7625764529952)
--(axis cs:447,14.7340770672634)
--(axis cs:448,14.8615386118127)
--(axis cs:449,14.8772726380028)
--(axis cs:450,14.9007450493318)
--(axis cs:451,14.766682983386)
--(axis cs:452,14.844778043184)
--(axis cs:453,14.8247448395535)
--(axis cs:454,14.8904779510632)
--(axis cs:455,15.0388323087647)
--(axis cs:456,14.941861960118)
--(axis cs:457,14.9699261619413)
--(axis cs:458,14.9035539572436)
--(axis cs:459,14.9931599389817)
--(axis cs:460,15.0629834958412)
--(axis cs:461,14.9548159636628)
--(axis cs:462,14.9401659019622)
--(axis cs:463,15.0048363406015)
--(axis cs:464,15.0611989590916)
--(axis cs:465,15.0561880819535)
--(axis cs:466,15.0353493320653)
--(axis cs:467,14.9637241204665)
--(axis cs:468,14.9696227032532)
--(axis cs:469,15.0864663321994)
--(axis cs:470,14.9570712675073)
--(axis cs:471,15.1620999544537)
--(axis cs:472,15.0737646271765)
--(axis cs:473,15.0039758639257)
--(axis cs:474,14.9504391092823)
--(axis cs:475,14.9189555378347)
--(axis cs:476,15.1103476410124)
--(axis cs:477,14.9208307919299)
--(axis cs:478,15.0002872726854)
--(axis cs:479,15.0386382127915)
--(axis cs:480,15.1036096318656)
--(axis cs:481,15.1710940463623)
--(axis cs:482,14.9608742647923)
--(axis cs:483,15.1049742401202)
--(axis cs:484,15.0715231786947)
--(axis cs:485,14.9790850563819)
--(axis cs:486,14.9855307423612)
--(axis cs:487,15.0005568414273)
--(axis cs:488,14.995173469903)
--(axis cs:489,14.9869024239846)
--(axis cs:490,15.1806675749734)
--(axis cs:491,14.9841681323442)
--(axis cs:492,15.1542816541156)
--(axis cs:493,14.9928802600422)
--(axis cs:494,15.0022659217873)
--(axis cs:495,14.9827374146772)
--(axis cs:496,15.0019996103592)
--(axis cs:497,14.9961004416457)
--(axis cs:498,14.9813263962215)
--(axis cs:499,14.9954541022468)
--(axis cs:500,14.967613856101)
--(axis cs:501,14.9924796840358)
--(axis cs:502,15.003022970997)
--(axis cs:503,14.9867036241056)
--(axis cs:504,14.9872620887616)
--(axis cs:505,15.0097012712957)
--(axis cs:506,14.9917193299885)
--(axis cs:507,14.9716570713969)
--(axis cs:508,15.0964679778844)
--(axis cs:509,15.0266827785001)
--(axis cs:510,14.9931315586254)
--(axis cs:511,15.1283591235404)
--(axis cs:512,15.010290261826)
--(axis cs:513,15.034225677254)
--(axis cs:514,15.0097002449214)
--(axis cs:515,15.0032617284064)
--(axis cs:516,15.0019602027001)
--(axis cs:517,14.97804790417)
--(axis cs:518,14.9924117463355)
--(axis cs:519,15.0031686051208)
--(axis cs:520,15.0028225130611)
--(axis cs:521,15.0103186770656)
--(axis cs:522,15.0027420176552)
--(axis cs:523,15.0160062259704)
--(axis cs:524,14.9982191797062)
--(axis cs:525,15.0967947739186)
--(axis cs:526,15)
--(axis cs:527,15.0102079099788)
--(axis cs:528,14.9982058656648)
--(axis cs:529,15.0140436319416)
--(axis cs:530,15.0085152693537)
--(axis cs:531,15.0046654642697)
--(axis cs:532,15.0030585626251)
--(axis cs:533,14.9933027320871)
--(axis cs:534,15.0091695642814)
--(axis cs:535,15.0025975570189)
--(axis cs:536,15.0113664413923)
--(axis cs:537,15.006859004819)
--(axis cs:538,14.9980823936974)
--(axis cs:539,14.9981513267994)
--(axis cs:540,15.0097002449214)
--(axis cs:541,15.0067006079209)
--(axis cs:542,15.0096320715819)
--(axis cs:543,15.0030383143535)
--(axis cs:544,15.0157256947258)
--(axis cs:545,15.0218738574616)
--(axis cs:546,15.0162545655756)
--(axis cs:547,15.0228197950791)
--(axis cs:548,15.0144112255372)
--(axis cs:549,15.0085152693537)
--(axis cs:550,15.044279400639)
--(axis cs:551,15.0075061020722)
--(axis cs:552,14.998059065329)
--(axis cs:553,14.995625492356)
--(axis cs:554,15.0105747851604)
--(axis cs:555,15)
--(axis cs:556,15.0548643584433)
--(axis cs:557,15.0020825580708)
--(axis cs:558,15.0066988188558)
--(axis cs:559,15.0553492507988)
--(axis cs:560,15.1388758786971)
--(axis cs:561,15.0107147479595)
--(axis cs:562,15.0817200366446)
--(axis cs:563,15.0366615769998)
--(axis cs:564,15.0026381135415)
--(axis cs:565,15.0097002449214)
--(axis cs:566,15.002501352663)
--(axis cs:567,15.0085152693537)
--(axis cs:568,15.0221397003195)
--(axis cs:569,15.0069747570531)
--(axis cs:570,15.0067006079209)
--(axis cs:571,15.0098306218192)
--(axis cs:572,15.0072570916553)
--(axis cs:573,15.0076847337813)
--(axis cs:574,15)
--(axis cs:575,15.0088558801278)
--(axis cs:576,15)
--(axis cs:577,15)
--(axis cs:578,15.0100635001452)
--(axis cs:579,15.0064234384263)
--(axis cs:580,15.009019453974)
--(axis cs:581,15.0088558801278)
--(axis cs:582,15.0074617793471)
--(axis cs:583,15)
--(axis cs:584,15.0032742571574)
--(axis cs:585,15.0096259566607)
--(axis cs:586,15.0088558801278)
--(axis cs:587,15.0085152693537)
--(axis cs:588,15.0233049477047)
--(axis cs:589,15.0860700208615)
--(axis cs:590,15.0068269175417)
--(axis cs:591,15)
--(axis cs:592,15.0061245774802)
--(axis cs:593,15.0085152693537)
--(axis cs:594,15.0085152693537)
--(axis cs:595,15.0151159436539)
--(axis cs:596,14.9989790808922)
--(axis cs:597,15.0067006079209)
--(axis cs:598,15)
--(axis cs:599,15)
--(axis cs:599,15)
--(axis cs:599,15)
--(axis cs:598,15)
--(axis cs:597,14.9565205459252)
--(axis cs:596,14.9210435435422)
--(axis cs:595,14.9214911992033)
--(axis cs:594,14.9734558844925)
--(axis cs:593,14.9734558844925)
--(axis cs:592,14.960725608624)
--(axis cs:591,15)
--(axis cs:590,14.9556730824582)
--(axis cs:589,14.6942871219956)
--(axis cs:588,14.9273529470321)
--(axis cs:587,14.9734558844925)
--(axis cs:586,14.9723941198722)
--(axis cs:585,14.9699936085567)
--(axis cs:584,14.9368442181173)
--(axis cs:583,15)
--(axis cs:582,14.9517001524711)
--(axis cs:581,14.9723941198722)
--(axis cs:580,14.9446570166143)
--(axis cs:579,14.9586627684703)
--(axis cs:578,14.9686296816729)
--(axis cs:577,15)
--(axis cs:576,15)
--(axis cs:575,14.9723941198722)
--(axis cs:574,15)
--(axis cs:573,14.9504625876473)
--(axis cs:572,14.953407244009)
--(axis cs:571,14.9072366858731)
--(axis cs:570,14.9565205459252)
--(axis cs:569,14.9547439929469)
--(axis cs:568,14.9309852996805)
--(axis cs:567,14.9734558844925)
--(axis cs:566,14.8980480978864)
--(axis cs:565,14.9066459089247)
--(axis cs:564,14.8986324908541)
--(axis cs:563,14.8515595768463)
--(axis cs:562,14.6174987133554)
--(axis cs:561,14.9038854544697)
--(axis cs:560,14.4082738465776)
--(axis cs:559,14.8274632492012)
--(axis cs:558,14.8783973349904)
--(axis cs:557,14.8961153038499)
--(axis cs:556,14.7924570701281)
--(axis cs:555,15)
--(axis cs:554,14.9033042634226)
--(axis cs:553,14.8806638433083)
--(axis cs:552,14.921739071317)
--(axis cs:551,14.9521436232025)
--(axis cs:550,14.861970599361)
--(axis cs:549,14.9734558844925)
--(axis cs:548,14.9249462919453)
--(axis cs:547,14.8178052049209)
--(axis cs:546,14.8944597201387)
--(axis cs:545,14.8665189996813)
--(axis cs:544,14.9245104803894)
--(axis cs:543,14.9416674577605)
--(axis cs:542,14.9125752573795)
--(axis cs:541,14.9565205459252)
--(axis cs:540,14.9066459089247)
--(axis cs:539,14.9221627341397)
--(axis cs:538,14.9260762601487)
--(axis cs:537,14.9555808990271)
--(axis cs:536,14.8708147964874)
--(axis cs:535,14.8583872914659)
--(axis cs:534,14.9038986175368)
--(axis cs:533,14.9196412040172)
--(axis cs:532,14.9390610025923)
--(axis cs:531,14.8877223562431)
--(axis cs:530,14.9734558844925)
--(axis cs:529,14.8905837719045)
--(axis cs:528,14.9294641205989)
--(axis cs:527,14.908724450764)
--(axis cs:526,15)
--(axis cs:525,14.5436585227847)
--(axis cs:524,14.9293670271904)
--(axis cs:523,14.9183687740296)
--(axis cs:522,14.9472232214994)
--(axis cs:521,14.9089353551925)
--(axis cs:520,14.8634603816757)
--(axis cs:519,14.8951487025715)
--(axis cs:518,14.910266825093)
--(axis cs:517,14.773795044548)
--(axis cs:516,14.8894460472999)
--(axis cs:515,14.8946854856698)
--(axis cs:514,14.9066459089247)
--(axis cs:513,14.8182655814872)
--(axis cs:512,14.9020863829109)
--(axis cs:511,14.3942370303057)
--(axis cs:510,14.908141065809)
--(axis cs:509,14.472424364357)
--(axis cs:508,14.5472219259618)
--(axis cs:507,14.7586231742312)
--(axis cs:506,14.8401402167148)
--(axis cs:505,14.8992693704767)
--(axis cs:504,14.8464640469686)
--(axis cs:503,14.9013439186294)
--(axis cs:502,14.8945731828491)
--(axis cs:501,14.9120797850277)
--(axis cs:500,14.787406350142)
--(axis cs:499,14.8798206230279)
--(axis cs:498,14.7508562237901)
--(axis cs:497,14.8530334106132)
--(axis cs:496,14.868201282498)
--(axis cs:495,14.8620496443368)
--(axis cs:494,14.8909775534875)
--(axis cs:493,14.915767819764)
--(axis cs:492,14.2397762505401)
--(axis cs:491,14.8054163793451)
--(axis cs:490,14.2682306622061)
--(axis cs:489,14.8351728965283)
--(axis cs:488,14.6934941125146)
--(axis cs:487,14.8248418149988)
--(axis cs:486,14.6162332565076)
--(axis cs:485,14.8093886755022)
--(axis cs:484,14.3319992993755)
--(axis cs:483,14.1536195098798)
--(axis cs:482,14.7744823057206)
--(axis cs:481,14.252864286971)
--(axis cs:480,14.1805895023335)
--(axis cs:479,14.4533658357915)
--(axis cs:478,14.2214138491249)
--(axis cs:477,14.6220184712211)
--(axis cs:476,14.1874941972319)
--(axis cs:475,13.6788507000451)
--(axis cs:474,14.3906151489595)
--(axis cs:473,14.4452354099978)
--(axis cs:472,13.9105767787002)
--(axis cs:471,14.2472149493925)
--(axis cs:470,14.5598812700867)
--(axis cs:469,14.1483222487849)
--(axis cs:468,14.4480484276992)
--(axis cs:467,14.349384974386)
--(axis cs:466,14.0742580972379)
--(axis cs:465,14.1097675157576)
--(axis cs:464,13.7819735972523)
--(axis cs:463,13.9318064026202)
--(axis cs:462,13.8474280830002)
--(axis cs:461,14.4692554076742)
--(axis cs:460,13.9706989675288)
--(axis cs:459,14.0600657005104)
--(axis cs:458,13.4944754949219)
--(axis cs:457,14.1441979951516)
--(axis cs:456,13.74826156767)
--(axis cs:455,13.7534522449169)
--(axis cs:454,13.8635614150882)
--(axis cs:453,13.666806906398)
--(axis cs:452,13.2338448963764)
--(axis cs:451,13.3163320145237)
--(axis cs:450,13.3604671759429)
--(axis cs:449,13.8972958179221)
--(axis cs:448,13.2855467247257)
--(axis cs:447,12.9520006570955)
--(axis cs:446,13.0654275742276)
--(axis cs:445,13.2266482986703)
--(axis cs:444,12.6707955396791)
--(axis cs:443,13.0933916598324)
--(axis cs:442,12.9095296135545)
--(axis cs:441,13.242996626383)
--(axis cs:440,12.8190855047723)
--(axis cs:439,12.5888118886838)
--(axis cs:438,12.9043142339529)
--(axis cs:437,13.4090644204346)
--(axis cs:436,12.4035306347433)
--(axis cs:435,13.43437029847)
--(axis cs:434,12.9091675015721)
--(axis cs:433,13.0980704474988)
--(axis cs:432,12.7561919607346)
--(axis cs:431,12.8369773470786)
--(axis cs:430,12.6982077583165)
--(axis cs:429,12.8069525424249)
--(axis cs:428,13.4719526371022)
--(axis cs:427,12.7531379108822)
--(axis cs:426,11.958479358725)
--(axis cs:425,12.433558452764)
--(axis cs:424,12.1929456457568)
--(axis cs:423,12.2104309300394)
--(axis cs:422,12.6466401766871)
--(axis cs:421,12.5074548794199)
--(axis cs:420,11.7958242981175)
--(axis cs:419,12.0001922791919)
--(axis cs:418,12.2247223689931)
--(axis cs:417,12.3775231925175)
--(axis cs:416,12.612594551425)
--(axis cs:415,12.1379929063814)
--(axis cs:414,11.7137764291939)
--(axis cs:413,12.5613192301887)
--(axis cs:412,11.2265452247353)
--(axis cs:411,11.2678672875229)
--(axis cs:410,11.2254686374559)
--(axis cs:409,11.6911069003882)
--(axis cs:408,11.5200961552776)
--(axis cs:407,11.561278049777)
--(axis cs:406,11.6449538535498)
--(axis cs:405,11.1185214467531)
--(axis cs:404,11.1446994055319)
--(axis cs:403,10.5387433074139)
--(axis cs:402,10.9729398686576)
--(axis cs:401,10.371450037693)
--(axis cs:400,10.7079333242559)
--(axis cs:399,10.2596353421232)
--(axis cs:398,10.8974335289665)
--(axis cs:397,9.83164974540359)
--(axis cs:396,10.5843551197265)
--(axis cs:395,10.2608745596085)
--(axis cs:394,9.15714933280034)
--(axis cs:393,9.46516668972179)
--(axis cs:392,8.20306210149749)
--(axis cs:391,8.76239297518564)
--(axis cs:390,9.1017757544674)
--(axis cs:389,8.21804086614006)
--(axis cs:388,8.40896412191117)
--(axis cs:387,7.81391038267695)
--(axis cs:386,8.52246776009683)
--(axis cs:385,6.71496895801004)
--(axis cs:384,7.20631485794996)
--(axis cs:383,7.43497911015769)
--(axis cs:382,7.51095190527046)
--(axis cs:381,6.77016489266202)
--(axis cs:380,6.96514022565167)
--(axis cs:379,6.78365841924932)
--(axis cs:378,6.267362831429)
--(axis cs:377,6.49770398806532)
--(axis cs:376,5.73376379449094)
--(axis cs:375,5.96805483758304)
--(axis cs:374,5.86680046889996)
--(axis cs:373,6.55325285307734)
--(axis cs:372,5.11562559874123)
--(axis cs:371,5.36304745736362)
--(axis cs:370,5.27077002802611)
--(axis cs:369,4.86643837090138)
--(axis cs:368,4.55098446313047)
--(axis cs:367,5.21444475174097)
--(axis cs:366,4.5705224297184)
--(axis cs:365,4.98399182649712)
--(axis cs:364,4.8701162940498)
--(axis cs:363,4.45833052039753)
--(axis cs:362,4.70119820601757)
--(axis cs:361,4.10103107274296)
--(axis cs:360,4.47972437699277)
--(axis cs:359,3.60211828517345)
--(axis cs:358,3.9578011376487)
--(axis cs:357,3.42419358889942)
--(axis cs:356,3.30991618464757)
--(axis cs:355,3.35195895690859)
--(axis cs:354,3.45859721155804)
--(axis cs:353,3.23720757510197)
--(axis cs:352,2.94531188686372)
--(axis cs:351,3.83091566534196)
--(axis cs:350,3.05697952654606)
--(axis cs:349,2.40841453370632)
--(axis cs:348,2.9629468989425)
--(axis cs:347,2.35080808358725)
--(axis cs:346,1.98245681684686)
--(axis cs:345,2.37304058535379)
--(axis cs:344,2.08386751842676)
--(axis cs:343,1.60241565820684)
--(axis cs:342,1.21229868821136)
--(axis cs:341,2.06500370119263)
--(axis cs:340,1.37389872527243)
--(axis cs:339,2.17555239894576)
--(axis cs:338,1.03261474907216)
--(axis cs:337,1.28400375147839)
--(axis cs:336,1.15644156439266)
--(axis cs:335,1.75870392087608)
--(axis cs:334,1.34519398119965)
--(axis cs:333,0.740914052950442)
--(axis cs:332,0.652568227665678)
--(axis cs:331,1.32729447031111)
--(axis cs:330,0.865985542828283)
--(axis cs:329,0.351189448122643)
--(axis cs:328,0.683974976792041)
--(axis cs:327,0.473801696787882)
--(axis cs:326,0.9444091310892)
--(axis cs:325,0.468933387861851)
--(axis cs:324,0.44627554394547)
--(axis cs:323,0.239310446695348)
--(axis cs:322,0.690185943981785)
--(axis cs:321,0.390521505924814)
--(axis cs:320,0.492432076160233)
--(axis cs:319,0.259386057366858)
--(axis cs:318,0.837336202774715)
--(axis cs:317,0.163500776856243)
--(axis cs:316,0.497281891395007)
--(axis cs:315,0.429713424430975)
--(axis cs:314,0.612075695913541)
--(axis cs:313,0.0287556920296519)
--(axis cs:312,0.298613405526352)
--(axis cs:311,0.251595142849519)
--(axis cs:310,0.202938391903649)
--(axis cs:309,0.0627970111489411)
--(axis cs:308,0)
--(axis cs:307,0.206852007532737)
--(axis cs:306,0.25722452508653)
--(axis cs:305,0.231463623003743)
--(axis cs:304,0.178910752315966)
--(axis cs:303,-0.0681464549757463)
--(axis cs:302,0.0739615586116683)
--(axis cs:301,-0.0337942138362773)
--(axis cs:300,-0.0106469091592827)
--(axis cs:299,0.187414296185457)
--(axis cs:298,-0.0707236004098211)
--(axis cs:297,0.139336361021846)
--(axis cs:296,-0.0641971180192488)
--(axis cs:295,0.170326544766323)
--(axis cs:294,-0.112712712588854)
--(axis cs:293,0.00756406686301625)
--(axis cs:292,-0.0548423584583073)
--(axis cs:291,0.032964261978622)
--(axis cs:290,-0.0635780285685648)
--(axis cs:289,-0.101854462100454)
--(axis cs:288,0.165220703324875)
--(axis cs:287,0.0640906407479902)
--(axis cs:286,-0.0294921601725092)
--(axis cs:285,-0.0254738404377994)
--(axis cs:284,0.0287191521144143)
--(axis cs:283,-0.0918702013386317)
--(axis cs:282,-0.134257111392468)
--(axis cs:281,-0.0786304886592307)
--(axis cs:280,-0.115313137857743)
--(axis cs:279,-0.110698501597556)
--(axis cs:278,-0.132838201917067)
--(axis cs:277,-0.140057896386635)
--(axis cs:276,-0.00182372490444221)
--(axis cs:275,-0.158636590339302)
--(axis cs:274,-0.0352022039629545)
--(axis cs:273,-0.0919654907804526)
--(axis cs:272,-0.061904159141923)
--(axis cs:271,-0.0341345877087521)
--(axis cs:270,-0.189768859881524)
--(axis cs:269,-0.0442794006390222)
--(axis cs:268,-0.0306551635934844)
--(axis cs:267,-0.182127098010829)
--(axis cs:266,-0.221397003195111)
--(axis cs:265,-0.0478434668545049)
--(axis cs:264,-0.122390363346778)
--(axis cs:263,-0.112712712588854)
--(axis cs:262,-0.00182372490444221)
--(axis cs:261,-0.0341345877087521)
--(axis cs:260,0)
--(axis cs:259,-0.0442794006390222)
--(axis cs:258,-0.0442794006390222)
--(axis cs:257,-0.0885588012780444)
--(axis cs:256,-0.0786304886592307)
--(axis cs:255,-0.0147327453902263)
--(axis cs:254,-0.0885588012780444)
--(axis cs:253,-0.182127098010829)
--(axis cs:252,-0.0748487201769177)
--(axis cs:251,-0.113781959029174)
--(axis cs:250,0)
--(axis cs:249,-0.153030676624786)
--(axis cs:248,0)
--(axis cs:247,-0.0442794006390222)
--(axis cs:246,-0.160062259703804)
--(axis cs:245,-0.0753927879903306)
--(axis cs:244,-0.0442794006390222)
--(axis cs:243,-0.189768859881524)
--(axis cs:242,-0.0442794006390222)
--(axis cs:241,-0.061904159141923)
--(axis cs:240,-0.0802169653783137)
--(axis cs:239,-0.080031129851902)
--(axis cs:238,-0.0442794006390222)
--(axis cs:237,-0.0885588012780444)
--(axis cs:236,-0.061904159141923)
--(axis cs:235,0)
--(axis cs:234,-0.0442794006390222)
--(axis cs:233,0)
--(axis cs:232,-0.182127098010829)
--(axis cs:231,0)
--(axis cs:230,-0.0862205133199658)
--(axis cs:229,-0.0855976535490173)
--(axis cs:228,-0.189768859881524)
--(axis cs:227,-0.132838201917067)
--(axis cs:226,-0.0341345877087521)
--(axis cs:225,0)
--(axis cs:224,0)
--(axis cs:223,0)
--(axis cs:222,-0.0442794006390222)
--(axis cs:221,-0.061904159141923)
--(axis cs:220,-0.0442794006390222)
--(axis cs:219,0)
--(axis cs:218,-0.0341345877087521)
--(axis cs:217,0)
--(axis cs:216,0)
--(axis cs:215,0)
--(axis cs:214,-0.0442794006390222)
--(axis cs:213,-0.189768859881524)
--(axis cs:212,-0.073799001065037)
--(axis cs:211,0)
--(axis cs:210,0)
--(axis cs:209,-0.0147598002130074)
--(axis cs:208,0)
--(axis cs:207,0)
--(axis cs:206,0)
--(axis cs:205,0)
--(axis cs:204,0)
--(axis cs:203,-0.110698501597556)
--(axis cs:202,-0.184497502662593)
--(axis cs:201,-0.110698501597556)
--(axis cs:200,0)
--(axis cs:199,0)
--(axis cs:198,0)
--(axis cs:197,0)
--(axis cs:196,-0.0442794006390222)
--(axis cs:195,0)
--(axis cs:194,0)
--(axis cs:193,0)
--(axis cs:192,-0.0442794006390222)
--(axis cs:191,0)
--(axis cs:190,0)
--(axis cs:189,0)
--(axis cs:188,0)
--(axis cs:187,0)
--(axis cs:186,0)
--(axis cs:185,0)
--(axis cs:184,0)
--(axis cs:183,0)
--(axis cs:182,0)
--(axis cs:181,0)
--(axis cs:180,0)
--(axis cs:179,-0.073799001065037)
--(axis cs:178,-0.0442794006390222)
--(axis cs:177,0)
--(axis cs:176,0)
--(axis cs:175,0)
--(axis cs:174,0)
--(axis cs:173,0)
--(axis cs:172,0)
--(axis cs:171,0)
--(axis cs:170,0)
--(axis cs:169,0)
--(axis cs:168,0)
--(axis cs:167,0)
--(axis cs:166,0)
--(axis cs:165,-0.0442794006390222)
--(axis cs:164,0)
--(axis cs:163,0)
--(axis cs:162,-0.0885588012780444)
--(axis cs:161,0)
--(axis cs:160,0)
--(axis cs:159,0)
--(axis cs:158,0)
--(axis cs:157,0)
--(axis cs:156,-0.0442794006390222)
--(axis cs:155,0)
--(axis cs:154,0)
--(axis cs:153,0)
--(axis cs:152,0)
--(axis cs:151,-0.00885588012780444)
--(axis cs:150,0)
--(axis cs:149,-0.110698501597556)
--(axis cs:148,0)
--(axis cs:147,0)
--(axis cs:146,0)
--(axis cs:145,0)
--(axis cs:144,0)
--(axis cs:143,0)
--(axis cs:142,0)
--(axis cs:141,0)
--(axis cs:140,0)
--(axis cs:139,0)
--(axis cs:138,0)
--(axis cs:137,0)
--(axis cs:136,0)
--(axis cs:135,-0.0442794006390222)
--(axis cs:134,0)
--(axis cs:133,0)
--(axis cs:132,0)
--(axis cs:131,0)
--(axis cs:130,0)
--(axis cs:129,0)
--(axis cs:128,0)
--(axis cs:127,0)
--(axis cs:126,0)
--(axis cs:125,0)
--(axis cs:124,0)
--(axis cs:123,0)
--(axis cs:122,0)
--(axis cs:121,0)
--(axis cs:120,0)
--(axis cs:119,0)
--(axis cs:118,0)
--(axis cs:117,0)
--(axis cs:116,0)
--(axis cs:115,0)
--(axis cs:114,-0.0368995005325185)
--(axis cs:113,0)
--(axis cs:112,-0.0442794006390222)
--(axis cs:111,0)
--(axis cs:110,0)
--(axis cs:109,0)
--(axis cs:108,0)
--(axis cs:107,0)
--(axis cs:106,0)
--(axis cs:105,-0.00885588012780444)
--(axis cs:104,-0.00885588012780444)
--(axis cs:103,0)
--(axis cs:102,0)
--(axis cs:101,0)
--(axis cs:100,0)
--(axis cs:99,0)
--(axis cs:98,0)
--(axis cs:97,0)
--(axis cs:96,0)
--(axis cs:95,0)
--(axis cs:94,0)
--(axis cs:93,0)
--(axis cs:92,0)
--(axis cs:91,0)
--(axis cs:90,0)
--(axis cs:89,0)
--(axis cs:88,0)
--(axis cs:87,0)
--(axis cs:86,0)
--(axis cs:85,0)
--(axis cs:84,0)
--(axis cs:83,0)
--(axis cs:82,0)
--(axis cs:81,0)
--(axis cs:80,0)
--(axis cs:79,-0.0101854462100454)
--(axis cs:78,-0.0147598002130074)
--(axis cs:77,-0.0147598002130074)
--(axis cs:76,-0.0442794006390222)
--(axis cs:75,-0.0147598002130074)
--(axis cs:74,0)
--(axis cs:73,0)
--(axis cs:72,0)
--(axis cs:71,-0.00885588012780444)
--(axis cs:70,0)
--(axis cs:69,-0.00295196004260148)
--(axis cs:68,0)
--(axis cs:67,0)
--(axis cs:66,-0.0221397003195111)
--(axis cs:65,0)
--(axis cs:64,0)
--(axis cs:63,-0.00590392008520296)
--(axis cs:62,0)
--(axis cs:61,0)
--(axis cs:60,0)
--(axis cs:59,-0.0993795955514264)
--(axis cs:58,-0.0442794006390222)
--(axis cs:57,0)
--(axis cs:56,0)
--(axis cs:55,-0.00295196004260148)
--(axis cs:54,0)
--(axis cs:53,0)
--(axis cs:52,0)
--(axis cs:51,0)
--(axis cs:50,-0.00295196004260148)
--(axis cs:49,0)
--(axis cs:48,0)
--(axis cs:47,0)
--(axis cs:46,0)
--(axis cs:45,0)
--(axis cs:44,0)
--(axis cs:43,0)
--(axis cs:42,0)
--(axis cs:41,0)
--(axis cs:40,0)
--(axis cs:39,0)
--(axis cs:38,-0.0073799001065037)
--(axis cs:37,-0.00885588012780441)
--(axis cs:36,-0.00885588012780441)
--(axis cs:35,-0.00885588012780441)
--(axis cs:34,-0.0221397003195111)
--(axis cs:33,-0.0123808318283846)
--(axis cs:32,-0.00452709121198022)
--(axis cs:31,-0.0105427144378625)
--(axis cs:30,-0.00955209817750164)
--(axis cs:29,-0.00819988900722631)
--(axis cs:28,-0.0368848343962641)
--(axis cs:27,-0.0107344001549145)
--(axis cs:26,-0.00491993340433579)
--(axis cs:25,0)
--(axis cs:24,0)
--(axis cs:23,0)
--(axis cs:22,0)
--(axis cs:21,-0.0049199334043358)
--(axis cs:20,0)
--(axis cs:19,-0.0109656518643766)
--(axis cs:18,-0.0147598002130074)
--(axis cs:17,-0.0147598002130074)
--(axis cs:16,-0.0147598002130074)
--(axis cs:15,-0.0101854462100454)
--(axis cs:14,-0.000982183026015094)
--(axis cs:13,-0.00295196004260148)
--(axis cs:12,0)
--(axis cs:11,-0.00295196004260148)
--(axis cs:10,-0.00491993340433579)
--(axis cs:9,-0.00295196004260148)
--(axis cs:8,0)
--(axis cs:7,-0.0122892057989671)
--(axis cs:6,-0.00885588012780443)
--(axis cs:5,-0.0049199334043358)
--(axis cs:4,0)
--(axis cs:3,-0.00944964178040325)
--(axis cs:2,-0.00679029747336362)
--(axis cs:1,0)
--(axis cs:0,0)
--cycle;

\addplot [semithick, color0, dash pattern=on 1pt off 3pt on 3pt off 3pt, forget plot]
table [row sep=\\]{%
0	0 \\
1	0 \\
2	0 \\
3	0 \\
4	0 \\
5	0 \\
6	0.003125 \\
7	0.00260416666666667 \\
8	0 \\
9	0.003125 \\
10	0.0078125 \\
11	0.003125 \\
12	0 \\
13	0 \\
14	0 \\
15	0 \\
16	0.003125 \\
17	0.0109375 \\
18	0.00625 \\
19	0 \\
20	0 \\
21	0 \\
22	0 \\
23	0 \\
24	0 \\
25	0.01171875 \\
26	0.0201388888888889 \\
27	0.003125 \\
28	0 \\
29	0 \\
30	0 \\
31	0 \\
32	0 \\
33	0 \\
34	0 \\
35	0 \\
36	0 \\
37	0.00625 \\
38	0.0200892857142857 \\
39	0.003125 \\
40	0 \\
41	0.046875 \\
42	0.009375 \\
43	0 \\
44	0.021875 \\
45	0.009375 \\
46	0 \\
47	0.053125 \\
48	0 \\
49	0 \\
50	0.133928571428571 \\
51	0.009375 \\
52	0 \\
53	0 \\
54	0 \\
55	0 \\
56	0 \\
57	0.009375 \\
58	0 \\
59	0 \\
60	0 \\
61	0 \\
62	0.003125 \\
63	0 \\
64	0 \\
65	0 \\
66	0 \\
67	0 \\
68	0 \\
69	0 \\
70	0 \\
71	0 \\
72	0 \\
73	0 \\
74	0 \\
75	0 \\
76	0.0267857142857143 \\
77	0 \\
78	0 \\
79	0 \\
80	0 \\
81	0.003125 \\
82	0 \\
83	0.009375 \\
84	0.00520833333333333 \\
85	0.00625 \\
86	0.0104166666666667 \\
87	0.015625 \\
88	0.012890625 \\
89	0.00625 \\
90	0.03125 \\
91	0.0333333333333333 \\
92	0.0078125 \\
93	0.00625 \\
94	0 \\
95	0.0364583333333333 \\
96	0.05 \\
97	0.0111607142857143 \\
98	0.0375 \\
99	0 \\
100	0 \\
101	0 \\
102	0.0435267857142857 \\
103	0.15625 \\
104	0.015625 \\
105	0.0796875 \\
106	0 \\
107	0 \\
108	0.0625 \\
109	0.046875 \\
110	0.234375 \\
111	0.233203125 \\
112	0.234375 \\
113	0.24375 \\
114	0 \\
115	0 \\
116	0.1875 \\
117	0.167410714285714 \\
118	0.28125 \\
119	0 \\
120	0.234375 \\
121	0.234375 \\
122	0.17578125 \\
123	0.1640625 \\
124	0 \\
125	0.003125 \\
126	0 \\
127	0.09375 \\
128	0.009375 \\
129	0 \\
130	0 \\
131	0 \\
132	0.009375 \\
133	0 \\
134	0.136607142857143 \\
135	0.237890625 \\
136	0.05 \\
137	0.028125 \\
138	0.009375 \\
139	0.290625 \\
140	0.234375 \\
141	0.28125 \\
142	0.24375 \\
143	0.24375 \\
144	0.234375 \\
145	0.290625 \\
146	0.309375 \\
147	0.234375 \\
148	0.234375 \\
149	0.234375 \\
150	0.234375 \\
151	0.290625 \\
152	0.234375 \\
153	0.24375 \\
154	0.290625 \\
155	0.234375 \\
156	0.25 \\
157	0.240625 \\
158	0.234375 \\
159	0.315178571428571 \\
160	0.2671875 \\
161	0.234375 \\
162	0.253125 \\
163	0.35 \\
164	0.277644230769231 \\
165	0.24375 \\
166	0.267857142857143 \\
167	0.375 \\
168	0.284765625 \\
169	0.310212053571429 \\
170	0.234375 \\
171	0.234375 \\
172	0.4078125 \\
173	0.505078125 \\
174	0.505729166666667 \\
175	0.234375 \\
176	0.41328125 \\
177	0.3 \\
178	0.253125 \\
179	0.571875 \\
180	0.4 \\
181	0.323863636363636 \\
182	0.4859375 \\
183	0.545982142857143 \\
184	0.396875 \\
185	0.295738636363636 \\
186	0.636160714285714 \\
187	0.328125 \\
188	0.353125 \\
189	0.515792410714286 \\
190	0.537890625 \\
191	0.36640625 \\
192	0.65726131735589 \\
193	0.652752130681818 \\
194	0.681135110294118 \\
195	0.559440559440559 \\
196	0.61640625 \\
197	0.585128348214286 \\
198	0.500520833333333 \\
199	0.628901397515528 \\
200	0.3375 \\
201	0.558664772727273 \\
202	0.70859375 \\
203	0.654241071428571 \\
204	0.623325892857143 \\
205	0.704595588235294 \\
206	0.85053228021978 \\
207	0.787946428571429 \\
208	0.7203125 \\
209	0.645833333333333 \\
210	0.795535714285714 \\
211	0.591706730769231 \\
212	0.6046875 \\
213	0.515625 \\
214	0.659895833333333 \\
215	0.679138764880952 \\
216	0.77890625 \\
217	0.939713541666667 \\
218	0.63125 \\
219	0.769663345410628 \\
220	0.877008928571429 \\
221	0.851106770833333 \\
222	0.765705128205128 \\
223	0.77265625 \\
224	0.724981945903361 \\
225	0.860100446428571 \\
226	0.903571428571429 \\
227	0.974330357142857 \\
228	0.7140625 \\
229	0.762326388888889 \\
230	0.84038318452381 \\
231	0.854613095238095 \\
232	0.88046875 \\
233	1.06607142857143 \\
234	1.31473214285714 \\
235	0.987710336538462 \\
236	1.66755681818182 \\
237	1.66041666666667 \\
238	1.76986799568966 \\
239	1.56392721861472 \\
240	1.41376488095238 \\
241	1.58340336134454 \\
242	1.79603794642857 \\
243	1.48376116071429 \\
244	1.415625 \\
245	0.893136160714286 \\
246	1.63946632584485 \\
247	1.4677306547619 \\
248	1.278125 \\
249	1.16030505952381 \\
250	1.68855857683983 \\
251	1.66032196969697 \\
252	1.733984375 \\
253	1.73405708874459 \\
254	1.69326923076923 \\
255	2.03319805194805 \\
256	1.76806988374529 \\
257	1.49671474358974 \\
258	1.5644798136646 \\
259	1.56458333333333 \\
260	1.79237351190476 \\
261	1.47061298076923 \\
262	1.63597132034632 \\
263	1.65971474527311 \\
264	1.74730113636364 \\
265	1.53296130952381 \\
266	1.84479166666667 \\
267	1.69371565934066 \\
268	1.7544333921287 \\
269	1.96755022321429 \\
270	1.93461174242424 \\
271	1.84072265625 \\
272	1.80535714285714 \\
273	2.13715657554063 \\
274	2.35920902014652 \\
275	2.44488463309884 \\
276	2.1361328125 \\
277	1.89744088898101 \\
278	2.16502343450794 \\
279	2.23076754385965 \\
280	2.41381243315865 \\
281	2.26412545787546 \\
282	2.42676711309524 \\
283	2.39149957306207 \\
284	2.1372081043956 \\
285	2.36973802060009 \\
286	2.77074922360248 \\
287	2.52056640625 \\
288	2.4546875 \\
289	2.39137620192308 \\
290	2.43301221804511 \\
291	2.2041907979408 \\
292	2.62277644230769 \\
293	2.28956473214286 \\
294	2.83009672619048 \\
295	2.91227870997537 \\
296	2.71104481456044 \\
297	3.01579591735403 \\
298	3.35363713603207 \\
299	3.10683035714286 \\
300	2.86490076013513 \\
301	2.97943131087662 \\
302	3.20484928772998 \\
303	2.91809760551948 \\
304	3.04348082983193 \\
305	3.22093285657273 \\
306	3.00265456989247 \\
307	3.32791548295455 \\
308	3.06048153672224 \\
309	3.83632618616118 \\
310	3.58220933161666 \\
311	3.70041993596681 \\
312	3.73639680631868 \\
313	3.88541587089381 \\
314	3.81515827922078 \\
315	4.15346760699472 \\
316	3.94650822079491 \\
317	4.20846497252747 \\
318	4.2897907239819 \\
319	4.36190927128427 \\
320	4.27726496848739 \\
321	3.91049952651515 \\
322	4.04651157680089 \\
323	4.67428504954098 \\
324	4.39580614697802 \\
325	4.72185296474359 \\
326	4.55845072735058 \\
327	4.47158756004714 \\
328	4.48127545426065 \\
329	4.18764443277311 \\
330	4.61080013736264 \\
331	4.0828403173883 \\
332	4.83159551266568 \\
333	4.25318431568432 \\
334	4.42255280127402 \\
335	4.28234825835929 \\
336	4.411328125 \\
337	4.57784598214286 \\
338	4.48584206780538 \\
339	4.45147942590498 \\
340	4.65042730706793 \\
341	4.72465128621379 \\
342	4.9419932606456 \\
343	4.7211168514785 \\
344	4.43693169575522 \\
345	4.89139365767947 \\
346	5.12680297742264 \\
347	4.85851901185675 \\
348	5.57329178158311 \\
349	4.98151458810069 \\
350	5.66931559669976 \\
351	5.70315320530164 \\
352	5.63131744294708 \\
353	5.84290267850911 \\
354	5.6075633725934 \\
355	5.48764267530715 \\
356	5.84077539381493 \\
357	5.83804493204164 \\
358	5.92472550243215 \\
359	6.07356085526316 \\
360	5.98380795002683 \\
361	5.52093039772727 \\
362	5.88589787714328 \\
363	5.68061372221529 \\
364	5.88167534722222 \\
365	5.95169298240504 \\
366	6.13234192341634 \\
367	6.48451702739639 \\
368	6.33872917107802 \\
369	6.10187337363923 \\
370	6.41903130463759 \\
371	6.38927021329365 \\
372	6.2558866867958 \\
373	5.97206744868035 \\
374	6.21081516631252 \\
375	6.39542547487745 \\
376	6.43855924186907 \\
377	6.36321775071775 \\
378	6.66652969426407 \\
379	6.2355716765873 \\
380	6.53598748858947 \\
381	6.84089087320784 \\
382	6.95044059340598 \\
383	7.05552079841759 \\
384	7.03125310019841 \\
385	6.93667283237596 \\
386	6.9135765440857 \\
387	6.75426138981385 \\
388	7.27605477544117 \\
389	7.52181709954049 \\
390	7.6995216926915 \\
391	7.68938232990967 \\
392	7.64382061761968 \\
393	7.44375210726773 \\
394	7.5542353759716 \\
395	8.15379982864358 \\
396	7.78679202393726 \\
397	8.40380129448678 \\
398	8.22250861118049 \\
399	7.90677485812267 \\
400	8.04049227122564 \\
401	8.25319801726052 \\
402	8.41233202561328 \\
403	8.79610234434683 \\
404	8.66695699112979 \\
405	8.88942522321429 \\
406	9.04170659229883 \\
407	9.02612179487179 \\
408	8.81436631944444 \\
409	8.62115147783251 \\
410	9.19534105149237 \\
411	9.23296130952381 \\
412	9.02806417693826 \\
413	8.90428114672066 \\
414	9.58610245601173 \\
415	9.24109796453546 \\
416	9.27721315599605 \\
417	9.85997810782967 \\
418	9.69556691628264 \\
419	9.72533954949036 \\
420	9.70246299467619 \\
421	9.89796706989247 \\
422	9.93178615196078 \\
423	9.80981652462121 \\
424	9.8820110251762 \\
425	9.70667021150892 \\
426	9.68521416083916 \\
427	9.87104616398095 \\
428	9.80284350495082 \\
429	9.88245382662721 \\
430	10.037328778758 \\
431	10.1858928493163 \\
432	10.0619460434942 \\
433	10.2507616341991 \\
434	10.4855059776457 \\
435	10.2839216146797 \\
436	10.8151430892988 \\
437	10.678773529371 \\
438	10.4837357954545 \\
439	10.6076707063098 \\
440	10.7456940628816 \\
441	10.8561142521272 \\
442	10.9790625 \\
443	10.9412514568765 \\
444	10.8832209696434 \\
445	11.0657707734674 \\
446	11.2491376678877 \\
447	11.1402882438443 \\
448	11.2061832264957 \\
449	11.3918784340659 \\
450	11.283879913769 \\
451	11.2746735848928 \\
452	11.3835636697861 \\
453	11.3744246971066 \\
454	11.3861218944099 \\
455	11.6129550291219 \\
456	11.7565172697368 \\
457	11.8440600198413 \\
458	11.8699596774194 \\
459	12.0093060661765 \\
460	11.8724022952854 \\
461	11.7211038961039 \\
462	11.8720766129032 \\
463	11.9005837912088 \\
464	11.8427884615385 \\
465	11.8659358003108 \\
466	11.90625 \\
467	11.8289124668435 \\
468	11.8756944444444 \\
469	11.9542452819564 \\
470	11.8995535714286 \\
471	11.9498345135468 \\
472	11.953125 \\
473	11.9606971153846 \\
474	11.859375 \\
475	11.9648101898102 \\
476	11.90478515625 \\
477	12.1211965460526 \\
478	11.9608966503268 \\
479	12.0220645680147 \\
480	12.2917564655172 \\
481	12.286006773399 \\
482	12.2373060966811 \\
483	12.3121582365003 \\
484	12.2674553571429 \\
485	12.3431500204248 \\
486	12.2721128994361 \\
487	11.9948705808081 \\
488	12.4661842105263 \\
489	12.4770984927235 \\
490	12.463125 \\
491	12.4566644293207 \\
492	12.5120192307692 \\
493	12.5194741410819 \\
494	12.6476862980769 \\
495	12.6965183002481 \\
496	12.668471834829 \\
497	12.6596804584305 \\
498	12.7220982142857 \\
499	12.7453125 \\
500	12.7772321428571 \\
501	12.7410224780702 \\
502	12.8579050457179 \\
503	12.9915329126519 \\
504	13.0178977272727 \\
505	12.8935489766082 \\
506	12.9463235294118 \\
507	12.8958333333333 \\
508	13.1590711805556 \\
509	13.2074353448276 \\
510	13.2388392857143 \\
511	13.246875 \\
512	13.275 \\
513	13.265625 \\
514	13.265625 \\
515	13.265625 \\
516	13.265625 \\
517	13.265625 \\
518	13.2552989130435 \\
519	13.265625 \\
520	13.2625 \\
521	13.265625 \\
522	13.2307573891626 \\
523	13.2564453125 \\
524	13.21875 \\
525	13.21875 \\
526	13.25625 \\
527	13.234375 \\
528	13.3005712365591 \\
529	13.3080357142857 \\
530	13.2244318181818 \\
531	13.453125 \\
532	13.4003434065934 \\
533	13.2946428571429 \\
534	13.21875 \\
535	13.3990384615385 \\
536	13.40625 \\
537	13.4460304054054 \\
538	13.5386424731183 \\
539	13.5648370726496 \\
540	13.5667162698413 \\
541	13.734375 \\
542	13.7326388888889 \\
543	13.7836805555556 \\
544	13.784765625 \\
545	13.8 \\
546	13.806326486014 \\
547	13.8038194444444 \\
548	13.796875 \\
549	13.821875 \\
550	13.7941277472527 \\
551	13.84375 \\
552	13.8103632478632 \\
553	13.828125 \\
554	13.828125 \\
555	13.828125 \\
556	13.828125 \\
557	13.828125 \\
558	14.0526315789474 \\
559	13.828125 \\
560	13.875 \\
561	13.8970810602467 \\
562	14.04375 \\
563	13.875 \\
564	13.8713942307692 \\
565	13.9080141129032 \\
566	14.0086805555556 \\
567	14.0086805555556 \\
568	14.0538194444444 \\
569	14.0534855769231 \\
570	14.0625 \\
571	14.0222355769231 \\
572	14.015625 \\
573	14.015625 \\
574	14.03125 \\
575	14.0625 \\
576	14.0625 \\
577	14.0625 \\
578	14.0625 \\
579	13.828125 \\
580	14.0523073476703 \\
581	14.0625 \\
582	14.0625 \\
583	14.0625 \\
584	14.0625 \\
585	14.0534855769231 \\
586	14.0572916666667 \\
587	14.0625 \\
588	14.034375 \\
589	14.0491071428571 \\
590	14.025 \\
591	14.0625 \\
592	14.0541294642857 \\
593	14.0625 \\
594	14.0625 \\
595	14.0625 \\
596	14.0534855769231 \\
597	14.0625 \\
598	14.0625 \\
599	14.0625 \\
};
\addplot [semithick, color1, forget plot]
table [row sep=\\]{%
0	0 \\
1	0.00260416666666667 \\
2	0 \\
3	0 \\
4	0 \\
5	0 \\
6	0 \\
7	0.003125 \\
8	0 \\
9	0 \\
10	0 \\
11	0 \\
12	0.00669642857142857 \\
13	0 \\
14	0 \\
15	0 \\
16	0 \\
17	0.015625 \\
18	0.015625 \\
19	0.015625 \\
20	0.0133928571428571 \\
21	0 \\
22	0 \\
23	0 \\
24	0 \\
25	0.003125 \\
26	0 \\
27	0 \\
28	0 \\
29	0 \\
30	0 \\
31	0 \\
32	0 \\
33	0 \\
34	0 \\
35	0 \\
36	0 \\
37	0 \\
38	0 \\
39	0 \\
40	0 \\
41	0 \\
42	0 \\
43	0 \\
44	0 \\
45	0 \\
46	0 \\
47	0 \\
48	0 \\
49	0 \\
50	0 \\
51	0 \\
52	0 \\
53	0 \\
54	0 \\
55	0 \\
56	0 \\
57	0 \\
58	0 \\
59	0 \\
60	0 \\
61	0 \\
62	0 \\
63	0 \\
64	0 \\
65	0 \\
66	0 \\
67	0 \\
68	0 \\
69	0.003125 \\
70	0.003125 \\
71	0.003125 \\
72	0.003125 \\
73	0 \\
74	0 \\
75	0 \\
76	0 \\
77	0 \\
78	0 \\
79	0 \\
80	0 \\
81	0 \\
82	0 \\
83	0 \\
84	0 \\
85	0 \\
86	0.046875 \\
87	0 \\
88	0.046875 \\
89	0.470913461538462 \\
90	0.703125 \\
91	0.703125 \\
92	0.691964285714286 \\
93	0.7125 \\
94	0.903172348484849 \\
95	0.928125 \\
96	0.9375 \\
97	0.95625 \\
98	1.17223557692308 \\
99	1.18125 \\
100	1.18125 \\
101	1.1625 \\
102	1.171875 \\
103	1.171875 \\
104	1.3960597826087 \\
105	1.396484375 \\
106	1.396484375 \\
107	1.40625 \\
108	1.40625 \\
109	1.40625 \\
110	1.603125 \\
111	1.640625 \\
112	1.63161057692308 \\
113	1.63161057692308 \\
114	1.63161057692308 \\
115	1.640625 \\
116	1.640625 \\
117	1.65104166666667 \\
118	1.640625 \\
119	1.65451388888889 \\
120	1.63161057692308 \\
121	1.65625 \\
122	1.63125 \\
123	1.63161057692308 \\
124	1.640625 \\
125	1.83250343406593 \\
126	1.875 \\
127	1.884375 \\
128	1.921875 \\
129	1.88571428571429 \\
130	1.884375 \\
131	1.93125 \\
132	1.90661057692308 \\
133	1.8984375 \\
134	1.921875 \\
135	2.04397321428571 \\
136	2.20657894736842 \\
137	2.29935064935065 \\
138	2.35714285714286 \\
139	2.56506774475524 \\
140	2.53665865384615 \\
141	2.82217548076923 \\
142	2.81422991071429 \\
143	2.99646577380952 \\
144	3.07267992424242 \\
145	2.92327008928571 \\
146	3.29765625 \\
147	3.06749131944444 \\
148	3.15055147058824 \\
149	3.08180147058824 \\
150	3.27281709558823 \\
151	3.4469696969697 \\
152	3.55133928571429 \\
153	3.50497159090909 \\
154	3.95073265550239 \\
155	3.97058823529412 \\
156	4.15211397058824 \\
157	4.19587053571429 \\
158	4.21875 \\
159	4.39778645833333 \\
160	4.66462053571429 \\
161	4.89391447368421 \\
162	4.88934659090909 \\
163	4.69456845238095 \\
164	4.83475274725275 \\
165	5.2092803030303 \\
166	5.16110491071429 \\
167	5.4109681372549 \\
168	5.35748106060606 \\
169	5.59134615384615 \\
170	5.625 \\
171	5.85587528280543 \\
172	5.95912796442688 \\
173	6.00326236263736 \\
174	6.31882440476191 \\
175	6.52853094362745 \\
176	6.46205357142857 \\
177	6.55184659090909 \\
178	6.65801646270396 \\
179	6.59637784090909 \\
180	6.77472897302861 \\
181	6.92848501801907 \\
182	6.63576121144481 \\
183	6.7749660326087 \\
184	7.00053097943723 \\
185	6.89228766025641 \\
186	7.11959212662338 \\
187	7.19364500988142 \\
188	7.14697712418301 \\
189	7.41263640873016 \\
190	7.4078125 \\
191	7.41090959821429 \\
192	7.49413819875776 \\
193	7.97391141288631 \\
194	8.31328125 \\
195	8.19079748376623 \\
196	8.33805147058824 \\
197	8.63398591897233 \\
198	8.53989806149733 \\
199	8.67278814935065 \\
200	9.02933784965035 \\
201	9.53913352272727 \\
202	9.79439831002331 \\
203	9.99135218934412 \\
204	9.96965479373568 \\
205	10.0489955357143 \\
206	10.2193004261364 \\
207	9.81828067765568 \\
208	10.4303952205882 \\
209	10.500625 \\
210	10.9073353390269 \\
211	11.202496689891 \\
212	11.0794526667587 \\
213	11.2632830039584 \\
214	11.2682276604974 \\
215	11.3703125 \\
216	11.3275072109152 \\
217	11.569689728164 \\
218	11.6893735079259 \\
219	11.7959007352941 \\
220	12.0727870475113 \\
221	12.1158024267399 \\
222	12.1009830013736 \\
223	12.1077223557692 \\
224	12.1309987001223 \\
225	12.2916666666667 \\
226	12.416015625 \\
227	12.7246428571429 \\
228	12.7200592376374 \\
229	12.4944886363636 \\
230	12.9522569444444 \\
231	12.9807291666667 \\
232	13.0319994490537 \\
233	13.102785326087 \\
234	13.2604166666667 \\
235	13.198828125 \\
236	13.1850231046366 \\
237	13.2067750506073 \\
238	13.2085597826087 \\
239	13.2043707271788 \\
240	13.3918269230769 \\
241	13.40625 \\
242	13.634375 \\
243	13.6180491727941 \\
244	13.65 \\
245	13.4208639705882 \\
246	13.479175015027 \\
247	13.712576486014 \\
248	13.6697544642857 \\
249	13.8400005225753 \\
250	13.8704633295626 \\
251	13.8842903020892 \\
252	13.6719123803828 \\
253	13.9034127237852 \\
254	13.9585597826087 \\
255	14.015625 \\
256	13.9453072390572 \\
257	13.8555803571429 \\
258	14.1010044642857 \\
259	14.0839920948617 \\
260	14.0945338968097 \\
261	14.1003605769231 \\
262	14.1107954545455 \\
263	14.1368534482759 \\
264	14.1105846774194 \\
265	13.7364309210526 \\
266	13.8444484147609 \\
267	14.0971433080808 \\
268	14.0913722826087 \\
269	13.8954326923077 \\
270	13.9751820958647 \\
271	13.983913208502 \\
272	14.0831823079937 \\
273	14.1260080645161 \\
274	14.113779047976 \\
275	14.1450892857143 \\
276	14.15625 \\
277	14.15625 \\
278	14.15 \\
279	14.015625 \\
280	14.1264022435897 \\
281	14.15625 \\
282	14.1215193576585 \\
283	14.15625 \\
284	14.3804347826087 \\
285	14.3816105769231 \\
286	14.3816105769231 \\
287	14.390625 \\
288	14.3758960573477 \\
289	14.3871527777778 \\
290	14.6005859375 \\
291	14.7393465909091 \\
292	14.6266998626374 \\
293	14.7641129032258 \\
294	14.7801724137931 \\
295	14.6283235178868 \\
296	14.6710136003427 \\
297	14.7641129032258 \\
298	14.7603794642857 \\
299	14.8058035714286 \\
300	14.8125 \\
301	14.8125 \\
302	14.8038194444444 \\
303	14.8125 \\
304	14.8125 \\
305	14.8125 \\
306	14.8018465909091 \\
307	14.8125 \\
308	14.8046875 \\
309	14.8041294642857 \\
310	14.8018465909091 \\
311	14.8125 \\
312	14.8125 \\
313	14.8041294642857 \\
314	14.8125 \\
315	14.8125 \\
316	14.8125 \\
317	14.8125 \\
318	14.8125 \\
319	14.8125 \\
320	14.8125 \\
321	14.8125 \\
322	14.8034855769231 \\
323	14.8125 \\
324	14.8125 \\
325	14.8125 \\
326	14.578125 \\
327	14.8125 \\
328	14.8125 \\
329	14.8125 \\
330	14.8125 \\
331	14.8125 \\
332	14.80625 \\
333	14.8125 \\
334	14.8125 \\
335	14.5651041666667 \\
336	14.8125 \\
337	14.8125 \\
338	14.8125 \\
339	14.8125 \\
340	14.8125 \\
341	14.804939516129 \\
342	14.8125 \\
343	14.8125 \\
344	14.8038194444444 \\
345	14.8125 \\
346	14.8125 \\
347	14.8064516129032 \\
348	14.8046875 \\
349	14.8125 \\
350	14.8125 \\
351	14.8125 \\
352	14.8125 \\
353	14.8125 \\
354	14.765625 \\
355	14.8125 \\
356	14.8125 \\
357	14.8125 \\
358	14.8125 \\
359	14.8125 \\
360	14.8046875 \\
361	14.8125 \\
362	14.8125 \\
363	14.8038194444444 \\
364	14.8125 \\
365	14.8125 \\
366	14.8125 \\
367	14.7890625 \\
368	14.8125 \\
369	14.8125 \\
370	14.8125 \\
371	14.8125 \\
372	14.8038194444444 \\
373	14.578125 \\
374	14.8125 \\
375	14.8125 \\
376	14.8125 \\
377	14.8125 \\
378	14.8125 \\
379	14.8125 \\
380	14.8125 \\
381	14.8125 \\
382	14.8125 \\
383	14.8060344827586 \\
384	14.8125 \\
385	14.578125 \\
386	14.8125 \\
387	14.8125 \\
388	14.8125 \\
389	14.8125 \\
390	14.8125 \\
391	14.8125 \\
392	14.8125 \\
393	14.8125 \\
394	14.8125 \\
395	14.8125 \\
396	14.8125 \\
397	14.8125 \\
398	14.8125 \\
399	14.8125 \\
400	14.8125 \\
401	14.8125 \\
402	14.8125 \\
403	14.8034855769231 \\
404	14.578125 \\
405	14.578125 \\
406	14.8125 \\
407	14.8125 \\
408	14.8125 \\
409	14.8125 \\
410	14.8125 \\
411	14.8125 \\
412	14.80078125 \\
413	14.8125 \\
414	14.8125 \\
415	14.804939516129 \\
416	14.8125 \\
417	14.8125 \\
418	14.8125 \\
419	14.8125 \\
420	14.8125 \\
421	14.8125 \\
422	14.8125 \\
423	14.8125 \\
424	14.8125 \\
425	14.8125 \\
426	14.8125 \\
427	14.8125 \\
428	14.8125 \\
429	14.8125 \\
430	14.8125 \\
431	14.8125 \\
432	14.8125 \\
433	14.8125 \\
434	14.8125 \\
435	14.8125 \\
436	14.8125 \\
437	14.8125 \\
438	14.8125 \\
439	14.8038194444444 \\
440	14.8125 \\
441	14.8125 \\
442	14.8018465909091 \\
443	14.8125 \\
444	14.8125 \\
445	14.8125 \\
446	14.8125 \\
447	14.8125 \\
448	14.8125 \\
449	14.8125 \\
450	14.8125 \\
451	14.8125 \\
452	14.8125 \\
453	14.8125 \\
454	14.8125 \\
455	14.8038194444444 \\
456	14.8125 \\
457	14.8125 \\
458	14.8023097826087 \\
459	14.8125 \\
460	14.8125 \\
461	14.8125 \\
462	14.8125 \\
463	14.8125 \\
464	14.8125 \\
465	14.8125 \\
466	14.8125 \\
467	14.8046875 \\
468	14.8125 \\
469	14.8125 \\
470	14.8125 \\
471	14.8125 \\
472	14.8125 \\
473	14.804939516129 \\
474	14.8125 \\
475	14.8125 \\
476	14.8125 \\
477	14.8125 \\
478	14.8125 \\
479	14.8125 \\
480	14.8041294642857 \\
481	14.8055555555556 \\
482	14.8125 \\
483	14.8125 \\
484	14.8125 \\
485	14.8125 \\
486	14.8125 \\
487	14.8034855769231 \\
488	14.8125 \\
489	14.8125 \\
490	14.8125 \\
491	14.8125 \\
492	14.8125 \\
493	14.8125 \\
494	14.8125 \\
495	14.80625 \\
496	14.8125 \\
497	14.578125 \\
498	14.8125 \\
499	14.8125 \\
500	14.804939516129 \\
501	14.8125 \\
502	14.8125 \\
503	14.8125 \\
504	14.8125 \\
505	14.8125 \\
506	14.8052884615385 \\
507	14.7604166666667 \\
508	14.8125 \\
509	14.8125 \\
510	14.8125 \\
511	14.7960069444444 \\
512	14.8125 \\
513	14.8125 \\
514	14.7960069444444 \\
515	14.8125 \\
516	14.8125 \\
517	14.8125 \\
518	14.8125 \\
519	14.8125 \\
520	14.8125 \\
521	14.8125 \\
522	14.8125 \\
523	14.8125 \\
524	14.8125 \\
525	14.8125 \\
526	14.8125 \\
527	14.8125 \\
528	14.8125 \\
529	14.8125 \\
530	14.8125 \\
531	14.8125 \\
532	14.8125 \\
533	14.8125 \\
534	14.8125 \\
535	14.8125 \\
536	14.8125 \\
537	14.8125 \\
538	14.8125 \\
539	14.8125 \\
540	14.8125 \\
541	14.8125 \\
542	14.8125 \\
543	14.804939516129 \\
544	14.8125 \\
545	14.8125 \\
546	14.8125 \\
547	14.8125 \\
548	14.8125 \\
549	14.8125 \\
550	14.8125 \\
551	14.8125 \\
552	14.8125 \\
553	14.8125 \\
554	14.8125 \\
555	14.8125 \\
556	14.8125 \\
557	14.8125 \\
558	14.8125 \\
559	14.8125 \\
560	14.8125 \\
561	14.8125 \\
562	14.8125 \\
563	14.8125 \\
564	14.8125 \\
565	14.8125 \\
566	14.8125 \\
567	14.8125 \\
568	14.8125 \\
569	14.8125 \\
570	14.8125 \\
571	14.8125 \\
572	14.8125 \\
573	14.8125 \\
574	14.8125 \\
575	14.8125 \\
576	14.8125 \\
577	14.8125 \\
578	14.8125 \\
579	14.8125 \\
580	14.8125 \\
581	14.8125 \\
582	14.8125 \\
583	14.8125 \\
584	14.8125 \\
585	14.8046875 \\
586	14.8125 \\
587	14.8125 \\
588	14.8018465909091 \\
589	14.8125 \\
590	14.8018465909091 \\
591	14.8125 \\
592	14.8125 \\
593	14.8125 \\
594	14.8125 \\
595	14.8125 \\
596	14.8125 \\
597	14.8125 \\
598	14.8125 \\
599	14.8125 \\
};
\addplot [semithick, color2, dashed, forget plot]
table [row sep=\\]{%
0	0 \\
1	0 \\
2	0.01875 \\
3	0.015625 \\
4	0.0143229166666667 \\
5	0.0173295454545455 \\
6	0.01875 \\
7	0.0144230769230769 \\
8	0.0078125 \\
9	0.0170138888888889 \\
10	0.0078125 \\
11	0 \\
12	0 \\
13	0 \\
14	0 \\
15	0.003125 \\
16	0 \\
17	0 \\
18	0 \\
19	0 \\
20	0 \\
21	0.00625 \\
22	0.00669642857142857 \\
23	0.00260416666666667 \\
24	0.0078125 \\
25	0 \\
26	0.003125 \\
27	0 \\
28	0 \\
29	0 \\
30	0 \\
31	0 \\
32	0.003125 \\
33	0.003125 \\
34	0.003125 \\
35	0 \\
36	0.009375 \\
37	0.00520833333333333 \\
38	0.0125 \\
39	0.003125 \\
40	0.0182291666666667 \\
41	0.0182291666666667 \\
42	0.00520833333333333 \\
43	0.003125 \\
44	0 \\
45	0 \\
46	0 \\
47	0 \\
48	0 \\
49	0.009375 \\
50	0 \\
51	0.0125 \\
52	0 \\
53	0 \\
54	0 \\
55	0 \\
56	0.003125 \\
57	0 \\
58	0 \\
59	0 \\
60	0 \\
61	0 \\
62	0 \\
63	0.003125 \\
64	0 \\
65	0 \\
66	0.00520833333333333 \\
67	0 \\
68	0 \\
69	0 \\
70	0 \\
71	0.003125 \\
72	0 \\
73	0 \\
74	0.003125 \\
75	0.01875 \\
76	0 \\
77	0.009375 \\
78	0.003125 \\
79	0.00625 \\
80	0 \\
81	0.0114583333333333 \\
82	0.0368303571428571 \\
83	0.00625 \\
84	0 \\
85	0.0104166666666667 \\
86	0.00892857142857143 \\
87	0.009375 \\
88	0.003125 \\
89	0 \\
90	0 \\
91	0.0111607142857143 \\
92	0.003125 \\
93	0.00625 \\
94	0.00892857142857143 \\
95	0 \\
96	0 \\
97	0.00520833333333333 \\
98	0.01171875 \\
99	0.003125 \\
100	0 \\
101	0.003125 \\
102	0.003125 \\
103	0.0109375 \\
104	0.00625 \\
105	0 \\
106	0.015625 \\
107	0.003125 \\
108	0 \\
109	0.00625 \\
110	0.013671875 \\
111	0.015625 \\
112	0.0147569444444444 \\
113	0.015625 \\
114	0.015625 \\
115	0.015625 \\
116	0.015625 \\
117	0 \\
118	0.0328125 \\
119	0 \\
120	0 \\
121	0 \\
122	0 \\
123	0 \\
124	0.003125 \\
125	0.003125 \\
126	0 \\
127	0 \\
128	0 \\
129	0 \\
130	0.025 \\
131	0.021875 \\
132	0.025 \\
133	0.01875 \\
134	0.0331730769230769 \\
135	0.003125 \\
136	0.003125 \\
137	0.00520833333333333 \\
138	0.0241776315789474 \\
139	0.03125 \\
140	0.015625 \\
141	0.015625 \\
142	0.015625 \\
143	0.025 \\
144	0.01875 \\
145	0 \\
146	0.0457386363636364 \\
147	0.0375 \\
148	0.0375 \\
149	0.0421875 \\
150	0.034375 \\
151	0.0622767857142857 \\
152	0.0578125 \\
153	0.009375 \\
154	0.040625 \\
155	0.0510416666666667 \\
156	0.0328125 \\
157	0.009375 \\
158	0 \\
159	0 \\
160	0.01875 \\
161	0.009375 \\
162	0 \\
163	0 \\
164	0 \\
165	0 \\
166	0 \\
167	0.00669642857142857 \\
168	0.0364583333333333 \\
169	0.009375 \\
170	0 \\
171	0.009375 \\
172	0.05 \\
173	0.003125 \\
174	0 \\
175	0.078125 \\
176	0.03125 \\
177	0.046875 \\
178	0.288616071428571 \\
179	0.234375 \\
180	0.26953125 \\
181	0.234375 \\
182	0.24375 \\
183	0.2625 \\
184	0.234375 \\
185	0.234375 \\
186	0.274553571428571 \\
187	0.234375 \\
188	0.234375 \\
189	0.234375 \\
190	0.234375 \\
191	0.234375 \\
192	0.234375 \\
193	0.234375 \\
194	0.2625 \\
195	0.267857142857143 \\
196	0.24375 \\
197	0.2375 \\
198	0.234375 \\
199	0.234375 \\
200	0.24375 \\
201	0.265625 \\
202	0.234375 \\
203	0.234375 \\
204	0.234375 \\
205	0.261160714285714 \\
206	0.234375 \\
207	0.25 \\
208	0.25625 \\
209	0.253125 \\
210	0.234375 \\
211	0.2375 \\
212	0.234375 \\
213	0.234375 \\
214	0.24375 \\
215	0.234375 \\
216	0.234375 \\
217	0.2671875 \\
218	0.234735576923077 \\
219	0.24375 \\
220	0.234375 \\
221	0.234375 \\
222	0.272569444444444 \\
223	0.28125 \\
224	0.234375 \\
225	0.234375 \\
226	0.24375 \\
227	0.234375 \\
228	0.234375 \\
229	0.225834061771562 \\
230	0.25 \\
231	0.248579545454545 \\
232	0.25 \\
233	0.291015625 \\
234	0.291666666666667 \\
235	0.283482142857143 \\
236	0.296875 \\
237	0.296875 \\
238	0.25 \\
239	0.2490234375 \\
240	0.296875 \\
241	0.24375 \\
242	0.24375 \\
243	0.234375 \\
244	0.253125 \\
245	0.24375 \\
246	0.50625 \\
247	0.478125 \\
248	0.657366071428571 \\
249	0.63046875 \\
250	0.550852272727273 \\
251	0.5390625 \\
252	0.528125 \\
253	0.603125 \\
254	0.755078125 \\
255	0.58125 \\
256	0.6 \\
257	0.5546875 \\
258	0.549080141129032 \\
259	0.578125 \\
260	0.7375 \\
261	0.695673076923077 \\
262	0.814006696428571 \\
263	0.55625 \\
264	0.6 \\
265	0.694419642857143 \\
266	0.578571428571429 \\
267	0.5625 \\
268	0.696875 \\
269	0.614423076923077 \\
270	0.645932163187856 \\
271	0.553819444444444 \\
272	0.55 \\
273	0.571875 \\
274	0.525 \\
275	0.5625 \\
276	0.614488636363636 \\
277	0.553819444444444 \\
278	0.328125 \\
279	0.605769230769231 \\
280	0.633238636363636 \\
281	0.7015625 \\
282	0.665625 \\
283	0.93125 \\
284	0.981730769230769 \\
285	1.069921875 \\
286	1.11869612068966 \\
287	1.08241914335664 \\
288	0.859821428571429 \\
289	1.29665178571429 \\
290	1.05625 \\
291	1.21875 \\
292	0.998322533444816 \\
293	1.27611607142857 \\
294	1.27081512237762 \\
295	1.23948754789272 \\
296	1.219921875 \\
297	1.33385416666667 \\
298	1.3 \\
299	1.3489025297619 \\
300	1.16629464285714 \\
301	1.37268415178571 \\
302	1.28214285714286 \\
303	1.15456730769231 \\
304	1.2125 \\
305	1.35446100315126 \\
306	1.32872023809524 \\
307	1.3859375 \\
308	1.3375 \\
309	1.25758928571429 \\
310	1.24375 \\
311	1.3375 \\
312	1.26398841594828 \\
313	1.25625 \\
314	1.509375 \\
315	1.59375 \\
316	1.91953125 \\
317	2.10420903110048 \\
318	1.984375 \\
319	1.96875 \\
320	2.015625 \\
321	1.8234375 \\
322	2.06587301587302 \\
323	2.11628289473684 \\
324	2.34375 \\
325	2.1875 \\
326	2.1198795995671 \\
327	2.23928571428571 \\
328	2.05308357007576 \\
329	2.15754310344828 \\
330	1.8891369047619 \\
331	2.0734375 \\
332	2.04840959821429 \\
333	2.23860294117647 \\
334	2.32449776785714 \\
335	2.31328125 \\
336	2.79401633522727 \\
337	3.07434875677245 \\
338	2.8578125 \\
339	2.9587890625 \\
340	3.12632511483614 \\
341	3.30909707731784 \\
342	3.30322485902256 \\
343	3.40122767857143 \\
344	3.39732142857143 \\
345	3.35625 \\
346	3.91492946708464 \\
347	3.90350274725275 \\
348	3.99776785714286 \\
349	4.17372159090909 \\
350	4.1796875 \\
351	4.12901785714286 \\
352	4.18839550395257 \\
353	4.40680803571429 \\
354	4.24526631773399 \\
355	4.20390139751553 \\
356	4.303125 \\
357	4.165625 \\
358	4.25104166666667 \\
359	4.25754310344828 \\
360	4.325 \\
361	4.171875 \\
362	4.171875 \\
363	4.61024305555556 \\
364	4.425 \\
365	4.40625 \\
366	4.23255208333333 \\
367	4.2375 \\
368	4.4967032967033 \\
369	4.315625 \\
370	4.44514802631579 \\
371	4.58432904411765 \\
372	4.21875 \\
373	4.38351370442171 \\
374	4.2375 \\
375	4.2578125 \\
376	4.372265625 \\
377	4.74012276785714 \\
378	4.359375 \\
379	4.52483912483913 \\
380	4.40140739889706 \\
381	4.62937614339994 \\
382	4.61082732371795 \\
383	4.83482142857143 \\
384	5.04095052083333 \\
385	4.64944534632035 \\
386	5.08080142513736 \\
387	5.33961038961039 \\
388	5.01337972689076 \\
389	5.71315104166667 \\
390	5.5565459280303 \\
391	5.47966213474026 \\
392	5.33215144230769 \\
393	5.53134057971014 \\
394	5.32157315340909 \\
395	5.2974609375 \\
396	5.51862922705314 \\
397	5.33452928046218 \\
398	5.72281588246712 \\
399	5.64000064270749 \\
400	5.57316176470588 \\
401	5.58177861225462 \\
402	5.58911037846313 \\
403	5.48768575174825 \\
404	5.59986669939715 \\
405	5.79025297619048 \\
406	5.4940452125999 \\
407	5.67731584821429 \\
408	5.93465166598249 \\
409	5.89913553414787 \\
410	5.88832070707071 \\
411	5.83784954322638 \\
412	5.81455067920585 \\
413	5.85682773109244 \\
414	5.81241243635994 \\
415	5.95521039126408 \\
416	5.89339815139473 \\
417	5.9557034223858 \\
418	6.15956739387331 \\
419	5.94056919642857 \\
420	6.07473269650419 \\
421	6.17544952671221 \\
422	6.46885674075744 \\
423	6.44475741506777 \\
424	6.6346748664063 \\
425	6.44870127178856 \\
426	6.46917613636364 \\
427	6.78125 \\
428	7.0039853828237 \\
429	6.90841346153846 \\
430	7.15015024038462 \\
431	6.84986979166667 \\
432	7.11841613247863 \\
433	7.14708153735632 \\
434	6.95178571428571 \\
435	7.12610294117647 \\
436	6.93180753373313 \\
437	7.19002840909091 \\
438	7.28543719211823 \\
439	7.15656081989247 \\
440	7.19191491578345 \\
441	7.21128849637681 \\
442	7.23295454545454 \\
443	7.1625 \\
444	7.198125 \\
445	7.11006944444444 \\
446	7.13203125 \\
447	7.22706512237762 \\
448	7.45145089285714 \\
449	7.43541666666667 \\
450	7.25492345205745 \\
451	7.3203125 \\
452	7.9714292478355 \\
453	7.77385664607132 \\
454	7.8129933557127 \\
455	7.7704107505071 \\
456	7.64849015478056 \\
457	7.84600535343353 \\
458	7.732086966893 \\
459	7.95756560727579 \\
460	8.16547012204623 \\
461	8.04846325549451 \\
462	8.18446800595238 \\
463	8.30809837092732 \\
464	8.12467948717949 \\
465	8.1734375 \\
466	8.29476258894362 \\
467	8.29494047619048 \\
468	8.38753292624521 \\
469	8.3768062232906 \\
470	8.2669134059874 \\
471	8.31317137200084 \\
472	8.22093906641604 \\
473	8.46417410714286 \\
474	8.40692616959064 \\
475	8.44050099206349 \\
476	8.40342261904762 \\
477	8.3796875 \\
478	8.93924851190476 \\
479	8.82970085470085 \\
480	8.63248570261438 \\
481	8.59040178571429 \\
482	9.13716517857143 \\
483	9.51593318668047 \\
484	9.3666015625 \\
485	9.28008740980475 \\
486	9.43379210884881 \\
487	9.32080933179724 \\
488	9.51014441287879 \\
489	9.51618700592886 \\
490	9.76163295830244 \\
491	9.75356045343805 \\
492	9.83401100852273 \\
493	9.78697222915973 \\
494	9.70957988664216 \\
495	9.69928751803752 \\
496	9.83881392045455 \\
497	9.87963286713287 \\
498	9.9548572954823 \\
499	9.94587862318841 \\
500	9.87333333333333 \\
501	9.86838102729072 \\
502	9.89447737068966 \\
503	9.96899801587302 \\
504	9.86572265625 \\
505	9.92267586504847 \\
506	10.1139873451901 \\
507	10.132579985119 \\
508	10.0843158577534 \\
509	10.2307852372408 \\
510	10.1171336206897 \\
511	10.1500437062937 \\
512	10.251033423813 \\
513	10.2448918269231 \\
514	10.4481584821429 \\
515	10.575338421659 \\
516	10.5875 \\
517	10.7503219436813 \\
518	11.0192395541287 \\
519	10.6846875 \\
520	10.7705555555556 \\
521	10.7379241071429 \\
522	10.8172611177885 \\
523	10.810702537594 \\
524	10.8341517857143 \\
525	10.9828082238643 \\
526	10.8100234885621 \\
527	10.8203562769397 \\
528	10.8726362179487 \\
529	10.8667279411765 \\
530	10.8325520833333 \\
531	10.9625534188034 \\
532	11.0012977065826 \\
533	11.2622316919192 \\
534	11.2421875 \\
535	11.23125 \\
536	11.4026353599516 \\
537	11.38671875 \\
538	11.4025 \\
539	11.4313368055556 \\
540	11.3990207489879 \\
541	11.3867312927544 \\
542	11.4455592105263 \\
543	11.409375 \\
544	11.4069831227572 \\
545	11.4140625 \\
546	11.4610576923077 \\
547	11.4 \\
548	11.384375 \\
549	11.4 \\
550	11.4272073412698 \\
551	11.4362664473684 \\
552	11.6367228317659 \\
553	11.730459057072 \\
554	11.6703629032258 \\
555	11.6867221025487 \\
556	11.7011911073207 \\
557	11.7232189849624 \\
558	11.9696707431381 \\
559	12.1429855872845 \\
560	12.1029761904762 \\
561	12.1183810763889 \\
562	11.9879464285714 \\
563	12.1740032327586 \\
564	12.3746170948617 \\
565	12.3273838141026 \\
566	12.15390625 \\
567	12.3553592545308 \\
568	12.3311083447802 \\
569	12.3912378246753 \\
570	12.3052734375 \\
571	12.2477790717736 \\
572	12.3110795454545 \\
573	12.2426716780462 \\
574	12.1628652283282 \\
575	12.2120303199405 \\
576	12.3169232536765 \\
577	12.2524639423077 \\
578	12.3573529411765 \\
579	12.3790386025265 \\
580	12.3149840669087 \\
581	12.2867587375859 \\
582	12.4460707720588 \\
583	12.4113636363636 \\
584	12.4380600649351 \\
585	12.434375 \\
586	12.2515065579582 \\
587	12.2364350912779 \\
588	12.406352124183 \\
589	12.1805555555556 \\
590	12.234375 \\
591	12.4251693349754 \\
592	12.4319196428571 \\
593	12.4002665303851 \\
594	12.3736213235294 \\
595	12.4162109375 \\
596	12.4260416666667 \\
597	12.384375 \\
598	12.4083333333333 \\
599	12.4132564484127 \\
};
\addplot [semithick, color3, dotted, forget plot]
table [row sep=\\]{%
0	0 \\
1	0 \\
2	0.0166666666666667 \\
3	0.020900974025974 \\
4	0 \\
5	0.00520833333333333 \\
6	0.009375 \\
7	0.0317708333333333 \\
8	0 \\
9	0.003125 \\
10	0.00520833333333333 \\
11	0.003125 \\
12	0 \\
13	0.003125 \\
14	0.009375 \\
15	0.025 \\
16	0.015625 \\
17	0.015625 \\
18	0.015625 \\
19	0.0174479166666667 \\
20	0 \\
21	0.00520833333333333 \\
22	0 \\
23	0 \\
24	0 \\
25	0 \\
26	0.00520833333333333 \\
27	0.0113636363636364 \\
28	0.0555555555555556 \\
29	0.00868055555555556 \\
30	0.0240234375 \\
31	0.0111607142857143 \\
32	0.0573563664596273 \\
33	0.028125 \\
34	0.0234375 \\
35	0.009375 \\
36	0.009375 \\
37	0.009375 \\
38	0.0078125 \\
39	0 \\
40	0 \\
41	0 \\
42	0 \\
43	0 \\
44	0 \\
45	0 \\
46	0 \\
47	0 \\
48	0 \\
49	0 \\
50	0.003125 \\
51	0 \\
52	0 \\
53	0 \\
54	0 \\
55	0.003125 \\
56	0 \\
57	0 \\
58	0.046875 \\
59	0.1875 \\
60	0 \\
61	0 \\
62	0 \\
63	0.00625 \\
64	0 \\
65	0 \\
66	0.0234375 \\
67	0 \\
68	0 \\
69	0.003125 \\
70	0 \\
71	0.009375 \\
72	0 \\
73	0 \\
74	0 \\
75	0.015625 \\
76	0.046875 \\
77	0.015625 \\
78	0.015625 \\
79	0.025 \\
80	0 \\
81	0 \\
82	0 \\
83	0 \\
84	0 \\
85	0 \\
86	0 \\
87	0 \\
88	0 \\
89	0 \\
90	0 \\
91	0 \\
92	0 \\
93	0 \\
94	0 \\
95	0 \\
96	0 \\
97	0 \\
98	0 \\
99	0 \\
100	0 \\
101	0 \\
102	0 \\
103	0 \\
104	0.009375 \\
105	0.009375 \\
106	0 \\
107	0 \\
108	0 \\
109	0 \\
110	0 \\
111	0 \\
112	0.046875 \\
113	0 \\
114	0.0390625 \\
115	0 \\
116	0 \\
117	0 \\
118	0 \\
119	0 \\
120	0 \\
121	0 \\
122	0 \\
123	0 \\
124	0 \\
125	0 \\
126	0 \\
127	0 \\
128	0 \\
129	0 \\
130	0 \\
131	0 \\
132	0 \\
133	0 \\
134	0 \\
135	0.046875 \\
136	0 \\
137	0 \\
138	0 \\
139	0 \\
140	0 \\
141	0 \\
142	0 \\
143	0 \\
144	0 \\
145	0 \\
146	0 \\
147	0 \\
148	0 \\
149	0.1171875 \\
150	0 \\
151	0.009375 \\
152	0 \\
153	0 \\
154	0 \\
155	0 \\
156	0.046875 \\
157	0 \\
158	0 \\
159	0 \\
160	0 \\
161	0 \\
162	0.09375 \\
163	0 \\
164	0 \\
165	0.046875 \\
166	0 \\
167	0 \\
168	0 \\
169	0 \\
170	0 \\
171	0 \\
172	0 \\
173	0 \\
174	0 \\
175	0 \\
176	0 \\
177	0 \\
178	0.046875 \\
179	0.078125 \\
180	0 \\
181	0 \\
182	0 \\
183	0 \\
184	0 \\
185	0 \\
186	0 \\
187	0 \\
188	0 \\
189	0 \\
190	0 \\
191	0 \\
192	0.046875 \\
193	0 \\
194	0 \\
195	0 \\
196	0.046875 \\
197	0 \\
198	0 \\
199	0 \\
200	0 \\
201	0.1171875 \\
202	0.1953125 \\
203	0.1171875 \\
204	0 \\
205	0 \\
206	0 \\
207	0 \\
208	0 \\
209	0.015625 \\
210	0 \\
211	0 \\
212	0.078125 \\
213	0.200892857142857 \\
214	0.046875 \\
215	0 \\
216	0 \\
217	0 \\
218	0.09375 \\
219	0 \\
220	0.046875 \\
221	0.140625 \\
222	0.046875 \\
223	0 \\
224	0 \\
225	0 \\
226	0.09375 \\
227	0.140625 \\
228	0.200892857142857 \\
229	0.495535714285714 \\
230	0.50390625 \\
231	0 \\
232	0.28125 \\
233	0 \\
234	0.046875 \\
235	0 \\
236	0.140625 \\
237	0.09375 \\
238	0.046875 \\
239	0.1640625 \\
240	0.30078125 \\
241	0.140625 \\
242	0.046875 \\
243	0.200892857142857 \\
244	0.046875 \\
245	0.328125 \\
246	0.328125 \\
247	0.046875 \\
248	0 \\
249	0.368303571428571 \\
250	0 \\
251	0.3125 \\
252	0.341517857142857 \\
253	0.28125 \\
254	0.09375 \\
255	0.140625 \\
256	0.2109375 \\
257	0.09375 \\
258	0.046875 \\
259	0.046875 \\
260	0 \\
261	0.09375 \\
262	0.234375 \\
263	0.203125 \\
264	0.214285714285714 \\
265	0.2109375 \\
266	0.234375 \\
267	0.28125 \\
268	0.0729166666666667 \\
269	0.046875 \\
270	0.200892857142857 \\
271	0.09375 \\
272	0.140625 \\
273	0.21875 \\
274	0.234375 \\
275	0.255208333333333 \\
276	0.234375 \\
277	0.234375 \\
278	0.140625 \\
279	0.1171875 \\
280	0.26953125 \\
281	0.2109375 \\
282	0.345703125 \\
283	0.234375 \\
284	0.2109375 \\
285	0.375 \\
286	0.181640625 \\
287	0.56640625 \\
288	0.6328125 \\
289	0.25 \\
290	0.171875 \\
291	0.334821428571429 \\
292	0.375 \\
293	0.630208333333333 \\
294	0.203125 \\
295	0.940104166666667 \\
296	0.328125 \\
297	0.874720982142857 \\
298	0.469866071428571 \\
299	0.862379807692308 \\
300	0.522321428571429 \\
301	0.505974264705882 \\
302	0.515625 \\
303	0.509486607142857 \\
304	0.808035714285714 \\
305	0.941824776785714 \\
306	1.03069196428571 \\
307	1.03292410714286 \\
308	0 \\
309	0.804408482142857 \\
310	0.955729166666667 \\
311	0.883864182692308 \\
312	1.13932291666667 \\
313	0.662286931818182 \\
314	1.55171130952381 \\
315	1.28185096153846 \\
316	1.40625 \\
317	0.981119791666667 \\
318	1.87577266483516 \\
319	1.0357082201087 \\
320	1.42914117132867 \\
321	1.18220413165266 \\
322	1.7421875 \\
323	0.9375 \\
324	1.20740327380952 \\
325	1.34341809506283 \\
326	1.84774210164835 \\
327	1.33872767857143 \\
328	1.708984375 \\
329	1.1376953125 \\
330	1.83421266233766 \\
331	2.55620092989119 \\
332	1.69921875 \\
333	1.80049377705628 \\
334	2.53774350649351 \\
335	3.0706058264652 \\
336	2.32142857142857 \\
337	2.37129667207792 \\
338	2.17699032738095 \\
339	3.59691220238095 \\
340	2.71932234432234 \\
341	3.43782298430736 \\
342	2.41717026745899 \\
343	2.81630214340326 \\
344	3.56019631410256 \\
345	3.79808407738095 \\
346	3.39593219280719 \\
347	3.80222393096034 \\
348	4.48474442224442 \\
349	3.93813631221719 \\
350	4.56033508706206 \\
351	5.45638884356166 \\
352	4.50705910421234 \\
353	4.83829463109354 \\
354	5.09827108839272 \\
355	4.97027815934066 \\
356	4.9220712125576 \\
357	5.0564946771978 \\
358	5.60968335050366 \\
359	5.25659572297787 \\
360	6.13833276098901 \\
361	5.79852134932921 \\
362	6.33615424795518 \\
363	6.16155133928571 \\
364	6.54509943181818 \\
365	6.69718276515151 \\
366	6.262832332975 \\
367	6.83277529761905 \\
368	6.20647321428571 \\
369	6.59101055194805 \\
370	7.04281655844156 \\
371	7.06662608225108 \\
372	6.83084952731092 \\
373	8.20616883116883 \\
374	7.60830543154762 \\
375	7.66462053571429 \\
376	7.40806027853725 \\
377	8.175737482493 \\
378	7.9419818576891 \\
379	8.48431061126374 \\
380	8.68137591575092 \\
381	8.47899502840909 \\
382	9.12159921713251 \\
383	9.07535082105395 \\
384	8.89109390150291 \\
385	8.42008634868421 \\
386	10.0505404139366 \\
387	9.42332589285714 \\
388	9.92065662584964 \\
389	9.82412795027685 \\
390	10.6105242932818 \\
391	10.2925323374542 \\
392	9.82809355886533 \\
393	10.9446856230737 \\
394	10.6482624390154 \\
395	11.7042668269231 \\
396	11.9020524842482 \\
397	11.2658133533134 \\
398	12.1458913482351 \\
399	11.6648082309847 \\
400	11.9953473545289 \\
401	11.8118724207455 \\
402	12.2147552967865 \\
403	11.9098792928563 \\
404	12.4121355857684 \\
405	12.3925874255952 \\
406	12.8206117895263 \\
407	12.7885230654762 \\
408	12.723103459041 \\
409	12.8647937660166 \\
410	12.4494041115135 \\
411	12.54035713643 \\
412	12.5041513004658 \\
413	13.509412202381 \\
414	12.8785294408288 \\
415	13.1916969379062 \\
416	13.531544089968 \\
417	13.3845359240659 \\
418	13.2461732278139 \\
419	13.1378976286871 \\
420	12.9518327067669 \\
421	13.4434266974921 \\
422	13.5511095063025 \\
423	13.251062071986 \\
424	13.2726777129121 \\
425	13.4498517796542 \\
426	13.1060711423993 \\
427	13.7013804385153 \\
428	14.1820304489918 \\
429	13.7563237543706 \\
430	13.6844625804687 \\
431	13.757864010989 \\
432	13.6768290133779 \\
433	13.9736283936652 \\
434	13.7650003552482 \\
435	14.1680986121573 \\
436	13.4313161928651 \\
437	14.1241112933614 \\
438	13.8229867788462 \\
439	13.556604011797 \\
440	13.7395651223776 \\
441	13.9940720737596 \\
442	13.8034539473684 \\
443	13.9288234827244 \\
444	13.6607142857143 \\
445	14.0277672847985 \\
446	13.9140020136114 \\
447	13.8430388621795 \\
448	14.0735426682692 \\
449	14.3872842279624 \\
450	14.1306061126374 \\
451	14.0415074989549 \\
452	14.0393114697802 \\
453	14.2457758729757 \\
454	14.3770196830757 \\
455	14.3961422768408 \\
456	14.345061763894 \\
457	14.5570620785465 \\
458	14.1990147260827 \\
459	14.526612819746 \\
460	14.516841231685 \\
461	14.7120356856685 \\
462	14.3937969924812 \\
463	14.4683213716108 \\
464	14.421586278172 \\
465	14.5829777988556 \\
466	14.5548037146516 \\
467	14.6565545474263 \\
468	14.7088355654762 \\
469	14.6173942904921 \\
470	14.758476268797 \\
471	14.7046574519231 \\
472	14.4921707029384 \\
473	14.7246056369617 \\
474	14.6705271291209 \\
475	14.2989031189399 \\
476	14.6489209191222 \\
477	14.7714246315755 \\
478	14.6108505609051 \\
479	14.7460020242915 \\
480	14.6420995670996 \\
481	14.7119791666667 \\
482	14.8676782852564 \\
483	14.629296875 \\
484	14.7017612390351 \\
485	14.894236865942 \\
486	14.8008819994344 \\
487	14.9126993282131 \\
488	14.8443337912088 \\
489	14.9110376602564 \\
490	14.7244491185897 \\
491	14.8947922558446 \\
492	14.6970289523279 \\
493	14.9543240399031 \\
494	14.9466217376374 \\
495	14.922393529507 \\
496	14.9351004464286 \\
497	14.9245669261294 \\
498	14.8660913100058 \\
499	14.9376373626374 \\
500	14.8775101031215 \\
501	14.9522797345318 \\
502	14.9487980769231 \\
503	14.9440237713675 \\
504	14.9168630678651 \\
505	14.9544853208862 \\
506	14.9159297733516 \\
507	14.865140122814 \\
508	14.8218449519231 \\
509	14.7495535714286 \\
510	14.9506363122172 \\
511	14.7612980769231 \\
512	14.9561883223684 \\
513	14.9262456293706 \\
514	14.9581730769231 \\
515	14.9489736070381 \\
516	14.945703125 \\
517	14.875921474359 \\
518	14.9513392857143 \\
519	14.9491586538462 \\
520	14.9331414473684 \\
521	14.959627016129 \\
522	14.9749826195773 \\
523	14.9671875 \\
524	14.9637931034483 \\
525	14.8202266483516 \\
526	15 \\
527	14.9594661803714 \\
528	14.9638349931319 \\
529	14.9523137019231 \\
530	14.9909855769231 \\
531	14.9461939102564 \\
532	14.9710597826087 \\
533	14.9564719680521 \\
534	14.9565340909091 \\
535	14.9304924242424 \\
536	14.9410906189399 \\
537	14.9812199519231 \\
538	14.9620793269231 \\
539	14.9601570304695 \\
540	14.9581730769231 \\
541	14.9816105769231 \\
542	14.9611036644807 \\
543	14.972352886057 \\
544	14.9701180875576 \\
545	14.9441964285714 \\
546	14.9553571428571 \\
547	14.9203125 \\
548	14.9696787587413 \\
549	14.9909855769231 \\
550	14.953125 \\
551	14.9798248626374 \\
552	14.959899068323 \\
553	14.9381446678322 \\
554	14.9569395242915 \\
555	15 \\
556	14.9236607142857 \\
557	14.9490989309603 \\
558	14.9425480769231 \\
559	14.94140625 \\
560	14.7735748626374 \\
561	14.9573001012146 \\
562	14.849609375 \\
563	14.9441105769231 \\
564	14.9506353021978 \\
565	14.9581730769231 \\
566	14.9502747252747 \\
567	14.9909855769231 \\
568	14.9765625 \\
569	14.980859375 \\
570	14.9816105769231 \\
571	14.9585336538462 \\
572	14.9803321678322 \\
573	14.9790736607143 \\
574	15 \\
575	14.990625 \\
576	15 \\
577	15 \\
578	14.9893465909091 \\
579	14.9825431034483 \\
580	14.9768382352941 \\
581	14.990625 \\
582	14.9795809659091 \\
583	15 \\
584	14.9700592376374 \\
585	14.9898097826087 \\
586	14.990625 \\
587	14.9909855769231 \\
588	14.9753289473684 \\
589	14.8901785714286 \\
590	14.98125 \\
591	15 \\
592	14.9834250930521 \\
593	14.9909855769231 \\
594	14.9909855769231 \\
595	14.9683035714286 \\
596	14.9600113122172 \\
597	14.9816105769231 \\
598	15 \\
599	15 \\
};
\nextgroupplot[
height=\figureheight,
tick align=outside,
tick pos=left,
title={Frozen Lake},
title style={font=\small, yshift=-1.5ex},
xlabel style={font=\scriptsize},
yticklabel style={font=\scriptsize},
xticklabel style={font=\scriptsize},
grid=both,
width=\figurewidth,
x grid style={white!69.01960784313725!black},
xmin=-29.95, xmax=628.95,
y grid style={white!69.01960784313725!black},
ymin=-0.0518577065595288, ymax=0.909409980509189
]
\path [fill=color0, fill opacity=0.3] (axis cs:0,-0.00204367757289895)
--(axis cs:0,0.0117658997951212)
--(axis cs:1,0.0165344023444498)
--(axis cs:2,0.0163277950301242)
--(axis cs:3,0.023340324229687)
--(axis cs:4,0.0381584491093549)
--(axis cs:5,0.0244960846204818)
--(axis cs:6,0.0179915449990367)
--(axis cs:7,0.0203724858440405)
--(axis cs:8,0.0254874665083166)
--(axis cs:9,0.0444426553395264)
--(axis cs:10,0.0474757263770509)
--(axis cs:11,0.044891053167075)
--(axis cs:12,0.0764097730778421)
--(axis cs:13,0.0803191876993313)
--(axis cs:14,0.101985177008258)
--(axis cs:15,0.0758186053155144)
--(axis cs:16,0.101694481378228)
--(axis cs:17,0.0838749303341177)
--(axis cs:18,0.11332459215839)
--(axis cs:19,0.0866283444357827)
--(axis cs:20,0.0816463604726814)
--(axis cs:21,0.056027243856223)
--(axis cs:22,0.081751500772377)
--(axis cs:23,0.0660946013646747)
--(axis cs:24,0.0860607910233702)
--(axis cs:25,0.0644528649969039)
--(axis cs:26,0.0730343080037283)
--(axis cs:27,0.0494827505469379)
--(axis cs:28,0.0874952939075677)
--(axis cs:29,0.0773203986121187)
--(axis cs:30,0.0836613449015999)
--(axis cs:31,0.0555609883797782)
--(axis cs:32,0.0739527717220133)
--(axis cs:33,0.0611221757784639)
--(axis cs:34,0.111734589107945)
--(axis cs:35,0.0908696493357729)
--(axis cs:36,0.0642849760977448)
--(axis cs:37,0.0535137189831827)
--(axis cs:38,0.0605928288376791)
--(axis cs:39,0.0959489752127421)
--(axis cs:40,0.0848576821384124)
--(axis cs:41,0.0812879710308176)
--(axis cs:42,0.109320406263312)
--(axis cs:43,0.0893281685178395)
--(axis cs:44,0.103328807533458)
--(axis cs:45,0.100017675624244)
--(axis cs:46,0.113794524512457)
--(axis cs:47,0.079848631616894)
--(axis cs:48,0.0688926436048793)
--(axis cs:49,0.105634807846741)
--(axis cs:50,0.0907312832821322)
--(axis cs:51,0.0509972822721018)
--(axis cs:52,0.0981358064656958)
--(axis cs:53,0.135515218762523)
--(axis cs:54,0.125340923232453)
--(axis cs:55,0.0994796038209319)
--(axis cs:56,0.117497525642796)
--(axis cs:57,0.12062869434965)
--(axis cs:58,0.0713302277247916)
--(axis cs:59,0.120515242682076)
--(axis cs:60,0.105820136806443)
--(axis cs:61,0.124918165937164)
--(axis cs:62,0.124388508582682)
--(axis cs:63,0.10836840220365)
--(axis cs:64,0.10660509428033)
--(axis cs:65,0.14658456820678)
--(axis cs:66,0.148643846119715)
--(axis cs:67,0.145799874656724)
--(axis cs:68,0.149982111642517)
--(axis cs:69,0.0627312375933519)
--(axis cs:70,0.159185073903176)
--(axis cs:71,0.172606045340871)
--(axis cs:72,0.103281584417257)
--(axis cs:73,0.151207700445186)
--(axis cs:74,0.162743482283637)
--(axis cs:75,0.16087866619699)
--(axis cs:76,0.168969781819715)
--(axis cs:77,0.168408155116213)
--(axis cs:78,0.207994801686757)
--(axis cs:79,0.188671934648749)
--(axis cs:80,0.210522981261239)
--(axis cs:81,0.226254003847877)
--(axis cs:82,0.185345184300247)
--(axis cs:83,0.175038007298551)
--(axis cs:84,0.20380755786953)
--(axis cs:85,0.171253672594289)
--(axis cs:86,0.208549614305592)
--(axis cs:87,0.22655136223233)
--(axis cs:88,0.203671821546728)
--(axis cs:89,0.286795321286953)
--(axis cs:90,0.28413382346682)
--(axis cs:91,0.223917116879688)
--(axis cs:92,0.321463987982251)
--(axis cs:93,0.295086988912824)
--(axis cs:94,0.273738870098192)
--(axis cs:95,0.231184893025456)
--(axis cs:96,0.298114696464294)
--(axis cs:97,0.292121760924282)
--(axis cs:98,0.392601315199719)
--(axis cs:99,0.350478522361755)
--(axis cs:100,0.329944263692262)
--(axis cs:101,0.380287829233426)
--(axis cs:102,0.33763651784552)
--(axis cs:103,0.325322217370241)
--(axis cs:104,0.348215680345744)
--(axis cs:105,0.318618161223164)
--(axis cs:106,0.340407421201956)
--(axis cs:107,0.367547621486816)
--(axis cs:108,0.414112435478443)
--(axis cs:109,0.393543062376245)
--(axis cs:110,0.445268649681621)
--(axis cs:111,0.420276354953608)
--(axis cs:112,0.408398756872188)
--(axis cs:113,0.443913640778142)
--(axis cs:114,0.426438124275997)
--(axis cs:115,0.458362106811315)
--(axis cs:116,0.412183589273279)
--(axis cs:117,0.432257822657908)
--(axis cs:118,0.467415766169173)
--(axis cs:119,0.44897050542176)
--(axis cs:120,0.480163888909332)
--(axis cs:121,0.443707132340006)
--(axis cs:122,0.512622926784563)
--(axis cs:123,0.51223592876784)
--(axis cs:124,0.50968591397312)
--(axis cs:125,0.49799315801781)
--(axis cs:126,0.521074678978135)
--(axis cs:127,0.498318640995199)
--(axis cs:128,0.471283117671835)
--(axis cs:129,0.518834834631784)
--(axis cs:130,0.578365509235217)
--(axis cs:131,0.524324919193419)
--(axis cs:132,0.537793980453249)
--(axis cs:133,0.517122459501709)
--(axis cs:134,0.523956810770211)
--(axis cs:135,0.582244237527837)
--(axis cs:136,0.512135897063353)
--(axis cs:137,0.612266737716199)
--(axis cs:138,0.579608048587216)
--(axis cs:139,0.649352706579652)
--(axis cs:140,0.604193151031242)
--(axis cs:141,0.614397586186541)
--(axis cs:142,0.630851180238841)
--(axis cs:143,0.605716620169719)
--(axis cs:144,0.632150210060859)
--(axis cs:145,0.639119624051483)
--(axis cs:146,0.644437260788268)
--(axis cs:147,0.641345275271552)
--(axis cs:148,0.633950840280932)
--(axis cs:149,0.671741190705441)
--(axis cs:150,0.649180709150995)
--(axis cs:151,0.634557732522475)
--(axis cs:152,0.624436575612263)
--(axis cs:153,0.640760928772559)
--(axis cs:154,0.667453416981696)
--(axis cs:155,0.631522470275546)
--(axis cs:156,0.679837649576249)
--(axis cs:157,0.678500046928444)
--(axis cs:158,0.634032424907884)
--(axis cs:159,0.701814636101949)
--(axis cs:160,0.675664359458349)
--(axis cs:161,0.627942849644275)
--(axis cs:162,0.649635399354198)
--(axis cs:163,0.698211240255738)
--(axis cs:164,0.673288183379332)
--(axis cs:165,0.684158007371664)
--(axis cs:166,0.747455069618185)
--(axis cs:167,0.688122291469744)
--(axis cs:168,0.716275716981493)
--(axis cs:169,0.71433817818022)
--(axis cs:170,0.717486606046807)
--(axis cs:171,0.673130307251636)
--(axis cs:172,0.718535035185065)
--(axis cs:173,0.721424010766024)
--(axis cs:174,0.717358108522411)
--(axis cs:175,0.682267179363788)
--(axis cs:176,0.741875557531878)
--(axis cs:177,0.74450434956547)
--(axis cs:178,0.761684010388142)
--(axis cs:179,0.731048999309099)
--(axis cs:180,0.745685995802265)
--(axis cs:181,0.749434602989535)
--(axis cs:182,0.762675558180003)
--(axis cs:183,0.736386438773564)
--(axis cs:184,0.728282548769837)
--(axis cs:185,0.786182330540201)
--(axis cs:186,0.73722620726403)
--(axis cs:187,0.757081906202749)
--(axis cs:188,0.755541129789131)
--(axis cs:189,0.757822792404309)
--(axis cs:190,0.753195723866786)
--(axis cs:191,0.738243452363913)
--(axis cs:192,0.732361732148598)
--(axis cs:193,0.730009188563139)
--(axis cs:194,0.75156142917177)
--(axis cs:195,0.76867591450138)
--(axis cs:196,0.769226337014162)
--(axis cs:197,0.746429280020758)
--(axis cs:198,0.770162388447194)
--(axis cs:199,0.745759462445297)
--(axis cs:200,0.787407069721002)
--(axis cs:201,0.751598521118559)
--(axis cs:202,0.776892023262691)
--(axis cs:203,0.72255221384873)
--(axis cs:204,0.762054809082328)
--(axis cs:205,0.7553890143685)
--(axis cs:206,0.779966130027122)
--(axis cs:207,0.766771340550775)
--(axis cs:208,0.78015640950672)
--(axis cs:209,0.770420344880366)
--(axis cs:210,0.757702210615009)
--(axis cs:211,0.75885285546546)
--(axis cs:212,0.760075023932477)
--(axis cs:213,0.749125437185424)
--(axis cs:214,0.763066333478888)
--(axis cs:215,0.743789089346922)
--(axis cs:216,0.771051286896196)
--(axis cs:217,0.781644518690513)
--(axis cs:218,0.79126595695683)
--(axis cs:219,0.808308820883671)
--(axis cs:220,0.812503853315609)
--(axis cs:221,0.778310495711738)
--(axis cs:222,0.778616432660408)
--(axis cs:223,0.803982846296547)
--(axis cs:224,0.720115894808392)
--(axis cs:225,0.744326027091033)
--(axis cs:226,0.787726748118081)
--(axis cs:227,0.780922049300711)
--(axis cs:228,0.776035863480208)
--(axis cs:229,0.765221720206423)
--(axis cs:230,0.788442091884939)
--(axis cs:231,0.76001138821318)
--(axis cs:232,0.778851810125674)
--(axis cs:233,0.783362195427918)
--(axis cs:234,0.793553979281114)
--(axis cs:235,0.780789082110004)
--(axis cs:236,0.75471723834748)
--(axis cs:237,0.798355720424518)
--(axis cs:238,0.776803269625401)
--(axis cs:239,0.761065932553262)
--(axis cs:240,0.754016244268002)
--(axis cs:241,0.763065933396418)
--(axis cs:242,0.795868617721263)
--(axis cs:243,0.759127517367109)
--(axis cs:244,0.802148702650714)
--(axis cs:245,0.8016861719865)
--(axis cs:246,0.783254684168226)
--(axis cs:247,0.777572350106003)
--(axis cs:248,0.777359132122686)
--(axis cs:249,0.773319536436241)
--(axis cs:250,0.810620729149686)
--(axis cs:251,0.81760563372404)
--(axis cs:252,0.809866639320897)
--(axis cs:253,0.788055469127466)
--(axis cs:254,0.831476439995034)
--(axis cs:255,0.774496393061393)
--(axis cs:256,0.797208911911454)
--(axis cs:257,0.803442210546262)
--(axis cs:258,0.79330859521119)
--(axis cs:259,0.785261211715962)
--(axis cs:260,0.830253554758823)
--(axis cs:261,0.791069940219274)
--(axis cs:262,0.822193154801019)
--(axis cs:263,0.819046842665235)
--(axis cs:264,0.786751129211809)
--(axis cs:265,0.798692058329281)
--(axis cs:266,0.793388935214798)
--(axis cs:267,0.811409966874434)
--(axis cs:268,0.795520067919323)
--(axis cs:269,0.783643092242567)
--(axis cs:270,0.816333319783076)
--(axis cs:271,0.792383755652755)
--(axis cs:272,0.823153111911283)
--(axis cs:273,0.767732135375597)
--(axis cs:274,0.787933805989921)
--(axis cs:275,0.77234843329011)
--(axis cs:276,0.80424633431452)
--(axis cs:277,0.755921614491501)
--(axis cs:278,0.778475116493247)
--(axis cs:279,0.781976281886047)
--(axis cs:280,0.791408178456275)
--(axis cs:281,0.777932753858509)
--(axis cs:282,0.751937957865671)
--(axis cs:283,0.773862572618018)
--(axis cs:284,0.800557703115418)
--(axis cs:285,0.81231785776738)
--(axis cs:286,0.790302779447989)
--(axis cs:287,0.800439019562531)
--(axis cs:288,0.777026155638864)
--(axis cs:289,0.804485762442977)
--(axis cs:290,0.781279911281497)
--(axis cs:291,0.806166659678494)
--(axis cs:292,0.768212989827872)
--(axis cs:293,0.773106415759672)
--(axis cs:294,0.806663813406349)
--(axis cs:295,0.809785121014745)
--(axis cs:296,0.786874488996793)
--(axis cs:297,0.772872453678726)
--(axis cs:298,0.787985654264206)
--(axis cs:299,0.789775043669894)
--(axis cs:300,0.780909294516779)
--(axis cs:301,0.823756145476674)
--(axis cs:302,0.794412858950953)
--(axis cs:303,0.797131538336396)
--(axis cs:304,0.807810451060082)
--(axis cs:305,0.761969800085972)
--(axis cs:306,0.76578815855114)
--(axis cs:307,0.781788611388)
--(axis cs:308,0.799796435723176)
--(axis cs:309,0.791271407061595)
--(axis cs:310,0.7537079711005)
--(axis cs:311,0.787453439539666)
--(axis cs:312,0.783629667186767)
--(axis cs:313,0.793146277491093)
--(axis cs:314,0.825120472875505)
--(axis cs:315,0.802410388980749)
--(axis cs:316,0.826648997042636)
--(axis cs:317,0.803699135354119)
--(axis cs:318,0.81726049772505)
--(axis cs:319,0.798005426577398)
--(axis cs:320,0.783840725050433)
--(axis cs:321,0.817098966233064)
--(axis cs:322,0.803031842371718)
--(axis cs:323,0.766855767496531)
--(axis cs:324,0.768688571594406)
--(axis cs:325,0.782514326169545)
--(axis cs:326,0.772222550220051)
--(axis cs:327,0.7824074517293)
--(axis cs:328,0.743413262198044)
--(axis cs:329,0.794143981069877)
--(axis cs:330,0.772947363987588)
--(axis cs:331,0.797570317002506)
--(axis cs:332,0.796851768851717)
--(axis cs:333,0.805819970380978)
--(axis cs:334,0.786037501958447)
--(axis cs:335,0.792770765255183)
--(axis cs:336,0.794158676843233)
--(axis cs:337,0.773937078088183)
--(axis cs:338,0.842904259663665)
--(axis cs:339,0.806495087679084)
--(axis cs:340,0.823173279791118)
--(axis cs:341,0.812321330951598)
--(axis cs:342,0.819954682423967)
--(axis cs:343,0.777976055580538)
--(axis cs:344,0.79340348785663)
--(axis cs:345,0.825442222718477)
--(axis cs:346,0.794532733252324)
--(axis cs:347,0.791693855341771)
--(axis cs:348,0.8008833895109)
--(axis cs:349,0.803142056842947)
--(axis cs:350,0.817340551823662)
--(axis cs:351,0.800055336361106)
--(axis cs:352,0.809542455862504)
--(axis cs:353,0.807638123747808)
--(axis cs:354,0.800657866401113)
--(axis cs:355,0.785790559180301)
--(axis cs:356,0.79957859496406)
--(axis cs:357,0.805956749996219)
--(axis cs:358,0.815966834404311)
--(axis cs:359,0.781419079335391)
--(axis cs:360,0.841401458259318)
--(axis cs:361,0.806087843063364)
--(axis cs:362,0.814479955777769)
--(axis cs:363,0.817200068750304)
--(axis cs:364,0.809385912410183)
--(axis cs:365,0.817587183362891)
--(axis cs:366,0.805291962426812)
--(axis cs:367,0.809815216394976)
--(axis cs:368,0.80588041841954)
--(axis cs:369,0.810791060354968)
--(axis cs:370,0.784163709894276)
--(axis cs:371,0.826357272251834)
--(axis cs:372,0.821698991798287)
--(axis cs:373,0.818287688167154)
--(axis cs:374,0.799045609911943)
--(axis cs:375,0.792828977957445)
--(axis cs:376,0.817593302324768)
--(axis cs:377,0.786757948133673)
--(axis cs:378,0.794616860477034)
--(axis cs:379,0.823511630144)
--(axis cs:380,0.826672922685338)
--(axis cs:381,0.839101309510038)
--(axis cs:382,0.862485196639844)
--(axis cs:383,0.80239968032886)
--(axis cs:384,0.823046144297947)
--(axis cs:385,0.799154281081264)
--(axis cs:386,0.806362059593001)
--(axis cs:387,0.812308988048454)
--(axis cs:388,0.797971102329471)
--(axis cs:389,0.810973383171792)
--(axis cs:390,0.796661516551718)
--(axis cs:391,0.819214697663784)
--(axis cs:392,0.797916012043862)
--(axis cs:393,0.813864559142601)
--(axis cs:394,0.807033643472324)
--(axis cs:395,0.787849768722421)
--(axis cs:396,0.772551617527138)
--(axis cs:397,0.780825358990167)
--(axis cs:398,0.766079759055902)
--(axis cs:399,0.789418443308481)
--(axis cs:400,0.810136146213351)
--(axis cs:401,0.823208693226791)
--(axis cs:402,0.784490322931199)
--(axis cs:403,0.797939910444914)
--(axis cs:404,0.788649188072142)
--(axis cs:405,0.816917823815123)
--(axis cs:406,0.828053600052723)
--(axis cs:407,0.774504060674973)
--(axis cs:408,0.824335890130876)
--(axis cs:409,0.822745396471988)
--(axis cs:410,0.824110326604047)
--(axis cs:411,0.813338344764641)
--(axis cs:412,0.825602752623332)
--(axis cs:413,0.812372248325601)
--(axis cs:414,0.824835207275368)
--(axis cs:415,0.846150122238534)
--(axis cs:416,0.812766133212158)
--(axis cs:417,0.83823870868202)
--(axis cs:418,0.823385413438237)
--(axis cs:419,0.809431842754849)
--(axis cs:420,0.837349524540577)
--(axis cs:421,0.816205775713108)
--(axis cs:422,0.807791991660772)
--(axis cs:423,0.836897931558894)
--(axis cs:424,0.857006102557034)
--(axis cs:425,0.839788714869084)
--(axis cs:426,0.791915592204222)
--(axis cs:427,0.825861370175043)
--(axis cs:428,0.814417862500491)
--(axis cs:429,0.807176583683908)
--(axis cs:430,0.817655605897888)
--(axis cs:431,0.7742054716087)
--(axis cs:432,0.812259212755983)
--(axis cs:433,0.785141393307805)
--(axis cs:434,0.811602906303903)
--(axis cs:435,0.787705694751809)
--(axis cs:436,0.813091041509059)
--(axis cs:437,0.787628324628892)
--(axis cs:438,0.774973493644587)
--(axis cs:439,0.795863288794595)
--(axis cs:440,0.833171882160346)
--(axis cs:441,0.748002435027668)
--(axis cs:442,0.773726354769667)
--(axis cs:443,0.765775274326007)
--(axis cs:444,0.805785701837739)
--(axis cs:445,0.816039912724557)
--(axis cs:446,0.77908057364906)
--(axis cs:447,0.797236456289678)
--(axis cs:448,0.827430235227199)
--(axis cs:449,0.793931116417274)
--(axis cs:450,0.814864820461148)
--(axis cs:451,0.841586938374343)
--(axis cs:452,0.799633511613422)
--(axis cs:453,0.802564962846537)
--(axis cs:454,0.831516964429859)
--(axis cs:455,0.814839993767557)
--(axis cs:456,0.815890030227163)
--(axis cs:457,0.85652677328143)
--(axis cs:458,0.811080744728514)
--(axis cs:459,0.812419709603587)
--(axis cs:460,0.813381673786077)
--(axis cs:461,0.7877980036318)
--(axis cs:462,0.798239671925105)
--(axis cs:463,0.843558953398895)
--(axis cs:464,0.816558546996249)
--(axis cs:465,0.805484876219849)
--(axis cs:466,0.79926021003286)
--(axis cs:467,0.809937071083679)
--(axis cs:468,0.79708707400092)
--(axis cs:469,0.794930116012429)
--(axis cs:470,0.765421209363933)
--(axis cs:471,0.803827216859579)
--(axis cs:472,0.80376299691113)
--(axis cs:473,0.810804219978)
--(axis cs:474,0.778112702997723)
--(axis cs:475,0.838281065837618)
--(axis cs:476,0.772287849306945)
--(axis cs:477,0.805353010230175)
--(axis cs:478,0.822559353187737)
--(axis cs:479,0.819756422903662)
--(axis cs:480,0.849464049616781)
--(axis cs:481,0.809283731201502)
--(axis cs:482,0.827657727689124)
--(axis cs:483,0.807738696096474)
--(axis cs:484,0.833544041576071)
--(axis cs:485,0.79650970239308)
--(axis cs:486,0.813973810381143)
--(axis cs:487,0.795333271674586)
--(axis cs:488,0.848381009139493)
--(axis cs:489,0.805169450267186)
--(axis cs:490,0.817986266082605)
--(axis cs:491,0.793055793443421)
--(axis cs:492,0.810445123117893)
--(axis cs:493,0.820163925302544)
--(axis cs:494,0.852234462597711)
--(axis cs:495,0.816483252832769)
--(axis cs:496,0.792911972819814)
--(axis cs:497,0.84897139488277)
--(axis cs:498,0.832398489363923)
--(axis cs:499,0.835246778939361)
--(axis cs:500,0.817060831512291)
--(axis cs:501,0.846789036523322)
--(axis cs:502,0.836113781703332)
--(axis cs:503,0.840755406490029)
--(axis cs:504,0.821616449846461)
--(axis cs:505,0.78090355751658)
--(axis cs:506,0.831200932572917)
--(axis cs:507,0.804460511042023)
--(axis cs:508,0.800897491248601)
--(axis cs:509,0.785141171285657)
--(axis cs:510,0.792087768340642)
--(axis cs:511,0.80716770525207)
--(axis cs:512,0.816274238024733)
--(axis cs:513,0.821786614526546)
--(axis cs:514,0.835998292288475)
--(axis cs:515,0.793197158074768)
--(axis cs:516,0.832609475844691)
--(axis cs:517,0.81970002450785)
--(axis cs:518,0.804805755653462)
--(axis cs:519,0.795227507296374)
--(axis cs:520,0.819177010922306)
--(axis cs:521,0.833522380692451)
--(axis cs:522,0.788794587254514)
--(axis cs:523,0.828047804210221)
--(axis cs:524,0.811415274158603)
--(axis cs:525,0.81604446014343)
--(axis cs:526,0.825700201001131)
--(axis cs:527,0.778503851453159)
--(axis cs:528,0.796980475906514)
--(axis cs:529,0.793227433599579)
--(axis cs:530,0.783041029321115)
--(axis cs:531,0.80459494803621)
--(axis cs:532,0.806456624096433)
--(axis cs:533,0.809307584959877)
--(axis cs:534,0.81870616514579)
--(axis cs:535,0.819256681154078)
--(axis cs:536,0.82541870820321)
--(axis cs:537,0.820331598296804)
--(axis cs:538,0.823980283490877)
--(axis cs:539,0.847613374986406)
--(axis cs:540,0.817053607529946)
--(axis cs:541,0.839259825241018)
--(axis cs:542,0.865715994733338)
--(axis cs:543,0.838432031605911)
--(axis cs:544,0.786071058654247)
--(axis cs:545,0.835104457222026)
--(axis cs:546,0.837092808483811)
--(axis cs:547,0.791348792437604)
--(axis cs:548,0.797602203023919)
--(axis cs:549,0.824600146229029)
--(axis cs:550,0.803327033910313)
--(axis cs:551,0.83562466460345)
--(axis cs:552,0.81729650482839)
--(axis cs:553,0.814487276574091)
--(axis cs:554,0.79286781082789)
--(axis cs:555,0.792214644518366)
--(axis cs:556,0.794994386044992)
--(axis cs:557,0.816228091787093)
--(axis cs:558,0.817681254126085)
--(axis cs:559,0.834768806773669)
--(axis cs:560,0.78352896114391)
--(axis cs:561,0.796929489191881)
--(axis cs:562,0.820436404943409)
--(axis cs:563,0.814114714100515)
--(axis cs:564,0.818908523809177)
--(axis cs:565,0.805408156347311)
--(axis cs:566,0.816018490204523)
--(axis cs:567,0.797748830849205)
--(axis cs:568,0.813204838826091)
--(axis cs:569,0.790053327973984)
--(axis cs:570,0.820214255663831)
--(axis cs:571,0.812826604375458)
--(axis cs:572,0.825807214498239)
--(axis cs:573,0.756218874627347)
--(axis cs:574,0.818255659200642)
--(axis cs:575,0.821596685437282)
--(axis cs:576,0.78556908182859)
--(axis cs:577,0.803312522154636)
--(axis cs:578,0.845331657853621)
--(axis cs:579,0.823157574257443)
--(axis cs:580,0.820298822754437)
--(axis cs:581,0.8174244042734)
--(axis cs:582,0.821459842582565)
--(axis cs:583,0.829456302061392)
--(axis cs:584,0.837236123689455)
--(axis cs:585,0.817079304936731)
--(axis cs:586,0.826789370441816)
--(axis cs:587,0.790781606163787)
--(axis cs:588,0.79892471644266)
--(axis cs:589,0.777355567333996)
--(axis cs:590,0.807588374558097)
--(axis cs:591,0.793622447906024)
--(axis cs:592,0.809005096385429)
--(axis cs:593,0.800741655212164)
--(axis cs:594,0.814103785637689)
--(axis cs:595,0.822072800672717)
--(axis cs:596,0.785304449854246)
--(axis cs:597,0.813420720163708)
--(axis cs:598,0.820653303501281)
--(axis cs:599,0.823988423159427)
--(axis cs:599,0.740586585165581)
--(axis cs:599,0.740586585165581)
--(axis cs:598,0.750552140360413)
--(axis cs:597,0.727653570129333)
--(axis cs:596,0.689962730881685)
--(axis cs:595,0.749000687726138)
--(axis cs:594,0.738765567231664)
--(axis cs:593,0.709288765755757)
--(axis cs:592,0.71800409234251)
--(axis cs:591,0.703084404269578)
--(axis cs:590,0.708951465809869)
--(axis cs:589,0.685879981839053)
--(axis cs:588,0.724373513105569)
--(axis cs:587,0.691710467734536)
--(axis cs:586,0.751025844367883)
--(axis cs:585,0.733464248731822)
--(axis cs:584,0.745694920960339)
--(axis cs:583,0.737412628640259)
--(axis cs:582,0.73852895334373)
--(axis cs:581,0.727546163353417)
--(axis cs:580,0.735950636528111)
--(axis cs:579,0.742245797498724)
--(axis cs:578,0.763957707295119)
--(axis cs:577,0.737288456727593)
--(axis cs:576,0.704112243790235)
--(axis cs:575,0.740295588954992)
--(axis cs:574,0.71618429320806)
--(axis cs:573,0.649343189942071)
--(axis cs:572,0.747928438554848)
--(axis cs:571,0.722035408611555)
--(axis cs:570,0.734917461577261)
--(axis cs:569,0.704400352026571)
--(axis cs:568,0.749182513405011)
--(axis cs:567,0.731153083208959)
--(axis cs:566,0.729747202436169)
--(axis cs:565,0.721214786681882)
--(axis cs:564,0.737050621212469)
--(axis cs:563,0.734850903615103)
--(axis cs:562,0.734304557219427)
--(axis cs:561,0.699746751234359)
--(axis cs:560,0.698539491237043)
--(axis cs:559,0.738545899353537)
--(axis cs:558,0.71122746736076)
--(axis cs:557,0.736161879352878)
--(axis cs:556,0.711776421782802)
--(axis cs:555,0.681826852179381)
--(axis cs:554,0.712042383144803)
--(axis cs:553,0.726987210948191)
--(axis cs:552,0.721736424048289)
--(axis cs:551,0.747771626605341)
--(axis cs:550,0.734641594683315)
--(axis cs:549,0.741571139082881)
--(axis cs:548,0.69626595139111)
--(axis cs:547,0.701825351127164)
--(axis cs:546,0.754058835011582)
--(axis cs:545,0.759619395314326)
--(axis cs:544,0.687324071542353)
--(axis cs:543,0.751601589287083)
--(axis cs:542,0.783425662220819)
--(axis cs:541,0.750861511755319)
--(axis cs:540,0.745018435010846)
--(axis cs:539,0.7793681070732)
--(axis cs:538,0.731863785634442)
--(axis cs:537,0.730555578762248)
--(axis cs:536,0.755638949416948)
--(axis cs:535,0.742811846723272)
--(axis cs:534,0.736337402397777)
--(axis cs:533,0.724775282099963)
--(axis cs:532,0.731024818072509)
--(axis cs:531,0.734055175839098)
--(axis cs:530,0.691339971622386)
--(axis cs:529,0.722457012162992)
--(axis cs:528,0.699719005167967)
--(axis cs:527,0.6969565822104)
--(axis cs:526,0.750066750593026)
--(axis cs:525,0.743233718275373)
--(axis cs:524,0.710286895199815)
--(axis cs:523,0.755244944707528)
--(axis cs:522,0.696834280405604)
--(axis cs:521,0.759721007863438)
--(axis cs:520,0.75042639935206)
--(axis cs:519,0.710265523977907)
--(axis cs:518,0.730017450654119)
--(axis cs:517,0.744060798513083)
--(axis cs:516,0.754139972623507)
--(axis cs:515,0.698020941716522)
--(axis cs:514,0.781803220434912)
--(axis cs:513,0.734643934091502)
--(axis cs:512,0.731129002117476)
--(axis cs:511,0.721902963662349)
--(axis cs:510,0.712024785771912)
--(axis cs:509,0.685017393650868)
--(axis cs:508,0.711019541371556)
--(axis cs:507,0.71103094194943)
--(axis cs:506,0.743407011566277)
--(axis cs:505,0.710393253311481)
--(axis cs:504,0.740399962560577)
--(axis cs:503,0.759812862046989)
--(axis cs:502,0.750371087594037)
--(axis cs:501,0.766172784641624)
--(axis cs:500,0.731104371340412)
--(axis cs:499,0.768706906264324)
--(axis cs:498,0.751096574599891)
--(axis cs:497,0.768920348790223)
--(axis cs:496,0.722200518855178)
--(axis cs:495,0.730719370932355)
--(axis cs:494,0.79136372842695)
--(axis cs:493,0.728612948864955)
--(axis cs:492,0.716472056924287)
--(axis cs:491,0.685472579225576)
--(axis cs:490,0.735669585826923)
--(axis cs:489,0.715183807586072)
--(axis cs:488,0.773433981425497)
--(axis cs:487,0.710071081937488)
--(axis cs:486,0.72174715268357)
--(axis cs:485,0.710868943266815)
--(axis cs:484,0.738275233919675)
--(axis cs:483,0.720092934078906)
--(axis cs:482,0.750753184315538)
--(axis cs:481,0.738623090326643)
--(axis cs:480,0.773964865218384)
--(axis cs:479,0.734421387133523)
--(axis cs:478,0.730253285562402)
--(axis cs:477,0.730149508084843)
--(axis cs:476,0.675492408629563)
--(axis cs:475,0.758203959553656)
--(axis cs:474,0.677546828443058)
--(axis cs:473,0.724086583662803)
--(axis cs:472,0.723681867252484)
--(axis cs:471,0.711252755251642)
--(axis cs:470,0.679731558834501)
--(axis cs:469,0.700145799697677)
--(axis cs:468,0.701123482803386)
--(axis cs:467,0.724307937380079)
--(axis cs:466,0.707314465291816)
--(axis cs:465,0.71232988153057)
--(axis cs:464,0.7276896527207)
--(axis cs:463,0.760535346398529)
--(axis cs:462,0.71078949335406)
--(axis cs:461,0.687836864971819)
--(axis cs:460,0.730765455126676)
--(axis cs:459,0.725748372314495)
--(axis cs:458,0.726275128564859)
--(axis cs:457,0.789024030081873)
--(axis cs:456,0.727054134982627)
--(axis cs:455,0.75213851383595)
--(axis cs:454,0.759215566927672)
--(axis cs:453,0.733097378284554)
--(axis cs:452,0.700471351864603)
--(axis cs:451,0.75799169479804)
--(axis cs:450,0.740088669166054)
--(axis cs:449,0.709128844455187)
--(axis cs:448,0.75209605445534)
--(axis cs:447,0.722666470644499)
--(axis cs:446,0.70382705691482)
--(axis cs:445,0.725491385837992)
--(axis cs:444,0.726629972765435)
--(axis cs:443,0.680446039082806)
--(axis cs:442,0.681467108833723)
--(axis cs:441,0.651883009389026)
--(axis cs:440,0.755416352521638)
--(axis cs:439,0.72361992407097)
--(axis cs:438,0.678477005509037)
--(axis cs:437,0.699686877530059)
--(axis cs:436,0.716651871827604)
--(axis cs:435,0.694064921784432)
--(axis cs:434,0.727716975203478)
--(axis cs:433,0.686549415883004)
--(axis cs:432,0.734224598571578)
--(axis cs:431,0.684325147078169)
--(axis cs:430,0.727926320399499)
--(axis cs:429,0.716877357166907)
--(axis cs:428,0.725851243393615)
--(axis cs:427,0.737435749372076)
--(axis cs:426,0.70421614064626)
--(axis cs:425,0.763487522782153)
--(axis cs:424,0.769032875825694)
--(axis cs:423,0.763423361731149)
--(axis cs:422,0.726607116175836)
--(axis cs:421,0.729467431288224)
--(axis cs:420,0.765948744796295)
--(axis cs:419,0.734972213666671)
--(axis cs:418,0.760080859871786)
--(axis cs:417,0.761334851438415)
--(axis cs:416,0.727847419901395)
--(axis cs:415,0.759572367077705)
--(axis cs:414,0.751195229301943)
--(axis cs:413,0.738507897661452)
--(axis cs:412,0.737113437436608)
--(axis cs:411,0.714784920858625)
--(axis cs:410,0.751207454573109)
--(axis cs:409,0.741471029463188)
--(axis cs:408,0.750825927218441)
--(axis cs:407,0.676798196096034)
--(axis cs:406,0.739171709360086)
--(axis cs:405,0.736338902097853)
--(axis cs:404,0.706111216453887)
--(axis cs:403,0.710319104845352)
--(axis cs:402,0.701248504213878)
--(axis cs:401,0.756885084523236)
--(axis cs:400,0.72313882534287)
--(axis cs:399,0.687051711332629)
--(axis cs:398,0.681462542392649)
--(axis cs:397,0.709402382018824)
--(axis cs:396,0.684387840584194)
--(axis cs:395,0.70499513147542)
--(axis cs:394,0.705593919936489)
--(axis cs:393,0.727482808767266)
--(axis cs:392,0.723051128454529)
--(axis cs:391,0.721761556984345)
--(axis cs:390,0.716162508381682)
--(axis cs:389,0.718050110695452)
--(axis cs:388,0.707799732738239)
--(axis cs:387,0.730373936665721)
--(axis cs:386,0.717395779789838)
--(axis cs:385,0.715171323869341)
--(axis cs:384,0.744620087040159)
--(axis cs:383,0.713321057266877)
--(axis cs:382,0.792730421075772)
--(axis cs:381,0.774280621059392)
--(axis cs:380,0.730281667863003)
--(axis cs:379,0.742226971532101)
--(axis cs:378,0.698049847814674)
--(axis cs:377,0.696932493369268)
--(axis cs:376,0.733050368240777)
--(axis cs:375,0.708287180189963)
--(axis cs:374,0.712566409604488)
--(axis cs:373,0.72723137610191)
--(axis cs:372,0.746695027376982)
--(axis cs:371,0.754313272481211)
--(axis cs:370,0.679446291215724)
--(axis cs:369,0.735022674521267)
--(axis cs:368,0.725759989783368)
--(axis cs:367,0.716416581321196)
--(axis cs:366,0.72246929541257)
--(axis cs:365,0.728665088670631)
--(axis cs:364,0.732163640467494)
--(axis cs:363,0.726353556443946)
--(axis cs:362,0.727791054461991)
--(axis cs:361,0.724794180856343)
--(axis cs:360,0.746195437287945)
--(axis cs:359,0.685767206600895)
--(axis cs:358,0.736228062712462)
--(axis cs:357,0.725246850325028)
--(axis cs:356,0.725143557883092)
--(axis cs:355,0.711597469457727)
--(axis cs:354,0.717443555767404)
--(axis cs:353,0.719825655324427)
--(axis cs:352,0.740553414362042)
--(axis cs:351,0.725672831508606)
--(axis cs:350,0.739186254703144)
--(axis cs:349,0.72362561992473)
--(axis cs:348,0.717903205838195)
--(axis cs:347,0.712374645601729)
--(axis cs:346,0.710006304099214)
--(axis cs:345,0.738089558000804)
--(axis cs:344,0.709031577078434)
--(axis cs:343,0.688600857090124)
--(axis cs:342,0.747463351951787)
--(axis cs:341,0.735321347550456)
--(axis cs:340,0.750303156966569)
--(axis cs:339,0.732591173282176)
--(axis cs:338,0.766625056506276)
--(axis cs:337,0.682007281371802)
--(axis cs:336,0.700594269862838)
--(axis cs:335,0.710529320076153)
--(axis cs:334,0.699797496376551)
--(axis cs:333,0.726280654688397)
--(axis cs:332,0.697762427715605)
--(axis cs:331,0.718793440767501)
--(axis cs:330,0.691438250398026)
--(axis cs:329,0.705922366491875)
--(axis cs:328,0.651767749090504)
--(axis cs:327,0.680734874694277)
--(axis cs:326,0.69397584493055)
--(axis cs:325,0.697091744842776)
--(axis cs:324,0.677492050275553)
--(axis cs:323,0.690389894827256)
--(axis cs:322,0.704666275482649)
--(axis cs:321,0.712777320674472)
--(axis cs:320,0.70419281259248)
--(axis cs:319,0.702348559714088)
--(axis cs:318,0.732120503218451)
--(axis cs:317,0.711494351714368)
--(axis cs:316,0.744166125334986)
--(axis cs:315,0.711951691442912)
--(axis cs:314,0.715388323883291)
--(axis cs:313,0.696715341690379)
--(axis cs:312,0.677121647470798)
--(axis cs:311,0.6987333771159)
--(axis cs:310,0.670126137499233)
--(axis cs:309,0.695422528598333)
--(axis cs:308,0.693130038843923)
--(axis cs:307,0.689082964952327)
--(axis cs:306,0.668625127815271)
--(axis cs:305,0.670272064462877)
--(axis cs:304,0.73735003081915)
--(axis cs:303,0.723337055569697)
--(axis cs:302,0.69434451317039)
--(axis cs:301,0.740827881607353)
--(axis cs:300,0.698014082453472)
--(axis cs:299,0.708113140960018)
--(axis cs:298,0.70114759810434)
--(axis cs:297,0.695021154449257)
--(axis cs:296,0.699167247044943)
--(axis cs:295,0.724366101027102)
--(axis cs:294,0.729118026281741)
--(axis cs:293,0.668871007858376)
--(axis cs:292,0.668913027537116)
--(axis cs:291,0.722281541425957)
--(axis cs:290,0.700450944668108)
--(axis cs:289,0.720114203563239)
--(axis cs:288,0.682136036335828)
--(axis cs:287,0.719300044478004)
--(axis cs:286,0.718765279229444)
--(axis cs:285,0.71755834242132)
--(axis cs:284,0.72447273519627)
--(axis cs:283,0.671567518124573)
--(axis cs:282,0.654633422143209)
--(axis cs:281,0.686253144495588)
--(axis cs:280,0.697779066590345)
--(axis cs:279,0.689438674685159)
--(axis cs:278,0.695305616662486)
--(axis cs:277,0.669129202746816)
--(axis cs:276,0.712906564869629)
--(axis cs:275,0.686749596947626)
--(axis cs:274,0.70304615971431)
--(axis cs:273,0.689781725355175)
--(axis cs:272,0.740113916199495)
--(axis cs:271,0.719961191067191)
--(axis cs:270,0.729654885740974)
--(axis cs:269,0.698506171891991)
--(axis cs:268,0.71197811965699)
--(axis cs:267,0.72818602740906)
--(axis cs:266,0.702288347571859)
--(axis cs:265,0.707147935010712)
--(axis cs:264,0.698433261050705)
--(axis cs:263,0.74595823497771)
--(axis cs:262,0.74468276441865)
--(axis cs:261,0.704710364545405)
--(axis cs:260,0.751036792455238)
--(axis cs:259,0.706580843797968)
--(axis cs:258,0.697580593724873)
--(axis cs:257,0.726216238643437)
--(axis cs:256,0.717808275761983)
--(axis cs:255,0.713241043113542)
--(axis cs:254,0.755710433588898)
--(axis cs:253,0.681304077575831)
--(axis cs:252,0.724414184022426)
--(axis cs:251,0.74680908503351)
--(axis cs:250,0.738441758607655)
--(axis cs:249,0.706327118991664)
--(axis cs:248,0.686836524651095)
--(axis cs:247,0.6835855024581)
--(axis cs:246,0.688701051068759)
--(axis cs:245,0.720233141339063)
--(axis cs:244,0.711428171863661)
--(axis cs:243,0.693616929233762)
--(axis cs:242,0.71269677521913)
--(axis cs:241,0.671466591370482)
--(axis cs:240,0.667059077979195)
--(axis cs:239,0.678442735861656)
--(axis cs:238,0.702074710033828)
--(axis cs:237,0.710013774663727)
--(axis cs:236,0.681941805733438)
--(axis cs:235,0.694437565221423)
--(axis cs:234,0.705051711902702)
--(axis cs:233,0.696663854910632)
--(axis cs:232,0.697853444382889)
--(axis cs:231,0.66918070237597)
--(axis cs:230,0.697925232579812)
--(axis cs:229,0.692349181114478)
--(axis cs:228,0.684388521162926)
--(axis cs:227,0.700791195073378)
--(axis cs:226,0.704922374103652)
--(axis cs:225,0.659582760710219)
--(axis cs:224,0.615331600404728)
--(axis cs:223,0.731613328362127)
--(axis cs:222,0.680372729729489)
--(axis cs:221,0.703289761669769)
--(axis cs:220,0.746527133059128)
--(axis cs:219,0.73669123114763)
--(axis cs:218,0.703600027752905)
--(axis cs:217,0.702462866979372)
--(axis cs:216,0.67363284841294)
--(axis cs:215,0.669037667229834)
--(axis cs:214,0.684803635094206)
--(axis cs:213,0.650307378263017)
--(axis cs:212,0.651485377471674)
--(axis cs:211,0.674812386563752)
--(axis cs:210,0.674407780088732)
--(axis cs:209,0.687794053568407)
--(axis cs:208,0.701895132701072)
--(axis cs:207,0.678009226104792)
--(axis cs:206,0.694270830928589)
--(axis cs:205,0.657206159554799)
--(axis cs:204,0.677938173833448)
--(axis cs:203,0.629348012660871)
--(axis cs:202,0.688476635855968)
--(axis cs:201,0.668071305582518)
--(axis cs:200,0.702644978924796)
--(axis cs:199,0.661999098438263)
--(axis cs:198,0.682909192749388)
--(axis cs:197,0.661726009338576)
--(axis cs:196,0.680504082178574)
--(axis cs:195,0.695431332418367)
--(axis cs:194,0.680922593567187)
--(axis cs:193,0.626442840820875)
--(axis cs:192,0.646954567870827)
--(axis cs:191,0.643546030620422)
--(axis cs:190,0.671545333686772)
--(axis cs:189,0.659337217724451)
--(axis cs:188,0.671777489602099)
--(axis cs:187,0.66733714390564)
--(axis cs:186,0.642541204144006)
--(axis cs:185,0.684320389510216)
--(axis cs:184,0.631462220349932)
--(axis cs:183,0.624473708024082)
--(axis cs:182,0.676559910191494)
--(axis cs:181,0.658568548372992)
--(axis cs:180,0.648991594516773)
--(axis cs:179,0.62965264075835)
--(axis cs:178,0.673297098168334)
--(axis cs:177,0.645782793846673)
--(axis cs:176,0.663344309406739)
--(axis cs:175,0.589508252108334)
--(axis cs:174,0.63297014778011)
--(axis cs:173,0.632481415405027)
--(axis cs:172,0.617908989540209)
--(axis cs:171,0.586788126540879)
--(axis cs:170,0.607153643141139)
--(axis cs:169,0.621467835658882)
--(axis cs:168,0.625567639314988)
--(axis cs:167,0.591611067491556)
--(axis cs:166,0.645605428911063)
--(axis cs:165,0.573018019726238)
--(axis cs:164,0.580150953056128)
--(axis cs:163,0.604417988154741)
--(axis cs:162,0.559162920381622)
--(axis cs:161,0.512475282570732)
--(axis cs:160,0.567038857463617)
--(axis cs:159,0.585135566473253)
--(axis cs:158,0.523266222625138)
--(axis cs:157,0.56933818255041)
--(axis cs:156,0.591494176787747)
--(axis cs:155,0.529374500078662)
--(axis cs:154,0.573300543354466)
--(axis cs:153,0.552575702937234)
--(axis cs:152,0.528991547799413)
--(axis cs:151,0.537286472088273)
--(axis cs:150,0.544429543443633)
--(axis cs:149,0.571232415549415)
--(axis cs:148,0.534376196244634)
--(axis cs:147,0.543492022846996)
--(axis cs:146,0.531324043879286)
--(axis cs:145,0.519804611435252)
--(axis cs:144,0.540271465485817)
--(axis cs:143,0.506476030419072)
--(axis cs:142,0.531020801945641)
--(axis cs:141,0.514769253917799)
--(axis cs:140,0.504823899116322)
--(axis cs:139,0.546684402327522)
--(axis cs:138,0.469903792153099)
--(axis cs:137,0.503579005082669)
--(axis cs:136,0.40045359677614)
--(axis cs:135,0.475188857359302)
--(axis cs:134,0.427898105147205)
--(axis cs:133,0.409327009340852)
--(axis cs:132,0.430237794139333)
--(axis cs:131,0.411897278747529)
--(axis cs:130,0.47261127643826)
--(axis cs:129,0.412381325504612)
--(axis cs:128,0.343144024028913)
--(axis cs:127,0.378076361571678)
--(axis cs:126,0.380545705845375)
--(axis cs:125,0.376383058086347)
--(axis cs:124,0.404914439072956)
--(axis cs:123,0.388168571246035)
--(axis cs:122,0.384147403579517)
--(axis cs:121,0.326169355551187)
--(axis cs:120,0.362518224728664)
--(axis cs:119,0.328844260490432)
--(axis cs:118,0.342028695619039)
--(axis cs:117,0.299843843037507)
--(axis cs:116,0.296183227733244)
--(axis cs:115,0.322121125234416)
--(axis cs:114,0.28462333264943)
--(axis cs:113,0.322728397738896)
--(axis cs:112,0.295260466869756)
--(axis cs:111,0.302095977574975)
--(axis cs:110,0.31865770174473)
--(axis cs:109,0.275073737490555)
--(axis cs:108,0.30048754787154)
--(axis cs:107,0.246319631052312)
--(axis cs:106,0.212188073581039)
--(axis cs:105,0.207194654433401)
--(axis cs:104,0.245830984794671)
--(axis cs:103,0.21881838977588)
--(axis cs:102,0.223424729934478)
--(axis cs:101,0.231926534221747)
--(axis cs:100,0.210373534828662)
--(axis cs:99,0.212470507774775)
--(axis cs:98,0.261734106323202)
--(axis cs:97,0.173880899609628)
--(axis cs:96,0.17851665829206)
--(axis cs:95,0.122388876954564)
--(axis cs:94,0.169111508689686)
--(axis cs:93,0.165654613235028)
--(axis cs:92,0.188813494951482)
--(axis cs:91,0.104741117247297)
--(axis cs:90,0.162554210721214)
--(axis cs:89,0.160060340412459)
--(axis cs:88,0.1058610067361)
--(axis cs:87,0.111848218170927)
--(axis cs:86,0.103687454181476)
--(axis cs:85,0.0829997996279333)
--(axis cs:84,0.0953702608395391)
--(axis cs:83,0.0767343522863083)
--(axis cs:82,0.0805015557652432)
--(axis cs:81,0.107431928275556)
--(axis cs:80,0.103435152618769)
--(axis cs:79,0.0865332421189278)
--(axis cs:78,0.107444012970808)
--(axis cs:77,0.0687754078485998)
--(axis cs:76,0.0798589068527235)
--(axis cs:75,0.0626194910787409)
--(axis cs:74,0.0642925025648482)
--(axis cs:73,0.0591075051825194)
--(axis cs:72,0.0307732496375775)
--(axis cs:71,0.0728546571510815)
--(axis cs:70,0.062515447797346)
--(axis cs:69,0.0173198744890102)
--(axis cs:68,0.0620209027042473)
--(axis cs:67,0.0612360581604591)
--(axis cs:66,0.0642225409654223)
--(axis cs:65,0.0623172938200819)
--(axis cs:64,0.034878942516207)
--(axis cs:63,0.0313062460647482)
--(axis cs:62,0.039035914216741)
--(axis cs:61,0.0512174172296692)
--(axis cs:60,0.0373661907548843)
--(axis cs:59,0.0524262255718926)
--(axis cs:58,0.017437514949201)
--(axis cs:57,0.0329337154627599)
--(axis cs:56,0.0325299285206879)
--(axis cs:55,0.0267247473834193)
--(axis cs:54,0.0346293148627847)
--(axis cs:53,0.0539586753988709)
--(axis cs:52,0.0316339897103504)
--(axis cs:51,0.0103302790554596)
--(axis cs:50,0.0179819652748664)
--(axis cs:49,0.0245129957385626)
--(axis cs:48,0.0072989599304742)
--(axis cs:47,0.0171391028708405)
--(axis cs:46,0.0379741061624233)
--(axis cs:45,0.0261976021535339)
--(axis cs:44,0.0366321863337859)
--(axis cs:43,0.020858779950359)
--(axis cs:42,0.0310317762763706)
--(axis cs:41,0.020001711508865)
--(axis cs:40,0.0129597781790479)
--(axis cs:39,0.0239960103572435)
--(axis cs:38,0.0149922631536629)
--(axis cs:37,0.0135035519403382)
--(axis cs:36,0.016833384708116)
--(axis cs:35,0.0263676272765037)
--(axis cs:34,0.0402039237680683)
--(axis cs:33,0.0158495504120123)
--(axis cs:32,0.00762155672731509)
--(axis cs:31,0.00564967595588614)
--(axis cs:30,0.0237182546654997)
--(axis cs:29,0.0195720318740617)
--(axis cs:28,0.0220285156162418)
--(axis cs:27,0.0113727134335261)
--(axis cs:26,0.02198170029353)
--(axis cs:25,0.0160686967746579)
--(axis cs:24,0.0200857534044242)
--(axis cs:23,0.0139401208575475)
--(axis cs:22,0.018575429242053)
--(axis cs:21,0.00943420441772525)
--(axis cs:20,0.0200029450828742)
--(axis cs:19,0.0246067746118364)
--(axis cs:18,0.0402411996886516)
--(axis cs:17,0.0346270537928664)
--(axis cs:16,0.0401480510893047)
--(axis cs:15,0.0178885555791465)
--(axis cs:14,0.0272682952139645)
--(axis cs:13,0.0171391265558579)
--(axis cs:12,0.0174304145123455)
--(axis cs:11,0.00434122497145318)
--(axis cs:10,0.00680367477735022)
--(axis cs:9,0.00137264047576941)
--(axis cs:8,0.00224183543973531)
--(axis cs:7,0.00337616134210667)
--(axis cs:6,0.00101887166762993)
--(axis cs:5,0.000206296331899161)
--(axis cs:4,0.00830654423063848)
--(axis cs:3,-0.000119838118575884)
--(axis cs:2,0.00107277315169399)
--(axis cs:1,0.00247144560876655)
--(axis cs:0,-0.00204367757289895)
--cycle;

\path [fill=color1, fill opacity=0.3] (axis cs:0,-0.000410108335200871)
--(axis cs:0,0.0068089178590104)
--(axis cs:1,0)
--(axis cs:2,0.0166600678550971)
--(axis cs:3,0.0123708633066541)
--(axis cs:4,0.0159455207124802)
--(axis cs:5,0.0234905591530698)
--(axis cs:6,0.025786722623578)
--(axis cs:7,0.0131943825656744)
--(axis cs:8,0.0203976670579365)
--(axis cs:9,0.0368090838637354)
--(axis cs:10,0.0447572869494205)
--(axis cs:11,0.0360015799755838)
--(axis cs:12,0.0524855241357771)
--(axis cs:13,0.0425541359289862)
--(axis cs:14,0.0367181559603182)
--(axis cs:15,0.0658246794601793)
--(axis cs:16,0.0659299935354763)
--(axis cs:17,0.0723239873259948)
--(axis cs:18,0.0499887423070852)
--(axis cs:19,0.0670655076308767)
--(axis cs:20,0.0557446825826829)
--(axis cs:21,0.0689113960931311)
--(axis cs:22,0.0748232410338417)
--(axis cs:23,0.0482182842757415)
--(axis cs:24,0.0741236531419506)
--(axis cs:25,0.0702171114973329)
--(axis cs:26,0.0546896948666402)
--(axis cs:27,0.0621754256659684)
--(axis cs:28,0.0701888487102833)
--(axis cs:29,0.0557118074781051)
--(axis cs:30,0.0642751847698035)
--(axis cs:31,0.0958867521822996)
--(axis cs:32,0.0957632814226514)
--(axis cs:33,0.0874012809462982)
--(axis cs:34,0.0926207711458594)
--(axis cs:35,0.0618540596991398)
--(axis cs:36,0.0610790112116803)
--(axis cs:37,0.070977235174789)
--(axis cs:38,0.0676075816371331)
--(axis cs:39,0.106300743541389)
--(axis cs:40,0.058310066003358)
--(axis cs:41,0.101320308462189)
--(axis cs:42,0.128903297469261)
--(axis cs:43,0.0942967816073584)
--(axis cs:44,0.115113002250298)
--(axis cs:45,0.140648441289951)
--(axis cs:46,0.151080554497025)
--(axis cs:47,0.188644585878579)
--(axis cs:48,0.161545367801587)
--(axis cs:49,0.169179670662769)
--(axis cs:50,0.162061478349626)
--(axis cs:51,0.213037624247665)
--(axis cs:52,0.173286302294181)
--(axis cs:53,0.232190125540791)
--(axis cs:54,0.230787883529205)
--(axis cs:55,0.250028952159778)
--(axis cs:56,0.247111742231525)
--(axis cs:57,0.239956714781239)
--(axis cs:58,0.266097445969026)
--(axis cs:59,0.247380140983085)
--(axis cs:60,0.288369317788017)
--(axis cs:61,0.296970823747079)
--(axis cs:62,0.339057095121074)
--(axis cs:63,0.36981002447365)
--(axis cs:64,0.35400917203604)
--(axis cs:65,0.346406286945292)
--(axis cs:66,0.437006242987844)
--(axis cs:67,0.417643085621802)
--(axis cs:68,0.395324922237532)
--(axis cs:69,0.476314391665895)
--(axis cs:70,0.448991960327642)
--(axis cs:71,0.48628243472225)
--(axis cs:72,0.505858527528802)
--(axis cs:73,0.456562838013501)
--(axis cs:74,0.545998987756741)
--(axis cs:75,0.527746199053642)
--(axis cs:76,0.546840321104109)
--(axis cs:77,0.580078857540012)
--(axis cs:78,0.649982865246424)
--(axis cs:79,0.597335978193144)
--(axis cs:80,0.608329815985416)
--(axis cs:81,0.580508699797202)
--(axis cs:82,0.608664240535665)
--(axis cs:83,0.653604024409813)
--(axis cs:84,0.629726793839922)
--(axis cs:85,0.609604161926603)
--(axis cs:86,0.637336241473591)
--(axis cs:87,0.625964124964596)
--(axis cs:88,0.647954371974538)
--(axis cs:89,0.657515708656632)
--(axis cs:90,0.676424443551995)
--(axis cs:91,0.683558622269454)
--(axis cs:92,0.699258776508045)
--(axis cs:93,0.669643416587644)
--(axis cs:94,0.68234646121798)
--(axis cs:95,0.698002317525191)
--(axis cs:96,0.693500159918964)
--(axis cs:97,0.712482939669826)
--(axis cs:98,0.760813054237348)
--(axis cs:99,0.72420625541157)
--(axis cs:100,0.702650514578185)
--(axis cs:101,0.651041271585418)
--(axis cs:102,0.712243783743055)
--(axis cs:103,0.71210814035245)
--(axis cs:104,0.747119664574456)
--(axis cs:105,0.743629930762481)
--(axis cs:106,0.735078707050533)
--(axis cs:107,0.71965286557681)
--(axis cs:108,0.708789760665056)
--(axis cs:109,0.716123192461166)
--(axis cs:110,0.702026401670462)
--(axis cs:111,0.688891888645959)
--(axis cs:112,0.697919207967971)
--(axis cs:113,0.731192808544236)
--(axis cs:114,0.712727241424314)
--(axis cs:115,0.726294614982439)
--(axis cs:116,0.727779567191639)
--(axis cs:117,0.757681274047605)
--(axis cs:118,0.730579887733216)
--(axis cs:119,0.73277093609514)
--(axis cs:120,0.739839479933767)
--(axis cs:121,0.769878720066554)
--(axis cs:122,0.740901513044328)
--(axis cs:123,0.769869531219146)
--(axis cs:124,0.761022322280059)
--(axis cs:125,0.756292342111417)
--(axis cs:126,0.770773322257805)
--(axis cs:127,0.752869987750055)
--(axis cs:128,0.749593806134243)
--(axis cs:129,0.752892032661857)
--(axis cs:130,0.729994276699192)
--(axis cs:131,0.780669052933865)
--(axis cs:132,0.769289894532727)
--(axis cs:133,0.74322061055309)
--(axis cs:134,0.735382629940356)
--(axis cs:135,0.765052665574517)
--(axis cs:136,0.758922939124579)
--(axis cs:137,0.711472488778258)
--(axis cs:138,0.755508288531286)
--(axis cs:139,0.719460541779264)
--(axis cs:140,0.713631607686635)
--(axis cs:141,0.727183865313098)
--(axis cs:142,0.716245679575814)
--(axis cs:143,0.785922189900548)
--(axis cs:144,0.754258696825328)
--(axis cs:145,0.704476309498342)
--(axis cs:146,0.73640545655911)
--(axis cs:147,0.6930549852933)
--(axis cs:148,0.744274634948325)
--(axis cs:149,0.704510472188579)
--(axis cs:150,0.730428498655351)
--(axis cs:151,0.744660234754226)
--(axis cs:152,0.684110429035273)
--(axis cs:153,0.740272582748571)
--(axis cs:154,0.699077811597075)
--(axis cs:155,0.686892280143184)
--(axis cs:156,0.731716202181568)
--(axis cs:157,0.680876273748734)
--(axis cs:158,0.714331587308388)
--(axis cs:159,0.67889338708805)
--(axis cs:160,0.717391631705213)
--(axis cs:161,0.68515232437414)
--(axis cs:162,0.715873978890582)
--(axis cs:163,0.734238302361315)
--(axis cs:164,0.735700489350756)
--(axis cs:165,0.707288267165286)
--(axis cs:166,0.70609217406862)
--(axis cs:167,0.713468781893113)
--(axis cs:168,0.705466388042802)
--(axis cs:169,0.729284037343821)
--(axis cs:170,0.706806354472888)
--(axis cs:171,0.684242771352267)
--(axis cs:172,0.714258656084204)
--(axis cs:173,0.699517769587208)
--(axis cs:174,0.673699180890788)
--(axis cs:175,0.685310686797984)
--(axis cs:176,0.679791546031844)
--(axis cs:177,0.688618388708189)
--(axis cs:178,0.688470449629466)
--(axis cs:179,0.72584784766953)
--(axis cs:180,0.712267213434947)
--(axis cs:181,0.757999984916425)
--(axis cs:182,0.735684385494178)
--(axis cs:183,0.738537211101595)
--(axis cs:184,0.79253610944752)
--(axis cs:185,0.749249086914005)
--(axis cs:186,0.734285084347949)
--(axis cs:187,0.735858779099917)
--(axis cs:188,0.772548937838245)
--(axis cs:189,0.774870946002634)
--(axis cs:190,0.743268736051666)
--(axis cs:191,0.721824077728481)
--(axis cs:192,0.76269127797052)
--(axis cs:193,0.760391693479028)
--(axis cs:194,0.768913173891316)
--(axis cs:195,0.72663822965522)
--(axis cs:196,0.766997612345046)
--(axis cs:197,0.740234523590079)
--(axis cs:198,0.75190208768416)
--(axis cs:199,0.766701395806317)
--(axis cs:200,0.765843055733979)
--(axis cs:201,0.803779810913583)
--(axis cs:202,0.769345988311063)
--(axis cs:203,0.745326184700006)
--(axis cs:204,0.728809670223832)
--(axis cs:205,0.743405551950973)
--(axis cs:206,0.760270963158347)
--(axis cs:207,0.752918252689318)
--(axis cs:208,0.729072852613612)
--(axis cs:209,0.75763388269715)
--(axis cs:210,0.778992990523765)
--(axis cs:211,0.775330029093593)
--(axis cs:212,0.820574386734958)
--(axis cs:213,0.791577113832601)
--(axis cs:214,0.756848533444647)
--(axis cs:215,0.73870889926778)
--(axis cs:216,0.750513643584364)
--(axis cs:217,0.783600359539598)
--(axis cs:218,0.773235899835938)
--(axis cs:219,0.775630130037618)
--(axis cs:220,0.760055288313046)
--(axis cs:221,0.763199474551916)
--(axis cs:222,0.77527576131193)
--(axis cs:223,0.791270936737478)
--(axis cs:224,0.757657142893991)
--(axis cs:225,0.765137348664268)
--(axis cs:226,0.749349997240364)
--(axis cs:227,0.740983698969324)
--(axis cs:228,0.766777308132916)
--(axis cs:229,0.790606185964017)
--(axis cs:230,0.724545349316702)
--(axis cs:231,0.806913130284059)
--(axis cs:232,0.783439359013358)
--(axis cs:233,0.771698358552473)
--(axis cs:234,0.776097921827807)
--(axis cs:235,0.761642490195175)
--(axis cs:236,0.763724309423796)
--(axis cs:237,0.761482199705537)
--(axis cs:238,0.783532061897446)
--(axis cs:239,0.751508376174849)
--(axis cs:240,0.778002879246758)
--(axis cs:241,0.762979410008904)
--(axis cs:242,0.801098955383187)
--(axis cs:243,0.785596881090782)
--(axis cs:244,0.784486789182566)
--(axis cs:245,0.808610599002682)
--(axis cs:246,0.792909862920984)
--(axis cs:247,0.787691017987266)
--(axis cs:248,0.775935309894784)
--(axis cs:249,0.751833781288438)
--(axis cs:250,0.796138506711153)
--(axis cs:251,0.803130219247238)
--(axis cs:252,0.801807082877544)
--(axis cs:253,0.79314035626387)
--(axis cs:254,0.761615863288085)
--(axis cs:255,0.791115188409927)
--(axis cs:256,0.823053866293007)
--(axis cs:257,0.788243289571273)
--(axis cs:258,0.791875810724429)
--(axis cs:259,0.772494908703943)
--(axis cs:260,0.752843075678707)
--(axis cs:261,0.788545470119077)
--(axis cs:262,0.767279380164818)
--(axis cs:263,0.788632774863959)
--(axis cs:264,0.792144631775004)
--(axis cs:265,0.780498460643758)
--(axis cs:266,0.770238829484463)
--(axis cs:267,0.782247191688527)
--(axis cs:268,0.810295166751078)
--(axis cs:269,0.780516765139306)
--(axis cs:270,0.773984879147702)
--(axis cs:271,0.78638454735354)
--(axis cs:272,0.799132182327974)
--(axis cs:273,0.767704750450842)
--(axis cs:274,0.760096182717413)
--(axis cs:275,0.772810945675209)
--(axis cs:276,0.802562190887597)
--(axis cs:277,0.754189533791098)
--(axis cs:278,0.769485087672501)
--(axis cs:279,0.74398686192236)
--(axis cs:280,0.757929347133836)
--(axis cs:281,0.790770083656786)
--(axis cs:282,0.740261837429256)
--(axis cs:283,0.776408606149277)
--(axis cs:284,0.785976950461456)
--(axis cs:285,0.797406292904342)
--(axis cs:286,0.77579046395249)
--(axis cs:287,0.769388187783829)
--(axis cs:288,0.805564672370623)
--(axis cs:289,0.79978602965928)
--(axis cs:290,0.76455708888856)
--(axis cs:291,0.778750555152729)
--(axis cs:292,0.803525487306493)
--(axis cs:293,0.793015841644078)
--(axis cs:294,0.790577273484417)
--(axis cs:295,0.794923562997524)
--(axis cs:296,0.800562673569624)
--(axis cs:297,0.807630201909498)
--(axis cs:298,0.809574220222344)
--(axis cs:299,0.78232650510347)
--(axis cs:300,0.802802710699573)
--(axis cs:301,0.801997660221703)
--(axis cs:302,0.776676165679338)
--(axis cs:303,0.808495990413308)
--(axis cs:304,0.830000618607156)
--(axis cs:305,0.810693252581738)
--(axis cs:306,0.811965991479481)
--(axis cs:307,0.804856945859905)
--(axis cs:308,0.804268436558273)
--(axis cs:309,0.77902189999318)
--(axis cs:310,0.783895313255212)
--(axis cs:311,0.790846201067883)
--(axis cs:312,0.832614353864969)
--(axis cs:313,0.821777395297661)
--(axis cs:314,0.818735397301753)
--(axis cs:315,0.82228523521531)
--(axis cs:316,0.771484362401156)
--(axis cs:317,0.750777252167188)
--(axis cs:318,0.775195509698692)
--(axis cs:319,0.803958401314414)
--(axis cs:320,0.805423595011026)
--(axis cs:321,0.796284346734377)
--(axis cs:322,0.793683612383339)
--(axis cs:323,0.777652742125816)
--(axis cs:324,0.812421257908776)
--(axis cs:325,0.820191807696048)
--(axis cs:326,0.794631596408679)
--(axis cs:327,0.796150611965933)
--(axis cs:328,0.822701749204234)
--(axis cs:329,0.767155442783729)
--(axis cs:330,0.773945552327461)
--(axis cs:331,0.81370318262085)
--(axis cs:332,0.822570527883476)
--(axis cs:333,0.843622507911965)
--(axis cs:334,0.7873037414566)
--(axis cs:335,0.808395487650403)
--(axis cs:336,0.81471577305054)
--(axis cs:337,0.758448955829835)
--(axis cs:338,0.788091665602842)
--(axis cs:339,0.802504328664135)
--(axis cs:340,0.820434571096888)
--(axis cs:341,0.78348285656239)
--(axis cs:342,0.781297545532159)
--(axis cs:343,0.792297982565312)
--(axis cs:344,0.791061153041813)
--(axis cs:345,0.774881779678084)
--(axis cs:346,0.805857101763307)
--(axis cs:347,0.788384804404756)
--(axis cs:348,0.779098793043187)
--(axis cs:349,0.801409242740365)
--(axis cs:350,0.822831413285336)
--(axis cs:351,0.778074703971746)
--(axis cs:352,0.833780777704989)
--(axis cs:353,0.79990928204812)
--(axis cs:354,0.783549979350404)
--(axis cs:355,0.800544316730718)
--(axis cs:356,0.793759609337709)
--(axis cs:357,0.808369428983904)
--(axis cs:358,0.796927903480011)
--(axis cs:359,0.782814943296137)
--(axis cs:360,0.792615982256063)
--(axis cs:361,0.814375387170515)
--(axis cs:362,0.820227601242566)
--(axis cs:363,0.797882855232685)
--(axis cs:364,0.817200683282006)
--(axis cs:365,0.781989188889839)
--(axis cs:366,0.783469289827768)
--(axis cs:367,0.828708066039405)
--(axis cs:368,0.808160530344923)
--(axis cs:369,0.758768804687066)
--(axis cs:370,0.805580864158838)
--(axis cs:371,0.817475328670888)
--(axis cs:372,0.773950493726087)
--(axis cs:373,0.825381285590645)
--(axis cs:374,0.793698495253922)
--(axis cs:375,0.778880498896522)
--(axis cs:376,0.835931786463558)
--(axis cs:377,0.777275860025554)
--(axis cs:378,0.7910128176146)
--(axis cs:379,0.808504875973502)
--(axis cs:380,0.811898364970262)
--(axis cs:381,0.819886957757051)
--(axis cs:382,0.798520968923126)
--(axis cs:383,0.824085244082354)
--(axis cs:384,0.845085568877867)
--(axis cs:385,0.863618580950206)
--(axis cs:386,0.794015433263692)
--(axis cs:387,0.842985036848748)
--(axis cs:388,0.791956899191832)
--(axis cs:389,0.793693189967627)
--(axis cs:390,0.769888997308041)
--(axis cs:391,0.787835692408813)
--(axis cs:392,0.795596488700683)
--(axis cs:393,0.777050334919825)
--(axis cs:394,0.800834638859962)
--(axis cs:395,0.816249305693886)
--(axis cs:396,0.800067538912616)
--(axis cs:397,0.74499513548779)
--(axis cs:398,0.790716222931195)
--(axis cs:399,0.820276423992661)
--(axis cs:400,0.779666238401139)
--(axis cs:401,0.746724223476633)
--(axis cs:402,0.774804444560005)
--(axis cs:403,0.815818581653728)
--(axis cs:404,0.796138462084345)
--(axis cs:405,0.807571564952437)
--(axis cs:406,0.788104331036763)
--(axis cs:407,0.809542143429329)
--(axis cs:408,0.799507557378098)
--(axis cs:409,0.766919560388047)
--(axis cs:410,0.787525075658359)
--(axis cs:411,0.80433016164499)
--(axis cs:412,0.808794661539719)
--(axis cs:413,0.801122354812101)
--(axis cs:414,0.81094746898406)
--(axis cs:415,0.809319062180377)
--(axis cs:416,0.757871368935418)
--(axis cs:417,0.816614425652158)
--(axis cs:418,0.754420131501912)
--(axis cs:419,0.760771161470074)
--(axis cs:420,0.793033437606066)
--(axis cs:421,0.773704666633882)
--(axis cs:422,0.832114834279497)
--(axis cs:423,0.791695349540144)
--(axis cs:424,0.842667696702713)
--(axis cs:425,0.81562231397004)
--(axis cs:426,0.795121906423398)
--(axis cs:427,0.814541080587062)
--(axis cs:428,0.786714297543474)
--(axis cs:429,0.810158497426652)
--(axis cs:430,0.785095915237295)
--(axis cs:431,0.792209595391834)
--(axis cs:432,0.818445286944861)
--(axis cs:433,0.797309810776865)
--(axis cs:434,0.796637315310899)
--(axis cs:435,0.823820882259296)
--(axis cs:436,0.822713912130208)
--(axis cs:437,0.835258310638551)
--(axis cs:438,0.777963097912078)
--(axis cs:439,0.797552685852047)
--(axis cs:440,0.790002539518334)
--(axis cs:441,0.785292047545658)
--(axis cs:442,0.810498213577874)
--(axis cs:443,0.818541902214018)
--(axis cs:444,0.830250011336422)
--(axis cs:445,0.824010483922572)
--(axis cs:446,0.813816306383504)
--(axis cs:447,0.7883476956612)
--(axis cs:448,0.798511307327647)
--(axis cs:449,0.812953214259704)
--(axis cs:450,0.828133904215131)
--(axis cs:451,0.815499956553049)
--(axis cs:452,0.836715600512613)
--(axis cs:453,0.816484065806888)
--(axis cs:454,0.833897468004743)
--(axis cs:455,0.80096858907649)
--(axis cs:456,0.821054380700988)
--(axis cs:457,0.81538700669339)
--(axis cs:458,0.821602000192718)
--(axis cs:459,0.846981475680053)
--(axis cs:460,0.769777434449577)
--(axis cs:461,0.817778027847405)
--(axis cs:462,0.818179657389021)
--(axis cs:463,0.849378675275289)
--(axis cs:464,0.807809620282788)
--(axis cs:465,0.78064112610689)
--(axis cs:466,0.831803531027462)
--(axis cs:467,0.784729470509035)
--(axis cs:468,0.806839423986501)
--(axis cs:469,0.782271988281041)
--(axis cs:470,0.795984478161106)
--(axis cs:471,0.808578027531633)
--(axis cs:472,0.792811542097108)
--(axis cs:473,0.802079515750347)
--(axis cs:474,0.797005682760698)
--(axis cs:475,0.838029748961699)
--(axis cs:476,0.80924273177399)
--(axis cs:477,0.801257280605356)
--(axis cs:478,0.82156145941383)
--(axis cs:479,0.801954222969625)
--(axis cs:480,0.800970412347812)
--(axis cs:481,0.826032623552051)
--(axis cs:482,0.810617295588073)
--(axis cs:483,0.825159578203261)
--(axis cs:484,0.809441889859354)
--(axis cs:485,0.825031285409626)
--(axis cs:486,0.82350013962536)
--(axis cs:487,0.80619716478954)
--(axis cs:488,0.779659359736355)
--(axis cs:489,0.82566256302247)
--(axis cs:490,0.821412564104324)
--(axis cs:491,0.807797207140826)
--(axis cs:492,0.822759234902582)
--(axis cs:493,0.823369107985069)
--(axis cs:494,0.817194763178061)
--(axis cs:495,0.832003999690561)
--(axis cs:496,0.841585243233694)
--(axis cs:497,0.799327873844942)
--(axis cs:498,0.837455332553046)
--(axis cs:499,0.798479715730267)
--(axis cs:500,0.841486243671241)
--(axis cs:501,0.812502123639841)
--(axis cs:502,0.768527036391735)
--(axis cs:503,0.797544406686986)
--(axis cs:504,0.820896014676311)
--(axis cs:505,0.801466264412285)
--(axis cs:506,0.793868247587002)
--(axis cs:507,0.835762945457602)
--(axis cs:508,0.848169525622776)
--(axis cs:509,0.812847174664098)
--(axis cs:510,0.818818411047491)
--(axis cs:511,0.844995410851153)
--(axis cs:512,0.796913206353357)
--(axis cs:513,0.790918160518376)
--(axis cs:514,0.812792940416996)
--(axis cs:515,0.802004842284373)
--(axis cs:516,0.782212332580719)
--(axis cs:517,0.774968123461125)
--(axis cs:518,0.826795723002176)
--(axis cs:519,0.81824703846775)
--(axis cs:520,0.808008862786293)
--(axis cs:521,0.827242670139546)
--(axis cs:522,0.806184648268339)
--(axis cs:523,0.818504648656228)
--(axis cs:524,0.808262183781118)
--(axis cs:525,0.811960627727838)
--(axis cs:526,0.833925612197525)
--(axis cs:527,0.775821787459261)
--(axis cs:528,0.791001317136105)
--(axis cs:529,0.819821723363966)
--(axis cs:530,0.820830116671495)
--(axis cs:531,0.782533003871454)
--(axis cs:532,0.835201916768099)
--(axis cs:533,0.814089562167518)
--(axis cs:534,0.804936777192787)
--(axis cs:535,0.805951964029)
--(axis cs:536,0.815270316703129)
--(axis cs:537,0.803006959066178)
--(axis cs:538,0.776819291560041)
--(axis cs:539,0.815964520345033)
--(axis cs:540,0.821934125496311)
--(axis cs:541,0.78268520131336)
--(axis cs:542,0.805284438883558)
--(axis cs:543,0.831661337245995)
--(axis cs:544,0.833587925340282)
--(axis cs:545,0.812263473575479)
--(axis cs:546,0.833186911828518)
--(axis cs:547,0.825298175605655)
--(axis cs:548,0.810628052359351)
--(axis cs:549,0.789824793922164)
--(axis cs:550,0.795562947961443)
--(axis cs:551,0.820456226693076)
--(axis cs:552,0.834275048816595)
--(axis cs:553,0.798445090804643)
--(axis cs:554,0.822977622021402)
--(axis cs:555,0.800567695544033)
--(axis cs:556,0.849708901088634)
--(axis cs:557,0.835034831541628)
--(axis cs:558,0.774116175624186)
--(axis cs:559,0.820355668012448)
--(axis cs:560,0.840877408192995)
--(axis cs:561,0.799047072740498)
--(axis cs:562,0.81400748170241)
--(axis cs:563,0.782228994551825)
--(axis cs:564,0.803536723068493)
--(axis cs:565,0.816708230151834)
--(axis cs:566,0.818233269523076)
--(axis cs:567,0.77998648132737)
--(axis cs:568,0.792955885561176)
--(axis cs:569,0.818165498982487)
--(axis cs:570,0.79512722139268)
--(axis cs:571,0.793092567736689)
--(axis cs:572,0.758299272590043)
--(axis cs:573,0.809447582373198)
--(axis cs:574,0.799332819168278)
--(axis cs:575,0.801352368857866)
--(axis cs:576,0.803915370678331)
--(axis cs:577,0.809569476310063)
--(axis cs:578,0.819206652763181)
--(axis cs:579,0.798790737787804)
--(axis cs:580,0.825137086471227)
--(axis cs:581,0.802474746199687)
--(axis cs:582,0.836619943599417)
--(axis cs:583,0.825277886449925)
--(axis cs:584,0.839595042982072)
--(axis cs:585,0.786859212421237)
--(axis cs:586,0.805794926286023)
--(axis cs:587,0.820988252989703)
--(axis cs:588,0.787302762306384)
--(axis cs:589,0.815417185455989)
--(axis cs:590,0.782417473834498)
--(axis cs:591,0.774975081048355)
--(axis cs:592,0.791548561194831)
--(axis cs:593,0.807812011577525)
--(axis cs:594,0.806369827407747)
--(axis cs:595,0.822172345712546)
--(axis cs:596,0.799558459439609)
--(axis cs:597,0.850070582048315)
--(axis cs:598,0.784684357625154)
--(axis cs:599,0.818079531897339)
--(axis cs:599,0.721393104840923)
--(axis cs:599,0.721393104840923)
--(axis cs:598,0.689523205644909)
--(axis cs:597,0.777844642535109)
--(axis cs:596,0.722873535648636)
--(axis cs:595,0.739768838103638)
--(axis cs:594,0.732867584485915)
--(axis cs:593,0.728137195465432)
--(axis cs:592,0.703263137523118)
--(axis cs:591,0.688077274582125)
--(axis cs:590,0.691180412654013)
--(axis cs:589,0.72498477369597)
--(axis cs:588,0.708730588023841)
--(axis cs:587,0.754779295434094)
--(axis cs:586,0.721529485413388)
--(axis cs:585,0.689814070892671)
--(axis cs:584,0.762964463483685)
--(axis cs:583,0.737160022862984)
--(axis cs:582,0.756580128897531)
--(axis cs:581,0.71971721114852)
--(axis cs:580,0.746294259960119)
--(axis cs:579,0.707462210652644)
--(axis cs:578,0.7250602782225)
--(axis cs:577,0.722808628632381)
--(axis cs:576,0.727064413426453)
--(axis cs:575,0.724499730431733)
--(axis cs:574,0.717555051947534)
--(axis cs:573,0.731337327161711)
--(axis cs:572,0.667861892633622)
--(axis cs:571,0.72149748630649)
--(axis cs:570,0.702342496389537)
--(axis cs:569,0.756662190688952)
--(axis cs:568,0.702264345596555)
--(axis cs:567,0.678413842307328)
--(axis cs:566,0.721440771713465)
--(axis cs:565,0.715737102293498)
--(axis cs:564,0.717822906884887)
--(axis cs:563,0.687409700086869)
--(axis cs:562,0.733215781145852)
--(axis cs:561,0.717022621454196)
--(axis cs:560,0.750143983609646)
--(axis cs:559,0.749737953643674)
--(axis cs:558,0.686574002253492)
--(axis cs:557,0.73886449727498)
--(axis cs:556,0.769779509806026)
--(axis cs:555,0.710968985707722)
--(axis cs:554,0.747650274387744)
--(axis cs:553,0.727543850809299)
--(axis cs:552,0.763264130753835)
--(axis cs:551,0.738126086622693)
--(axis cs:550,0.704271679216933)
--(axis cs:549,0.709480067882698)
--(axis cs:548,0.710242526714354)
--(axis cs:547,0.73432960980338)
--(axis cs:546,0.748556128117646)
--(axis cs:545,0.73343678970291)
--(axis cs:544,0.75288627754017)
--(axis cs:543,0.744182437035279)
--(axis cs:542,0.711406437641877)
--(axis cs:541,0.683671787286275)
--(axis cs:540,0.735609046201273)
--(axis cs:539,0.733379148139261)
--(axis cs:538,0.701206762247263)
--(axis cs:537,0.721285805460962)
--(axis cs:536,0.731720840628102)
--(axis cs:535,0.721447286026499)
--(axis cs:534,0.729618077674568)
--(axis cs:533,0.723945018651989)
--(axis cs:532,0.752515443996136)
--(axis cs:531,0.707194542887342)
--(axis cs:530,0.718605361045232)
--(axis cs:529,0.740614802855987)
--(axis cs:528,0.714259325509648)
--(axis cs:527,0.680073486588587)
--(axis cs:526,0.757579497276334)
--(axis cs:525,0.72842047554389)
--(axis cs:524,0.720421589043279)
--(axis cs:523,0.734626725178271)
--(axis cs:522,0.714940147075206)
--(axis cs:521,0.750972690888315)
--(axis cs:520,0.723419084266654)
--(axis cs:519,0.732382606601233)
--(axis cs:518,0.750800292093839)
--(axis cs:517,0.688174419759543)
--(axis cs:516,0.711555531499645)
--(axis cs:515,0.71227775012322)
--(axis cs:514,0.71900529606874)
--(axis cs:513,0.679360258978793)
--(axis cs:512,0.711402783212633)
--(axis cs:511,0.770203799660558)
--(axis cs:510,0.711320113622285)
--(axis cs:509,0.731462768864595)
--(axis cs:508,0.768262826747075)
--(axis cs:507,0.761809187270781)
--(axis cs:506,0.701583175989421)
--(axis cs:505,0.715006845810826)
--(axis cs:504,0.742502219034422)
--(axis cs:503,0.70637155837328)
--(axis cs:502,0.673683357381159)
--(axis cs:501,0.719911469711252)
--(axis cs:500,0.769651599796565)
--(axis cs:499,0.696720098376764)
--(axis cs:498,0.755593632370919)
--(axis cs:497,0.700519396026999)
--(axis cs:496,0.767152460035259)
--(axis cs:495,0.756889918890857)
--(axis cs:494,0.732844390376717)
--(axis cs:493,0.741938978026142)
--(axis cs:492,0.736752260546413)
--(axis cs:491,0.722312190202028)
--(axis cs:490,0.750864160906776)
--(axis cs:489,0.738415383132681)
--(axis cs:488,0.686553507257762)
--(axis cs:487,0.736572999074374)
--(axis cs:486,0.735452019326799)
--(axis cs:485,0.739281745309654)
--(axis cs:484,0.727256567932853)
--(axis cs:483,0.747989234789302)
--(axis cs:482,0.71786468250328)
--(axis cs:481,0.747027562789385)
--(axis cs:480,0.709004946722941)
--(axis cs:479,0.715549237111959)
--(axis cs:478,0.737171135095419)
--(axis cs:477,0.718050668390092)
--(axis cs:476,0.732997276481643)
--(axis cs:475,0.738086495904546)
--(axis cs:474,0.717534102413278)
--(axis cs:473,0.712949118809537)
--(axis cs:472,0.714391151043085)
--(axis cs:471,0.725604543369688)
--(axis cs:470,0.708474941096107)
--(axis cs:469,0.692859737631935)
--(axis cs:468,0.706284518043691)
--(axis cs:467,0.683176841928527)
--(axis cs:466,0.76809596184078)
--(axis cs:465,0.68567367887979)
--(axis cs:464,0.721586244269327)
--(axis cs:463,0.766920313725906)
--(axis cs:462,0.730757689985826)
--(axis cs:461,0.735458232420105)
--(axis cs:460,0.669450004777862)
--(axis cs:459,0.762339021531068)
--(axis cs:458,0.734342178177931)
--(axis cs:457,0.730326784723527)
--(axis cs:456,0.753059959819602)
--(axis cs:455,0.713652130277684)
--(axis cs:454,0.760342319985045)
--(axis cs:453,0.729332357353286)
--(axis cs:452,0.758339561339424)
--(axis cs:451,0.733795886492794)
--(axis cs:450,0.748225586988109)
--(axis cs:449,0.739870481142116)
--(axis cs:448,0.727545361697771)
--(axis cs:447,0.7053536550714)
--(axis cs:446,0.738086998644801)
--(axis cs:445,0.760259790329319)
--(axis cs:444,0.749300472901562)
--(axis cs:443,0.741096983206117)
--(axis cs:442,0.7230513098154)
--(axis cs:441,0.695964063581777)
--(axis cs:440,0.691744410427145)
--(axis cs:439,0.701831220172483)
--(axis cs:438,0.68607434034411)
--(axis cs:437,0.764332116295626)
--(axis cs:436,0.747309293830497)
--(axis cs:435,0.749752796921)
--(axis cs:434,0.701345622803617)
--(axis cs:433,0.718360109580555)
--(axis cs:432,0.72723216927947)
--(axis cs:431,0.710521657730044)
--(axis cs:430,0.706625471015341)
--(axis cs:429,0.731672848310943)
--(axis cs:428,0.700702582451532)
--(axis cs:427,0.732800987067505)
--(axis cs:426,0.717173523840782)
--(axis cs:425,0.738928422772792)
--(axis cs:424,0.770281983825092)
--(axis cs:423,0.707138715700172)
--(axis cs:422,0.746526816145241)
--(axis cs:421,0.685179489446966)
--(axis cs:420,0.717225400777772)
--(axis cs:419,0.672816033606091)
--(axis cs:418,0.670880642581031)
--(axis cs:417,0.748562151243168)
--(axis cs:416,0.656865673536)
--(axis cs:415,0.73945152487146)
--(axis cs:414,0.72196124556528)
--(axis cs:413,0.708127770063024)
--(axis cs:412,0.72503328543161)
--(axis cs:411,0.708708806127433)
--(axis cs:410,0.694862068447535)
--(axis cs:409,0.674899037680551)
--(axis cs:408,0.722762207171078)
--(axis cs:407,0.731165219061121)
--(axis cs:406,0.699614200431768)
--(axis cs:405,0.70255966666317)
--(axis cs:404,0.715427836200702)
--(axis cs:403,0.736973851607455)
--(axis cs:402,0.670752255684195)
--(axis cs:401,0.653475923598514)
--(axis cs:400,0.700069520631495)
--(axis cs:399,0.743628941475204)
--(axis cs:398,0.705995962452865)
--(axis cs:397,0.651671711247991)
--(axis cs:396,0.728011046105308)
--(axis cs:395,0.746140214508134)
--(axis cs:394,0.722548531788833)
--(axis cs:393,0.681379082884592)
--(axis cs:392,0.698179613669169)
--(axis cs:391,0.704707484899989)
--(axis cs:390,0.68982953936362)
--(axis cs:389,0.701758407046469)
--(axis cs:388,0.677564638298455)
--(axis cs:387,0.757940860170898)
--(axis cs:386,0.714514795188412)
--(axis cs:385,0.780295187572938)
--(axis cs:384,0.758835249886702)
--(axis cs:383,0.741065967553232)
--(axis cs:382,0.727877762553731)
--(axis cs:381,0.737690220477296)
--(axis cs:380,0.705085605919959)
--(axis cs:379,0.726946314992395)
--(axis cs:378,0.706488639261857)
--(axis cs:377,0.689572005650437)
--(axis cs:376,0.753279869935599)
--(axis cs:375,0.68121378284298)
--(axis cs:374,0.710974193678877)
--(axis cs:373,0.740529062741578)
--(axis cs:372,0.690221488805271)
--(axis cs:371,0.727152526113216)
--(axis cs:370,0.713115456881232)
--(axis cs:369,0.646442693883808)
--(axis cs:368,0.72693961256772)
--(axis cs:367,0.738711468062114)
--(axis cs:366,0.699238150301547)
--(axis cs:365,0.680235239051501)
--(axis cs:364,0.740879028060572)
--(axis cs:363,0.722381990484463)
--(axis cs:362,0.734629607520893)
--(axis cs:361,0.746424056441428)
--(axis cs:360,0.709811234624279)
--(axis cs:359,0.694705886568443)
--(axis cs:358,0.715760709989853)
--(axis cs:357,0.713847327401051)
--(axis cs:356,0.712939594236494)
--(axis cs:355,0.725651061624585)
--(axis cs:354,0.705042637183389)
--(axis cs:353,0.704415776573814)
--(axis cs:352,0.759593469575508)
--(axis cs:351,0.689269542876177)
--(axis cs:350,0.737546648195665)
--(axis cs:349,0.717411953424582)
--(axis cs:348,0.67869662334598)
--(axis cs:347,0.707682384125668)
--(axis cs:346,0.728382361796009)
--(axis cs:345,0.674989583129859)
--(axis cs:344,0.703854978110531)
--(axis cs:343,0.704891286498957)
--(axis cs:342,0.695912473819404)
--(axis cs:341,0.687945031317703)
--(axis cs:340,0.730185121710305)
--(axis cs:339,0.705884527545444)
--(axis cs:338,0.711752023693972)
--(axis cs:337,0.667351063697207)
--(axis cs:336,0.729256617406225)
--(axis cs:335,0.743110168152128)
--(axis cs:334,0.698075809547951)
--(axis cs:333,0.771931989673783)
--(axis cs:332,0.740026467135394)
--(axis cs:331,0.731888335236293)
--(axis cs:330,0.684427498233089)
--(axis cs:329,0.683809564431278)
--(axis cs:328,0.734040770428911)
--(axis cs:327,0.719793686790866)
--(axis cs:326,0.688892416343274)
--(axis cs:325,0.743372546148636)
--(axis cs:324,0.728994791319774)
--(axis cs:323,0.693598662719339)
--(axis cs:322,0.702628755804029)
--(axis cs:321,0.712950602344322)
--(axis cs:320,0.7046167552637)
--(axis cs:319,0.719942763901089)
--(axis cs:318,0.676515149433842)
--(axis cs:317,0.660614004145944)
--(axis cs:316,0.692079677725384)
--(axis cs:315,0.758759934892359)
--(axis cs:314,0.746271384111278)
--(axis cs:313,0.752370505412739)
--(axis cs:312,0.764364708739093)
--(axis cs:311,0.702022128519196)
--(axis cs:310,0.683809413594736)
--(axis cs:309,0.689879467736313)
--(axis cs:308,0.728587943173107)
--(axis cs:307,0.724789102536143)
--(axis cs:306,0.725162601945068)
--(axis cs:305,0.746875307330571)
--(axis cs:304,0.733268404036866)
--(axis cs:303,0.719453148886934)
--(axis cs:302,0.680435992995321)
--(axis cs:301,0.720333654297111)
--(axis cs:300,0.728951385898273)
--(axis cs:299,0.695783030887499)
--(axis cs:298,0.712505913185914)
--(axis cs:297,0.737262145326599)
--(axis cs:296,0.724844558087608)
--(axis cs:295,0.71941158102512)
--(axis cs:294,0.709511396526127)
--(axis cs:293,0.707843761375451)
--(axis cs:292,0.707725241131735)
--(axis cs:291,0.687357382048958)
--(axis cs:290,0.684640414996444)
--(axis cs:289,0.702239505569381)
--(axis cs:288,0.728710774404823)
--(axis cs:287,0.683859692807801)
--(axis cs:286,0.698280829793968)
--(axis cs:285,0.703434793092994)
--(axis cs:284,0.708845414673409)
--(axis cs:283,0.692750706760035)
--(axis cs:282,0.629166507918207)
--(axis cs:281,0.70748009090313)
--(axis cs:280,0.650989511785023)
--(axis cs:279,0.645610172987175)
--(axis cs:278,0.685288575143442)
--(axis cs:277,0.652783267712954)
--(axis cs:276,0.710705826838142)
--(axis cs:275,0.697682189661676)
--(axis cs:274,0.679757389466122)
--(axis cs:273,0.661411836087619)
--(axis cs:272,0.713720773539688)
--(axis cs:271,0.723083588677095)
--(axis cs:270,0.690446522783699)
--(axis cs:269,0.688462866340325)
--(axis cs:268,0.719748752500466)
--(axis cs:267,0.686669351837392)
--(axis cs:266,0.686978692576809)
--(axis cs:265,0.696782117234246)
--(axis cs:264,0.703244267520145)
--(axis cs:263,0.693093094752536)
--(axis cs:262,0.692534026914215)
--(axis cs:261,0.6980565563267)
--(axis cs:260,0.675456800507199)
--(axis cs:259,0.678011401774793)
--(axis cs:258,0.703610473824355)
--(axis cs:257,0.71072878869768)
--(axis cs:256,0.74971186623835)
--(axis cs:255,0.708940051489063)
--(axis cs:254,0.67871299188452)
--(axis cs:253,0.719207105302341)
--(axis cs:252,0.709659445385859)
--(axis cs:251,0.730394199042805)
--(axis cs:250,0.706416880609859)
--(axis cs:249,0.657844561202405)
--(axis cs:248,0.674904422832284)
--(axis cs:247,0.69660584834085)
--(axis cs:246,0.710499646278966)
--(axis cs:245,0.737770118931344)
--(axis cs:244,0.710650574475018)
--(axis cs:243,0.700156254662354)
--(axis cs:242,0.724678825863343)
--(axis cs:241,0.678971911239292)
--(axis cs:240,0.687434721347092)
--(axis cs:239,0.658960842106869)
--(axis cs:238,0.696908981434222)
--(axis cs:237,0.68878745251724)
--(axis cs:236,0.673327570987459)
--(axis cs:235,0.664554056313872)
--(axis cs:234,0.684859350066965)
--(axis cs:233,0.692165208436094)
--(axis cs:232,0.700093687800939)
--(axis cs:231,0.725365339816102)
--(axis cs:230,0.627024817648685)
--(axis cs:229,0.692912132340485)
--(axis cs:228,0.666065977054119)
--(axis cs:227,0.64981534573597)
--(axis cs:226,0.652542814808697)
--(axis cs:225,0.680912131369586)
--(axis cs:224,0.670772291233972)
--(axis cs:223,0.712420077109786)
--(axis cs:222,0.685183078765476)
--(axis cs:221,0.669104332751891)
--(axis cs:220,0.672377250553316)
--(axis cs:219,0.685934780746043)
--(axis cs:218,0.695783204017724)
--(axis cs:217,0.692968358711105)
--(axis cs:216,0.669071805688585)
--(axis cs:215,0.651413065164846)
--(axis cs:214,0.663717055480317)
--(axis cs:213,0.708506743282506)
--(axis cs:212,0.74420068194636)
--(axis cs:211,0.688540863086123)
--(axis cs:210,0.685143230689662)
--(axis cs:209,0.656798118614446)
--(axis cs:208,0.638370712807895)
--(axis cs:207,0.670559796018509)
--(axis cs:206,0.665033465868508)
--(axis cs:205,0.642341166608246)
--(axis cs:204,0.635364941928722)
--(axis cs:203,0.662529338045222)
--(axis cs:202,0.702101747980422)
--(axis cs:201,0.715291395657623)
--(axis cs:200,0.691249366753664)
--(axis cs:199,0.684438130517033)
--(axis cs:198,0.648244002949062)
--(axis cs:197,0.647444203932398)
--(axis cs:196,0.688245235085302)
--(axis cs:195,0.639794130366295)
--(axis cs:194,0.687133036842394)
--(axis cs:193,0.677972633166548)
--(axis cs:192,0.679176209117555)
--(axis cs:191,0.636996355310702)
--(axis cs:190,0.656298904232886)
--(axis cs:189,0.692466003710235)
--(axis cs:188,0.682084633247532)
--(axis cs:187,0.649832959716821)
--(axis cs:186,0.648611790529624)
--(axis cs:185,0.649095277055359)
--(axis cs:184,0.702859640635729)
--(axis cs:183,0.640508309350176)
--(axis cs:182,0.652309434921518)
--(axis cs:181,0.68583093396045)
--(axis cs:180,0.605474218437919)
--(axis cs:179,0.641742201170519)
--(axis cs:178,0.593796892403502)
--(axis cs:177,0.589789401418351)
--(axis cs:176,0.577002005086155)
--(axis cs:175,0.590880294402188)
--(axis cs:174,0.586283674795192)
--(axis cs:173,0.602421531409998)
--(axis cs:172,0.613029333703786)
--(axis cs:171,0.581147781990589)
--(axis cs:170,0.607178815279745)
--(axis cs:169,0.644435862547953)
--(axis cs:168,0.600964941044777)
--(axis cs:167,0.621410227985897)
--(axis cs:166,0.596774490516795)
--(axis cs:165,0.609165851632582)
--(axis cs:164,0.627072022043079)
--(axis cs:163,0.658290769444073)
--(axis cs:162,0.61541132077707)
--(axis cs:161,0.581597980059988)
--(axis cs:160,0.637795228513816)
--(axis cs:159,0.575398328141166)
--(axis cs:158,0.61842621941662)
--(axis cs:157,0.581034401670973)
--(axis cs:156,0.640057053802215)
--(axis cs:155,0.578930785679881)
--(axis cs:154,0.598670360369847)
--(axis cs:153,0.661821430694126)
--(axis cs:152,0.588261554067408)
--(axis cs:151,0.645490796328984)
--(axis cs:150,0.633413215186363)
--(axis cs:149,0.608862012068435)
--(axis cs:148,0.644825694868916)
--(axis cs:147,0.596224270989632)
--(axis cs:146,0.65191336820273)
--(axis cs:145,0.625399934049777)
--(axis cs:144,0.656057462709581)
--(axis cs:143,0.698367045169937)
--(axis cs:142,0.638079060226867)
--(axis cs:141,0.63383137812402)
--(axis cs:140,0.621769102714076)
--(axis cs:139,0.620397475751451)
--(axis cs:138,0.658548834900838)
--(axis cs:137,0.611470446044273)
--(axis cs:136,0.670561898954009)
--(axis cs:135,0.685853199393848)
--(axis cs:134,0.641248439556668)
--(axis cs:133,0.657257817269087)
--(axis cs:132,0.68412405263122)
--(axis cs:131,0.686044225107537)
--(axis cs:130,0.636494037376439)
--(axis cs:129,0.65878847433115)
--(axis cs:128,0.65128185262285)
--(axis cs:127,0.662909024903957)
--(axis cs:126,0.685000331172099)
--(axis cs:125,0.674687962306388)
--(axis cs:124,0.654786626629219)
--(axis cs:123,0.680203659479044)
--(axis cs:122,0.645128429235614)
--(axis cs:121,0.661869739806905)
--(axis cs:120,0.6468066863999)
--(axis cs:119,0.630293343375389)
--(axis cs:118,0.627320024854197)
--(axis cs:117,0.642680560298604)
--(axis cs:116,0.618336816424744)
--(axis cs:115,0.641788063164577)
--(axis cs:114,0.619152018511013)
--(axis cs:113,0.623836450391273)
--(axis cs:112,0.598102478678715)
--(axis cs:111,0.577984408055705)
--(axis cs:110,0.590173238966679)
--(axis cs:109,0.601747825409852)
--(axis cs:108,0.600073307435145)
--(axis cs:107,0.605017689562496)
--(axis cs:106,0.631933447461622)
--(axis cs:105,0.64232339542522)
--(axis cs:104,0.647242483693942)
--(axis cs:103,0.596185996066686)
--(axis cs:102,0.592025412644707)
--(axis cs:101,0.53142431735995)
--(axis cs:100,0.587873804852384)
--(axis cs:99,0.608428817848504)
--(axis cs:98,0.643372170260376)
--(axis cs:97,0.594125556001169)
--(axis cs:96,0.584430589886786)
--(axis cs:95,0.582456026214403)
--(axis cs:94,0.545723309085247)
--(axis cs:93,0.532076480651555)
--(axis cs:92,0.569987818910425)
--(axis cs:91,0.542129578418747)
--(axis cs:90,0.541525618538692)
--(axis cs:89,0.523587432993385)
--(axis cs:88,0.51716283070797)
--(axis cs:87,0.485086889992669)
--(axis cs:86,0.502991027657132)
--(axis cs:85,0.476363169353228)
--(axis cs:84,0.480880353302152)
--(axis cs:83,0.512736926306137)
--(axis cs:82,0.473682145716972)
--(axis cs:81,0.432164998032746)
--(axis cs:80,0.468266111575586)
--(axis cs:79,0.443160267155075)
--(axis cs:78,0.499274041041732)
--(axis cs:77,0.439863710104762)
--(axis cs:76,0.40480703327762)
--(axis cs:75,0.374121161016844)
--(axis cs:74,0.386709362920359)
--(axis cs:73,0.31443691131805)
--(axis cs:72,0.355461575978802)
--(axis cs:71,0.32264856483375)
--(axis cs:70,0.302170679116247)
--(axis cs:69,0.319204065571312)
--(axis cs:68,0.249892360479751)
--(axis cs:67,0.268658373335907)
--(axis cs:66,0.288155357564381)
--(axis cs:65,0.19938004477854)
--(axis cs:64,0.212478940545822)
--(axis cs:63,0.214342194834819)
--(axis cs:62,0.195654980997252)
--(axis cs:61,0.156996649751644)
--(axis cs:60,0.154642138688513)
--(axis cs:59,0.120525345074974)
--(axis cs:58,0.136364588524259)
--(axis cs:57,0.108794187094663)
--(axis cs:56,0.114622790596758)
--(axis cs:55,0.115831662919587)
--(axis cs:54,0.114838972253901)
--(axis cs:53,0.103983984254642)
--(axis cs:52,0.0623927755723967)
--(axis cs:51,0.0859430825455415)
--(axis cs:50,0.0533880027248549)
--(axis cs:49,0.060030962454114)
--(axis cs:48,0.0573389596764901)
--(axis cs:47,0.0701317939915507)
--(axis cs:46,0.0457989548824842)
--(axis cs:45,0.0396248138454362)
--(axis cs:44,0.0382323503138044)
--(axis cs:43,0.0322657183926416)
--(axis cs:42,0.034489559673596)
--(axis cs:41,0.0225196536590233)
--(axis cs:40,0.00498471213517016)
--(axis cs:39,0.0152022326490875)
--(axis cs:38,0.0140888469342955)
--(axis cs:37,0.0112276259363221)
--(axis cs:36,0.00249985783593881)
--(axis cs:35,0.00879326172943161)
--(axis cs:34,0.0114579626203744)
--(axis cs:33,0.00912649683147959)
--(axis cs:32,0.0139093376249676)
--(axis cs:31,0.0183369581351607)
--(axis cs:30,0.0132545771349584)
--(axis cs:29,0.000207663675741061)
--(axis cs:28,0.00988307589289128)
--(axis cs:27,0.00292874100069825)
--(axis cs:26,-0.000771044072989391)
--(axis cs:25,0.00534092421695278)
--(axis cs:24,0.0126322992390018)
--(axis cs:23,-0.00816372078367799)
--(axis cs:22,0.00862169944234877)
--(axis cs:21,0.00587779041480539)
--(axis cs:20,0.0067553174173171)
--(axis cs:19,0.00867966733304823)
--(axis cs:18,0.00337300011715717)
--(axis cs:17,0.00926083410257665)
--(axis cs:16,0.00976445090896819)
--(axis cs:15,0.0105912657057658)
--(axis cs:14,0.00499811388095167)
--(axis cs:13,0.0013694751821249)
--(axis cs:12,0.00998967427692131)
--(axis cs:11,0.00715543661893279)
--(axis cs:10,0.00721556658593304)
--(axis cs:9,0.00767256296166142)
--(axis cs:8,0.00253252774725832)
--(axis cs:7,0.000187201128409279)
--(axis cs:6,0.00272270197959663)
--(axis cs:5,0.00365477814851749)
--(axis cs:4,0.000818444488984979)
--(axis cs:3,-0.00220221251300334)
--(axis cs:2,-0.000123609521763811)
--(axis cs:1,0)
--(axis cs:0,-0.000410108335200871)
--cycle;

\path [fill=color2, fill opacity=0.3] (axis cs:0,-0.00169097640834646)
--(axis cs:0,0.0114921127719828)
--(axis cs:1,0.0118268762920983)
--(axis cs:2,0.0169325622232983)
--(axis cs:3,0.00902764562352931)
--(axis cs:4,0.0179177874193454)
--(axis cs:5,0.0163793836291508)
--(axis cs:6,0.0238060911074813)
--(axis cs:7,0.0171659537915156)
--(axis cs:8,0.0181494056786938)
--(axis cs:9,0.0263506037076246)
--(axis cs:10,0.0150992219415435)
--(axis cs:11,0.0497571193378341)
--(axis cs:12,0.0717287652153059)
--(axis cs:13,0.068049703961235)
--(axis cs:14,0.0830440231742488)
--(axis cs:15,0.0766742123055612)
--(axis cs:16,0.0702548199882655)
--(axis cs:17,0.0817863529093466)
--(axis cs:18,0.0742515041646812)
--(axis cs:19,0.0495504696815526)
--(axis cs:20,0.0629750304547452)
--(axis cs:21,0.0651169026782729)
--(axis cs:22,0.0495481954382041)
--(axis cs:23,0.0566773093448171)
--(axis cs:24,0.0559281452197952)
--(axis cs:25,0.0813716260129729)
--(axis cs:26,0.0968001322310856)
--(axis cs:27,0.124230536573256)
--(axis cs:28,0.0629296466329784)
--(axis cs:29,0.110988742859271)
--(axis cs:30,0.112387596976685)
--(axis cs:31,0.0655349348503608)
--(axis cs:32,0.101140119878473)
--(axis cs:33,0.0794157336989348)
--(axis cs:34,0.14070958145887)
--(axis cs:35,0.136085943276675)
--(axis cs:36,0.119535356458616)
--(axis cs:37,0.0872074199585566)
--(axis cs:38,0.11482343127354)
--(axis cs:39,0.0898523112437024)
--(axis cs:40,0.127191390956268)
--(axis cs:41,0.14212710275034)
--(axis cs:42,0.124764837791991)
--(axis cs:43,0.148530283548039)
--(axis cs:44,0.0807758388875352)
--(axis cs:45,0.134556237028647)
--(axis cs:46,0.146653350106507)
--(axis cs:47,0.13992766332073)
--(axis cs:48,0.138470035697538)
--(axis cs:49,0.0950015501523019)
--(axis cs:50,0.123094542199551)
--(axis cs:51,0.136027690421189)
--(axis cs:52,0.145655642064886)
--(axis cs:53,0.142004233364375)
--(axis cs:54,0.14387586067605)
--(axis cs:55,0.148957099579647)
--(axis cs:56,0.114230061302879)
--(axis cs:57,0.145081078698346)
--(axis cs:58,0.152423505465286)
--(axis cs:59,0.1452142773582)
--(axis cs:60,0.126443074790003)
--(axis cs:61,0.121601484104041)
--(axis cs:62,0.185257871345577)
--(axis cs:63,0.197367816517677)
--(axis cs:64,0.150862907637001)
--(axis cs:65,0.138151228185736)
--(axis cs:66,0.166538372757172)
--(axis cs:67,0.169968833203688)
--(axis cs:68,0.123058142546051)
--(axis cs:69,0.138739779055975)
--(axis cs:70,0.163418710539563)
--(axis cs:71,0.159284817735905)
--(axis cs:72,0.154744559614325)
--(axis cs:73,0.1277736693624)
--(axis cs:74,0.137658782464735)
--(axis cs:75,0.121450068336433)
--(axis cs:76,0.172202034254192)
--(axis cs:77,0.179555209189656)
--(axis cs:78,0.196182992523073)
--(axis cs:79,0.142973814834472)
--(axis cs:80,0.159974430728253)
--(axis cs:81,0.146263978717962)
--(axis cs:82,0.158609914304527)
--(axis cs:83,0.178557594717472)
--(axis cs:84,0.1556753875244)
--(axis cs:85,0.191541562384088)
--(axis cs:86,0.176067941819625)
--(axis cs:87,0.142724365334779)
--(axis cs:88,0.1694673603588)
--(axis cs:89,0.174567923977033)
--(axis cs:90,0.175593430564253)
--(axis cs:91,0.136587595173479)
--(axis cs:92,0.173302089316034)
--(axis cs:93,0.196747077760326)
--(axis cs:94,0.171882078073079)
--(axis cs:95,0.161502985052094)
--(axis cs:96,0.210027307900014)
--(axis cs:97,0.164059814505082)
--(axis cs:98,0.189091982275782)
--(axis cs:99,0.192325769119912)
--(axis cs:100,0.201115053239112)
--(axis cs:101,0.215926452036907)
--(axis cs:102,0.220668911519403)
--(axis cs:103,0.20852419616794)
--(axis cs:104,0.23024325740053)
--(axis cs:105,0.212152025100571)
--(axis cs:106,0.166716428674731)
--(axis cs:107,0.2417513881168)
--(axis cs:108,0.245654723409275)
--(axis cs:109,0.226594907561067)
--(axis cs:110,0.245042151426606)
--(axis cs:111,0.300855209452614)
--(axis cs:112,0.277968304899343)
--(axis cs:113,0.247605172941067)
--(axis cs:114,0.278424201513836)
--(axis cs:115,0.326036915101441)
--(axis cs:116,0.24303693191979)
--(axis cs:117,0.352683231424799)
--(axis cs:118,0.285642825073821)
--(axis cs:119,0.302878620460397)
--(axis cs:120,0.393573588401776)
--(axis cs:121,0.249615115920613)
--(axis cs:122,0.32088541849958)
--(axis cs:123,0.285982192938045)
--(axis cs:124,0.352108837120674)
--(axis cs:125,0.347086124435944)
--(axis cs:126,0.337554892977104)
--(axis cs:127,0.364258141795487)
--(axis cs:128,0.335823570162518)
--(axis cs:129,0.346721615274791)
--(axis cs:130,0.336608668179136)
--(axis cs:131,0.351719751269298)
--(axis cs:132,0.364874466766869)
--(axis cs:133,0.353297372718841)
--(axis cs:134,0.42229998350385)
--(axis cs:135,0.458194185712736)
--(axis cs:136,0.420433047936317)
--(axis cs:137,0.481980158493728)
--(axis cs:138,0.434035923959596)
--(axis cs:139,0.417810137060194)
--(axis cs:140,0.395597573591963)
--(axis cs:141,0.480809545035638)
--(axis cs:142,0.442208656312754)
--(axis cs:143,0.489527461549633)
--(axis cs:144,0.47204767338758)
--(axis cs:145,0.431982602860166)
--(axis cs:146,0.478311302117788)
--(axis cs:147,0.485869179661468)
--(axis cs:148,0.489864694816127)
--(axis cs:149,0.462342208028914)
--(axis cs:150,0.526546427216149)
--(axis cs:151,0.521006050174829)
--(axis cs:152,0.496724059961349)
--(axis cs:153,0.508480853082039)
--(axis cs:154,0.546951941092075)
--(axis cs:155,0.500255139664299)
--(axis cs:156,0.524119996503704)
--(axis cs:157,0.541262354760362)
--(axis cs:158,0.56053402088636)
--(axis cs:159,0.515175010790998)
--(axis cs:160,0.515220129419687)
--(axis cs:161,0.544582428491123)
--(axis cs:162,0.559753555066503)
--(axis cs:163,0.584081975243067)
--(axis cs:164,0.574763805620325)
--(axis cs:165,0.596600074928653)
--(axis cs:166,0.574628793983737)
--(axis cs:167,0.56258434748972)
--(axis cs:168,0.585138363370304)
--(axis cs:169,0.577226933629558)
--(axis cs:170,0.584579311619693)
--(axis cs:171,0.613735148966545)
--(axis cs:172,0.599034553649861)
--(axis cs:173,0.619871031614876)
--(axis cs:174,0.631450859412816)
--(axis cs:175,0.631313285894779)
--(axis cs:176,0.599071455399938)
--(axis cs:177,0.579615555655721)
--(axis cs:178,0.659531853857292)
--(axis cs:179,0.62708053712352)
--(axis cs:180,0.593808321018065)
--(axis cs:181,0.659137651063167)
--(axis cs:182,0.656189712819873)
--(axis cs:183,0.625767434925999)
--(axis cs:184,0.653823846796745)
--(axis cs:185,0.661290680495591)
--(axis cs:186,0.654058401703632)
--(axis cs:187,0.669101027436038)
--(axis cs:188,0.685962252816572)
--(axis cs:189,0.629975066715566)
--(axis cs:190,0.653498565282392)
--(axis cs:191,0.706763913741834)
--(axis cs:192,0.668536724749002)
--(axis cs:193,0.686846337820544)
--(axis cs:194,0.693236399162586)
--(axis cs:195,0.720679687916988)
--(axis cs:196,0.719277205459824)
--(axis cs:197,0.69499205455064)
--(axis cs:198,0.684340866132118)
--(axis cs:199,0.697450905967485)
--(axis cs:200,0.708697881992851)
--(axis cs:201,0.71487444754857)
--(axis cs:202,0.682027544678105)
--(axis cs:203,0.726965339384933)
--(axis cs:204,0.666027619926344)
--(axis cs:205,0.697954923653947)
--(axis cs:206,0.736687812427061)
--(axis cs:207,0.700402901221609)
--(axis cs:208,0.701087129332099)
--(axis cs:209,0.721721661864648)
--(axis cs:210,0.725900715693941)
--(axis cs:211,0.702806634580146)
--(axis cs:212,0.748815970254367)
--(axis cs:213,0.733295977603191)
--(axis cs:214,0.730408603816706)
--(axis cs:215,0.733003852118257)
--(axis cs:216,0.786293659025268)
--(axis cs:217,0.736222826981359)
--(axis cs:218,0.729370459799142)
--(axis cs:219,0.7406327351484)
--(axis cs:220,0.783242246388161)
--(axis cs:221,0.737306124458295)
--(axis cs:222,0.687883911508043)
--(axis cs:223,0.742491338240757)
--(axis cs:224,0.755864914375744)
--(axis cs:225,0.693279468671708)
--(axis cs:226,0.760553133348431)
--(axis cs:227,0.75340207504916)
--(axis cs:228,0.777499794644813)
--(axis cs:229,0.744093684423658)
--(axis cs:230,0.769019559456173)
--(axis cs:231,0.739360226476238)
--(axis cs:232,0.771380788660259)
--(axis cs:233,0.723068221345265)
--(axis cs:234,0.752040867255925)
--(axis cs:235,0.79446679669769)
--(axis cs:236,0.762872254083327)
--(axis cs:237,0.777857163936274)
--(axis cs:238,0.791795883202511)
--(axis cs:239,0.77939766311125)
--(axis cs:240,0.73000972799175)
--(axis cs:241,0.765737749016741)
--(axis cs:242,0.738113688127031)
--(axis cs:243,0.729697162383653)
--(axis cs:244,0.756399770528697)
--(axis cs:245,0.768399037115913)
--(axis cs:246,0.786828380788425)
--(axis cs:247,0.7285758829162)
--(axis cs:248,0.765701972721135)
--(axis cs:249,0.781178544682372)
--(axis cs:250,0.732396208715512)
--(axis cs:251,0.729183647866625)
--(axis cs:252,0.752471070581692)
--(axis cs:253,0.77164240595279)
--(axis cs:254,0.741161323622522)
--(axis cs:255,0.737595001537164)
--(axis cs:256,0.784075898591998)
--(axis cs:257,0.73486552752891)
--(axis cs:258,0.744860570579659)
--(axis cs:259,0.749046410505783)
--(axis cs:260,0.787466982362123)
--(axis cs:261,0.799574086243874)
--(axis cs:262,0.7653889499118)
--(axis cs:263,0.759142642812278)
--(axis cs:264,0.806171087324301)
--(axis cs:265,0.784149643935363)
--(axis cs:266,0.773401701343249)
--(axis cs:267,0.750029024950023)
--(axis cs:268,0.771950630662828)
--(axis cs:269,0.785438935903921)
--(axis cs:270,0.770087842348086)
--(axis cs:271,0.788535902623003)
--(axis cs:272,0.775962754856017)
--(axis cs:273,0.732779023254132)
--(axis cs:274,0.801253863543714)
--(axis cs:275,0.778322687567799)
--(axis cs:276,0.767640529044329)
--(axis cs:277,0.758471986625108)
--(axis cs:278,0.780608090614871)
--(axis cs:279,0.784081627103872)
--(axis cs:280,0.788265681975294)
--(axis cs:281,0.789761895175913)
--(axis cs:282,0.801976529490361)
--(axis cs:283,0.791550846889517)
--(axis cs:284,0.796576385149061)
--(axis cs:285,0.748270814144402)
--(axis cs:286,0.796142878843813)
--(axis cs:287,0.757448877473002)
--(axis cs:288,0.792313878579603)
--(axis cs:289,0.752853119028597)
--(axis cs:290,0.781297776851868)
--(axis cs:291,0.771976946676299)
--(axis cs:292,0.777993800682639)
--(axis cs:293,0.766788444946871)
--(axis cs:294,0.789582214549136)
--(axis cs:295,0.75923804531305)
--(axis cs:296,0.790118071809399)
--(axis cs:297,0.814417675375371)
--(axis cs:298,0.761572530786379)
--(axis cs:299,0.813193562093768)
--(axis cs:300,0.787935547999159)
--(axis cs:301,0.798175284984724)
--(axis cs:302,0.80551295480874)
--(axis cs:303,0.800447539308337)
--(axis cs:304,0.76899783384502)
--(axis cs:305,0.77504058246379)
--(axis cs:306,0.788523025253675)
--(axis cs:307,0.78940599288893)
--(axis cs:308,0.785121974868406)
--(axis cs:309,0.781248717168661)
--(axis cs:310,0.811448110067998)
--(axis cs:311,0.798315844349287)
--(axis cs:312,0.753058241835847)
--(axis cs:313,0.82676620077123)
--(axis cs:314,0.786241158812566)
--(axis cs:315,0.837932121653986)
--(axis cs:316,0.789118634736788)
--(axis cs:317,0.791350597264497)
--(axis cs:318,0.775126909264908)
--(axis cs:319,0.798494390781756)
--(axis cs:320,0.777925293560444)
--(axis cs:321,0.778094154016003)
--(axis cs:322,0.760181699451463)
--(axis cs:323,0.778949224859103)
--(axis cs:324,0.775350230187828)
--(axis cs:325,0.809069587269252)
--(axis cs:326,0.774148352013406)
--(axis cs:327,0.780938758812116)
--(axis cs:328,0.771556851178583)
--(axis cs:329,0.796584363135433)
--(axis cs:330,0.771569948116805)
--(axis cs:331,0.813292952591155)
--(axis cs:332,0.799812383658011)
--(axis cs:333,0.808782570262454)
--(axis cs:334,0.76902004247537)
--(axis cs:335,0.792150033960362)
--(axis cs:336,0.781601541208608)
--(axis cs:337,0.785883551711532)
--(axis cs:338,0.773290474006647)
--(axis cs:339,0.77583519589644)
--(axis cs:340,0.784278648129956)
--(axis cs:341,0.807115079850003)
--(axis cs:342,0.778767920197619)
--(axis cs:343,0.777580241894076)
--(axis cs:344,0.778239473232929)
--(axis cs:345,0.801802123519804)
--(axis cs:346,0.782626470019458)
--(axis cs:347,0.81374776145416)
--(axis cs:348,0.812787823716065)
--(axis cs:349,0.779083742511167)
--(axis cs:350,0.781579925702019)
--(axis cs:351,0.781018104554898)
--(axis cs:352,0.800254402311557)
--(axis cs:353,0.795538581164466)
--(axis cs:354,0.799238971686867)
--(axis cs:355,0.785155896400086)
--(axis cs:356,0.810845992675176)
--(axis cs:357,0.770213762358171)
--(axis cs:358,0.814367181347486)
--(axis cs:359,0.776908432640242)
--(axis cs:360,0.781434687440358)
--(axis cs:361,0.810625243092834)
--(axis cs:362,0.779732589237698)
--(axis cs:363,0.7870732896891)
--(axis cs:364,0.803013173131349)
--(axis cs:365,0.834892343387689)
--(axis cs:366,0.772546459817306)
--(axis cs:367,0.817037368839263)
--(axis cs:368,0.789511257874343)
--(axis cs:369,0.789132654686952)
--(axis cs:370,0.812182124782191)
--(axis cs:371,0.786121184424202)
--(axis cs:372,0.788700358628001)
--(axis cs:373,0.812362262278947)
--(axis cs:374,0.817714046535276)
--(axis cs:375,0.797102593470786)
--(axis cs:376,0.803695670407216)
--(axis cs:377,0.807701461697406)
--(axis cs:378,0.776470776854072)
--(axis cs:379,0.749144747971862)
--(axis cs:380,0.79441976078643)
--(axis cs:381,0.793037427424445)
--(axis cs:382,0.766689911382408)
--(axis cs:383,0.824667893908052)
--(axis cs:384,0.778615649241492)
--(axis cs:385,0.789960030120791)
--(axis cs:386,0.807265913817699)
--(axis cs:387,0.794239466326914)
--(axis cs:388,0.801281417926478)
--(axis cs:389,0.815759118460197)
--(axis cs:390,0.817528885943405)
--(axis cs:391,0.784338244227932)
--(axis cs:392,0.744163279387018)
--(axis cs:393,0.789111782923281)
--(axis cs:394,0.786451067692371)
--(axis cs:395,0.834425231860522)
--(axis cs:396,0.797310460583971)
--(axis cs:397,0.849630985001005)
--(axis cs:398,0.817965196490995)
--(axis cs:399,0.840934177782381)
--(axis cs:400,0.826270130909161)
--(axis cs:401,0.794551366609288)
--(axis cs:402,0.797845765252569)
--(axis cs:403,0.78948827704671)
--(axis cs:404,0.822048473580022)
--(axis cs:405,0.815554004283779)
--(axis cs:406,0.81014062111213)
--(axis cs:407,0.82666239293611)
--(axis cs:408,0.80135663667091)
--(axis cs:409,0.806025373760421)
--(axis cs:410,0.824876220046316)
--(axis cs:411,0.825984795664666)
--(axis cs:412,0.823906737734451)
--(axis cs:413,0.786109271027964)
--(axis cs:414,0.791621362018052)
--(axis cs:415,0.787563583549181)
--(axis cs:416,0.782420110051094)
--(axis cs:417,0.799325260166192)
--(axis cs:418,0.79249666368636)
--(axis cs:419,0.819508436455964)
--(axis cs:420,0.80067804352395)
--(axis cs:421,0.818655339288647)
--(axis cs:422,0.812404283064786)
--(axis cs:423,0.789708072433477)
--(axis cs:424,0.849418316819499)
--(axis cs:425,0.792089518737412)
--(axis cs:426,0.774024816731321)
--(axis cs:427,0.806574155912584)
--(axis cs:428,0.779957604412127)
--(axis cs:429,0.80099061477717)
--(axis cs:430,0.797028243518084)
--(axis cs:431,0.770800053926245)
--(axis cs:432,0.811002660278547)
--(axis cs:433,0.790252344191877)
--(axis cs:434,0.800221388512332)
--(axis cs:435,0.801326086999364)
--(axis cs:436,0.791088521930885)
--(axis cs:437,0.818908119142924)
--(axis cs:438,0.807911806930546)
--(axis cs:439,0.793233259225775)
--(axis cs:440,0.798445846150861)
--(axis cs:441,0.786265182708974)
--(axis cs:442,0.795158313088806)
--(axis cs:443,0.784647302820928)
--(axis cs:444,0.792980220765386)
--(axis cs:445,0.824033520937028)
--(axis cs:446,0.832768820515457)
--(axis cs:447,0.787522469021993)
--(axis cs:448,0.792072727427403)
--(axis cs:449,0.774312258832743)
--(axis cs:450,0.757581985097867)
--(axis cs:451,0.791582346967655)
--(axis cs:452,0.768998577673078)
--(axis cs:453,0.762927766198011)
--(axis cs:454,0.816533618219175)
--(axis cs:455,0.784577373036719)
--(axis cs:456,0.775836551019852)
--(axis cs:457,0.786283809499243)
--(axis cs:458,0.749101961078821)
--(axis cs:459,0.794182932046401)
--(axis cs:460,0.801676529051869)
--(axis cs:461,0.797376560986268)
--(axis cs:462,0.811897119672503)
--(axis cs:463,0.801118876233222)
--(axis cs:464,0.780033334160361)
--(axis cs:465,0.798351396177491)
--(axis cs:466,0.788003797861751)
--(axis cs:467,0.803300835100765)
--(axis cs:468,0.794462406880927)
--(axis cs:469,0.79461246068973)
--(axis cs:470,0.832070943144175)
--(axis cs:471,0.839387364906387)
--(axis cs:472,0.825805092422084)
--(axis cs:473,0.807491050318181)
--(axis cs:474,0.811539461741385)
--(axis cs:475,0.812151643353287)
--(axis cs:476,0.780354590476113)
--(axis cs:477,0.810300598029116)
--(axis cs:478,0.81733144044679)
--(axis cs:479,0.813595109630096)
--(axis cs:480,0.794613487432626)
--(axis cs:481,0.809408501054571)
--(axis cs:482,0.817466063108659)
--(axis cs:483,0.806041814521704)
--(axis cs:484,0.829488650540721)
--(axis cs:485,0.82539477881249)
--(axis cs:486,0.84184541287605)
--(axis cs:487,0.793576090253872)
--(axis cs:488,0.830982665530597)
--(axis cs:489,0.772938195541492)
--(axis cs:490,0.797412052818351)
--(axis cs:491,0.794466040364525)
--(axis cs:492,0.826826887637274)
--(axis cs:493,0.791587105221118)
--(axis cs:494,0.802726755695995)
--(axis cs:495,0.831497290561116)
--(axis cs:496,0.798257876690603)
--(axis cs:497,0.831794085828461)
--(axis cs:498,0.790373601529062)
--(axis cs:499,0.824333362102995)
--(axis cs:500,0.795547729875281)
--(axis cs:501,0.796300519295383)
--(axis cs:502,0.799770036345542)
--(axis cs:503,0.80345313493873)
--(axis cs:504,0.769548456165696)
--(axis cs:505,0.806493416298173)
--(axis cs:506,0.801574378345336)
--(axis cs:507,0.824074750193027)
--(axis cs:508,0.812668056573508)
--(axis cs:509,0.820910778795822)
--(axis cs:510,0.827882505383569)
--(axis cs:511,0.842598176988298)
--(axis cs:512,0.808700773642864)
--(axis cs:513,0.819059760142634)
--(axis cs:514,0.820581931301642)
--(axis cs:515,0.787549838299758)
--(axis cs:516,0.777902477383213)
--(axis cs:517,0.790324965695527)
--(axis cs:518,0.819957783390807)
--(axis cs:519,0.824052751896612)
--(axis cs:520,0.843593511711739)
--(axis cs:521,0.780447773970965)
--(axis cs:522,0.78228659916362)
--(axis cs:523,0.775779610100093)
--(axis cs:524,0.803897449947799)
--(axis cs:525,0.798912401514729)
--(axis cs:526,0.785465127737267)
--(axis cs:527,0.78605527600112)
--(axis cs:528,0.778941130895729)
--(axis cs:529,0.814235270158497)
--(axis cs:530,0.816016650552985)
--(axis cs:531,0.820330894610239)
--(axis cs:532,0.839783163414163)
--(axis cs:533,0.801653105793455)
--(axis cs:534,0.809456581455242)
--(axis cs:535,0.798735922066098)
--(axis cs:536,0.767184587886526)
--(axis cs:537,0.778220739768104)
--(axis cs:538,0.81262801821487)
--(axis cs:539,0.835151094062503)
--(axis cs:540,0.790731983260543)
--(axis cs:541,0.821035232093269)
--(axis cs:542,0.821784973331534)
--(axis cs:543,0.830624603548725)
--(axis cs:544,0.822531227742328)
--(axis cs:545,0.799398396984241)
--(axis cs:546,0.805305198013901)
--(axis cs:547,0.830957629773504)
--(axis cs:548,0.83507317049554)
--(axis cs:549,0.771013918330538)
--(axis cs:550,0.817289104937974)
--(axis cs:551,0.820531350168089)
--(axis cs:552,0.797229093542992)
--(axis cs:553,0.808533369348086)
--(axis cs:554,0.806584570342985)
--(axis cs:555,0.806329816164388)
--(axis cs:556,0.807305729667172)
--(axis cs:557,0.793016772062756)
--(axis cs:558,0.816613780379074)
--(axis cs:559,0.806834787951231)
--(axis cs:560,0.809054158650479)
--(axis cs:561,0.790152461592458)
--(axis cs:562,0.808689098788942)
--(axis cs:563,0.796890604328531)
--(axis cs:564,0.803845418752)
--(axis cs:565,0.829438218688947)
--(axis cs:566,0.770537428354662)
--(axis cs:567,0.829732454445147)
--(axis cs:568,0.830602104595136)
--(axis cs:569,0.797432464992039)
--(axis cs:570,0.820046029672241)
--(axis cs:571,0.810450602845107)
--(axis cs:572,0.82663930731487)
--(axis cs:573,0.814302462576719)
--(axis cs:574,0.824152837584641)
--(axis cs:575,0.833111915964098)
--(axis cs:576,0.836969638639048)
--(axis cs:577,0.788454405616359)
--(axis cs:578,0.778831672685003)
--(axis cs:579,0.804987794618297)
--(axis cs:580,0.794417640702561)
--(axis cs:581,0.808396564581563)
--(axis cs:582,0.822872130273781)
--(axis cs:583,0.808077405213172)
--(axis cs:584,0.809250968694131)
--(axis cs:585,0.821170394629343)
--(axis cs:586,0.794283521395169)
--(axis cs:587,0.813658618156733)
--(axis cs:588,0.777586026894897)
--(axis cs:589,0.807534833610887)
--(axis cs:590,0.808304308835129)
--(axis cs:591,0.822110391222153)
--(axis cs:592,0.833190627859461)
--(axis cs:593,0.782217559407546)
--(axis cs:594,0.793975642040212)
--(axis cs:595,0.821256618585321)
--(axis cs:596,0.821117117810092)
--(axis cs:597,0.802454738188934)
--(axis cs:598,0.815925214303305)
--(axis cs:599,0.827775008413548)
--(axis cs:599,0.752624453235914)
--(axis cs:599,0.752624453235914)
--(axis cs:598,0.735276535335945)
--(axis cs:597,0.726096571612375)
--(axis cs:596,0.739596127278153)
--(axis cs:595,0.723582095600819)
--(axis cs:594,0.696760444789624)
--(axis cs:593,0.69879838645215)
--(axis cs:592,0.739033935771353)
--(axis cs:591,0.729983261652749)
--(axis cs:590,0.72815862059655)
--(axis cs:589,0.72104799848757)
--(axis cs:588,0.679330575701852)
--(axis cs:587,0.736061106562991)
--(axis cs:586,0.694259879648232)
--(axis cs:585,0.736320378486429)
--(axis cs:584,0.731949931100518)
--(axis cs:583,0.718788628569222)
--(axis cs:582,0.753436717128815)
--(axis cs:581,0.73542627924128)
--(axis cs:580,0.699227169348498)
--(axis cs:579,0.713584436867537)
--(axis cs:578,0.687847221337641)
--(axis cs:577,0.699783138089934)
--(axis cs:576,0.758171765096104)
--(axis cs:575,0.761584359982178)
--(axis cs:574,0.751584099089796)
--(axis cs:573,0.725998423220491)
--(axis cs:572,0.732139153807341)
--(axis cs:571,0.729914847676594)
--(axis cs:570,0.728992431866221)
--(axis cs:569,0.718829875098426)
--(axis cs:568,0.746884777395422)
--(axis cs:567,0.748736559180116)
--(axis cs:566,0.685292115365506)
--(axis cs:565,0.73883109879912)
--(axis cs:564,0.716125687041716)
--(axis cs:563,0.695271140176963)
--(axis cs:562,0.722118374987281)
--(axis cs:561,0.680863404689952)
--(axis cs:560,0.719748202183131)
--(axis cs:559,0.720328586868393)
--(axis cs:558,0.723354627948709)
--(axis cs:557,0.714534140566282)
--(axis cs:556,0.732658169281101)
--(axis cs:555,0.729273357214521)
--(axis cs:554,0.721948701940286)
--(axis cs:553,0.718106050260084)
--(axis cs:552,0.71254053582682)
--(axis cs:551,0.746933606749993)
--(axis cs:550,0.72370340949204)
--(axis cs:549,0.69048604004442)
--(axis cs:548,0.764518557221188)
--(axis cs:547,0.763448439035139)
--(axis cs:546,0.71778924226804)
--(axis cs:545,0.722040407267063)
--(axis cs:544,0.747054308249457)
--(axis cs:543,0.742553173535301)
--(axis cs:542,0.755433858184172)
--(axis cs:541,0.725754913525001)
--(axis cs:540,0.700382024259715)
--(axis cs:539,0.760197914567754)
--(axis cs:538,0.72589013744483)
--(axis cs:537,0.697993714180725)
--(axis cs:536,0.660256330570018)
--(axis cs:535,0.71404903603136)
--(axis cs:534,0.72939892068151)
--(axis cs:533,0.715509245743896)
--(axis cs:532,0.76692583026029)
--(axis cs:531,0.736122769678902)
--(axis cs:530,0.723780878307043)
--(axis cs:529,0.724621359323132)
--(axis cs:528,0.701463100289752)
--(axis cs:527,0.696990237513143)
--(axis cs:526,0.701573181176042)
--(axis cs:525,0.717751186633234)
--(axis cs:524,0.708923062872713)
--(axis cs:523,0.672913582940527)
--(axis cs:522,0.719608250778105)
--(axis cs:521,0.688683138519322)
--(axis cs:520,0.756013391801414)
--(axis cs:519,0.739121986967465)
--(axis cs:518,0.73308079363674)
--(axis cs:517,0.715152282750471)
--(axis cs:516,0.692084610181999)
--(axis cs:515,0.702374213343044)
--(axis cs:514,0.734310401141241)
--(axis cs:513,0.727895021187147)
--(axis cs:512,0.737613112124147)
--(axis cs:511,0.780938962955091)
--(axis cs:510,0.753394482028529)
--(axis cs:509,0.721807203283373)
--(axis cs:508,0.728773314555363)
--(axis cs:507,0.754212544109892)
--(axis cs:506,0.714296556275598)
--(axis cs:505,0.735879618652986)
--(axis cs:504,0.667888846115356)
--(axis cs:503,0.706853919118324)
--(axis cs:502,0.724532216242556)
--(axis cs:501,0.705964403282039)
--(axis cs:500,0.724365343162792)
--(axis cs:499,0.74645895322057)
--(axis cs:498,0.703165801775966)
--(axis cs:497,0.750348597876722)
--(axis cs:496,0.716026146577796)
--(axis cs:495,0.746903215183141)
--(axis cs:494,0.732528613934374)
--(axis cs:493,0.698402809378172)
--(axis cs:492,0.752099394414008)
--(axis cs:491,0.696658373416138)
--(axis cs:490,0.69795880895876)
--(axis cs:489,0.68038179176287)
--(axis cs:488,0.750338790790859)
--(axis cs:487,0.707655750821719)
--(axis cs:486,0.739948319331285)
--(axis cs:485,0.735513930533719)
--(axis cs:484,0.754099467835999)
--(axis cs:483,0.728833813322673)
--(axis cs:482,0.741414032490186)
--(axis cs:481,0.721435481351911)
--(axis cs:480,0.697443240561602)
--(axis cs:479,0.709313627364118)
--(axis cs:478,0.752092885227535)
--(axis cs:477,0.737402219986202)
--(axis cs:476,0.690508227261704)
--(axis cs:475,0.734380210209816)
--(axis cs:474,0.740905575859902)
--(axis cs:473,0.73375581045993)
--(axis cs:472,0.73744060285486)
--(axis cs:471,0.765199194367672)
--(axis cs:470,0.746929136637155)
--(axis cs:469,0.698208000099481)
--(axis cs:468,0.711575131969112)
--(axis cs:467,0.734275824350894)
--(axis cs:466,0.704471190385296)
--(axis cs:465,0.721241944509599)
--(axis cs:464,0.682885864696338)
--(axis cs:463,0.715482968102997)
--(axis cs:462,0.721603581015697)
--(axis cs:461,0.719254637676181)
--(axis cs:460,0.713076686482597)
--(axis cs:459,0.696610909528691)
--(axis cs:458,0.661077044964155)
--(axis cs:457,0.698906217487659)
--(axis cs:456,0.698895676515684)
--(axis cs:455,0.707662591859496)
--(axis cs:454,0.743616561462255)
--(axis cs:453,0.672691042077047)
--(axis cs:452,0.680290715550039)
--(axis cs:451,0.699775340483783)
--(axis cs:450,0.660340826739418)
--(axis cs:449,0.68775050479252)
--(axis cs:448,0.704733764691589)
--(axis cs:447,0.68130778926139)
--(axis cs:446,0.764701812804761)
--(axis cs:445,0.756003629412622)
--(axis cs:444,0.704296460886295)
--(axis cs:443,0.706007273403501)
--(axis cs:442,0.704027743909751)
--(axis cs:441,0.707353004659213)
--(axis cs:440,0.710943549176034)
--(axis cs:439,0.716818900114068)
--(axis cs:438,0.72655181972058)
--(axis cs:437,0.730824960277656)
--(axis cs:436,0.713877345534983)
--(axis cs:435,0.712469362342961)
--(axis cs:434,0.71431043383199)
--(axis cs:433,0.700756651918222)
--(axis cs:432,0.716221417726781)
--(axis cs:431,0.675805987601672)
--(axis cs:430,0.701594921511331)
--(axis cs:429,0.717199181537626)
--(axis cs:428,0.667612487873038)
--(axis cs:427,0.710317546135368)
--(axis cs:426,0.687899543630539)
--(axis cs:425,0.70119050852699)
--(axis cs:424,0.78268772817717)
--(axis cs:423,0.705427694577289)
--(axis cs:422,0.729670785963408)
--(axis cs:421,0.732460385264577)
--(axis cs:420,0.724345579707394)
--(axis cs:419,0.725921169681363)
--(axis cs:418,0.698934214463267)
--(axis cs:417,0.720773824776644)
--(axis cs:416,0.689195340470607)
--(axis cs:415,0.6957837982357)
--(axis cs:414,0.718291529420574)
--(axis cs:413,0.68406360251662)
--(axis cs:412,0.745218285526601)
--(axis cs:411,0.743632637555709)
--(axis cs:410,0.751751449159478)
--(axis cs:409,0.706330586251789)
--(axis cs:408,0.714243691820043)
--(axis cs:407,0.734141917132905)
--(axis cs:406,0.734159566547433)
--(axis cs:405,0.73475814016899)
--(axis cs:404,0.732843466077542)
--(axis cs:403,0.703276505148219)
--(axis cs:402,0.721991446657254)
--(axis cs:401,0.708059130532459)
--(axis cs:400,0.76209605764085)
--(axis cs:399,0.769436441181987)
--(axis cs:398,0.736379330509783)
--(axis cs:397,0.764193471792201)
--(axis cs:396,0.718029239544332)
--(axis cs:395,0.750820338541298)
--(axis cs:394,0.699022616609438)
--(axis cs:393,0.708491083981057)
--(axis cs:392,0.648240020438156)
--(axis cs:391,0.694587647588584)
--(axis cs:390,0.732614807169037)
--(axis cs:389,0.735146807721464)
--(axis cs:388,0.698042955616646)
--(axis cs:387,0.703172583428885)
--(axis cs:386,0.713259827801793)
--(axis cs:385,0.698627250653989)
--(axis cs:384,0.6947200428442)
--(axis cs:383,0.724534420321835)
--(axis cs:382,0.673460563143066)
--(axis cs:381,0.703176815848621)
--(axis cs:380,0.706873043631375)
--(axis cs:379,0.629817964716421)
--(axis cs:378,0.701130094496799)
--(axis cs:377,0.716789281933962)
--(axis cs:376,0.711070899828104)
--(axis cs:375,0.708508309107028)
--(axis cs:374,0.722498345239615)
--(axis cs:373,0.746565441961257)
--(axis cs:372,0.707553017929651)
--(axis cs:371,0.700973792698348)
--(axis cs:370,0.712387604406104)
--(axis cs:369,0.687670013131341)
--(axis cs:368,0.684256901050065)
--(axis cs:367,0.731429355408711)
--(axis cs:366,0.687515786963691)
--(axis cs:365,0.741754057086836)
--(axis cs:364,0.726610935399009)
--(axis cs:363,0.695700013396703)
--(axis cs:362,0.681509640754532)
--(axis cs:361,0.733824404481813)
--(axis cs:360,0.695411176560837)
--(axis cs:359,0.696043487784105)
--(axis cs:358,0.73241251465096)
--(axis cs:357,0.683284511222823)
--(axis cs:356,0.710483242325933)
--(axis cs:355,0.704669729363039)
--(axis cs:354,0.718564313940044)
--(axis cs:353,0.714666318102933)
--(axis cs:352,0.724759646139991)
--(axis cs:351,0.681911890796972)
--(axis cs:350,0.693703558175215)
--(axis cs:349,0.681508052846254)
--(axis cs:348,0.722628669512928)
--(axis cs:347,0.729753962516314)
--(axis cs:346,0.68351300088975)
--(axis cs:345,0.696722095316915)
--(axis cs:344,0.680850568669613)
--(axis cs:343,0.678906066935983)
--(axis cs:342,0.678472883320757)
--(axis cs:341,0.739185969247097)
--(axis cs:340,0.683824342629285)
--(axis cs:339,0.676710262117768)
--(axis cs:338,0.691000096533923)
--(axis cs:337,0.708411129301899)
--(axis cs:336,0.684881566917785)
--(axis cs:335,0.71063103913321)
--(axis cs:334,0.686262608901031)
--(axis cs:333,0.709739257174814)
--(axis cs:332,0.723527817807191)
--(axis cs:331,0.715743549101596)
--(axis cs:330,0.667406873672516)
--(axis cs:329,0.706276924882105)
--(axis cs:328,0.686379960273853)
--(axis cs:327,0.680946968307986)
--(axis cs:326,0.679029199601645)
--(axis cs:325,0.71899245485529)
--(axis cs:324,0.687700434425337)
--(axis cs:323,0.693310890944212)
--(axis cs:322,0.667592446337388)
--(axis cs:321,0.682649702821604)
--(axis cs:320,0.685269594680399)
--(axis cs:319,0.724917440442575)
--(axis cs:318,0.67935193501173)
--(axis cs:317,0.72787202039562)
--(axis cs:316,0.707845165352013)
--(axis cs:315,0.76804124525688)
--(axis cs:314,0.69969165837025)
--(axis cs:313,0.737038917721388)
--(axis cs:312,0.656859054409574)
--(axis cs:311,0.702663818349126)
--(axis cs:310,0.73154500792512)
--(axis cs:309,0.698764065187871)
--(axis cs:308,0.683169889517208)
--(axis cs:307,0.694195572008591)
--(axis cs:306,0.714103514872865)
--(axis cs:305,0.680534755188753)
--(axis cs:304,0.675548418513732)
--(axis cs:303,0.716189816391518)
--(axis cs:302,0.732292831112854)
--(axis cs:301,0.726380357493712)
--(axis cs:300,0.676830762079651)
--(axis cs:299,0.734539737592656)
--(axis cs:298,0.668616163280991)
--(axis cs:297,0.735335262658817)
--(axis cs:296,0.691050483899598)
--(axis cs:295,0.637402120077115)
--(axis cs:294,0.694813077658656)
--(axis cs:293,0.67749010636293)
--(axis cs:292,0.67452247389246)
--(axis cs:291,0.66567687978727)
--(axis cs:290,0.695269839559498)
--(axis cs:289,0.650459645509496)
--(axis cs:288,0.685251907350154)
--(axis cs:287,0.667851429858555)
--(axis cs:286,0.696991113899554)
--(axis cs:285,0.637785317224229)
--(axis cs:284,0.699454702839747)
--(axis cs:283,0.679311277097606)
--(axis cs:282,0.705089451403745)
--(axis cs:281,0.688571837064068)
--(axis cs:280,0.697212231145082)
--(axis cs:279,0.699547608754224)
--(axis cs:278,0.687407487556957)
--(axis cs:277,0.673350964729093)
--(axis cs:276,0.666087331636656)
--(axis cs:275,0.691037682964446)
--(axis cs:274,0.706459863701263)
--(axis cs:273,0.642374729243371)
--(axis cs:272,0.689477447302935)
--(axis cs:271,0.702510550923451)
--(axis cs:270,0.670062632177388)
--(axis cs:269,0.691613438804703)
--(axis cs:268,0.682297184430575)
--(axis cs:267,0.65711577061389)
--(axis cs:266,0.68812969261002)
--(axis cs:265,0.688930010784917)
--(axis cs:264,0.72382496849896)
--(axis cs:263,0.656796228365564)
--(axis cs:262,0.678625869011106)
--(axis cs:261,0.694757536993999)
--(axis cs:260,0.685040848348832)
--(axis cs:259,0.642795211413963)
--(axis cs:258,0.644255998615035)
--(axis cs:257,0.633800921422466)
--(axis cs:256,0.684310711626559)
--(axis cs:255,0.624356675258263)
--(axis cs:254,0.646363261167688)
--(axis cs:253,0.676580725082841)
--(axis cs:252,0.666346396673276)
--(axis cs:251,0.625564598678497)
--(axis cs:250,0.619754673490517)
--(axis cs:249,0.692221242825872)
--(axis cs:248,0.666635211132593)
--(axis cs:247,0.596294033727614)
--(axis cs:246,0.689453583774789)
--(axis cs:245,0.674342813047812)
--(axis cs:244,0.65510443706926)
--(axis cs:243,0.63942100498819)
--(axis cs:242,0.633697027709421)
--(axis cs:241,0.684371560076943)
--(axis cs:240,0.630510741122469)
--(axis cs:239,0.704322704042021)
--(axis cs:238,0.692513774154021)
--(axis cs:237,0.690269110830626)
--(axis cs:236,0.66265602176664)
--(axis cs:235,0.704702666396405)
--(axis cs:234,0.664724120717703)
--(axis cs:233,0.625866871339828)
--(axis cs:232,0.670309672124871)
--(axis cs:231,0.635565391582604)
--(axis cs:230,0.677540304291191)
--(axis cs:229,0.644758088803115)
--(axis cs:228,0.684686369885101)
--(axis cs:227,0.65046055537597)
--(axis cs:226,0.648313780987233)
--(axis cs:225,0.587315717392228)
--(axis cs:224,0.649929248059043)
--(axis cs:223,0.63208850899534)
--(axis cs:222,0.570874170687539)
--(axis cs:221,0.630483744985985)
--(axis cs:220,0.685281070572656)
--(axis cs:219,0.636993197946283)
--(axis cs:218,0.621177993483686)
--(axis cs:217,0.618823760418169)
--(axis cs:216,0.678306931356572)
--(axis cs:215,0.629628567545413)
--(axis cs:214,0.611596715210672)
--(axis cs:213,0.628310234167495)
--(axis cs:212,0.637708616270219)
--(axis cs:211,0.580783039634529)
--(axis cs:210,0.615035150931263)
--(axis cs:209,0.60859606983176)
--(axis cs:208,0.593023317512722)
--(axis cs:207,0.571292162293307)
--(axis cs:206,0.633628393527932)
--(axis cs:205,0.577827810768494)
--(axis cs:204,0.544225826586294)
--(axis cs:203,0.606223547751322)
--(axis cs:202,0.564877282037577)
--(axis cs:201,0.60312825515808)
--(axis cs:200,0.605840731805631)
--(axis cs:199,0.569285681587117)
--(axis cs:198,0.573097579817357)
--(axis cs:197,0.570261720365964)
--(axis cs:196,0.606267105723274)
--(axis cs:195,0.595254420705298)
--(axis cs:194,0.568782094496533)
--(axis cs:193,0.56268494686699)
--(axis cs:192,0.536672866007464)
--(axis cs:191,0.585499403624279)
--(axis cs:190,0.52686334711077)
--(axis cs:189,0.505087960534961)
--(axis cs:188,0.556278102314409)
--(axis cs:187,0.552570183297672)
--(axis cs:186,0.532890365446616)
--(axis cs:185,0.54019725043313)
--(axis cs:184,0.522136543915484)
--(axis cs:183,0.496810433998377)
--(axis cs:182,0.513580751091216)
--(axis cs:181,0.525577763730373)
--(axis cs:180,0.47133660350186)
--(axis cs:179,0.488836018585992)
--(axis cs:178,0.523246529126684)
--(axis cs:177,0.445931520327936)
--(axis cs:176,0.464706563456206)
--(axis cs:175,0.508859317237383)
--(axis cs:174,0.500367870774481)
--(axis cs:173,0.498764114118519)
--(axis cs:172,0.452681017849718)
--(axis cs:171,0.472387023897356)
--(axis cs:170,0.456343660478298)
--(axis cs:169,0.433001716618674)
--(axis cs:168,0.454131088582972)
--(axis cs:167,0.4340136725458)
--(axis cs:166,0.440526276640084)
--(axis cs:165,0.440132680163477)
--(axis cs:164,0.439468035863354)
--(axis cs:163,0.4300426001407)
--(axis cs:162,0.422066544768303)
--(axis cs:161,0.405100700879507)
--(axis cs:160,0.383597747454256)
--(axis cs:159,0.36870976762503)
--(axis cs:158,0.427436743997621)
--(axis cs:157,0.3959337977391)
--(axis cs:156,0.376515430671245)
--(axis cs:155,0.354350348029897)
--(axis cs:154,0.385308360740596)
--(axis cs:153,0.355532824212889)
--(axis cs:152,0.341048696285672)
--(axis cs:151,0.368218093736814)
--(axis cs:150,0.376002369473273)
--(axis cs:149,0.318576462653689)
--(axis cs:148,0.343744389651607)
--(axis cs:147,0.342435545127632)
--(axis cs:146,0.319564259820274)
--(axis cs:145,0.301017324602078)
--(axis cs:144,0.327074262682886)
--(axis cs:143,0.344880904798071)
--(axis cs:142,0.303298525059365)
--(axis cs:141,0.347660422496829)
--(axis cs:140,0.259148516071606)
--(axis cs:139,0.267661335911279)
--(axis cs:138,0.276964794072372)
--(axis cs:137,0.335536237997852)
--(axis cs:136,0.279480527614594)
--(axis cs:135,0.315191976735927)
--(axis cs:134,0.272450838590722)
--(axis cs:133,0.216203265531797)
--(axis cs:132,0.220976658721205)
--(axis cs:131,0.209568816919908)
--(axis cs:130,0.202495811389498)
--(axis cs:129,0.209976675201073)
--(axis cs:128,0.199208411744464)
--(axis cs:127,0.230884543870297)
--(axis cs:126,0.201071747947331)
--(axis cs:125,0.215566959816405)
--(axis cs:124,0.219395526571189)
--(axis cs:123,0.176543613567541)
--(axis cs:122,0.194661404469118)
--(axis cs:121,0.136739458324586)
--(axis cs:120,0.24423495721927)
--(axis cs:119,0.169224493439592)
--(axis cs:118,0.167094541726045)
--(axis cs:117,0.211597149652457)
--(axis cs:116,0.136095003852771)
--(axis cs:115,0.190417519965126)
--(axis cs:114,0.144736568475978)
--(axis cs:113,0.13274660277416)
--(axis cs:112,0.148705077886172)
--(axis cs:111,0.171132170508111)
--(axis cs:110,0.136285064267617)
--(axis cs:109,0.108386594854646)
--(axis cs:108,0.122362415701614)
--(axis cs:107,0.118163785396477)
--(axis cs:106,0.0714335421619659)
--(axis cs:105,0.0906734480374026)
--(axis cs:104,0.10519630303903)
--(axis cs:103,0.10207689085778)
--(axis cs:102,0.107683332641665)
--(axis cs:101,0.106981223926835)
--(axis cs:100,0.0940035434419844)
--(axis cs:99,0.0941649970583544)
--(axis cs:98,0.0731778659315666)
--(axis cs:97,0.0687558507324579)
--(axis cs:96,0.106906506345381)
--(axis cs:95,0.0692448261644676)
--(axis cs:94,0.0691710783293425)
--(axis cs:93,0.0836059842103832)
--(axis cs:92,0.0632805121103172)
--(axis cs:91,0.0535628135745496)
--(axis cs:90,0.0768282796699571)
--(axis cs:89,0.0706503024452372)
--(axis cs:88,0.0658755952966562)
--(axis cs:87,0.0524154341018953)
--(axis cs:86,0.0715670620653794)
--(axis cs:85,0.081204859902776)
--(axis cs:84,0.0669526788536668)
--(axis cs:83,0.072018295948879)
--(axis cs:82,0.0633499859871746)
--(axis cs:81,0.0549371715410929)
--(axis cs:80,0.0667278600365379)
--(axis cs:79,0.0490497891146835)
--(axis cs:78,0.0793409524861664)
--(axis cs:77,0.068230685471239)
--(axis cs:76,0.0730473344326764)
--(axis cs:75,0.0498016400246505)
--(axis cs:74,0.0498052702953706)
--(axis cs:73,0.039411808048261)
--(axis cs:72,0.058066414134149)
--(axis cs:71,0.0574460234204872)
--(axis cs:70,0.0616968372092573)
--(axis cs:69,0.0578288381927895)
--(axis cs:68,0.0535175030093737)
--(axis cs:67,0.0780736624950577)
--(axis cs:66,0.0556223913153565)
--(axis cs:65,0.046055764270337)
--(axis cs:64,0.0590331707713054)
--(axis cs:63,0.0878847503006509)
--(axis cs:62,0.0714183950963144)
--(axis cs:61,0.0426253480915412)
--(axis cs:60,0.0462675132330848)
--(axis cs:59,0.0549283751282029)
--(axis cs:58,0.0535175043118308)
--(axis cs:57,0.0543066863144191)
--(axis cs:56,0.0328334118865136)
--(axis cs:55,0.0656121533530928)
--(axis cs:54,0.0599984767920373)
--(axis cs:53,0.0511824257689752)
--(axis cs:52,0.0498113909021471)
--(axis cs:51,0.050393137015263)
--(axis cs:50,0.0355658945858854)
--(axis cs:49,0.0283956255886239)
--(axis cs:48,0.0548124630537107)
--(axis cs:47,0.0365392574396693)
--(axis cs:46,0.0426412095005526)
--(axis cs:45,0.0449032599533437)
--(axis cs:44,0.0159395984529021)
--(axis cs:43,0.0410085196626388)
--(axis cs:42,0.032538622636469)
--(axis cs:41,0.0510768287348411)
--(axis cs:40,0.0299345560134287)
--(axis cs:39,0.00936305672166553)
--(axis cs:38,0.0238358261357173)
--(axis cs:37,0.0179345734646868)
--(axis cs:36,0.0355285032927434)
--(axis cs:35,0.0408759836852515)
--(axis cs:34,0.0525769930777049)
--(axis cs:33,0.0205487115780105)
--(axis cs:32,0.027406507105654)
--(axis cs:31,0.0114077099231664)
--(axis cs:30,0.0174606797465918)
--(axis cs:29,0.0271583531003246)
--(axis cs:28,0.00718444066860887)
--(axis cs:27,0.0350178174119402)
--(axis cs:26,0.0227505427883394)
--(axis cs:25,0.0226488326949858)
--(axis cs:24,0.011811605832134)
--(axis cs:23,0.0108316573828997)
--(axis cs:22,0.0117500549225463)
--(axis cs:21,0.00745028707641687)
--(axis cs:20,0.0174899600286277)
--(axis cs:19,-0.00258866412599706)
--(axis cs:18,0.00344096688448321)
--(axis cs:17,0.0114568622088685)
--(axis cs:16,0.00996488626544816)
--(axis cs:15,0.0104308522581431)
--(axis cs:14,0.0197275824845851)
--(axis cs:13,0.00724791508638406)
--(axis cs:12,0.00837605133252537)
--(axis cs:11,0.00769632131873154)
--(axis cs:10,-0.00158516247160168)
--(axis cs:9,0.00285439268487179)
--(axis cs:8,0.00194608043241726)
--(axis cs:7,-0.00258262045818231)
--(axis cs:6,0.00173503443364427)
--(axis cs:5,0.00263229999166597)
--(axis cs:4,-0.00102452443233241)
--(axis cs:3,-0.000592188107189438)
--(axis cs:2,0.000446476122112287)
--(axis cs:1,0.00179245850322334)
--(axis cs:0,-0.00169097640834646)
--cycle;

\path [fill=color3, fill opacity=0.3] (axis cs:0,-0.000922487513312963)
--(axis cs:0,0.00287561251331296)
--(axis cs:1,0.0103457904898522)
--(axis cs:2,0.0113037054046848)
--(axis cs:3,0.0226118156996624)
--(axis cs:4,0.00924697944081956)
--(axis cs:5,0.0228762083267923)
--(axis cs:6,0.0333181604693584)
--(axis cs:7,0.0575200189527107)
--(axis cs:8,0.0474961069650268)
--(axis cs:9,0.0292960904166398)
--(axis cs:10,0.0649316946700002)
--(axis cs:11,0.0531114657478403)
--(axis cs:12,0.0689669589556661)
--(axis cs:13,0.0653081580920611)
--(axis cs:14,0.0774012558159464)
--(axis cs:15,0.0733187126727394)
--(axis cs:16,0.0944539512768733)
--(axis cs:17,0.0913230975207036)
--(axis cs:18,0.0531678036431022)
--(axis cs:19,0.0959347628609962)
--(axis cs:20,0.0617608705807584)
--(axis cs:21,0.0809833281011448)
--(axis cs:22,0.117155674814962)
--(axis cs:23,0.0954608100124769)
--(axis cs:24,0.114893577426067)
--(axis cs:25,0.154494659466935)
--(axis cs:26,0.137289162827972)
--(axis cs:27,0.169070508344479)
--(axis cs:28,0.13785500761859)
--(axis cs:29,0.123326677525253)
--(axis cs:30,0.140578704884217)
--(axis cs:31,0.101885815541691)
--(axis cs:32,0.135517230926543)
--(axis cs:33,0.0975472497757324)
--(axis cs:34,0.107171863299544)
--(axis cs:35,0.153035443305925)
--(axis cs:36,0.101369033908054)
--(axis cs:37,0.119509686717628)
--(axis cs:38,0.131242957927589)
--(axis cs:39,0.112221279444229)
--(axis cs:40,0.125517461631)
--(axis cs:41,0.101295446553685)
--(axis cs:42,0.109177338773711)
--(axis cs:43,0.122807982153387)
--(axis cs:44,0.103988301161651)
--(axis cs:45,0.0599313662982832)
--(axis cs:46,0.0942992828724209)
--(axis cs:47,0.0925637339066689)
--(axis cs:48,0.101159683993536)
--(axis cs:49,0.0701995533225892)
--(axis cs:50,0.120786936086997)
--(axis cs:51,0.124232638791467)
--(axis cs:52,0.181599380883603)
--(axis cs:53,0.137210421601029)
--(axis cs:54,0.138013338774187)
--(axis cs:55,0.127016406920032)
--(axis cs:56,0.138691301672063)
--(axis cs:57,0.109165701644072)
--(axis cs:58,0.0897726866285475)
--(axis cs:59,0.098267393776901)
--(axis cs:60,0.110097600675377)
--(axis cs:61,0.125709176968276)
--(axis cs:62,0.121875745598761)
--(axis cs:63,0.145481129919754)
--(axis cs:64,0.136277144374673)
--(axis cs:65,0.145924939995082)
--(axis cs:66,0.121024912526343)
--(axis cs:67,0.172845479962416)
--(axis cs:68,0.118971939244808)
--(axis cs:69,0.112867746299748)
--(axis cs:70,0.138619919152821)
--(axis cs:71,0.155810944970155)
--(axis cs:72,0.0864345716952725)
--(axis cs:73,0.133554346611359)
--(axis cs:74,0.0980309038556965)
--(axis cs:75,0.0942601771755303)
--(axis cs:76,0.0777566089379147)
--(axis cs:77,0.116301758117342)
--(axis cs:78,0.106871183503528)
--(axis cs:79,0.0972367478811257)
--(axis cs:80,0.0837196013784413)
--(axis cs:81,0.0922874940156037)
--(axis cs:82,0.0969012748060087)
--(axis cs:83,0.161446644949719)
--(axis cs:84,0.114543551489917)
--(axis cs:85,0.168378757826372)
--(axis cs:86,0.123193207010996)
--(axis cs:87,0.0930256611836022)
--(axis cs:88,0.106800157242358)
--(axis cs:89,0.11672376911905)
--(axis cs:90,0.129155360929575)
--(axis cs:91,0.148189502079958)
--(axis cs:92,0.132841204836274)
--(axis cs:93,0.153606331118504)
--(axis cs:94,0.10153023740659)
--(axis cs:95,0.152817473787452)
--(axis cs:96,0.134809958368293)
--(axis cs:97,0.128585704178524)
--(axis cs:98,0.134350657308328)
--(axis cs:99,0.127541409832743)
--(axis cs:100,0.190113276903392)
--(axis cs:101,0.156757610326057)
--(axis cs:102,0.165209226643649)
--(axis cs:103,0.173482018842549)
--(axis cs:104,0.194087287638423)
--(axis cs:105,0.216434506607144)
--(axis cs:106,0.1730828094177)
--(axis cs:107,0.175226221871752)
--(axis cs:108,0.170679447028022)
--(axis cs:109,0.189518111051881)
--(axis cs:110,0.143485245991831)
--(axis cs:111,0.197361919763227)
--(axis cs:112,0.256629952241584)
--(axis cs:113,0.166772594807539)
--(axis cs:114,0.169921116003187)
--(axis cs:115,0.183048376334882)
--(axis cs:116,0.209143957662712)
--(axis cs:117,0.231322113586514)
--(axis cs:118,0.212703978944002)
--(axis cs:119,0.1989199185899)
--(axis cs:120,0.247823982608075)
--(axis cs:121,0.21712396785288)
--(axis cs:122,0.21268601969457)
--(axis cs:123,0.252257093253568)
--(axis cs:124,0.191398796944765)
--(axis cs:125,0.272200041683078)
--(axis cs:126,0.217729133321955)
--(axis cs:127,0.275108238027573)
--(axis cs:128,0.263117986332056)
--(axis cs:129,0.285483598229306)
--(axis cs:130,0.291133736210753)
--(axis cs:131,0.312713970433359)
--(axis cs:132,0.321342596592402)
--(axis cs:133,0.259389762019911)
--(axis cs:134,0.317786648836558)
--(axis cs:135,0.275217358515012)
--(axis cs:136,0.334795965286336)
--(axis cs:137,0.301234248945675)
--(axis cs:138,0.341356672522138)
--(axis cs:139,0.324426313955277)
--(axis cs:140,0.358883287838881)
--(axis cs:141,0.359772327178973)
--(axis cs:142,0.38204034637985)
--(axis cs:143,0.34674207308424)
--(axis cs:144,0.341735824814626)
--(axis cs:145,0.370389851915457)
--(axis cs:146,0.323748687818243)
--(axis cs:147,0.342376776523578)
--(axis cs:148,0.400147050515698)
--(axis cs:149,0.37596175222476)
--(axis cs:150,0.349865285866325)
--(axis cs:151,0.386912956947368)
--(axis cs:152,0.375831558706579)
--(axis cs:153,0.407843277124077)
--(axis cs:154,0.386395974247148)
--(axis cs:155,0.413242269128311)
--(axis cs:156,0.479032908294347)
--(axis cs:157,0.418038540075431)
--(axis cs:158,0.44313330067814)
--(axis cs:159,0.473123307692306)
--(axis cs:160,0.417882834693478)
--(axis cs:161,0.435455735978738)
--(axis cs:162,0.454055277394591)
--(axis cs:163,0.513546142803088)
--(axis cs:164,0.484594544073182)
--(axis cs:165,0.469368634913727)
--(axis cs:166,0.506969434138309)
--(axis cs:167,0.489118249314318)
--(axis cs:168,0.554289914860458)
--(axis cs:169,0.472392808564545)
--(axis cs:170,0.541956737230276)
--(axis cs:171,0.571415553382822)
--(axis cs:172,0.521699861800876)
--(axis cs:173,0.512429959150349)
--(axis cs:174,0.511326578171714)
--(axis cs:175,0.519852833915706)
--(axis cs:176,0.578043920444742)
--(axis cs:177,0.535701241703967)
--(axis cs:178,0.560499507830436)
--(axis cs:179,0.5661139428651)
--(axis cs:180,0.581914872116168)
--(axis cs:181,0.635830576778805)
--(axis cs:182,0.586417860666249)
--(axis cs:183,0.586564402135235)
--(axis cs:184,0.592299347331573)
--(axis cs:185,0.587361893040558)
--(axis cs:186,0.582998699938316)
--(axis cs:187,0.60898340008234)
--(axis cs:188,0.627156649891788)
--(axis cs:189,0.651840488915617)
--(axis cs:190,0.602262855231492)
--(axis cs:191,0.633570352541339)
--(axis cs:192,0.630236130293426)
--(axis cs:193,0.585708692360222)
--(axis cs:194,0.625772653640035)
--(axis cs:195,0.607263495315493)
--(axis cs:196,0.559921159351462)
--(axis cs:197,0.621124157093196)
--(axis cs:198,0.652826278497088)
--(axis cs:199,0.638547201445681)
--(axis cs:200,0.666036716257728)
--(axis cs:201,0.614851894433989)
--(axis cs:202,0.67462109357921)
--(axis cs:203,0.657226379875323)
--(axis cs:204,0.682910501315163)
--(axis cs:205,0.672266259111101)
--(axis cs:206,0.661406183178678)
--(axis cs:207,0.651861690556353)
--(axis cs:208,0.682431143572049)
--(axis cs:209,0.667303178921486)
--(axis cs:210,0.695053297433375)
--(axis cs:211,0.654130096690758)
--(axis cs:212,0.66547422001614)
--(axis cs:213,0.674293409520383)
--(axis cs:214,0.708056793261702)
--(axis cs:215,0.718192183362482)
--(axis cs:216,0.679553879851763)
--(axis cs:217,0.698519032798858)
--(axis cs:218,0.668123630590696)
--(axis cs:219,0.663637625232151)
--(axis cs:220,0.683271714932269)
--(axis cs:221,0.696099807284582)
--(axis cs:222,0.700663718562079)
--(axis cs:223,0.713299725053409)
--(axis cs:224,0.659547075415236)
--(axis cs:225,0.69180862288917)
--(axis cs:226,0.706293251322965)
--(axis cs:227,0.714726080762241)
--(axis cs:228,0.722324942884168)
--(axis cs:229,0.739586634103113)
--(axis cs:230,0.733990790479791)
--(axis cs:231,0.680662334367439)
--(axis cs:232,0.723544203844795)
--(axis cs:233,0.737956473025467)
--(axis cs:234,0.703183055942388)
--(axis cs:235,0.714494753051679)
--(axis cs:236,0.732337835339385)
--(axis cs:237,0.719837959729732)
--(axis cs:238,0.735860080310612)
--(axis cs:239,0.755178216120254)
--(axis cs:240,0.765034990036946)
--(axis cs:241,0.749636966849114)
--(axis cs:242,0.76918387248513)
--(axis cs:243,0.789350250793199)
--(axis cs:244,0.80883119000266)
--(axis cs:245,0.771865920816765)
--(axis cs:246,0.753088680280736)
--(axis cs:247,0.760043518506279)
--(axis cs:248,0.732443366500968)
--(axis cs:249,0.732961229374719)
--(axis cs:250,0.745677352098749)
--(axis cs:251,0.737779025752227)
--(axis cs:252,0.762677832823936)
--(axis cs:253,0.730295098701461)
--(axis cs:254,0.73819691212141)
--(axis cs:255,0.721029269570711)
--(axis cs:256,0.746945043539303)
--(axis cs:257,0.700200524893913)
--(axis cs:258,0.748671739936609)
--(axis cs:259,0.72771623577739)
--(axis cs:260,0.737151282247064)
--(axis cs:261,0.711977652737433)
--(axis cs:262,0.718454597589624)
--(axis cs:263,0.720642298897842)
--(axis cs:264,0.749394219447244)
--(axis cs:265,0.762560361515293)
--(axis cs:266,0.777398682253479)
--(axis cs:267,0.738434237820644)
--(axis cs:268,0.728576672497342)
--(axis cs:269,0.760844276137049)
--(axis cs:270,0.76812565770352)
--(axis cs:271,0.722142889713421)
--(axis cs:272,0.740706770992805)
--(axis cs:273,0.760297551277487)
--(axis cs:274,0.784186524204692)
--(axis cs:275,0.716650084114476)
--(axis cs:276,0.744766576381812)
--(axis cs:277,0.787438506172417)
--(axis cs:278,0.746177360075949)
--(axis cs:279,0.776455693911989)
--(axis cs:280,0.734661790624749)
--(axis cs:281,0.735903567942578)
--(axis cs:282,0.751708688170579)
--(axis cs:283,0.705931811040865)
--(axis cs:284,0.714636375154925)
--(axis cs:285,0.772122129745553)
--(axis cs:286,0.769091458077065)
--(axis cs:287,0.723943918690015)
--(axis cs:288,0.746847815827878)
--(axis cs:289,0.739413329984658)
--(axis cs:290,0.769622200537819)
--(axis cs:291,0.745336553332598)
--(axis cs:292,0.766439503245875)
--(axis cs:293,0.760655480643085)
--(axis cs:294,0.743398339858858)
--(axis cs:295,0.745867879131741)
--(axis cs:296,0.772157067151449)
--(axis cs:297,0.765711038601922)
--(axis cs:298,0.795043312896671)
--(axis cs:299,0.768404000712646)
--(axis cs:300,0.716719970659823)
--(axis cs:301,0.798869722375084)
--(axis cs:302,0.762466709967804)
--(axis cs:303,0.801390651212142)
--(axis cs:304,0.776307388890398)
--(axis cs:305,0.743849216976121)
--(axis cs:306,0.770147956969054)
--(axis cs:307,0.809466126332333)
--(axis cs:308,0.781016277664699)
--(axis cs:309,0.784798413226544)
--(axis cs:310,0.815567769352899)
--(axis cs:311,0.804477545068677)
--(axis cs:312,0.794583773581412)
--(axis cs:313,0.797950735500721)
--(axis cs:314,0.750915408505233)
--(axis cs:315,0.787609298181485)
--(axis cs:316,0.768616235457904)
--(axis cs:317,0.809259528103011)
--(axis cs:318,0.816805582145071)
--(axis cs:319,0.818968978695177)
--(axis cs:320,0.759195505195616)
--(axis cs:321,0.773514153539566)
--(axis cs:322,0.786601544651969)
--(axis cs:323,0.800930456478392)
--(axis cs:324,0.78761734696012)
--(axis cs:325,0.775480022966846)
--(axis cs:326,0.77544414232)
--(axis cs:327,0.799152618823846)
--(axis cs:328,0.788537332931906)
--(axis cs:329,0.816569122988725)
--(axis cs:330,0.772911699108961)
--(axis cs:331,0.742595467643835)
--(axis cs:332,0.768894314279382)
--(axis cs:333,0.782844467921104)
--(axis cs:334,0.767819595981805)
--(axis cs:335,0.769583094694951)
--(axis cs:336,0.765836690695143)
--(axis cs:337,0.798754786412517)
--(axis cs:338,0.805823080218741)
--(axis cs:339,0.786949578274074)
--(axis cs:340,0.755707865409216)
--(axis cs:341,0.781272236498332)
--(axis cs:342,0.778956574562058)
--(axis cs:343,0.821772704935708)
--(axis cs:344,0.785970193815711)
--(axis cs:345,0.774157166199653)
--(axis cs:346,0.803743817139682)
--(axis cs:347,0.809744965993567)
--(axis cs:348,0.772624552590069)
--(axis cs:349,0.787621558443016)
--(axis cs:350,0.789001460424492)
--(axis cs:351,0.807392648518449)
--(axis cs:352,0.813314127474524)
--(axis cs:353,0.788058761803961)
--(axis cs:354,0.803542845332307)
--(axis cs:355,0.783238209309743)
--(axis cs:356,0.811285208682194)
--(axis cs:357,0.754252239915059)
--(axis cs:358,0.780021659136222)
--(axis cs:359,0.815899155238952)
--(axis cs:360,0.816798990082097)
--(axis cs:361,0.790378508652369)
--(axis cs:362,0.764901587784761)
--(axis cs:363,0.787472147234363)
--(axis cs:364,0.807070016172956)
--(axis cs:365,0.782869620744471)
--(axis cs:366,0.765580402010459)
--(axis cs:367,0.811472658163928)
--(axis cs:368,0.83392791260724)
--(axis cs:369,0.814072046504776)
--(axis cs:370,0.812910346633243)
--(axis cs:371,0.781351585913799)
--(axis cs:372,0.789382285300881)
--(axis cs:373,0.784429693709789)
--(axis cs:374,0.78411746075255)
--(axis cs:375,0.811464526961884)
--(axis cs:376,0.749610194294326)
--(axis cs:377,0.791147246102597)
--(axis cs:378,0.78717334388932)
--(axis cs:379,0.785689115556033)
--(axis cs:380,0.806055766468154)
--(axis cs:381,0.807270396913321)
--(axis cs:382,0.754791232235616)
--(axis cs:383,0.774255205768646)
--(axis cs:384,0.777241635338421)
--(axis cs:385,0.782114614694595)
--(axis cs:386,0.7734704625292)
--(axis cs:387,0.802452518481566)
--(axis cs:388,0.814332448560139)
--(axis cs:389,0.831156450944569)
--(axis cs:390,0.799883403893867)
--(axis cs:391,0.787064657525031)
--(axis cs:392,0.776291108772334)
--(axis cs:393,0.77586087764513)
--(axis cs:394,0.791059708180063)
--(axis cs:395,0.776194521914686)
--(axis cs:396,0.753189309904156)
--(axis cs:397,0.797526023925764)
--(axis cs:398,0.770300474704135)
--(axis cs:399,0.789834397008669)
--(axis cs:400,0.801694357147663)
--(axis cs:401,0.815653825642058)
--(axis cs:402,0.798857336571469)
--(axis cs:403,0.816911900920061)
--(axis cs:404,0.780104934106469)
--(axis cs:405,0.79846227625194)
--(axis cs:406,0.824380270487228)
--(axis cs:407,0.823410690788118)
--(axis cs:408,0.813938457261884)
--(axis cs:409,0.827557543715527)
--(axis cs:410,0.814915069089141)
--(axis cs:411,0.782236084171423)
--(axis cs:412,0.765774062283047)
--(axis cs:413,0.752890412222958)
--(axis cs:414,0.788822118181713)
--(axis cs:415,0.778845458450012)
--(axis cs:416,0.771425544310299)
--(axis cs:417,0.795102197549138)
--(axis cs:418,0.741513675626119)
--(axis cs:419,0.802167673644238)
--(axis cs:420,0.797941786367608)
--(axis cs:421,0.769080098452218)
--(axis cs:422,0.779894642157106)
--(axis cs:423,0.769684058470406)
--(axis cs:424,0.7965779398687)
--(axis cs:425,0.796886381372381)
--(axis cs:426,0.805153227432572)
--(axis cs:427,0.779654928844788)
--(axis cs:428,0.810296948542983)
--(axis cs:429,0.799339210858385)
--(axis cs:430,0.807455837025756)
--(axis cs:431,0.831032363958593)
--(axis cs:432,0.783415972417197)
--(axis cs:433,0.787111436867611)
--(axis cs:434,0.776480221972764)
--(axis cs:435,0.794868784883803)
--(axis cs:436,0.785071071372429)
--(axis cs:437,0.828945286515589)
--(axis cs:438,0.807452637220128)
--(axis cs:439,0.810615046895355)
--(axis cs:440,0.828021278530198)
--(axis cs:441,0.80305585612302)
--(axis cs:442,0.818364871462822)
--(axis cs:443,0.787232885279042)
--(axis cs:444,0.791383513534512)
--(axis cs:445,0.771481139554867)
--(axis cs:446,0.804259187322261)
--(axis cs:447,0.786156335057315)
--(axis cs:448,0.728995076552635)
--(axis cs:449,0.809630515464372)
--(axis cs:450,0.777582414311923)
--(axis cs:451,0.774189062998104)
--(axis cs:452,0.798318440411156)
--(axis cs:453,0.81960997013257)
--(axis cs:454,0.802690992988362)
--(axis cs:455,0.789015939325067)
--(axis cs:456,0.804454880983164)
--(axis cs:457,0.80172454202596)
--(axis cs:458,0.774511831825067)
--(axis cs:459,0.806667650472285)
--(axis cs:460,0.791223770518929)
--(axis cs:461,0.795709674450879)
--(axis cs:462,0.821543977472188)
--(axis cs:463,0.815619989060075)
--(axis cs:464,0.814711032752033)
--(axis cs:465,0.773784629182281)
--(axis cs:466,0.786865269760602)
--(axis cs:467,0.776470295703958)
--(axis cs:468,0.780386156002655)
--(axis cs:469,0.779717298719239)
--(axis cs:470,0.791363811684208)
--(axis cs:471,0.815012573536324)
--(axis cs:472,0.812588824043658)
--(axis cs:473,0.795492101960427)
--(axis cs:474,0.776382339967724)
--(axis cs:475,0.771545954626268)
--(axis cs:476,0.827023644092564)
--(axis cs:477,0.804299637854728)
--(axis cs:478,0.814962419697948)
--(axis cs:479,0.783851142881024)
--(axis cs:480,0.828033120520883)
--(axis cs:481,0.817053172380584)
--(axis cs:482,0.804857628071065)
--(axis cs:483,0.827417020661104)
--(axis cs:484,0.809164190919367)
--(axis cs:485,0.811295390154647)
--(axis cs:486,0.812220209856024)
--(axis cs:487,0.797242292920382)
--(axis cs:488,0.778008523679625)
--(axis cs:489,0.786682131436275)
--(axis cs:490,0.814025676561451)
--(axis cs:491,0.826555738063593)
--(axis cs:492,0.792861033160336)
--(axis cs:493,0.798146459999347)
--(axis cs:494,0.822967137478324)
--(axis cs:495,0.832368339900859)
--(axis cs:496,0.838529622731548)
--(axis cs:497,0.828763200334219)
--(axis cs:498,0.795579110430309)
--(axis cs:499,0.811220546729581)
--(axis cs:500,0.811493211974578)
--(axis cs:501,0.816121149673243)
--(axis cs:502,0.817792095198969)
--(axis cs:503,0.822293320136476)
--(axis cs:504,0.815676472293353)
--(axis cs:505,0.813164927778589)
--(axis cs:506,0.841944659091718)
--(axis cs:507,0.812943800287978)
--(axis cs:508,0.807470272864463)
--(axis cs:509,0.799864941233759)
--(axis cs:510,0.79531772942502)
--(axis cs:511,0.810777426953518)
--(axis cs:512,0.799999848086855)
--(axis cs:513,0.781268146648312)
--(axis cs:514,0.784186442921906)
--(axis cs:515,0.803136573416553)
--(axis cs:516,0.844828457865887)
--(axis cs:517,0.803259412831814)
--(axis cs:518,0.796419179768294)
--(axis cs:519,0.818799196007398)
--(axis cs:520,0.846551770065439)
--(axis cs:521,0.820994361511091)
--(axis cs:522,0.808547072886628)
--(axis cs:523,0.803506674542848)
--(axis cs:524,0.787418201940474)
--(axis cs:525,0.797155344872761)
--(axis cs:526,0.82069060426604)
--(axis cs:527,0.831016030281159)
--(axis cs:528,0.808526478378585)
--(axis cs:529,0.805443922189929)
--(axis cs:530,0.818995682758767)
--(axis cs:531,0.834998769354798)
--(axis cs:532,0.805136523255179)
--(axis cs:533,0.806435767074787)
--(axis cs:534,0.801196634677563)
--(axis cs:535,0.809572930663413)
--(axis cs:536,0.801538369275761)
--(axis cs:537,0.787848394844573)
--(axis cs:538,0.817259061285215)
--(axis cs:539,0.833456372563224)
--(axis cs:540,0.803562178795958)
--(axis cs:541,0.782609817953504)
--(axis cs:542,0.825827817703827)
--(axis cs:543,0.768103261954421)
--(axis cs:544,0.795920430537067)
--(axis cs:545,0.792862514423747)
--(axis cs:546,0.826056501077682)
--(axis cs:547,0.820223750638993)
--(axis cs:548,0.808389834240558)
--(axis cs:549,0.814072986512398)
--(axis cs:550,0.822888183703808)
--(axis cs:551,0.809988260477906)
--(axis cs:552,0.817018977052589)
--(axis cs:553,0.809442323870397)
--(axis cs:554,0.796101969600264)
--(axis cs:555,0.835853373896604)
--(axis cs:556,0.814741084341805)
--(axis cs:557,0.823526478471883)
--(axis cs:558,0.795122456322029)
--(axis cs:559,0.832356950140689)
--(axis cs:560,0.818870520878835)
--(axis cs:561,0.807189029287382)
--(axis cs:562,0.824431021298769)
--(axis cs:563,0.755215614135132)
--(axis cs:564,0.787004350870213)
--(axis cs:565,0.819968007353257)
--(axis cs:566,0.816744806739117)
--(axis cs:567,0.775010214157832)
--(axis cs:568,0.821521423863507)
--(axis cs:569,0.782010551260277)
--(axis cs:570,0.821107778199888)
--(axis cs:571,0.829633470733825)
--(axis cs:572,0.811348988357339)
--(axis cs:573,0.791425923121677)
--(axis cs:574,0.780309234541818)
--(axis cs:575,0.799578811402046)
--(axis cs:576,0.789183556593461)
--(axis cs:577,0.80357424424332)
--(axis cs:578,0.802607027402795)
--(axis cs:579,0.791994380027327)
--(axis cs:580,0.80083291525102)
--(axis cs:581,0.802218550385741)
--(axis cs:582,0.808029348274588)
--(axis cs:583,0.823729532511869)
--(axis cs:584,0.809186766031621)
--(axis cs:585,0.830045919543418)
--(axis cs:586,0.808232059196615)
--(axis cs:587,0.827405349086654)
--(axis cs:588,0.79958581014544)
--(axis cs:589,0.806087448822152)
--(axis cs:590,0.786716791214259)
--(axis cs:591,0.789832343503405)
--(axis cs:592,0.798364242875998)
--(axis cs:593,0.811741717423591)
--(axis cs:594,0.811110896858911)
--(axis cs:595,0.800339483715768)
--(axis cs:596,0.793654922357821)
--(axis cs:597,0.796542446072242)
--(axis cs:598,0.810692002817761)
--(axis cs:599,0.760384962929514)
--(axis cs:599,0.637706528911978)
--(axis cs:599,0.637706528911978)
--(axis cs:598,0.733917901167142)
--(axis cs:597,0.699602381322585)
--(axis cs:596,0.719466137546327)
--(axis cs:595,0.718190077001293)
--(axis cs:594,0.703750934018544)
--(axis cs:593,0.736194807856684)
--(axis cs:592,0.717483277659023)
--(axis cs:591,0.719064315393253)
--(axis cs:590,0.697202934892967)
--(axis cs:589,0.726107526341573)
--(axis cs:588,0.715230666780411)
--(axis cs:587,0.742359109822225)
--(axis cs:586,0.741475178010622)
--(axis cs:585,0.742147165149667)
--(axis cs:584,0.720355913511058)
--(axis cs:583,0.746522090864755)
--(axis cs:582,0.728233352464215)
--(axis cs:581,0.714080341694401)
--(axis cs:580,0.723902982531752)
--(axis cs:579,0.714160871991895)
--(axis cs:578,0.735436352828086)
--(axis cs:577,0.708110511972686)
--(axis cs:576,0.693668706543729)
--(axis cs:575,0.715375028161168)
--(axis cs:574,0.685637155669279)
--(axis cs:573,0.691027994488491)
--(axis cs:572,0.718832115261264)
--(axis cs:571,0.731942800217445)
--(axis cs:570,0.739656822293572)
--(axis cs:569,0.682166417604192)
--(axis cs:568,0.74621483293525)
--(axis cs:567,0.678430615845498)
--(axis cs:566,0.740061494598434)
--(axis cs:565,0.738881394878497)
--(axis cs:564,0.693259676360725)
--(axis cs:563,0.660528364108846)
--(axis cs:562,0.744119958970962)
--(axis cs:561,0.732323324678097)
--(axis cs:560,0.73666708387752)
--(axis cs:559,0.770690870094631)
--(axis cs:558,0.696613498695175)
--(axis cs:557,0.737098573559419)
--(axis cs:556,0.709922620009399)
--(axis cs:555,0.750816119804139)
--(axis cs:554,0.713055539557244)
--(axis cs:553,0.72576790860546)
--(axis cs:552,0.728129190751828)
--(axis cs:551,0.71995042063265)
--(axis cs:550,0.746438323122698)
--(axis cs:549,0.739005488831703)
--(axis cs:548,0.726306805924833)
--(axis cs:547,0.744070423834409)
--(axis cs:546,0.761972830701649)
--(axis cs:545,0.690175801427069)
--(axis cs:544,0.7180097365181)
--(axis cs:543,0.659042682379023)
--(axis cs:542,0.725737197142437)
--(axis cs:541,0.700024865071804)
--(axis cs:540,0.720736961847754)
--(axis cs:539,0.758957810475958)
--(axis cs:538,0.739059013001609)
--(axis cs:537,0.692525831623403)
--(axis cs:536,0.711245114601472)
--(axis cs:535,0.740684624281642)
--(axis cs:534,0.721105720850896)
--(axis cs:533,0.741160690287294)
--(axis cs:532,0.707236793277144)
--(axis cs:531,0.77479833458293)
--(axis cs:530,0.747374058610973)
--(axis cs:529,0.71802002705527)
--(axis cs:528,0.720995388643281)
--(axis cs:527,0.739329466236212)
--(axis cs:526,0.738612592537157)
--(axis cs:525,0.709447340636173)
--(axis cs:524,0.708788629422608)
--(axis cs:523,0.716194240861192)
--(axis cs:522,0.725394723164543)
--(axis cs:521,0.7408308792204)
--(axis cs:520,0.760038297852754)
--(axis cs:519,0.745439690106488)
--(axis cs:518,0.707600238313623)
--(axis cs:517,0.722207264587988)
--(axis cs:516,0.759454932355003)
--(axis cs:515,0.711690353129124)
--(axis cs:514,0.697825338699251)
--(axis cs:513,0.690988156076741)
--(axis cs:512,0.712890560116053)
--(axis cs:511,0.728529143212426)
--(axis cs:510,0.708138536531246)
--(axis cs:509,0.713972003071935)
--(axis cs:508,0.727908497670558)
--(axis cs:507,0.734526004213076)
--(axis cs:506,0.76986609959404)
--(axis cs:505,0.72021133137892)
--(axis cs:504,0.730044606056402)
--(axis cs:503,0.746062369312963)
--(axis cs:502,0.734501531313407)
--(axis cs:501,0.734348779702936)
--(axis cs:500,0.735107409626044)
--(axis cs:499,0.723303502546305)
--(axis cs:498,0.713472515027566)
--(axis cs:497,0.751859822476304)
--(axis cs:496,0.759264987172436)
--(axis cs:495,0.747435416901978)
--(axis cs:494,0.739048951008352)
--(axis cs:493,0.709347348275711)
--(axis cs:492,0.713679735099182)
--(axis cs:491,0.735883558750704)
--(axis cs:490,0.731941507296357)
--(axis cs:489,0.69618443775842)
--(axis cs:488,0.693766576220475)
--(axis cs:487,0.700370250785911)
--(axis cs:486,0.743388114111674)
--(axis cs:485,0.722924383127626)
--(axis cs:484,0.722454078474638)
--(axis cs:483,0.740450051502843)
--(axis cs:482,0.713185049737238)
--(axis cs:481,0.714465048981387)
--(axis cs:480,0.742201289166652)
--(axis cs:479,0.693183878931959)
--(axis cs:478,0.735246645724932)
--(axis cs:477,0.710223512605922)
--(axis cs:476,0.760804239829069)
--(axis cs:475,0.679935463261399)
--(axis cs:474,0.690183719155902)
--(axis cs:473,0.710282418657844)
--(axis cs:472,0.719477763882304)
--(axis cs:471,0.725703698039322)
--(axis cs:470,0.71720383594594)
--(axis cs:469,0.696068373941434)
--(axis cs:468,0.694010706836303)
--(axis cs:467,0.671137651903989)
--(axis cs:466,0.694721936435979)
--(axis cs:465,0.699809398317997)
--(axis cs:464,0.731919203331328)
--(axis cs:463,0.734951574792739)
--(axis cs:462,0.734174175902215)
--(axis cs:461,0.704773142874769)
--(axis cs:460,0.708758972432564)
--(axis cs:459,0.722190730729847)
--(axis cs:458,0.660974166551555)
--(axis cs:457,0.718573623860441)
--(axis cs:456,0.724087879361067)
--(axis cs:455,0.692408799130921)
--(axis cs:454,0.719402018167149)
--(axis cs:453,0.730203497649648)
--(axis cs:452,0.716539878353412)
--(axis cs:451,0.674235169019878)
--(axis cs:450,0.672243367545108)
--(axis cs:449,0.721271915438059)
--(axis cs:448,0.633118851186293)
--(axis cs:447,0.685859843208863)
--(axis cs:446,0.717281598281024)
--(axis cs:445,0.690419286408059)
--(axis cs:444,0.715528786680323)
--(axis cs:443,0.711021984850828)
--(axis cs:442,0.722265505105053)
--(axis cs:441,0.719786380276202)
--(axis cs:440,0.767814396211726)
--(axis cs:439,0.711684615248056)
--(axis cs:438,0.73153589161215)
--(axis cs:437,0.737499206491404)
--(axis cs:436,0.701261866523008)
--(axis cs:435,0.722507545383152)
--(axis cs:434,0.692524818126026)
--(axis cs:433,0.700014718227293)
--(axis cs:432,0.701247627871404)
--(axis cs:431,0.735619950740596)
--(axis cs:430,0.71226542924882)
--(axis cs:429,0.693875733919059)
--(axis cs:428,0.721235321270536)
--(axis cs:427,0.706836349994836)
--(axis cs:426,0.714199555201461)
--(axis cs:425,0.72005951994852)
--(axis cs:424,0.712620283782649)
--(axis cs:423,0.681995555318583)
--(axis cs:422,0.685530810147941)
--(axis cs:421,0.667874093189474)
--(axis cs:420,0.71187409219827)
--(axis cs:419,0.712454423009109)
--(axis cs:418,0.652335444004251)
--(axis cs:417,0.711092302395362)
--(axis cs:416,0.660735610507105)
--(axis cs:415,0.698381567995764)
--(axis cs:414,0.689968882484287)
--(axis cs:413,0.640007835362789)
--(axis cs:412,0.686283081961597)
--(axis cs:411,0.68446951925293)
--(axis cs:410,0.742210357567536)
--(axis cs:409,0.75411585335309)
--(axis cs:408,0.735069305808379)
--(axis cs:407,0.736528690704571)
--(axis cs:406,0.736486640379682)
--(axis cs:405,0.686469545398632)
--(axis cs:404,0.685427208276002)
--(axis cs:403,0.729644277276742)
--(axis cs:402,0.717363335117952)
--(axis cs:401,0.737177891252158)
--(axis cs:400,0.723899233289677)
--(axis cs:399,0.696502183077911)
--(axis cs:398,0.670919286437501)
--(axis cs:397,0.723032584132845)
--(axis cs:396,0.661203241988396)
--(axis cs:395,0.677325222230058)
--(axis cs:394,0.696350234308004)
--(axis cs:393,0.675873065495062)
--(axis cs:392,0.681422271597296)
--(axis cs:391,0.692548220915973)
--(axis cs:390,0.712241473373473)
--(axis cs:389,0.744553689609321)
--(axis cs:388,0.731659493525553)
--(axis cs:387,0.717247659812362)
--(axis cs:386,0.699259325044337)
--(axis cs:385,0.696366284512407)
--(axis cs:384,0.674080956999797)
--(axis cs:383,0.673682293823265)
--(axis cs:382,0.653211442173308)
--(axis cs:381,0.721344998108323)
--(axis cs:380,0.699459811703674)
--(axis cs:379,0.68368244170615)
--(axis cs:378,0.712229726998126)
--(axis cs:377,0.697814053014952)
--(axis cs:376,0.642255838977957)
--(axis cs:375,0.731408658880052)
--(axis cs:374,0.706320660154321)
--(axis cs:373,0.691098337286992)
--(axis cs:372,0.704360430316835)
--(axis cs:371,0.701821246156091)
--(axis cs:370,0.738090909055512)
--(axis cs:369,0.744534364445384)
--(axis cs:368,0.749832136065492)
--(axis cs:367,0.73264368556179)
--(axis cs:366,0.670083023106092)
--(axis cs:365,0.689499763343663)
--(axis cs:364,0.724470491930051)
--(axis cs:363,0.706566376537616)
--(axis cs:362,0.665767309290386)
--(axis cs:361,0.723585738905628)
--(axis cs:360,0.739540330319724)
--(axis cs:359,0.731676272805226)
--(axis cs:358,0.690205509528446)
--(axis cs:357,0.659252137651819)
--(axis cs:356,0.720600960703975)
--(axis cs:355,0.714753642588359)
--(axis cs:354,0.718600592826756)
--(axis cs:353,0.691440239501107)
--(axis cs:352,0.731584723674326)
--(axis cs:351,0.706601690475889)
--(axis cs:350,0.698800711362054)
--(axis cs:349,0.682437447093931)
--(axis cs:348,0.694234196299929)
--(axis cs:347,0.723841846499495)
--(axis cs:346,0.728539090924476)
--(axis cs:345,0.705828039566803)
--(axis cs:344,0.71143205671154)
--(axis cs:343,0.719957526638272)
--(axis cs:342,0.701411841743858)
--(axis cs:341,0.687177246219243)
--(axis cs:340,0.657389147871621)
--(axis cs:339,0.703044111145167)
--(axis cs:338,0.722999832979171)
--(axis cs:337,0.711943803192323)
--(axis cs:336,0.671357365248913)
--(axis cs:335,0.663900739179508)
--(axis cs:334,0.67985104934509)
--(axis cs:333,0.669440930504)
--(axis cs:332,0.684269188884121)
--(axis cs:331,0.6321176664443)
--(axis cs:330,0.698745452970985)
--(axis cs:329,0.741830290098189)
--(axis cs:328,0.685104376236118)
--(axis cs:327,0.724501168157147)
--(axis cs:326,0.681670367965466)
--(axis cs:325,0.678047742670294)
--(axis cs:324,0.71040675669358)
--(axis cs:323,0.731446975117789)
--(axis cs:322,0.717362633530959)
--(axis cs:321,0.682857212022976)
--(axis cs:320,0.668776349338739)
--(axis cs:319,0.742394657668459)
--(axis cs:318,0.737619264755106)
--(axis cs:317,0.719516903798421)
--(axis cs:316,0.672883584166915)
--(axis cs:315,0.698395842511156)
--(axis cs:314,0.672132749423439)
--(axis cs:313,0.728946389669429)
--(axis cs:312,0.704033076276246)
--(axis cs:311,0.718253283130901)
--(axis cs:310,0.72275896467696)
--(axis cs:309,0.702494675629046)
--(axis cs:308,0.700127491760321)
--(axis cs:307,0.721098578060496)
--(axis cs:306,0.676125647898302)
--(axis cs:305,0.657705043928953)
--(axis cs:304,0.680039389331379)
--(axis cs:303,0.716865918606927)
--(axis cs:302,0.666459378241372)
--(axis cs:301,0.708900724090215)
--(axis cs:300,0.623850743348391)
--(axis cs:299,0.687519849742454)
--(axis cs:298,0.705139186895203)
--(axis cs:297,0.668407569505656)
--(axis cs:296,0.682211384046047)
--(axis cs:295,0.639936657936583)
--(axis cs:294,0.65225908431419)
--(axis cs:293,0.671305904581167)
--(axis cs:292,0.669848016934109)
--(axis cs:291,0.642682329337021)
--(axis cs:290,0.658906799567467)
--(axis cs:289,0.647240987919659)
--(axis cs:288,0.651168837657526)
--(axis cs:287,0.610897439910535)
--(axis cs:286,0.679912292971724)
--(axis cs:285,0.676248553390755)
--(axis cs:284,0.623042265648715)
--(axis cs:283,0.617772953460629)
--(axis cs:282,0.655826745146104)
--(axis cs:281,0.643020155903021)
--(axis cs:280,0.636994114234281)
--(axis cs:279,0.693233476190784)
--(axis cs:278,0.661827173686856)
--(axis cs:277,0.699200411486721)
--(axis cs:276,0.657559087639624)
--(axis cs:275,0.635901314140048)
--(axis cs:274,0.695785979802738)
--(axis cs:273,0.68409634897216)
--(axis cs:272,0.633475947330538)
--(axis cs:271,0.611224350685069)
--(axis cs:270,0.696152035089798)
--(axis cs:269,0.661401360171088)
--(axis cs:268,0.626747517201847)
--(axis cs:267,0.63086742916025)
--(axis cs:266,0.688486517616096)
--(axis cs:265,0.674804010224079)
--(axis cs:264,0.654516366494592)
--(axis cs:263,0.630085662376994)
--(axis cs:262,0.629606455940179)
--(axis cs:261,0.61443829752962)
--(axis cs:260,0.632701643626082)
--(axis cs:259,0.633013746046342)
--(axis cs:258,0.658275913104794)
--(axis cs:257,0.601307775833137)
--(axis cs:256,0.629582428896886)
--(axis cs:255,0.606010227390661)
--(axis cs:254,0.619858619969049)
--(axis cs:253,0.618752266821088)
--(axis cs:252,0.641328317275964)
--(axis cs:251,0.626509376817426)
--(axis cs:250,0.633024154727757)
--(axis cs:249,0.638051237019181)
--(axis cs:248,0.636556287901647)
--(axis cs:247,0.649431424953039)
--(axis cs:246,0.648226931191125)
--(axis cs:245,0.67582532023385)
--(axis cs:244,0.717469944524576)
--(axis cs:243,0.702042909763564)
--(axis cs:242,0.682004175577918)
--(axis cs:241,0.649164994567737)
--(axis cs:240,0.665953932702896)
--(axis cs:239,0.649482235254352)
--(axis cs:238,0.619988888428982)
--(axis cs:237,0.617047593952697)
--(axis cs:236,0.639414240709566)
--(axis cs:235,0.603975928635915)
--(axis cs:234,0.584807399352876)
--(axis cs:233,0.629930768248481)
--(axis cs:232,0.622415131958291)
--(axis cs:231,0.562164046123415)
--(axis cs:230,0.627720198184322)
--(axis cs:229,0.63280649054185)
--(axis cs:228,0.612103933810817)
--(axis cs:227,0.594536011312351)
--(axis cs:226,0.602001790542849)
--(axis cs:225,0.588504272956814)
--(axis cs:224,0.538947776342483)
--(axis cs:223,0.602122393677533)
--(axis cs:222,0.586250011090621)
--(axis cs:221,0.58108064094366)
--(axis cs:220,0.563739721853984)
--(axis cs:219,0.550516761027015)
--(axis cs:218,0.554455100286699)
--(axis cs:217,0.588078822014457)
--(axis cs:216,0.565016286367251)
--(axis cs:215,0.60188478827699)
--(axis cs:214,0.587956517365579)
--(axis cs:213,0.564055134296911)
--(axis cs:212,0.543489776057231)
--(axis cs:211,0.538640169195317)
--(axis cs:210,0.580222393561647)
--(axis cs:209,0.545416112612144)
--(axis cs:208,0.565758860062518)
--(axis cs:207,0.541870197959543)
--(axis cs:206,0.544748642898023)
--(axis cs:205,0.544088046305705)
--(axis cs:204,0.537538514538265)
--(axis cs:203,0.538445350355231)
--(axis cs:202,0.555989498770692)
--(axis cs:201,0.491201666593691)
--(axis cs:200,0.547413760451021)
--(axis cs:199,0.526612472928148)
--(axis cs:198,0.527463025126919)
--(axis cs:197,0.489976461038672)
--(axis cs:196,0.443705205603028)
--(axis cs:195,0.480178489782742)
--(axis cs:194,0.497630010893143)
--(axis cs:193,0.465482413817097)
--(axis cs:192,0.498699343954548)
--(axis cs:191,0.497325140854154)
--(axis cs:190,0.482118605321843)
--(axis cs:189,0.519961111240762)
--(axis cs:188,0.486672365949611)
--(axis cs:187,0.478162771760523)
--(axis cs:186,0.458735678836503)
--(axis cs:185,0.457796710825476)
--(axis cs:184,0.472424921831068)
--(axis cs:183,0.451920076546128)
--(axis cs:182,0.465786228647215)
--(axis cs:181,0.507580449490678)
--(axis cs:180,0.454365004053414)
--(axis cs:179,0.437346786560703)
--(axis cs:178,0.433423019295216)
--(axis cs:177,0.406362571595092)
--(axis cs:176,0.43740476230633)
--(axis cs:175,0.376113222597225)
--(axis cs:174,0.388179290245001)
--(axis cs:173,0.378443186800922)
--(axis cs:172,0.379415255720491)
--(axis cs:171,0.431158080601658)
--(axis cs:170,0.413304581312293)
--(axis cs:169,0.345922779597937)
--(axis cs:168,0.42610027807569)
--(axis cs:167,0.357762831206468)
--(axis cs:166,0.371399505949381)
--(axis cs:165,0.349210335135831)
--(axis cs:164,0.358888015034377)
--(axis cs:163,0.379513098303028)
--(axis cs:162,0.321903225907662)
--(axis cs:161,0.299731610375888)
--(axis cs:160,0.283076780650328)
--(axis cs:159,0.336967372664919)
--(axis cs:158,0.323169820220569)
--(axis cs:157,0.281192003686363)
--(axis cs:156,0.348759091372653)
--(axis cs:155,0.278533341902815)
--(axis cs:154,0.265432404005862)
--(axis cs:153,0.270551551086928)
--(axis cs:152,0.227757205309333)
--(axis cs:151,0.261837470525485)
--(axis cs:150,0.234417011697222)
--(axis cs:149,0.248025203425798)
--(axis cs:148,0.267100821097983)
--(axis cs:147,0.208686743289942)
--(axis cs:146,0.196304635084345)
--(axis cs:145,0.238622821481408)
--(axis cs:144,0.198776180541129)
--(axis cs:143,0.208237464008784)
--(axis cs:142,0.239998276761715)
--(axis cs:141,0.224957143717778)
--(axis cs:140,0.226498622830859)
--(axis cs:139,0.190268062848475)
--(axis cs:138,0.21311538473025)
--(axis cs:137,0.190130019337856)
--(axis cs:136,0.207943413390543)
--(axis cs:135,0.161823834940655)
--(axis cs:134,0.185435866495333)
--(axis cs:133,0.153556590034897)
--(axis cs:132,0.193595053933242)
--(axis cs:131,0.190268550517911)
--(axis cs:130,0.164696690001107)
--(axis cs:129,0.172770468986158)
--(axis cs:128,0.150425674867855)
--(axis cs:127,0.149882656494571)
--(axis cs:126,0.115218915871499)
--(axis cs:125,0.15405725137109)
--(axis cs:124,0.085592686374118)
--(axis cs:123,0.129615678072329)
--(axis cs:122,0.095217231568056)
--(axis cs:121,0.101869303989657)
--(axis cs:120,0.126108595230753)
--(axis cs:119,0.10169509140011)
--(axis cs:118,0.102739431811909)
--(axis cs:117,0.108743376478976)
--(axis cs:116,0.105053675790436)
--(axis cs:115,0.0932920367242807)
--(axis cs:114,0.0787896350513139)
--(axis cs:113,0.0688682400520461)
--(axis cs:112,0.135998469683712)
--(axis cs:111,0.0846608751501932)
--(axis cs:110,0.0610646388659581)
--(axis cs:109,0.0864953719928524)
--(axis cs:108,0.0628966123292874)
--(axis cs:107,0.0727279506917959)
--(axis cs:106,0.0817949160023043)
--(axis cs:105,0.0982931268704892)
--(axis cs:104,0.083925238030353)
--(axis cs:103,0.0762317222774424)
--(axis cs:102,0.0645446999852772)
--(axis cs:101,0.0525141942582476)
--(axis cs:100,0.0747292131229218)
--(axis cs:99,0.046926049790967)
--(axis cs:98,0.0519672360770962)
--(axis cs:97,0.0542417981802283)
--(axis cs:96,0.0416333746467313)
--(axis cs:95,0.0671015828503547)
--(axis cs:94,0.0366394054505524)
--(axis cs:93,0.0612874784710295)
--(axis cs:92,0.0454873599812609)
--(axis cs:91,0.0607754912600356)
--(axis cs:90,0.0467707299321773)
--(axis cs:89,0.0398982546904742)
--(axis cs:88,0.0357515618718609)
--(axis cs:87,0.0265577935561026)
--(axis cs:86,0.0494110269902966)
--(axis cs:85,0.0724455654198264)
--(axis cs:84,0.042315821775706)
--(axis cs:83,0.0619840909810171)
--(axis cs:82,0.0243893269533431)
--(axis cs:81,0.0180048331425323)
--(axis cs:80,0.0232238891587992)
--(axis cs:79,0.031132034072215)
--(axis cs:78,0.0310837868733174)
--(axis cs:77,0.0332386320480485)
--(axis cs:76,0.014813754726199)
--(axis cs:75,0.0238455452270671)
--(axis cs:74,0.0222026820028893)
--(axis cs:73,0.0456314849182228)
--(axis cs:72,0.0214495443714094)
--(axis cs:71,0.0584562012822855)
--(axis cs:70,0.0436890300678934)
--(axis cs:69,0.036595628410337)
--(axis cs:68,0.0455102799249108)
--(axis cs:67,0.0716561989142626)
--(axis cs:66,0.0400386766622462)
--(axis cs:65,0.0497614049100126)
--(axis cs:64,0.0484822802285012)
--(axis cs:63,0.0441974640557151)
--(axis cs:62,0.0259672967755311)
--(axis cs:61,0.0415881090790104)
--(axis cs:60,0.0385558551771675)
--(axis cs:59,0.022303398278266)
--(axis cs:58,0.0181233644200771)
--(axis cs:57,0.0297198052101847)
--(axis cs:56,0.0432322384316237)
--(axis cs:55,0.0399850811752061)
--(axis cs:54,0.0411656693521898)
--(axis cs:53,0.0420104876980049)
--(axis cs:52,0.071976210191988)
--(axis cs:51,0.0386883811608033)
--(axis cs:50,0.0267859725467669)
--(axis cs:49,0.0159451955688702)
--(axis cs:48,0.0308658425632403)
--(axis cs:47,0.0192842271600422)
--(axis cs:46,0.0230327454793443)
--(axis cs:45,0.0177313355831687)
--(axis cs:44,0.0305384568026065)
--(axis cs:43,0.0337114532722986)
--(axis cs:42,0.0389559770526195)
--(axis cs:41,0.0370071362801479)
--(axis cs:40,0.045674280654492)
--(axis cs:39,0.0258015484223489)
--(axis cs:38,0.0460620287523974)
--(axis cs:37,0.0385357400777988)
--(axis cs:36,0.0275208695912242)
--(axis cs:35,0.0566239944785129)
--(axis cs:34,0.0304588219873914)
--(axis cs:33,0.0249829977545151)
--(axis cs:32,0.0458258183852563)
--(axis cs:31,0.0296444367835977)
--(axis cs:30,0.0545448347705724)
--(axis cs:29,0.0355769035750197)
--(axis cs:28,0.0549660715136309)
--(axis cs:27,0.0691796172448006)
--(axis cs:26,0.0493781629018541)
--(axis cs:25,0.0622399383870511)
--(axis cs:24,0.0306987916449205)
--(axis cs:23,0.0368797628563467)
--(axis cs:22,0.0485537299980609)
--(axis cs:21,0.0228928532832602)
--(axis cs:20,0.0226616467776143)
--(axis cs:19,0.025499817672849)
--(axis cs:18,0.0166775419834934)
--(axis cs:17,0.0269851261184024)
--(axis cs:16,0.0201560506493051)
--(axis cs:15,0.0220714006861239)
--(axis cs:14,0.0205590851953815)
--(axis cs:13,0.0222195919356889)
--(axis cs:12,0.0211520028036788)
--(axis cs:11,0.0134664016891603)
--(axis cs:10,0.0122479239569868)
--(axis cs:9,0.00334279847224912)
--(axis cs:8,0.00918727995249319)
--(axis cs:7,0.0061176021067853)
--(axis cs:6,0.00492403544746983)
--(axis cs:5,-0.000492775787109745)
--(axis cs:4,0.000166350285010169)
--(axis cs:3,0.000792774501802753)
--(axis cs:2,-0.00076684643032582)
--(axis cs:1,-0.00100311111040816)
--(axis cs:0,-0.000922487513312963)
--cycle;

\addplot [semithick, color0, dash pattern=on 1pt off 3pt on 3pt off 3pt, forget plot]
table [row sep=\\]{%
0	0.00486111111111111 \\
1	0.00950292397660819 \\
2	0.00870028409090909 \\
3	0.0116102430555556 \\
4	0.0232324966699967 \\
5	0.0123511904761905 \\
6	0.00950520833333333 \\
7	0.0118743235930736 \\
8	0.013864650974026 \\
9	0.0229076479076479 \\
10	0.0271397005772006 \\
11	0.0246161390692641 \\
12	0.0469200937950938 \\
13	0.0487291571275946 \\
14	0.0646267361111111 \\
15	0.0468535804473304 \\
16	0.0709212662337662 \\
17	0.0592509920634921 \\
18	0.0767828959235209 \\
19	0.0556175595238095 \\
20	0.0508246527777778 \\
21	0.0327307241369741 \\
22	0.050163465007215 \\
23	0.0400173611111111 \\
24	0.0530732722138972 \\
25	0.0402607808857809 \\
26	0.0475080041486291 \\
27	0.030427731990232 \\
28	0.0547619047619048 \\
29	0.0484462152430902 \\
30	0.0536897997835498 \\
31	0.0306053321678322 \\
32	0.0407871642246642 \\
33	0.0384858630952381 \\
34	0.0759692564380064 \\
35	0.0586186383061383 \\
36	0.0405591804029304 \\
37	0.0335086354617605 \\
38	0.037792545995671 \\
39	0.0599724927849928 \\
40	0.0489087301587302 \\
41	0.0506448412698413 \\
42	0.0701760912698413 \\
43	0.0550934742340992 \\
44	0.0699804969336219 \\
45	0.0631076388888889 \\
46	0.0758843153374403 \\
47	0.0484938672438672 \\
48	0.0380958017676768 \\
49	0.0650739017926518 \\
50	0.0543566242784993 \\
51	0.0306637806637807 \\
52	0.0648848980880231 \\
53	0.0947369470806971 \\
54	0.079985119047619 \\
55	0.0631021756021756 \\
56	0.0750137270817418 \\
57	0.0767812049062049 \\
58	0.0443838713369963 \\
59	0.0864707341269841 \\
60	0.0715931637806638 \\
61	0.0880677915834166 \\
62	0.0817122113997114 \\
63	0.0698373241341991 \\
64	0.0707420183982684 \\
65	0.104450931013431 \\
66	0.106433193542569 \\
67	0.103517966408591 \\
68	0.106001507173382 \\
69	0.040025556041181 \\
70	0.110850260850261 \\
71	0.122730351245976 \\
72	0.067027417027417 \\
73	0.105157602813853 \\
74	0.113517992424242 \\
75	0.111749078637865 \\
76	0.124414344336219 \\
77	0.118591781482406 \\
78	0.157719407328782 \\
79	0.137602588383838 \\
80	0.156979066940004 \\
81	0.166842966061716 \\
82	0.132923370032745 \\
83	0.12588617979243 \\
84	0.149588909354534 \\
85	0.127126736111111 \\
86	0.156118534243534 \\
87	0.169199790201628 \\
88	0.154766414141414 \\
89	0.223427830849706 \\
90	0.223344017094017 \\
91	0.164329117063492 \\
92	0.255138741466866 \\
93	0.230370801073926 \\
94	0.221425189393939 \\
95	0.17678688499001 \\
96	0.238315677378177 \\
97	0.233001330266955 \\
98	0.327167710761461 \\
99	0.281474515068265 \\
100	0.270158899260462 \\
101	0.306107181727586 \\
102	0.280530623889999 \\
103	0.272070303573061 \\
104	0.297023332570207 \\
105	0.262906407828283 \\
106	0.276297747391497 \\
107	0.306933626269564 \\
108	0.357299991674992 \\
109	0.3343083999334 \\
110	0.381963175713176 \\
111	0.361186166264291 \\
112	0.351829611870972 \\
113	0.383321019258519 \\
114	0.355530728462714 \\
115	0.390241616022866 \\
116	0.354183408503261 \\
117	0.366050832847708 \\
118	0.404722230894106 \\
119	0.388907382956096 \\
120	0.421341056818998 \\
121	0.384938243945597 \\
122	0.44838516518204 \\
123	0.450202250006937 \\
124	0.457300176523038 \\
125	0.437188108052079 \\
126	0.450810192411755 \\
127	0.438197501283439 \\
128	0.407213570850374 \\
129	0.465608080068198 \\
130	0.525488392836738 \\
131	0.468111098970474 \\
132	0.484015887296291 \\
133	0.46322473442128 \\
134	0.475927457958708 \\
135	0.528716547443569 \\
136	0.456294746919747 \\
137	0.557922871399434 \\
138	0.524755920370157 \\
139	0.598018554453587 \\
140	0.554508525073782 \\
141	0.56458342005217 \\
142	0.580935991092241 \\
143	0.556096325294395 \\
144	0.586210837773338 \\
145	0.579462117743368 \\
146	0.587880652333777 \\
147	0.592418649059274 \\
148	0.584163518262783 \\
149	0.621486803127428 \\
150	0.596805126297314 \\
151	0.585922102305374 \\
152	0.576714061705838 \\
153	0.596668315854897 \\
154	0.620376980168081 \\
155	0.580448485177104 \\
156	0.635665913181998 \\
157	0.623919114739427 \\
158	0.578649323766511 \\
159	0.643475101287601 \\
160	0.621351608460983 \\
161	0.570209066107504 \\
162	0.60439915986791 \\
163	0.651314614205239 \\
164	0.62671956821773 \\
165	0.628588013548951 \\
166	0.696530249264624 \\
167	0.63986667948065 \\
168	0.670921678148241 \\
169	0.667903006919551 \\
170	0.662320124593973 \\
171	0.629959216896257 \\
172	0.668222012362637 \\
173	0.676952713085526 \\
174	0.675164128151261 \\
175	0.635887715736061 \\
176	0.702609933469308 \\
177	0.695143571706072 \\
178	0.717490554278238 \\
179	0.680350820033724 \\
180	0.697338795159519 \\
181	0.704001575681263 \\
182	0.719617734185749 \\
183	0.680430073398823 \\
184	0.679872384559884 \\
185	0.735251360025209 \\
186	0.689883705704018 \\
187	0.712209525054194 \\
188	0.713659309695615 \\
189	0.70858000506438 \\
190	0.712370528776779 \\
191	0.690894741492168 \\
192	0.689658150009713 \\
193	0.678226014692007 \\
194	0.716242011369478 \\
195	0.732053623459874 \\
196	0.724865209596368 \\
197	0.704077644679667 \\
198	0.726535790598291 \\
199	0.70387928044178 \\
200	0.745026024322899 \\
201	0.709834913350538 \\
202	0.73268432955933 \\
203	0.675950113254801 \\
204	0.719996491457888 \\
205	0.706297586961649 \\
206	0.737118480477855 \\
207	0.722390283327783 \\
208	0.741025771103896 \\
209	0.729107199224387 \\
210	0.71605499535187 \\
211	0.716832621014606 \\
212	0.705780200702076 \\
213	0.69971640772422 \\
214	0.723934984286547 \\
215	0.706413378288378 \\
216	0.722342067654568 \\
217	0.742053692834943 \\
218	0.747432992354867 \\
219	0.772500026015651 \\
220	0.779515493187368 \\
221	0.740800128690754 \\
222	0.729494581194949 \\
223	0.767798087329337 \\
224	0.66772374760656 \\
225	0.701954393900626 \\
226	0.746324561110866 \\
227	0.740856622187045 \\
228	0.730212192321567 \\
229	0.72878545066045 \\
230	0.743183662232376 \\
231	0.714596045294575 \\
232	0.738352627254282 \\
233	0.740013025169275 \\
234	0.749302845591908 \\
235	0.737613323665713 \\
236	0.718329522040459 \\
237	0.754184747544122 \\
238	0.739438989829615 \\
239	0.719754334207459 \\
240	0.710537661123599 \\
241	0.71726626238345 \\
242	0.754282696470196 \\
243	0.726372223300435 \\
244	0.756788437257187 \\
245	0.760959656662782 \\
246	0.735977867618493 \\
247	0.730578926282051 \\
248	0.732097828386891 \\
249	0.739823327713953 \\
250	0.77453124387867 \\
251	0.782207359378775 \\
252	0.767140411671661 \\
253	0.734679773351648 \\
254	0.793593436791966 \\
255	0.743868718087468 \\
256	0.757508593836719 \\
257	0.76482922459485 \\
258	0.745444594468032 \\
259	0.745921027756965 \\
260	0.79064517360703 \\
261	0.74789015238234 \\
262	0.783437959609835 \\
263	0.782502538821473 \\
264	0.742592195131257 \\
265	0.752919996669997 \\
266	0.747838641393329 \\
267	0.769797997141747 \\
268	0.753749093788156 \\
269	0.741074632067279 \\
270	0.772994102762025 \\
271	0.756172473359973 \\
272	0.781633514055389 \\
273	0.728756930365386 \\
274	0.745489982852115 \\
275	0.729549015118868 \\
276	0.758576449592074 \\
277	0.712525408619158 \\
278	0.736890366577866 \\
279	0.735707478285603 \\
280	0.74459362252331 \\
281	0.732092949177048 \\
282	0.70328569000444 \\
283	0.722715045371295 \\
284	0.762515219155844 \\
285	0.76493810009435 \\
286	0.754534029338716 \\
287	0.759869532020267 \\
288	0.729581095987346 \\
289	0.762299983003108 \\
290	0.740865427974803 \\
291	0.764224100552225 \\
292	0.718563008682494 \\
293	0.720988711809024 \\
294	0.767890919844045 \\
295	0.767075611020923 \\
296	0.743020868020868 \\
297	0.733946804063992 \\
298	0.744566626184273 \\
299	0.748944092314956 \\
300	0.739461688485126 \\
301	0.782292013542013 \\
302	0.744378686060671 \\
303	0.760234296953047 \\
304	0.772580240939616 \\
305	0.716120932274425 \\
306	0.717206643183206 \\
307	0.735435788170163 \\
308	0.74646323728355 \\
309	0.743346967829964 \\
310	0.711917054299867 \\
311	0.743093408327783 \\
312	0.730375657328782 \\
313	0.744930809590736 \\
314	0.770254398379398 \\
315	0.757181040211831 \\
316	0.785407561188811 \\
317	0.757596743534243 \\
318	0.77469050047175 \\
319	0.750176993145743 \\
320	0.744016768821456 \\
321	0.764938143453768 \\
322	0.753849058927184 \\
323	0.728622831161893 \\
324	0.72309031093498 \\
325	0.73980303550616 \\
326	0.733099197575301 \\
327	0.731571163211788 \\
328	0.697590505644274 \\
329	0.750033173780876 \\
330	0.732192807192807 \\
331	0.758181878885004 \\
332	0.747307098283661 \\
333	0.766050312534687 \\
334	0.742917499167499 \\
335	0.751650042665668 \\
336	0.747376473353036 \\
337	0.727972179729992 \\
338	0.804764658084971 \\
339	0.76954313048063 \\
340	0.786738218378844 \\
341	0.773821339251027 \\
342	0.783709017187877 \\
343	0.733288456335331 \\
344	0.751217532467532 \\
345	0.78176589035964 \\
346	0.752269518675769 \\
347	0.75203425047175 \\
348	0.759393297674547 \\
349	0.763383838383838 \\
350	0.778263403263403 \\
351	0.762864083934856 \\
352	0.775047935112273 \\
353	0.763731889536118 \\
354	0.759050711084259 \\
355	0.748694014319014 \\
356	0.762361076423576 \\
357	0.765601800160624 \\
358	0.776097448558386 \\
359	0.733593142968143 \\
360	0.793798447773632 \\
361	0.765441011959854 \\
362	0.77113550511988 \\
363	0.771776812597125 \\
364	0.770774776438839 \\
365	0.773126136016761 \\
366	0.763880628919691 \\
367	0.763115898858086 \\
368	0.765820204101454 \\
369	0.772906867438117 \\
370	0.731805000555 \\
371	0.790335272366522 \\
372	0.784197009587635 \\
373	0.772759532134532 \\
374	0.755806009758215 \\
375	0.750558079073704 \\
376	0.775321835282772 \\
377	0.741845220751471 \\
378	0.746333354145854 \\
379	0.78286930083805 \\
380	0.77847729527417 \\
381	0.806690965284715 \\
382	0.827607808857808 \\
383	0.757860368797869 \\
384	0.783833115669053 \\
385	0.757162802475303 \\
386	0.76187891969142 \\
387	0.771341462357087 \\
388	0.752885417533855 \\
389	0.764511746933622 \\
390	0.7564120124667 \\
391	0.770488127324064 \\
392	0.760483570249195 \\
393	0.770673683954934 \\
394	0.756313781704406 \\
395	0.74642245009892 \\
396	0.728469729055666 \\
397	0.745113870504495 \\
398	0.723771150724276 \\
399	0.738235077320555 \\
400	0.766637485778111 \\
401	0.790046888875014 \\
402	0.742869413572538 \\
403	0.754129507645133 \\
404	0.747380202263015 \\
405	0.776628362956488 \\
406	0.783612654706405 \\
407	0.725651128385503 \\
408	0.787580908674659 \\
409	0.782108212967588 \\
410	0.787658890588578 \\
411	0.764061632811633 \\
412	0.78135809502997 \\
413	0.775440072993527 \\
414	0.788015218288656 \\
415	0.802861244658119 \\
416	0.770306776556776 \\
417	0.799786780060217 \\
418	0.791733136655011 \\
419	0.77220202821076 \\
420	0.801649134668436 \\
421	0.772836603500666 \\
422	0.767199553918304 \\
423	0.800160646645021 \\
424	0.813019489191364 \\
425	0.801638118825619 \\
426	0.748065866425241 \\
427	0.78164855977356 \\
428	0.770134552947053 \\
429	0.762026970425408 \\
430	0.772790963148693 \\
431	0.729265309343434 \\
432	0.773241905663781 \\
433	0.735845404595404 \\
434	0.76965994075369 \\
435	0.740885308268121 \\
436	0.764871456668332 \\
437	0.743657601079476 \\
438	0.726725249576812 \\
439	0.759741606432783 \\
440	0.794294117340992 \\
441	0.699942722208347 \\
442	0.727596731801695 \\
443	0.723110656704406 \\
444	0.766207837301587 \\
445	0.770765649281274 \\
446	0.74145381528194 \\
447	0.759951463467088 \\
448	0.78976314484127 \\
449	0.75152998043623 \\
450	0.777476744813601 \\
451	0.799789316586191 \\
452	0.750052431739013 \\
453	0.767831170565546 \\
454	0.795366265678765 \\
455	0.783489253801753 \\
456	0.771472082604895 \\
457	0.822775401681652 \\
458	0.768677936646687 \\
459	0.769084040959041 \\
460	0.772073564456377 \\
461	0.737817434301809 \\
462	0.754514582639582 \\
463	0.802047149898712 \\
464	0.772124099858475 \\
465	0.75890737887521 \\
466	0.753287337662338 \\
467	0.767122504231879 \\
468	0.749105278402153 \\
469	0.747537957855053 \\
470	0.722576384099217 \\
471	0.757539986055611 \\
472	0.763722432081807 \\
473	0.767445401820402 \\
474	0.72782976572039 \\
475	0.798242512695637 \\
476	0.723890128968254 \\
477	0.767751259157509 \\
478	0.77640631937507 \\
479	0.777088905018593 \\
480	0.811714457417582 \\
481	0.773953410764072 \\
482	0.789205456002331 \\
483	0.76391581508769 \\
484	0.785909637747873 \\
485	0.753689322829948 \\
486	0.767860481532356 \\
487	0.752702176806037 \\
488	0.810907495282495 \\
489	0.760176628926629 \\
490	0.776827925954764 \\
491	0.739264186334499 \\
492	0.76345859002109 \\
493	0.774388437083749 \\
494	0.821799095512331 \\
495	0.773601311882562 \\
496	0.757556245837496 \\
497	0.808945871836496 \\
498	0.791747531981907 \\
499	0.801976842601843 \\
500	0.774082601426351 \\
501	0.806480910582473 \\
502	0.793242434648685 \\
503	0.800284134268509 \\
504	0.781008206203519 \\
505	0.74564840541403 \\
506	0.787303972069597 \\
507	0.757745726495726 \\
508	0.755958516310079 \\
509	0.735079282468262 \\
510	0.752056277056277 \\
511	0.76453533445721 \\
512	0.773701620071105 \\
513	0.778215274309024 \\
514	0.808900756361694 \\
515	0.745609049895645 \\
516	0.793374724234099 \\
517	0.781880411510466 \\
518	0.76741160315379 \\
519	0.75274651563714 \\
520	0.784801705137183 \\
521	0.796621694277944 \\
522	0.742814433830059 \\
523	0.791646374458874 \\
524	0.760851084679209 \\
525	0.779639089209402 \\
526	0.787883475797078 \\
527	0.737730216831779 \\
528	0.748349740537241 \\
529	0.757842222881286 \\
530	0.73719050047175 \\
531	0.769325061937654 \\
532	0.768740721084471 \\
533	0.76704143352992 \\
534	0.777521783771784 \\
535	0.781034263938675 \\
536	0.790528828810079 \\
537	0.775443588529526 \\
538	0.77792203456266 \\
539	0.813490741029803 \\
540	0.781036021270396 \\
541	0.795060668498168 \\
542	0.824570828477079 \\
543	0.795016810446497 \\
544	0.7366975650983 \\
545	0.797361926268176 \\
546	0.795575821747696 \\
547	0.746587071782384 \\
548	0.746934077207515 \\
549	0.783085642655955 \\
550	0.768984314296814 \\
551	0.791698145604395 \\
552	0.769516464438339 \\
553	0.770737243761141 \\
554	0.752455096986347 \\
555	0.737020748348874 \\
556	0.753385403913897 \\
557	0.776194985569985 \\
558	0.764454360743423 \\
559	0.786657353063603 \\
560	0.741034226190476 \\
561	0.74833812021312 \\
562	0.777370481081418 \\
563	0.774482808857809 \\
564	0.777979572510823 \\
565	0.763311471514596 \\
566	0.772882846320346 \\
567	0.764450957029082 \\
568	0.781193676115551 \\
569	0.747226840000277 \\
570	0.777565858620546 \\
571	0.767431006493506 \\
572	0.786867826526544 \\
573	0.702781032284709 \\
574	0.767219976204351 \\
575	0.780946137196137 \\
576	0.744840662809413 \\
577	0.770300489441115 \\
578	0.80464468257437 \\
579	0.782701685878084 \\
580	0.778124729641274 \\
581	0.772485283813409 \\
582	0.779994397963148 \\
583	0.783434465350825 \\
584	0.791465522324897 \\
585	0.775271776834277 \\
586	0.78890760740485 \\
587	0.741246036949162 \\
588	0.761649114774115 \\
589	0.731617774586525 \\
590	0.758269920183983 \\
591	0.748353426087801 \\
592	0.763504594363969 \\
593	0.75501521048396 \\
594	0.776434676434676 \\
595	0.785536744199428 \\
596	0.737633590367965 \\
597	0.77053714514652 \\
598	0.785602721930847 \\
599	0.782287504162504 \\
};
\addplot [semithick, color1, forget plot]
table [row sep=\\]{%
0	0.00319940476190476 \\
1	0 \\
2	0.00826822916666667 \\
3	0.0050843253968254 \\
4	0.0083819826007326 \\
5	0.0135726686507937 \\
6	0.0142547123015873 \\
7	0.00669079184704185 \\
8	0.0114650974025974 \\
9	0.0222408234126984 \\
10	0.0259864267676768 \\
11	0.0215785082972583 \\
12	0.0312375992063492 \\
13	0.0219618055555556 \\
14	0.0208581349206349 \\
15	0.0382079725829726 \\
16	0.0378472222222222 \\
17	0.0407924107142857 \\
18	0.0266808712121212 \\
19	0.0378725874819625 \\
20	0.03125 \\
21	0.0373945932539683 \\
22	0.0417224702380952 \\
23	0.0200272817460317 \\
24	0.0433779761904762 \\
25	0.0377790178571429 \\
26	0.0269593253968254 \\
27	0.0325520833333333 \\
28	0.0400359623015873 \\
29	0.0279597355769231 \\
30	0.0387648809523809 \\
31	0.0571118551587302 \\
32	0.0548363095238095 \\
33	0.0482638888888889 \\
34	0.0520393668831169 \\
35	0.0353236607142857 \\
36	0.0317894345238095 \\
37	0.0411024305555556 \\
38	0.0408482142857143 \\
39	0.0607514880952381 \\
40	0.0316473890692641 \\
41	0.0619199810606061 \\
42	0.0816964285714286 \\
43	0.06328125 \\
44	0.0766726762820513 \\
45	0.0901366275676937 \\
46	0.0984397546897547 \\
47	0.129388189935065 \\
48	0.109442163739039 \\
49	0.114605316558442 \\
50	0.107724740537241 \\
51	0.149490353396603 \\
52	0.117839538933289 \\
53	0.168087054897717 \\
54	0.172813427891553 \\
55	0.182930307539683 \\
56	0.180867266414141 \\
57	0.174375450937951 \\
58	0.201231017246642 \\
59	0.18395274302903 \\
60	0.221505728238265 \\
61	0.226983736749362 \\
62	0.267356038059163 \\
63	0.292076109654235 \\
64	0.283244056290931 \\
65	0.272893165861916 \\
66	0.362580800276113 \\
67	0.343150729478854 \\
68	0.322608641358641 \\
69	0.397759228618603 \\
70	0.375581319721945 \\
71	0.404465499778 \\
72	0.430660051753802 \\
73	0.385499874665775 \\
74	0.46635417533855 \\
75	0.450933680035243 \\
76	0.475823677190865 \\
77	0.509971283822387 \\
78	0.574628453144078 \\
79	0.520248122674109 \\
80	0.538297963780501 \\
81	0.506336848914974 \\
82	0.541173193126318 \\
83	0.583170475357975 \\
84	0.555303573571037 \\
85	0.542983665639916 \\
86	0.570163634565362 \\
87	0.555525507478632 \\
88	0.582558601341254 \\
89	0.590551570825008 \\
90	0.608975031045343 \\
91	0.6128441003441 \\
92	0.634623297709235 \\
93	0.600859948619599 \\
94	0.614034885151613 \\
95	0.640229171869797 \\
96	0.638965374902875 \\
97	0.653304247835497 \\
98	0.702092612248862 \\
99	0.666317536630037 \\
100	0.645262159715285 \\
101	0.591232794472684 \\
102	0.652134598193881 \\
103	0.654147068209568 \\
104	0.697181074134199 \\
105	0.69297666309385 \\
106	0.683506077256077 \\
107	0.662335277569653 \\
108	0.6544315340501 \\
109	0.658935508935509 \\
110	0.64609982031857 \\
111	0.633438148350832 \\
112	0.648010843323343 \\
113	0.677514629467755 \\
114	0.665939629967663 \\
115	0.684041339073508 \\
116	0.673058191808192 \\
117	0.700180917173104 \\
118	0.678949956293706 \\
119	0.681532139735265 \\
120	0.693323083166833 \\
121	0.71587422993673 \\
122	0.693014971139971 \\
123	0.725036595349095 \\
124	0.707904474454639 \\
125	0.715490152208902 \\
126	0.727886826714952 \\
127	0.707889506327006 \\
128	0.700437829378546 \\
129	0.705840253496503 \\
130	0.683244157037815 \\
131	0.733356639020701 \\
132	0.726706973581973 \\
133	0.700239213911089 \\
134	0.688315534748512 \\
135	0.725452932484183 \\
136	0.714742419039294 \\
137	0.661471467411265 \\
138	0.707028561716062 \\
139	0.669929008765357 \\
140	0.667700355200355 \\
141	0.680507621718559 \\
142	0.67716236990134 \\
143	0.742144617535242 \\
144	0.705158079767454 \\
145	0.66493812177406 \\
146	0.69415941238092 \\
147	0.644639628141466 \\
148	0.694550164908621 \\
149	0.656686242128507 \\
150	0.681920856920857 \\
151	0.695075515541605 \\
152	0.636185991551341 \\
153	0.701047006721349 \\
154	0.648874085983461 \\
155	0.632911532911533 \\
156	0.685886627991891 \\
157	0.630955337709853 \\
158	0.666378903362504 \\
159	0.627145857614608 \\
160	0.677593430109515 \\
161	0.633375152217064 \\
162	0.665642649833826 \\
163	0.696264535902694 \\
164	0.681386255696917 \\
165	0.658227059398934 \\
166	0.651433332292707 \\
167	0.667439504939505 \\
168	0.65321566454379 \\
169	0.686859949945887 \\
170	0.656992584876316 \\
171	0.632695276671428 \\
172	0.663643994893995 \\
173	0.650969650498603 \\
174	0.62999142784299 \\
175	0.638095490600086 \\
176	0.628396775559 \\
177	0.63920389506327 \\
178	0.641133671016484 \\
179	0.683795024420024 \\
180	0.658870715936433 \\
181	0.721915459438437 \\
182	0.693996910207848 \\
183	0.689522760225885 \\
184	0.747697875041625 \\
185	0.699172181984682 \\
186	0.691448437438787 \\
187	0.692845869408369 \\
188	0.727316785542889 \\
189	0.733668474856434 \\
190	0.699783820142276 \\
191	0.679410216519592 \\
192	0.720933743544037 \\
193	0.719182163322788 \\
194	0.728023105366855 \\
195	0.683216180010757 \\
196	0.727621423715174 \\
197	0.693839363761239 \\
198	0.700073045316611 \\
199	0.725569763161675 \\
200	0.728546211243821 \\
201	0.759535603285603 \\
202	0.735723868145743 \\
203	0.703927761372614 \\
204	0.682087306076277 \\
205	0.692873359279609 \\
206	0.712652214513428 \\
207	0.711739024353914 \\
208	0.683721782710753 \\
209	0.707216000655798 \\
210	0.732068110606713 \\
211	0.731935446089858 \\
212	0.782387534340659 \\
213	0.750041928557553 \\
214	0.710282794462482 \\
215	0.695060982216313 \\
216	0.709792724636475 \\
217	0.738284359125352 \\
218	0.734509551926831 \\
219	0.73078245539183 \\
220	0.716216269433181 \\
221	0.716151903651904 \\
222	0.730229420038703 \\
223	0.751845506923632 \\
224	0.714214717063981 \\
225	0.723024740016927 \\
226	0.700946406024531 \\
227	0.695399522352647 \\
228	0.716421642593517 \\
229	0.741759159152251 \\
230	0.675785083482694 \\
231	0.766139235050081 \\
232	0.741766523407148 \\
233	0.731931783494283 \\
234	0.730478635947386 \\
235	0.713098273254523 \\
236	0.718525940205628 \\
237	0.725134826111389 \\
238	0.740220521665834 \\
239	0.705234609140859 \\
240	0.732718800296925 \\
241	0.720975660624098 \\
242	0.762888890623265 \\
243	0.742876567876568 \\
244	0.747568681828792 \\
245	0.773190358967013 \\
246	0.751704754599975 \\
247	0.742148433164058 \\
248	0.725419866363534 \\
249	0.704839171245421 \\
250	0.751277693660506 \\
251	0.766762209145021 \\
252	0.755733264131701 \\
253	0.756173730783106 \\
254	0.720164427586302 \\
255	0.750027619949495 \\
256	0.786382866265679 \\
257	0.749486039134477 \\
258	0.747743142274392 \\
259	0.725253155239368 \\
260	0.714149938092953 \\
261	0.743301013222888 \\
262	0.729906703539516 \\
263	0.740862934808247 \\
264	0.747694449647574 \\
265	0.738640288939002 \\
266	0.728608761030636 \\
267	0.734458271762959 \\
268	0.765021959625772 \\
269	0.734489815739816 \\
270	0.732215700965701 \\
271	0.754734068015318 \\
272	0.756426477933831 \\
273	0.714558293269231 \\
274	0.719926786091768 \\
275	0.735246567668443 \\
276	0.756634008862869 \\
277	0.703486400752026 \\
278	0.727386831407971 \\
279	0.694798517454767 \\
280	0.704459429459429 \\
281	0.749125087279958 \\
282	0.684714172673731 \\
283	0.734579656454656 \\
284	0.747411182567432 \\
285	0.750420542998668 \\
286	0.737035646873229 \\
287	0.726623940295815 \\
288	0.767137723387723 \\
289	0.75101276761433 \\
290	0.724598751942502 \\
291	0.733053968600843 \\
292	0.755625364219114 \\
293	0.750429801509765 \\
294	0.750044335005272 \\
295	0.757167572011322 \\
296	0.762703615828616 \\
297	0.772446173618049 \\
298	0.761040066704129 \\
299	0.739054767995485 \\
300	0.765877048298923 \\
301	0.761165657259407 \\
302	0.728556079337329 \\
303	0.763974569650121 \\
304	0.781634511322011 \\
305	0.778784279956155 \\
306	0.768564296712274 \\
307	0.764823024198024 \\
308	0.76642818986569 \\
309	0.734450683864746 \\
310	0.733852363424974 \\
311	0.74643416479354 \\
312	0.798489531302031 \\
313	0.7870739503552 \\
314	0.782503390706516 \\
315	0.790522585053835 \\
316	0.73178202006327 \\
317	0.705695628156566 \\
318	0.725855329566267 \\
319	0.761950582607752 \\
320	0.755020175137363 \\
321	0.754617474539349 \\
322	0.748156184093684 \\
323	0.735625702422577 \\
324	0.770708024614275 \\
325	0.781782176922342 \\
326	0.741762006375977 \\
327	0.757972149378399 \\
328	0.778371259816572 \\
329	0.725482503607503 \\
330	0.729186525280275 \\
331	0.772795758928571 \\
332	0.781298497509435 \\
333	0.807777248792874 \\
334	0.742689775502276 \\
335	0.775752827901265 \\
336	0.771986195228383 \\
337	0.712900009763521 \\
338	0.749921844648407 \\
339	0.75419442810479 \\
340	0.775309846403596 \\
341	0.735713943940047 \\
342	0.738605009675782 \\
343	0.748594634532134 \\
344	0.747458065576172 \\
345	0.724935681403972 \\
346	0.767119731779658 \\
347	0.748033594265212 \\
348	0.728897708194583 \\
349	0.759410598082473 \\
350	0.780189030740501 \\
351	0.733672123423961 \\
352	0.796687123640248 \\
353	0.752162529310967 \\
354	0.744296308266896 \\
355	0.763097689177652 \\
356	0.753349601787102 \\
357	0.761108378192477 \\
358	0.756344306734932 \\
359	0.73876041493229 \\
360	0.751213608440171 \\
361	0.780399721805972 \\
362	0.777428604381729 \\
363	0.760132422858574 \\
364	0.779039855671289 \\
365	0.73111221397067 \\
366	0.741353720064657 \\
367	0.783709767050759 \\
368	0.767550071456321 \\
369	0.702605749285437 \\
370	0.759348160520035 \\
371	0.772313927392052 \\
372	0.732085991265679 \\
373	0.782955174166112 \\
374	0.752336344466399 \\
375	0.730047140869751 \\
376	0.794605828199578 \\
377	0.733423932837995 \\
378	0.748750728438228 \\
379	0.767725595482948 \\
380	0.75849198544511 \\
381	0.778788589117174 \\
382	0.763199365738428 \\
383	0.782575605817793 \\
384	0.801960409382284 \\
385	0.821956884261572 \\
386	0.754265114226052 \\
387	0.800462948509823 \\
388	0.734760768745144 \\
389	0.747725798507048 \\
390	0.729859268335831 \\
391	0.746271588654401 \\
392	0.746888051184926 \\
393	0.729214708902209 \\
394	0.761691585324398 \\
395	0.78119476010101 \\
396	0.764039292508962 \\
397	0.69833342336789 \\
398	0.74835609269203 \\
399	0.781952682733932 \\
400	0.739867879516317 \\
401	0.700100073537573 \\
402	0.7227783501221 \\
403	0.776396216630591 \\
404	0.755783149142524 \\
405	0.755065615807803 \\
406	0.743859265734266 \\
407	0.770353681245225 \\
408	0.761134882274588 \\
409	0.720909299034299 \\
410	0.741193572052947 \\
411	0.756519483886212 \\
412	0.766913973485665 \\
413	0.754625062437562 \\
414	0.76645435727467 \\
415	0.774385293525919 \\
416	0.707368521235709 \\
417	0.782588288447663 \\
418	0.712650387041472 \\
419	0.716793597538083 \\
420	0.755129419191919 \\
421	0.729442078040424 \\
422	0.789320825212369 \\
423	0.749417032620158 \\
424	0.806474840263903 \\
425	0.777275368371416 \\
426	0.75614771513209 \\
427	0.773671033827284 \\
428	0.743708439997503 \\
429	0.770915672868797 \\
430	0.745860693126318 \\
431	0.751365626560939 \\
432	0.772838728112165 \\
433	0.75783496017871 \\
434	0.748991469057258 \\
435	0.786786839590148 \\
436	0.785011602980353 \\
437	0.799795213467088 \\
438	0.732018719128094 \\
439	0.749691953012265 \\
440	0.74087347497274 \\
441	0.740628055563717 \\
442	0.766774761696637 \\
443	0.779819442710068 \\
444	0.789775242118992 \\
445	0.792135137125946 \\
446	0.775951652514152 \\
447	0.7468506753663 \\
448	0.763028334512709 \\
449	0.77641184770091 \\
450	0.78817974560162 \\
451	0.774647921522922 \\
452	0.797527580926018 \\
453	0.772908211580087 \\
454	0.797119893994894 \\
455	0.757310359677087 \\
456	0.787057170260295 \\
457	0.772856895708458 \\
458	0.777972089185325 \\
459	0.804660248605561 \\
460	0.71961371961372 \\
461	0.776618130133755 \\
462	0.774468673687423 \\
463	0.808149494500597 \\
464	0.764697932276057 \\
465	0.73315740249334 \\
466	0.799949746434121 \\
467	0.733953156218781 \\
468	0.756561971015096 \\
469	0.737565862956488 \\
470	0.752229709628607 \\
471	0.76709128545066 \\
472	0.753601346570097 \\
473	0.757514317279942 \\
474	0.757269892586988 \\
475	0.788058122433122 \\
476	0.771120004127817 \\
477	0.759653974497724 \\
478	0.779366297254625 \\
479	0.758751730040792 \\
480	0.754987679535377 \\
481	0.786530093170718 \\
482	0.764240989045676 \\
483	0.786574406496281 \\
484	0.768349228896104 \\
485	0.78215651535964 \\
486	0.779476079476079 \\
487	0.771385081931957 \\
488	0.733106433497059 \\
489	0.782038973077576 \\
490	0.78613836250555 \\
491	0.765054698671427 \\
492	0.779755747724498 \\
493	0.782654043005605 \\
494	0.775019576777389 \\
495	0.794446959290709 \\
496	0.804368851634477 \\
497	0.74992363493597 \\
498	0.796524482461982 \\
499	0.747599907053516 \\
500	0.805568921733903 \\
501	0.766206796675546 \\
502	0.721105196886447 \\
503	0.751957982530133 \\
504	0.781699116855367 \\
505	0.758236555111555 \\
506	0.747725711788212 \\
507	0.798786066364191 \\
508	0.808216176184926 \\
509	0.772154971764347 \\
510	0.765069262334888 \\
511	0.807599605255855 \\
512	0.754157994782995 \\
513	0.735139209748584 \\
514	0.765899118242868 \\
515	0.757141296203796 \\
516	0.746883932040182 \\
517	0.731571271610334 \\
518	0.788798007548007 \\
519	0.775314822534491 \\
520	0.765713973526474 \\
521	0.78910768051393 \\
522	0.760562397671773 \\
523	0.776565686917249 \\
524	0.764341886412199 \\
525	0.770190551635864 \\
526	0.79575255473693 \\
527	0.727947637023924 \\
528	0.752630321322876 \\
529	0.780218263109976 \\
530	0.769717738858364 \\
531	0.744863773379398 \\
532	0.793858680382117 \\
533	0.769017290409753 \\
534	0.767277427433677 \\
535	0.76369962502775 \\
536	0.773495578665615 \\
537	0.76214638226357 \\
538	0.739013026903652 \\
539	0.774671834242147 \\
540	0.778771585848792 \\
541	0.733178494299818 \\
542	0.758345438262718 \\
543	0.787921887140637 \\
544	0.793237101440226 \\
545	0.772850131639194 \\
546	0.790871519973082 \\
547	0.779813892704518 \\
548	0.760435289536852 \\
549	0.749652430902431 \\
550	0.749917313589188 \\
551	0.779291156657885 \\
552	0.798769589785215 \\
553	0.762994470806971 \\
554	0.785313948204573 \\
555	0.755768340625877 \\
556	0.80974420544733 \\
557	0.786949664408304 \\
558	0.730345088938839 \\
559	0.785046810828061 \\
560	0.795510695901321 \\
561	0.758034847097347 \\
562	0.773611631424131 \\
563	0.734819347319347 \\
564	0.76067981497669 \\
565	0.766222666222666 \\
566	0.76983702061827 \\
567	0.729200161817349 \\
568	0.747610115578865 \\
569	0.78741384483572 \\
570	0.748734858891109 \\
571	0.75729502702159 \\
572	0.713080582611833 \\
573	0.770392454767455 \\
574	0.758443935557906 \\
575	0.7629260496448 \\
576	0.765489892052392 \\
577	0.766189052471222 \\
578	0.77213346549284 \\
579	0.753126474220224 \\
580	0.785715673215673 \\
581	0.761095978674104 \\
582	0.796600036248474 \\
583	0.781218954656454 \\
584	0.801279753232878 \\
585	0.738336641656954 \\
586	0.763662205849706 \\
587	0.787883774211899 \\
588	0.748016675165113 \\
589	0.770200979575979 \\
590	0.736798943244256 \\
591	0.73152617781524 \\
592	0.747405849358974 \\
593	0.767974603521478 \\
594	0.769618705946831 \\
595	0.780970591908092 \\
596	0.761215997544123 \\
597	0.813957612291712 \\
598	0.737103781635032 \\
599	0.769736318369131 \\
};
\addplot [semithick, color2, dashed, forget plot]
table [row sep=\\]{%
0	0.00490056818181818 \\
1	0.00680966739766082 \\
2	0.00868951917270531 \\
3	0.00421772875816994 \\
4	0.00844663149350649 \\
5	0.00950584181040837 \\
6	0.0127705627705628 \\
7	0.00729166666666667 \\
8	0.0100477430555556 \\
9	0.0146024981962482 \\
10	0.00675702973497091 \\
11	0.0287267203282828 \\
12	0.0400524082739156 \\
13	0.0376488095238095 \\
14	0.051385802829417 \\
15	0.0435525322818521 \\
16	0.0401098531268568 \\
17	0.0466216075591076 \\
18	0.0388462355245822 \\
19	0.0234809027777778 \\
20	0.0402324952416864 \\
21	0.0362835948773449 \\
22	0.0306491251803752 \\
23	0.0337544833638584 \\
24	0.0338698755259646 \\
25	0.0520102293539794 \\
26	0.0597753375097125 \\
27	0.0796241769925981 \\
28	0.0350570436507937 \\
29	0.069073547979798 \\
30	0.0649241383616384 \\
31	0.0384713223867636 \\
32	0.0642733134920635 \\
33	0.0499822226384726 \\
34	0.0966432872682873 \\
35	0.0884809634809635 \\
36	0.0775319298756799 \\
37	0.0525709967116217 \\
38	0.0693296287046287 \\
39	0.049607683982684 \\
40	0.0785629734848485 \\
41	0.0966019657425907 \\
42	0.0786517302142302 \\
43	0.0947694016053391 \\
44	0.0483577186702187 \\
45	0.0897297484909956 \\
46	0.0946472798035298 \\
47	0.0882334603801995 \\
48	0.0966412493756244 \\
49	0.0616985878704629 \\
50	0.0793302183927184 \\
51	0.0932104137182262 \\
52	0.0977335164835165 \\
53	0.0965933295666752 \\
54	0.101937168734044 \\
55	0.10728462646637 \\
56	0.0735317365946961 \\
57	0.0996938825063825 \\
58	0.102970504888558 \\
59	0.100071326243201 \\
60	0.086355294011544 \\
61	0.0821134160977911 \\
62	0.128338133220946 \\
63	0.142626283409164 \\
64	0.104948039204153 \\
65	0.0921034962280366 \\
66	0.111080382036264 \\
67	0.124021247849373 \\
68	0.0882878227777125 \\
69	0.0982843086243822 \\
70	0.11255777387441 \\
71	0.108365420578196 \\
72	0.106405486874237 \\
73	0.0835927387053306 \\
74	0.0937320263800527 \\
75	0.0856258541805417 \\
76	0.122624684343434 \\
77	0.123892947330447 \\
78	0.13776197250462 \\
79	0.0960118019745777 \\
80	0.113351145382395 \\
81	0.100600575129527 \\
82	0.110979950145851 \\
83	0.125287945333176 \\
84	0.111314033189033 \\
85	0.136373211143432 \\
86	0.123817501942502 \\
87	0.0975698997183372 \\
88	0.117671477827728 \\
89	0.122609113211135 \\
90	0.126210855117105 \\
91	0.0950752043740143 \\
92	0.118291300713176 \\
93	0.140176530985355 \\
94	0.120526578201211 \\
95	0.115373905608281 \\
96	0.158466907122698 \\
97	0.11640783261877 \\
98	0.131134924103674 \\
99	0.143245383089133 \\
100	0.147559298340548 \\
101	0.161453837981871 \\
102	0.164176122080534 \\
103	0.15530054351286 \\
104	0.16771978021978 \\
105	0.151412736568987 \\
106	0.119074985418348 \\
107	0.179957586756639 \\
108	0.184008569555445 \\
109	0.167490751207856 \\
110	0.190663607847112 \\
111	0.235993689980363 \\
112	0.213336691392758 \\
113	0.190175887857613 \\
114	0.211580384994907 \\
115	0.258227217533284 \\
116	0.18956596788628 \\
117	0.282140190538628 \\
118	0.226368683399933 \\
119	0.236051556949994 \\
120	0.318904272810523 \\
121	0.1931772871226 \\
122	0.257773411484349 \\
123	0.231262903252793 \\
124	0.285752181845932 \\
125	0.281326542126174 \\
126	0.269313320462217 \\
127	0.297571342832892 \\
128	0.267515990953491 \\
129	0.278349145237932 \\
130	0.269552239784317 \\
131	0.280644284094603 \\
132	0.292925562744037 \\
133	0.284750319125319 \\
134	0.347375411047286 \\
135	0.386693081224331 \\
136	0.349956787775455 \\
137	0.40875819824579 \\
138	0.355500359015984 \\
139	0.342735736485737 \\
140	0.327373044831784 \\
141	0.414234983766234 \\
142	0.37275359068606 \\
143	0.417204183173852 \\
144	0.399560968035233 \\
145	0.366499963731122 \\
146	0.398937780969031 \\
147	0.41415236239455 \\
148	0.416804542233867 \\
149	0.390459335341301 \\
150	0.451274398344711 \\
151	0.444612071955822 \\
152	0.41888637812351 \\
153	0.432006838647464 \\
154	0.466130150916335 \\
155	0.427302743847098 \\
156	0.450317713587475 \\
157	0.468598076249731 \\
158	0.493985382441991 \\
159	0.441942389208014 \\
160	0.449408938436971 \\
161	0.474841564685315 \\
162	0.490910049917403 \\
163	0.507062287691883 \\
164	0.50711592074184 \\
165	0.518366377546065 \\
166	0.50757753531191 \\
167	0.49829901001776 \\
168	0.519634725976638 \\
169	0.505114325124116 \\
170	0.520461486048996 \\
171	0.543061086431951 \\
172	0.52585778574979 \\
173	0.559317572866697 \\
174	0.565909365093648 \\
175	0.570086301566081 \\
176	0.531889009428072 \\
177	0.512773537991828 \\
178	0.591389191491988 \\
179	0.557958277854756 \\
180	0.532572462259962 \\
181	0.59235770739677 \\
182	0.584885231955544 \\
183	0.561288934462188 \\
184	0.587980195356114 \\
185	0.60074396546436 \\
186	0.593474383575124 \\
187	0.610835605366855 \\
188	0.62112017756549 \\
189	0.567531513625264 \\
190	0.590180956196581 \\
191	0.646131658683057 \\
192	0.602604795378233 \\
193	0.624765642343767 \\
194	0.631009246829559 \\
195	0.657967054311143 \\
196	0.662772155591549 \\
197	0.632626887458302 \\
198	0.628719222974738 \\
199	0.633368293777301 \\
200	0.657269306899241 \\
201	0.659001351353325 \\
202	0.623452413357841 \\
203	0.666594443568128 \\
204	0.605126723256319 \\
205	0.63789136721122 \\
206	0.685158102977496 \\
207	0.635847531757458 \\
208	0.647055223422411 \\
209	0.665158865848204 \\
210	0.670467933312602 \\
211	0.641794837107337 \\
212	0.693262293262293 \\
213	0.680803105885343 \\
214	0.671002659513689 \\
215	0.681316209831835 \\
216	0.73230029519092 \\
217	0.677523293699764 \\
218	0.675274226641414 \\
219	0.688812966547342 \\
220	0.734261658480408 \\
221	0.68389493472214 \\
222	0.629379041097791 \\
223	0.687289923618049 \\
224	0.702897081217394 \\
225	0.640297593031968 \\
226	0.704433457167832 \\
227	0.701931315212565 \\
228	0.731093082264957 \\
229	0.694425886613387 \\
230	0.723279931873682 \\
231	0.687462809029421 \\
232	0.720845230392565 \\
233	0.674467546342546 \\
234	0.708382493986814 \\
235	0.749584731547047 \\
236	0.712764137924983 \\
237	0.73406313738345 \\
238	0.742154828678266 \\
239	0.741860183576636 \\
240	0.680260234557109 \\
241	0.725054654546842 \\
242	0.685905357918226 \\
243	0.684559083685922 \\
244	0.705752103798979 \\
245	0.721370925081863 \\
246	0.738140982281607 \\
247	0.662434958321907 \\
248	0.716168591926864 \\
249	0.736699893754122 \\
250	0.676075441103015 \\
251	0.677374123272561 \\
252	0.709408733627484 \\
253	0.724111565517816 \\
254	0.693762292395105 \\
255	0.680975838397714 \\
256	0.734193305109278 \\
257	0.684333224475688 \\
258	0.694558284597347 \\
259	0.695920810959873 \\
260	0.736253915355478 \\
261	0.747165811618937 \\
262	0.722007409461453 \\
263	0.707969435588921 \\
264	0.764998027911631 \\
265	0.73653982736014 \\
266	0.730765696976635 \\
267	0.703572397781956 \\
268	0.727123907546702 \\
269	0.738526187354312 \\
270	0.720075237262737 \\
271	0.745523226773227 \\
272	0.732720101079476 \\
273	0.687576876248751 \\
274	0.753856863622488 \\
275	0.734680185266123 \\
276	0.716863930340493 \\
277	0.7159114756771 \\
278	0.734007789085914 \\
279	0.741814617929048 \\
280	0.742738956560188 \\
281	0.739166866119991 \\
282	0.753532990447053 \\
283	0.735431061993562 \\
284	0.748015543994404 \\
285	0.693028065684316 \\
286	0.746566996371684 \\
287	0.712650153665779 \\
288	0.738782892964878 \\
289	0.701656382269047 \\
290	0.738283808205683 \\
291	0.718826913231785 \\
292	0.726258137287549 \\
293	0.7221392756549 \\
294	0.742197646103896 \\
295	0.698320082695083 \\
296	0.740584277854499 \\
297	0.774876469017094 \\
298	0.715094347033685 \\
299	0.773866649843212 \\
300	0.732383155039405 \\
301	0.762277821239218 \\
302	0.768902892960797 \\
303	0.758318677849928 \\
304	0.722273126179376 \\
305	0.727787668826272 \\
306	0.75131327006327 \\
307	0.74180078244876 \\
308	0.734145932192807 \\
309	0.740006391178266 \\
310	0.771496558996559 \\
311	0.750489831349206 \\
312	0.70495864812271 \\
313	0.781902559246309 \\
314	0.742966408591408 \\
315	0.802986683455433 \\
316	0.7484819000444 \\
317	0.759611308830059 \\
318	0.727239422138319 \\
319	0.761705915612166 \\
320	0.731597444120422 \\
321	0.730371928418804 \\
322	0.713887072894425 \\
323	0.736130057901657 \\
324	0.731525332306582 \\
325	0.764031021062271 \\
326	0.726588775807526 \\
327	0.730942863560051 \\
328	0.728968405726218 \\
329	0.751430644008769 \\
330	0.719488410894661 \\
331	0.764518250846376 \\
332	0.761670100732601 \\
333	0.759260913718634 \\
334	0.727641325688201 \\
335	0.751390536546786 \\
336	0.733241554063197 \\
337	0.747147340506715 \\
338	0.732145285270285 \\
339	0.726272729007104 \\
340	0.73405149537962 \\
341	0.77315052454855 \\
342	0.728620401759188 \\
343	0.728243154415029 \\
344	0.729545020951271 \\
345	0.749262109418359 \\
346	0.733069735454604 \\
347	0.771750861985237 \\
348	0.767708246614496 \\
349	0.73029589767871 \\
350	0.737641741938617 \\
351	0.731464997675935 \\
352	0.762507024225774 \\
353	0.755102449633699 \\
354	0.758901642813456 \\
355	0.744912812881563 \\
356	0.760664617500555 \\
357	0.726749136790497 \\
358	0.773389847999223 \\
359	0.736475960212173 \\
360	0.738422932000598 \\
361	0.772224823787323 \\
362	0.730621114996115 \\
363	0.741386651542902 \\
364	0.764812054265179 \\
365	0.788323200237263 \\
366	0.730031123390498 \\
367	0.774233362123987 \\
368	0.736884079462204 \\
369	0.738401333909146 \\
370	0.762284864594147 \\
371	0.743547488561275 \\
372	0.748126688278826 \\
373	0.779463852120102 \\
374	0.770106195887446 \\
375	0.752805451288907 \\
376	0.75738328511766 \\
377	0.762245371815684 \\
378	0.738800435675436 \\
379	0.689481356344142 \\
380	0.750646402208902 \\
381	0.748107121636533 \\
382	0.720075237262737 \\
383	0.774601157114944 \\
384	0.736667846042846 \\
385	0.74429364038739 \\
386	0.760262870809746 \\
387	0.7487060248779 \\
388	0.749662186771562 \\
389	0.775452963090831 \\
390	0.775071846556221 \\
391	0.739462945908258 \\
392	0.696201649912587 \\
393	0.748801433452169 \\
394	0.742736842150905 \\
395	0.79262278520091 \\
396	0.757669850064152 \\
397	0.806912228396603 \\
398	0.777172263500389 \\
399	0.805185309482184 \\
400	0.794183094275006 \\
401	0.751305248570873 \\
402	0.759918605954911 \\
403	0.746382391097464 \\
404	0.777445969828782 \\
405	0.775156072226385 \\
406	0.772150093829782 \\
407	0.780402155034508 \\
408	0.757800164245477 \\
409	0.756177980006105 \\
410	0.788313834602897 \\
411	0.784808716610187 \\
412	0.784562511630526 \\
413	0.735086436772292 \\
414	0.754956445719313 \\
415	0.741673690892441 \\
416	0.73580772526085 \\
417	0.760049542471418 \\
418	0.745715439074814 \\
419	0.772714803068663 \\
420	0.762511811615672 \\
421	0.775557862276612 \\
422	0.771037534514097 \\
423	0.747567883505383 \\
424	0.816053022498335 \\
425	0.746640013632201 \\
426	0.73096218018093 \\
427	0.758445851023976 \\
428	0.723785046142583 \\
429	0.759094898157398 \\
430	0.749311582514707 \\
431	0.723303020763958 \\
432	0.763612039002664 \\
433	0.745504498055049 \\
434	0.757265911172161 \\
435	0.756897724671162 \\
436	0.752482933732934 \\
437	0.77486653971029 \\
438	0.767231813325563 \\
439	0.755026079669922 \\
440	0.754694697663448 \\
441	0.746809093684093 \\
442	0.749593028499278 \\
443	0.745327288112215 \\
444	0.748638340825841 \\
445	0.790018575174825 \\
446	0.798735316660109 \\
447	0.734415129141692 \\
448	0.748403246059496 \\
449	0.731031381812631 \\
450	0.708961405918643 \\
451	0.745678843725719 \\
452	0.724644646611558 \\
453	0.717809404137529 \\
454	0.780075089840715 \\
455	0.746119982448107 \\
456	0.737366113767768 \\
457	0.742595013493451 \\
458	0.705089503021488 \\
459	0.745396920787546 \\
460	0.757376607767233 \\
461	0.758315599331224 \\
462	0.7667503503441 \\
463	0.758300922168109 \\
464	0.731459599428349 \\
465	0.759796670343545 \\
466	0.746237494123524 \\
467	0.76878832972583 \\
468	0.753018769425019 \\
469	0.746410230394605 \\
470	0.789500039890665 \\
471	0.80229327963703 \\
472	0.781622847638472 \\
473	0.770623430389055 \\
474	0.776222518800644 \\
475	0.773265926781552 \\
476	0.735431408868909 \\
477	0.773851409007659 \\
478	0.784712162837163 \\
479	0.761454368497107 \\
480	0.746028363997114 \\
481	0.765421991203241 \\
482	0.779440047799423 \\
483	0.767437813922189 \\
484	0.79179405918836 \\
485	0.780454354673104 \\
486	0.790896866103667 \\
487	0.750615920537795 \\
488	0.790660728160728 \\
489	0.726659993652181 \\
490	0.747685430888556 \\
491	0.745562206890332 \\
492	0.789463141025641 \\
493	0.744994957299645 \\
494	0.767627684815185 \\
495	0.789200252872128 \\
496	0.757142011634199 \\
497	0.791071341852592 \\
498	0.746769701652514 \\
499	0.785396157661783 \\
500	0.759956536519036 \\
501	0.751132461288711 \\
502	0.762151126294049 \\
503	0.755153527028527 \\
504	0.718718651140526 \\
505	0.77118651747558 \\
506	0.757935467310467 \\
507	0.789143647151459 \\
508	0.770720685564436 \\
509	0.771358991039598 \\
510	0.790638493706049 \\
511	0.811768569971695 \\
512	0.773156942883505 \\
513	0.773477390664891 \\
514	0.777446166221441 \\
515	0.744962025821401 \\
516	0.734993543782606 \\
517	0.752738624222999 \\
518	0.776519288513773 \\
519	0.781587369432039 \\
520	0.799803451756577 \\
521	0.734565456245144 \\
522	0.750947424970862 \\
523	0.72434659652031 \\
524	0.756410256410256 \\
525	0.758331794073981 \\
526	0.743519154456654 \\
527	0.741522756757131 \\
528	0.740202115592741 \\
529	0.769428314740815 \\
530	0.769898764430014 \\
531	0.778226832144571 \\
532	0.803354496837227 \\
533	0.758581175768676 \\
534	0.769427751068376 \\
535	0.756392479048729 \\
536	0.713720459228272 \\
537	0.738107226974415 \\
538	0.76925907782985 \\
539	0.797674504315129 \\
540	0.745557003760129 \\
541	0.773395072809135 \\
542	0.788609415757853 \\
543	0.786588888542013 \\
544	0.784792767995893 \\
545	0.760719402125652 \\
546	0.76154722014097 \\
547	0.797203034404321 \\
548	0.799795863858364 \\
549	0.730749979187479 \\
550	0.770496257215007 \\
551	0.783732478459041 \\
552	0.754884814684906 \\
553	0.763319709804085 \\
554	0.764266636141636 \\
555	0.767801586689455 \\
556	0.769981949474137 \\
557	0.753775456314519 \\
558	0.769984204163891 \\
559	0.763581687409812 \\
560	0.764401180416805 \\
561	0.735507933141205 \\
562	0.765403736888112 \\
563	0.746080872252747 \\
564	0.759985552896858 \\
565	0.784134658744034 \\
566	0.727914771860084 \\
567	0.789234506812632 \\
568	0.788743440995279 \\
569	0.758131170045232 \\
570	0.774519230769231 \\
571	0.77018272526085 \\
572	0.779389230561105 \\
573	0.770150442898605 \\
574	0.787868468337218 \\
575	0.797348137973138 \\
576	0.797570701867576 \\
577	0.744118771853147 \\
578	0.733339447011322 \\
579	0.759286115742917 \\
580	0.74682240502553 \\
581	0.771911421911422 \\
582	0.788154423701298 \\
583	0.763433016891197 \\
584	0.770600449897325 \\
585	0.778745386557886 \\
586	0.7442717005217 \\
587	0.774859862359862 \\
588	0.728458301298375 \\
589	0.764291416049228 \\
590	0.76823146471584 \\
591	0.776046826437451 \\
592	0.786112281815407 \\
593	0.740507972929848 \\
594	0.745368043414918 \\
595	0.77241935709307 \\
596	0.780356622544123 \\
597	0.764275654900655 \\
598	0.775600874819625 \\
599	0.790199730824731 \\
};
\addplot [semithick, color3, dotted, forget plot]
table [row sep=\\]{%
0	0.0009765625 \\
1	0.00467133968972204 \\
2	0.00526842948717949 \\
3	0.0117022951007326 \\
4	0.00470666486291486 \\
5	0.0111917162698413 \\
6	0.0191210979584141 \\
7	0.031818810529748 \\
8	0.02834169345876 \\
9	0.0163194444444444 \\
10	0.0385898093134935 \\
11	0.0332889337185003 \\
12	0.0450594808796724 \\
13	0.043763875013875 \\
14	0.0489801705056639 \\
15	0.0476950566794317 \\
16	0.0573050009630892 \\
17	0.059154111819553 \\
18	0.0349226728132978 \\
19	0.0607172902669226 \\
20	0.0422112586791863 \\
21	0.0519380906922025 \\
22	0.0828547024065116 \\
23	0.0661702864344118 \\
24	0.0727961845354937 \\
25	0.108367298926993 \\
26	0.0933336628649129 \\
27	0.11912506279464 \\
28	0.0964105395661102 \\
29	0.0794517905501362 \\
30	0.0975617698273948 \\
31	0.0657651261626445 \\
32	0.0906715246558996 \\
33	0.0612651237651238 \\
34	0.0688153426434676 \\
35	0.104829718892219 \\
36	0.0644449517496392 \\
37	0.0790227133977134 \\
38	0.0886524933399933 \\
39	0.0690114139332889 \\
40	0.0855958711427461 \\
41	0.0691512914169164 \\
42	0.0740666579131653 \\
43	0.0782597177128427 \\
44	0.067263378982129 \\
45	0.0388313509407259 \\
46	0.0586660141758826 \\
47	0.0559239805333555 \\
48	0.0660127632783883 \\
49	0.0430723744457297 \\
50	0.0737864543168819 \\
51	0.081460509976135 \\
52	0.126787795537796 \\
53	0.0896104546495172 \\
54	0.0895895040631883 \\
55	0.083500744047619 \\
56	0.0909617700518436 \\
57	0.0694427534271284 \\
58	0.0539480255243123 \\
59	0.0602853960275835 \\
60	0.0743267279262724 \\
61	0.083648643023643 \\
62	0.0739215211871462 \\
63	0.0948392969877345 \\
64	0.0923797123015873 \\
65	0.0978431724525475 \\
66	0.0805317945942946 \\
67	0.122250839438339 \\
68	0.0822411095848596 \\
69	0.0747316873550426 \\
70	0.091154474610357 \\
71	0.10713357312622 \\
72	0.0539420580333409 \\
73	0.0895929157647908 \\
74	0.0601167929292929 \\
75	0.0590528612012987 \\
76	0.0462851818320568 \\
77	0.0747701950826951 \\
78	0.0689774851884227 \\
79	0.0641843909766704 \\
80	0.0534717452686203 \\
81	0.055146163579068 \\
82	0.0606453008796759 \\
83	0.111715367965368 \\
84	0.0784296866328116 \\
85	0.120412161623099 \\
86	0.0863021170006464 \\
87	0.0597917273698524 \\
88	0.0712758595571096 \\
89	0.0783110119047619 \\
90	0.0879630454308763 \\
91	0.104482496669997 \\
92	0.0891642824087677 \\
93	0.107446904794767 \\
94	0.0690848214285714 \\
95	0.109959528318903 \\
96	0.0882216665075121 \\
97	0.0914137511793762 \\
98	0.0931589466927122 \\
99	0.0872337298118548 \\
100	0.132421245013157 \\
101	0.104635902292152 \\
102	0.114876963314463 \\
103	0.124856870559996 \\
104	0.139006262834388 \\
105	0.157363816738817 \\
106	0.127438862710002 \\
107	0.123977086281774 \\
108	0.116788029678655 \\
109	0.138006741522366 \\
110	0.102274942428895 \\
111	0.14101139745671 \\
112	0.196314210962648 \\
113	0.117820417429792 \\
114	0.124355375527251 \\
115	0.138170206529582 \\
116	0.157098816726574 \\
117	0.170032745032745 \\
118	0.157721705377955 \\
119	0.150307504995005 \\
120	0.186966288919414 \\
121	0.159496635921268 \\
122	0.153951625631313 \\
123	0.190936385662948 \\
124	0.138495741659441 \\
125	0.213128646527084 \\
126	0.166474024596727 \\
127	0.212495447261072 \\
128	0.206771830599956 \\
129	0.229127033607732 \\
130	0.22791521310593 \\
131	0.251491260475635 \\
132	0.257468825262822 \\
133	0.206473176027404 \\
134	0.251611257665945 \\
135	0.218520596727834 \\
136	0.271369689338439 \\
137	0.245682134141766 \\
138	0.277236028626194 \\
139	0.257347188401876 \\
140	0.29269095533487 \\
141	0.292364735448375 \\
142	0.311019311570782 \\
143	0.277489768546512 \\
144	0.270256002677878 \\
145	0.304506336698432 \\
146	0.260026661451294 \\
147	0.27553175990676 \\
148	0.33362393580684 \\
149	0.311993477825279 \\
150	0.292141148781774 \\
151	0.324375213736427 \\
152	0.301794382007956 \\
153	0.339197414105502 \\
154	0.325914189126505 \\
155	0.345887805515563 \\
156	0.4138959998335 \\
157	0.349615271880897 \\
158	0.383151560449355 \\
159	0.405045340178612 \\
160	0.350479807671903 \\
161	0.367593673177313 \\
162	0.387979251651127 \\
163	0.446529620553058 \\
164	0.42174127955378 \\
165	0.409289485024779 \\
166	0.439184470043845 \\
167	0.423440540260393 \\
168	0.490195096468074 \\
169	0.409157794081241 \\
170	0.477630659271284 \\
171	0.50128681699224 \\
172	0.450557558760684 \\
173	0.445436572975636 \\
174	0.449752934208357 \\
175	0.447983028256466 \\
176	0.507724341375536 \\
177	0.471031906649529 \\
178	0.496961263562826 \\
179	0.501730364712901 \\
180	0.518139938084791 \\
181	0.571705513134741 \\
182	0.526102044656732 \\
183	0.519242239340682 \\
184	0.53236213458132 \\
185	0.522579301933017 \\
186	0.52086718938741 \\
187	0.543573085921431 \\
188	0.5569145079207 \\
189	0.58590080007819 \\
190	0.542190730276668 \\
191	0.565447746697747 \\
192	0.564467737123987 \\
193	0.52559555308866 \\
194	0.561701332266589 \\
195	0.543720992549118 \\
196	0.501813182477245 \\
197	0.555550309065934 \\
198	0.590144651812004 \\
199	0.582579837186915 \\
200	0.606725238354374 \\
201	0.55302678051384 \\
202	0.615305296174951 \\
203	0.597835865115277 \\
204	0.610224507926714 \\
205	0.608177152708403 \\
206	0.60307741303835 \\
207	0.596865944257948 \\
208	0.624095001817283 \\
209	0.606359645766815 \\
210	0.637637845497511 \\
211	0.596385132943038 \\
212	0.604481998036686 \\
213	0.619174271908647 \\
214	0.64800665531364 \\
215	0.660038485819736 \\
216	0.622285083109507 \\
217	0.643298927406658 \\
218	0.611289365438698 \\
219	0.607077193129583 \\
220	0.623505718393127 \\
221	0.638590224114121 \\
222	0.64345686482635 \\
223	0.657711059365471 \\
224	0.59924742587886 \\
225	0.640156447922992 \\
226	0.654147520932907 \\
227	0.654631046037296 \\
228	0.667214438347493 \\
229	0.686196562322481 \\
230	0.680855494332057 \\
231	0.621413190245427 \\
232	0.672979667901543 \\
233	0.683943620636974 \\
234	0.643995227647632 \\
235	0.659235340843797 \\
236	0.685876038024476 \\
237	0.668442776841214 \\
238	0.677924484369797 \\
239	0.702330225687303 \\
240	0.715494461369921 \\
241	0.699400980708425 \\
242	0.725594024031524 \\
243	0.745696580278382 \\
244	0.763150567263618 \\
245	0.723845620525308 \\
246	0.700657805735931 \\
247	0.704737471729659 \\
248	0.684499827201307 \\
249	0.68550623319695 \\
250	0.689350753413253 \\
251	0.682144201284826 \\
252	0.70200307504995 \\
253	0.674523682761275 \\
254	0.679027766045229 \\
255	0.663519748480686 \\
256	0.688263736218095 \\
257	0.650754150363525 \\
258	0.703473826520701 \\
259	0.680364990911866 \\
260	0.684926462936573 \\
261	0.663207975133526 \\
262	0.674030526764902 \\
263	0.675363980637418 \\
264	0.701955292970918 \\
265	0.718682185869686 \\
266	0.732942599934787 \\
267	0.684650833490447 \\
268	0.677662094849595 \\
269	0.711122818154068 \\
270	0.732138846396659 \\
271	0.666683620199245 \\
272	0.687091359161672 \\
273	0.722196950124824 \\
274	0.739986252003715 \\
275	0.676275699127262 \\
276	0.701162832010718 \\
277	0.743319458829569 \\
278	0.704002266881403 \\
279	0.734844585051386 \\
280	0.685827952429515 \\
281	0.689461861922799 \\
282	0.703767716658341 \\
283	0.661852382250747 \\
284	0.66883932040182 \\
285	0.724185341568154 \\
286	0.724501875524394 \\
287	0.667420679300275 \\
288	0.699008326742702 \\
289	0.693327158952159 \\
290	0.714264500052643 \\
291	0.694009441334809 \\
292	0.718143760089992 \\
293	0.715980692612126 \\
294	0.697828712086524 \\
295	0.692902268534162 \\
296	0.727184225598748 \\
297	0.717059304053789 \\
298	0.750091249895937 \\
299	0.72796192522755 \\
300	0.670285357004107 \\
301	0.75388522323265 \\
302	0.714463044104588 \\
303	0.759128284909535 \\
304	0.728173389110889 \\
305	0.700777130452537 \\
306	0.723136802433678 \\
307	0.765282352196414 \\
308	0.74057188471251 \\
309	0.743646544427795 \\
310	0.769163367014929 \\
311	0.761365414099789 \\
312	0.749308424928829 \\
313	0.763448562585075 \\
314	0.711524078964336 \\
315	0.74300257034632 \\
316	0.72074990981241 \\
317	0.764388215950716 \\
318	0.777212423450088 \\
319	0.780681818181818 \\
320	0.713985927267177 \\
321	0.728185682781271 \\
322	0.751982089091464 \\
323	0.766188715798091 \\
324	0.74901205182685 \\
325	0.72676388281857 \\
326	0.728557255142733 \\
327	0.761826893490496 \\
328	0.736820854584012 \\
329	0.779199706543457 \\
330	0.735828576039973 \\
331	0.687356567044067 \\
332	0.726581751581752 \\
333	0.726142699212552 \\
334	0.723835322663447 \\
335	0.71674191693723 \\
336	0.718597027972028 \\
337	0.75534929480242 \\
338	0.764411456598956 \\
339	0.74499684470962 \\
340	0.706548506640418 \\
341	0.734224741358787 \\
342	0.740184208152958 \\
343	0.77086511578699 \\
344	0.748701125263625 \\
345	0.739992602883228 \\
346	0.766141454032079 \\
347	0.766793406246531 \\
348	0.733429374444999 \\
349	0.735029502768473 \\
350	0.743901085893273 \\
351	0.756997169497169 \\
352	0.772449425574425 \\
353	0.739749500652534 \\
354	0.761071719079531 \\
355	0.748995925949051 \\
356	0.765943084693084 \\
357	0.706752188783439 \\
358	0.735113584332334 \\
359	0.773787714022089 \\
360	0.77816966020091 \\
361	0.756982123778999 \\
362	0.715334448537574 \\
363	0.74701926188599 \\
364	0.765770254051504 \\
365	0.736184692044067 \\
366	0.717831712558275 \\
367	0.772058171862859 \\
368	0.791880024336366 \\
369	0.77930320547508 \\
370	0.775500627844378 \\
371	0.741586416034945 \\
372	0.746871357808858 \\
373	0.73776401549839 \\
374	0.745219060453435 \\
375	0.771436592920968 \\
376	0.695933016636142 \\
377	0.744480649558774 \\
378	0.749701535443723 \\
379	0.734685778631091 \\
380	0.752757789085914 \\
381	0.764307697510822 \\
382	0.704001337204462 \\
383	0.723968749795956 \\
384	0.725661296169109 \\
385	0.739240449603501 \\
386	0.736364893786768 \\
387	0.759850089146964 \\
388	0.772995971042846 \\
389	0.787855070276945 \\
390	0.75606243863367 \\
391	0.739806439220502 \\
392	0.728856690184815 \\
393	0.725866971570096 \\
394	0.743704971244034 \\
395	0.726759872072372 \\
396	0.707196275946276 \\
397	0.760279304029304 \\
398	0.720609880570818 \\
399	0.74316829004329 \\
400	0.76279679521867 \\
401	0.776415858447108 \\
402	0.75811033584471 \\
403	0.773278089098402 \\
404	0.732766071191236 \\
405	0.742465910825286 \\
406	0.780433455433455 \\
407	0.779969690746345 \\
408	0.774503881535131 \\
409	0.790836698534308 \\
410	0.778562713328338 \\
411	0.733352801712177 \\
412	0.726028572122322 \\
413	0.696449123792874 \\
414	0.739395500333 \\
415	0.738613513222888 \\
416	0.716080577408702 \\
417	0.75309724997225 \\
418	0.696924559815185 \\
419	0.757311048326673 \\
420	0.754907939282939 \\
421	0.718477095820846 \\
422	0.732712726152524 \\
423	0.725839806894495 \\
424	0.754599111825674 \\
425	0.758472950660451 \\
426	0.759676391317016 \\
427	0.743245639419812 \\
428	0.76576613490676 \\
429	0.746607472388722 \\
430	0.759860633137288 \\
431	0.783326157349595 \\
432	0.7423318001443 \\
433	0.743563077547452 \\
434	0.734502520049395 \\
435	0.758688165133477 \\
436	0.743166468947719 \\
437	0.783222246503496 \\
438	0.769494264416139 \\
439	0.761149831071706 \\
440	0.797917837370962 \\
441	0.761421118199611 \\
442	0.770315188283938 \\
443	0.749127435064935 \\
444	0.753456150107418 \\
445	0.730950212981463 \\
446	0.760770392801643 \\
447	0.736008089133089 \\
448	0.681056963869464 \\
449	0.765451215451215 \\
450	0.724912890928516 \\
451	0.724212116008991 \\
452	0.757429159382284 \\
453	0.774906733891109 \\
454	0.761046505577755 \\
455	0.740712369227994 \\
456	0.764271380172115 \\
457	0.7601490829432 \\
458	0.717742999188311 \\
459	0.764429190601066 \\
460	0.749991371475747 \\
461	0.750241408662824 \\
462	0.777859076687201 \\
463	0.775285781926407 \\
464	0.77331511804168 \\
465	0.736797013750139 \\
466	0.74079360309829 \\
467	0.723803973803974 \\
468	0.737198431419479 \\
469	0.737892836330336 \\
470	0.754283823815074 \\
471	0.770358135787823 \\
472	0.766033293962981 \\
473	0.752887260309135 \\
474	0.733283029561813 \\
475	0.725740708943834 \\
476	0.793913941960817 \\
477	0.757261575230325 \\
478	0.77510453271144 \\
479	0.738517510906491 \\
480	0.785117204843767 \\
481	0.765759110680985 \\
482	0.759021338904151 \\
483	0.783933536081973 \\
484	0.765809134697003 \\
485	0.767109886641137 \\
486	0.777804161983849 \\
487	0.748806271853147 \\
488	0.73588754995005 \\
489	0.741433284597347 \\
490	0.772983591928904 \\
491	0.781219648407148 \\
492	0.753270384129759 \\
493	0.753746904137529 \\
494	0.781008044243338 \\
495	0.789901878401419 \\
496	0.798897304951992 \\
497	0.790311511405261 \\
498	0.754525812728938 \\
499	0.767262024637943 \\
500	0.773300310800311 \\
501	0.775234964688089 \\
502	0.776146813256188 \\
503	0.784177844724719 \\
504	0.772860539174877 \\
505	0.766688129578754 \\
506	0.805905379342879 \\
507	0.773734902250527 \\
508	0.76768938526751 \\
509	0.756918472152847 \\
510	0.751728132978133 \\
511	0.769653285082972 \\
512	0.756445204101454 \\
513	0.736128151362526 \\
514	0.741005890810578 \\
515	0.757413463272838 \\
516	0.802141695110445 \\
517	0.762733338709901 \\
518	0.752009709040959 \\
519	0.782119443056943 \\
520	0.803295033959096 \\
521	0.780912620365745 \\
522	0.766970898025585 \\
523	0.75985045770202 \\
524	0.748103415681541 \\
525	0.753301342754467 \\
526	0.779651598401598 \\
527	0.785172748258686 \\
528	0.764760933510933 \\
529	0.7617319746226 \\
530	0.78318487068487 \\
531	0.804898551968864 \\
532	0.756186658266162 \\
533	0.773798228681041 \\
534	0.761151177764229 \\
535	0.775128777472527 \\
536	0.756391741938617 \\
537	0.740187113233988 \\
538	0.778159037143412 \\
539	0.796207091519591 \\
540	0.762149570321856 \\
541	0.741317341512654 \\
542	0.775782507423132 \\
543	0.713572972166722 \\
544	0.756965083527583 \\
545	0.741519157925408 \\
546	0.794014665889665 \\
547	0.782147087236701 \\
548	0.767348320082695 \\
549	0.77653923767205 \\
550	0.784663253413253 \\
551	0.764969340555278 \\
552	0.772574083902209 \\
553	0.767605116237929 \\
554	0.754578754578754 \\
555	0.793334746850372 \\
556	0.762331852175602 \\
557	0.780312526015651 \\
558	0.745867977508602 \\
559	0.80152391011766 \\
560	0.777768802378177 \\
561	0.769756176982739 \\
562	0.784275490134865 \\
563	0.707871989121989 \\
564	0.740132013615469 \\
565	0.779424701115877 \\
566	0.778403150668775 \\
567	0.726720415001665 \\
568	0.783868128399378 \\
569	0.732088484432235 \\
570	0.78038230024673 \\
571	0.780788135475635 \\
572	0.765090551809301 \\
573	0.741226958805084 \\
574	0.732973195105548 \\
575	0.757476919781607 \\
576	0.741426131568595 \\
577	0.755842378108003 \\
578	0.76902169011544 \\
579	0.753077626009611 \\
580	0.762367948891386 \\
581	0.758149446040071 \\
582	0.768131350369402 \\
583	0.785125811688312 \\
584	0.76477133977134 \\
585	0.786096542346542 \\
586	0.774853618603618 \\
587	0.78488222945444 \\
588	0.757408238462926 \\
589	0.766097487581862 \\
590	0.741959863053613 \\
591	0.754448329448329 \\
592	0.75792376026751 \\
593	0.773968262640137 \\
594	0.757430915438728 \\
595	0.759264780358531 \\
596	0.756560529952074 \\
597	0.748072413697414 \\
598	0.772304951992452 \\
599	0.699045745920746 \\
};
\nextgroupplot[
height=\figureheight,
tick align=outside,
tick pos=left,
title={Taxi},
title style={font=\small, yshift=-1.5ex},
xlabel style={font=\scriptsize},
yticklabel style={font=\scriptsize},
xticklabel style={font=\scriptsize},
grid=both,
width=\figurewidth,
x grid style={white!69.01960784313725!black},
xmin=-3.95, xmax=82.95,
y grid style={white!69.01960784313725!black},
ymin=-0.842166523256524, ymax=14.693118866764
]
\path [fill=color4, fill opacity=0.3] (axis cs:0,-0.0578822502191382)
--(axis cs:0,0.557882250219138)
--(axis cs:1,0.154325569076052)
--(axis cs:2,0.493182614516675)
--(axis cs:3,0.808445081701291)
--(axis cs:4,0.813814247990215)
--(axis cs:5,1.26810404360514)
--(axis cs:6,2.01234347082933)
--(axis cs:7,2.39600996794375)
--(axis cs:8,3.21851724139069)
--(axis cs:9,4.10659104691206)
--(axis cs:10,3.04714258101641)
--(axis cs:11,4.93980143160838)
--(axis cs:12,6.00482907607551)
--(axis cs:13,5.35037068692427)
--(axis cs:14,6.76174773371808)
--(axis cs:15,7.34157737508907)
--(axis cs:16,7.49433455467016)
--(axis cs:17,7.94678548461946)
--(axis cs:18,7.99057509037283)
--(axis cs:19,8.32494135135068)
--(axis cs:20,8.65027247799843)
--(axis cs:21,8.5734801324271)
--(axis cs:22,8.53443454133938)
--(axis cs:23,9.20236778822754)
--(axis cs:24,9.01057745093765)
--(axis cs:25,8.78220818312175)
--(axis cs:26,8.94580153656458)
--(axis cs:27,9.19893011712477)
--(axis cs:28,9.41232750658658)
--(axis cs:29,9.43104855856236)
--(axis cs:30,9.81026412300801)
--(axis cs:31,10.2781159484712)
--(axis cs:32,10.1175445987875)
--(axis cs:33,10.2114658808971)
--(axis cs:34,10.3937353177459)
--(axis cs:35,10.165258338939)
--(axis cs:36,10.3019542892585)
--(axis cs:37,10.2527810418139)
--(axis cs:38,10.1095264679169)
--(axis cs:39,10.7702415003976)
--(axis cs:40,10.461746576595)
--(axis cs:41,9.88270240841323)
--(axis cs:42,10.6533921040688)
--(axis cs:43,10.1232231731258)
--(axis cs:44,10.6006613645807)
--(axis cs:45,10.6269739320555)
--(axis cs:46,10.9678981281377)
--(axis cs:47,10.8234570186607)
--(axis cs:48,10.8113378140873)
--(axis cs:49,10.7312210583149)
--(axis cs:50,10.8348286579641)
--(axis cs:51,10.9919084903942)
--(axis cs:52,11.3579724337967)
--(axis cs:53,10.9564379955033)
--(axis cs:54,11.2102735746507)
--(axis cs:55,10.7207945029477)
--(axis cs:56,11.0551776376721)
--(axis cs:57,11.0276085150545)
--(axis cs:58,10.7488422251645)
--(axis cs:59,11.0358833956249)
--(axis cs:60,10.8722745113448)
--(axis cs:61,11.3261477898439)
--(axis cs:62,10.9856520775433)
--(axis cs:63,11.1422393791172)
--(axis cs:64,11.0855211824659)
--(axis cs:65,11.275008787664)
--(axis cs:66,10.8177540515904)
--(axis cs:67,11.2978769098854)
--(axis cs:68,11.5532207616025)
--(axis cs:69,11.1726426906567)
--(axis cs:70,11.2978769098854)
--(axis cs:71,11.6009639022236)
--(axis cs:72,11.4298922837027)
--(axis cs:73,11.4217621394313)
--(axis cs:74,11.6053078833237)
--(axis cs:75,11.8426016248657)
--(axis cs:76,12.0857645829394)
--(axis cs:77,11.9499840023451)
--(axis cs:78,11.1191549435796)
--(axis cs:79,12.0857645829394)
--(axis cs:79,9.49423541706064)
--(axis cs:79,9.49423541706064)
--(axis cs:78,8.36084505642042)
--(axis cs:77,9.33001599765493)
--(axis cs:76,9.49423541706064)
--(axis cs:75,9.21739837513433)
--(axis cs:74,8.95469211667628)
--(axis cs:73,8.71823786056866)
--(axis cs:72,8.73010771629733)
--(axis cs:71,8.94903609777641)
--(axis cs:70,8.58212309011457)
--(axis cs:69,8.44735730934335)
--(axis cs:68,8.8667792383975)
--(axis cs:67,8.58212309011457)
--(axis cs:66,8.07724594840957)
--(axis cs:65,8.54499121233598)
--(axis cs:64,8.33947881753412)
--(axis cs:63,8.39776062088279)
--(axis cs:62,8.21434792245672)
--(axis cs:61,8.62385221015612)
--(axis cs:60,8.11772548865522)
--(axis cs:59,8.29411660437512)
--(axis cs:58,7.97115777483546)
--(axis cs:57,8.28239148494554)
--(axis cs:56,8.32482236232789)
--(axis cs:55,7.94920549705232)
--(axis cs:54,8.50972642534929)
--(axis cs:53,8.23356200449672)
--(axis cs:52,8.68202756620325)
--(axis cs:51,8.28809150960577)
--(axis cs:50,8.10517134203593)
--(axis cs:49,8.00877894168508)
--(axis cs:48,8.07032885257938)
--(axis cs:47,8.08904298133933)
--(axis cs:46,8.25210187186225)
--(axis cs:45,7.87802606794454)
--(axis cs:44,7.84933863541931)
--(axis cs:43,7.33011016020754)
--(axis cs:42,7.88660789593116)
--(axis cs:41,7.09729759158677)
--(axis cs:40,7.71825342340504)
--(axis cs:39,8.05975849960238)
--(axis cs:38,7.31047353208309)
--(axis cs:37,7.48721895818607)
--(axis cs:36,7.53804571074149)
--(axis cs:35,7.39474166106101)
--(axis cs:34,7.63626468225406)
--(axis cs:33,7.44996269053143)
--(axis cs:32,7.32245540121254)
--(axis cs:31,7.50188405152884)
--(axis cs:30,7.05873587699199)
--(axis cs:29,6.66395144143763)
--(axis cs:28,6.65767249341342)
--(axis cs:27,6.47106988287523)
--(axis cs:26,6.25562703486399)
--(axis cs:25,6.02922038830682)
--(axis cs:24,6.28942254906235)
--(axis cs:23,6.47763221177246)
--(axis cs:22,5.80556545866062)
--(axis cs:21,5.8545198675729)
--(axis cs:20,5.92972752200157)
--(axis cs:19,5.61505864864932)
--(axis cs:18,5.28085348105574)
--(axis cs:17,5.26954784871387)
--(axis cs:16,4.86566544532984)
--(axis cs:15,4.66842262491093)
--(axis cs:14,4.20025226628192)
--(axis cs:13,3.05462931307573)
--(axis cs:12,3.55517092392449)
--(axis cs:11,2.72019856839162)
--(axis cs:10,1.30085741898359)
--(axis cs:9,2.05340895308794)
--(axis cs:8,1.39203831416486)
--(axis cs:7,0.842990032056252)
--(axis cs:6,0.627656529170672)
--(axis cs:5,0.231895956394857)
--(axis cs:4,0.0161857520097849)
--(axis cs:3,-0.0284450817012911)
--(axis cs:2,-0.103182614516675)
--(axis cs:1,-0.0143255690760518)
--(axis cs:0,-0.0578822502191382)
--cycle;

\path [fill=color5, fill opacity=0.3] (axis cs:0,-0.136017187346499)
--(axis cs:0,0.449350520679832)
--(axis cs:1,0.101482797594124)
--(axis cs:2,0.04744)
--(axis cs:3,0.629665336717145)
--(axis cs:4,0.867615815070709)
--(axis cs:5,1.24948088631119)
--(axis cs:6,1.41149478350667)
--(axis cs:7,1.41730557213062)
--(axis cs:8,2.41110752475003)
--(axis cs:9,2.77583764939311)
--(axis cs:10,4.05950959001036)
--(axis cs:11,6.22247910150132)
--(axis cs:12,5.77590065216208)
--(axis cs:13,7.05644486748916)
--(axis cs:14,8.2390181736363)
--(axis cs:15,8.28261157008725)
--(axis cs:16,9.13624502086519)
--(axis cs:17,9.48064666221928)
--(axis cs:18,9.88004168291205)
--(axis cs:19,10.0972702639628)
--(axis cs:20,10.636972417877)
--(axis cs:21,10.6389827183351)
--(axis cs:22,10.7726225636806)
--(axis cs:23,11.577181320618)
--(axis cs:24,11.9529079627373)
--(axis cs:25,11.7431283264023)
--(axis cs:26,12.0137346185136)
--(axis cs:27,11.9024577930068)
--(axis cs:28,11.8178189987822)
--(axis cs:29,12.6894479390119)
--(axis cs:30,12.213838900174)
--(axis cs:31,12.5373716561824)
--(axis cs:32,12.2252424762595)
--(axis cs:33,12.6894479390119)
--(axis cs:34,12.3836816380565)
--(axis cs:35,13.0182570093948)
--(axis cs:36,12.201363000874)
--(axis cs:37,12.1776184163943)
--(axis cs:38,12.5383472066431)
--(axis cs:39,12.5750239796862)
--(axis cs:40,12.3075733298699)
--(axis cs:41,12.8169192377289)
--(axis cs:42,12.7865443853802)
--(axis cs:43,12.5140994592848)
--(axis cs:44,12.6070239971803)
--(axis cs:45,13.0951768446433)
--(axis cs:46,12.7179349161906)
--(axis cs:47,12.6569208619956)
--(axis cs:48,12.8913340202489)
--(axis cs:49,12.8283240810228)
--(axis cs:50,12.9092149056046)
--(axis cs:51,13.6015121319868)
--(axis cs:52,13.1267172107304)
--(axis cs:53,13.1919538884684)
--(axis cs:54,13.4497035956626)
--(axis cs:55,13.5225964572009)
--(axis cs:56,12.6009306464886)
--(axis cs:57,13.6093843342784)
--(axis cs:58,13.7334484913272)
--(axis cs:59,13.8108271291287)
--(axis cs:60,13.986969530854)
--(axis cs:61,13.1685316624933)
--(axis cs:62,13.8347239152084)
--(axis cs:63,13.986969530854)
--(axis cs:64,13.4484217530073)
--(axis cs:65,13.1267172107304)
--(axis cs:66,13.4596082351234)
--(axis cs:67,13.2031421706857)
--(axis cs:68,13.664913292245)
--(axis cs:69,13.467190104002)
--(axis cs:70,13.3418198153416)
--(axis cs:71,13.8347239152084)
--(axis cs:72,13.1981067023583)
--(axis cs:73,13.7137511235581)
--(axis cs:74,13.2110423693815)
--(axis cs:75,13.5912007391784)
--(axis cs:76,13.1070874270593)
--(axis cs:77,13.467190104002)
--(axis cs:78,13.7837372808044)
--(axis cs:79,13.5912007391784)
--(axis cs:79,11.4287992608216)
--(axis cs:79,11.4287992608216)
--(axis cs:78,11.7522627191956)
--(axis cs:77,11.252809895998)
--(axis cs:76,10.7729125729407)
--(axis cs:75,11.4287992608216)
--(axis cs:74,10.8969576306185)
--(axis cs:73,11.6062488764419)
--(axis cs:72,10.9018932976417)
--(axis cs:71,11.7852760847916)
--(axis cs:70,11.0781801846584)
--(axis cs:69,11.252809895998)
--(axis cs:68,11.535086707755)
--(axis cs:67,10.9168578293143)
--(axis cs:66,11.2703917648766)
--(axis cs:65,10.8132827892696)
--(axis cs:64,11.2115782469927)
--(axis cs:63,12.053030469146)
--(axis cs:62,11.7852760847916)
--(axis cs:61,10.9314683375067)
--(axis cs:60,12.053030469146)
--(axis cs:59,11.8091728708713)
--(axis cs:58,11.6865515086728)
--(axis cs:57,11.4706156657216)
--(axis cs:56,10.1490693535114)
--(axis cs:55,11.3774035427991)
--(axis cs:54,11.2702964043374)
--(axis cs:53,10.9051889686745)
--(axis cs:52,10.8132827892696)
--(axis cs:51,11.5118212013466)
--(axis cs:50,10.5507850943954)
--(axis cs:49,10.3916759189772)
--(axis cs:48,10.5186659797511)
--(axis cs:47,10.2030791380044)
--(axis cs:46,10.2620650838094)
--(axis cs:45,10.7848231553567)
--(axis cs:44,10.1329760028197)
--(axis cs:43,10.0359005407152)
--(axis cs:42,10.3534556146198)
--(axis cs:41,10.4030807622711)
--(axis cs:40,9.79242667013013)
--(axis cs:39,10.1049760203138)
--(axis cs:38,10.0316527933569)
--(axis cs:37,9.64738158360571)
--(axis cs:36,9.64863699912603)
--(axis cs:35,10.6817429906052)
--(axis cs:34,9.87631836194351)
--(axis cs:33,10.2905520609881)
--(axis cs:32,9.66618609516904)
--(axis cs:31,10.0826283438176)
--(axis cs:30,9.646161099826)
--(axis cs:29,10.2905520609881)
--(axis cs:28,9.22218100121781)
--(axis cs:27,9.33354220699316)
--(axis cs:26,9.4662653814864)
--(axis cs:25,9.11687167359774)
--(axis cs:24,9.40709203726271)
--(axis cs:23,9.086818679382)
--(axis cs:22,8.11137743631937)
--(axis cs:21,7.96101728166492)
--(axis cs:20,8.01331329640869)
--(axis cs:19,7.44272973603716)
--(axis cs:18,7.17995831708795)
--(axis cs:17,6.75235333778072)
--(axis cs:16,6.44875497913481)
--(axis cs:15,5.55338842991275)
--(axis cs:14,5.54431515969704)
--(axis cs:13,4.47098370393941)
--(axis cs:12,3.36591752965611)
--(axis cs:11,3.7118066127844)
--(axis cs:10,2.076126773626)
--(axis cs:9,1.22116235060689)
--(axis cs:8,0.903178189535681)
--(axis cs:7,0.36269442786938)
--(axis cs:6,0.318505216493331)
--(axis cs:5,0.0905191136888123)
--(axis cs:4,-0.00761581507070852)
--(axis cs:3,-0.0396653367171447)
--(axis cs:2,-0.00743999999999998)
--(axis cs:1,-0.0214827975941238)
--(axis cs:0,-0.136017187346499)
--cycle;

\addplot [semithick, color4, dash pattern=on 1pt off 3pt on 3pt off 3pt, forget plot]
table [row sep=\\]{%
0	0.25 \\
1	0.07 \\
2	0.195 \\
3	0.39 \\
4	0.415 \\
5	0.75 \\
6	1.32 \\
7	1.6195 \\
8	2.30527777777778 \\
9	3.08 \\
10	2.174 \\
11	3.83 \\
12	4.78 \\
13	4.2025 \\
14	5.481 \\
15	6.005 \\
16	6.18 \\
17	6.60816666666667 \\
18	6.63571428571429 \\
19	6.97 \\
20	7.29 \\
21	7.214 \\
22	7.17 \\
23	7.84 \\
24	7.65 \\
25	7.40571428571429 \\
26	7.60071428571429 \\
27	7.835 \\
28	8.035 \\
29	8.0475 \\
30	8.4345 \\
31	8.89 \\
32	8.72 \\
33	8.83071428571429 \\
34	9.015 \\
35	8.78 \\
36	8.92 \\
37	8.87 \\
38	8.71 \\
39	9.415 \\
40	9.09 \\
41	8.49 \\
42	9.27 \\
43	8.72666666666667 \\
44	9.225 \\
45	9.2525 \\
46	9.61 \\
47	9.45625 \\
48	9.44083333333333 \\
49	9.37 \\
50	9.47 \\
51	9.64 \\
52	10.02 \\
53	9.595 \\
54	9.86 \\
55	9.335 \\
56	9.69 \\
57	9.655 \\
58	9.36 \\
59	9.665 \\
60	9.495 \\
61	9.975 \\
62	9.6 \\
63	9.77 \\
64	9.7125 \\
65	9.91 \\
66	9.4475 \\
67	9.94 \\
68	10.21 \\
69	9.81 \\
70	9.94 \\
71	10.275 \\
72	10.08 \\
73	10.07 \\
74	10.28 \\
75	10.53 \\
76	10.79 \\
77	10.64 \\
78	9.74 \\
79	10.79 \\
};
\addplot [semithick, color5, forget plot]
table [row sep=\\]{%
0	0.156666666666667 \\
1	0.04 \\
2	0.02 \\
3	0.295 \\
4	0.43 \\
5	0.67 \\
6	0.865 \\
7	0.89 \\
8	1.65714285714286 \\
9	1.9985 \\
10	3.06781818181818 \\
11	4.96714285714286 \\
12	4.57090909090909 \\
13	5.76371428571429 \\
14	6.89166666666667 \\
15	6.918 \\
16	7.7925 \\
17	8.1165 \\
18	8.53 \\
19	8.77 \\
20	9.32514285714286 \\
21	9.3 \\
22	9.442 \\
23	10.332 \\
24	10.68 \\
25	10.43 \\
26	10.74 \\
27	10.618 \\
28	10.52 \\
29	11.49 \\
30	10.93 \\
31	11.31 \\
32	10.9457142857143 \\
33	11.49 \\
34	11.13 \\
35	11.85 \\
36	10.925 \\
37	10.9125 \\
38	11.285 \\
39	11.34 \\
40	11.05 \\
41	11.61 \\
42	11.57 \\
43	11.275 \\
44	11.37 \\
45	11.94 \\
46	11.49 \\
47	11.43 \\
48	11.705 \\
49	11.61 \\
50	11.73 \\
51	12.5566666666667 \\
52	11.97 \\
53	12.0485714285714 \\
54	12.36 \\
55	12.45 \\
56	11.375 \\
57	12.54 \\
58	12.71 \\
59	12.81 \\
60	13.02 \\
61	12.05 \\
62	12.81 \\
63	13.02 \\
64	12.33 \\
65	11.97 \\
66	12.365 \\
67	12.06 \\
68	12.6 \\
69	12.36 \\
70	12.21 \\
71	12.81 \\
72	12.05 \\
73	12.66 \\
74	12.054 \\
75	12.51 \\
76	11.94 \\
77	12.36 \\
78	12.768 \\
79	12.51 \\
};
\nextgroupplot[
height=\figureheight,
tick align=outside,
title={50-Chain},
title style={font=\small, yshift=-1.5ex},
xlabel style={font=\scriptsize},
yticklabel style={font=\scriptsize},
xticklabel style={font=\scriptsize},
width=\figurewidth,
x grid style={white!69.01960784313725!black},
xmajorticks=false,
xmin=0.5, xmax=4.5,
y grid style={white!69.01960784313725!black},
ymin=34644.8, ymax=206315.2,
ytick pos=left
]
\addplot [black, forget plot]
table [row sep=\\]{%
1	113225 \\
1	93906 \\
};
\addplot [black, forget plot]
table [row sep=\\]{%
1	142017.5 \\
1	173680 \\
};
\addplot [black, forget plot]
table [row sep=\\]{%
0.8875	93906 \\
1.1125	93906 \\
};
\addplot [black, forget plot]
table [row sep=\\]{%
0.8875	173680 \\
1.1125	173680 \\
};
\addplot [black, mark=*, mark size=1, mark options={solid,fill opacity=0}, only marks, forget plot]
table [row sep=\\]{%
1	67503 \\
1	191511 \\
};
\addplot [black, forget plot]
table [row sep=\\]{%
2	75953.5 \\
2	42448 \\
};
\addplot [black, forget plot]
table [row sep=\\]{%
2	104767.25 \\
2	143169 \\
};
\addplot [black, forget plot]
table [row sep=\\]{%
1.8875	42448 \\
2.1125	42448 \\
};
\addplot [black, forget plot]
table [row sep=\\]{%
1.8875	143169 \\
2.1125	143169 \\
};
\addplot [black, mark=*, mark size=1, mark options={solid,fill opacity=0}, only marks, forget plot]
table [row sep=\\]{%
2	158106 \\
2	159452 \\
2	171682 \\
};
\addplot [black, forget plot]
table [row sep=\\]{%
3	115805 \\
3	84454 \\
};
\addplot [black, forget plot]
table [row sep=\\]{%
3	142457 \\
3	175340 \\
};
\addplot [black, forget plot]
table [row sep=\\]{%
2.8875	84454 \\
3.1125	84454 \\
};
\addplot [black, forget plot]
table [row sep=\\]{%
2.8875	175340 \\
3.1125	175340 \\
};
\addplot [black, mark=*, mark size=1, mark options={solid,fill opacity=0}, only marks, forget plot]
table [row sep=\\]{%
3	62167 \\
3	71853 \\
3	198512 \\
3	187167 \\
3	193737 \\
};
\addplot [black, forget plot]
table [row sep=\\]{%
4	92112.5 \\
4	59053 \\
};
\addplot [black, forget plot]
table [row sep=\\]{%
4	117640 \\
4	150340 \\
};
\addplot [black, forget plot]
table [row sep=\\]{%
3.8875	59053 \\
4.1125	59053 \\
};
\addplot [black, forget plot]
table [row sep=\\]{%
3.8875	150340 \\
4.1125	150340 \\
};
\path [draw=black, fill=color0] (axis cs:0.775,113225)
--(axis cs:1.225,113225)
--(axis cs:1.225,142017.5)
--(axis cs:0.775,142017.5)
--(axis cs:0.775,113225)
--cycle;

\path [draw=black, fill=color1] (axis cs:1.775,75953.5)
--(axis cs:2.225,75953.5)
--(axis cs:2.225,104767.25)
--(axis cs:1.775,104767.25)
--(axis cs:1.775,75953.5)
--cycle;

\path [draw=black, fill=color2] (axis cs:2.775,115805)
--(axis cs:3.225,115805)
--(axis cs:3.225,142457)
--(axis cs:2.775,142457)
--(axis cs:2.775,115805)
--cycle;

\path [draw=black, fill=color3] (axis cs:3.775,92112.5)
--(axis cs:4.225,92112.5)
--(axis cs:4.225,117640)
--(axis cs:3.775,117640)
--(axis cs:3.775,92112.5)
--cycle;

\addplot [black, forget plot]
table [row sep=\\]{%
0.775	128682 \\
1.225	128682 \\
};
\addplot [black, forget plot]
table [row sep=\\]{%
1.775	88011 \\
2.225	88011 \\
};
\addplot [black, forget plot]
table [row sep=\\]{%
2.775	128453 \\
3.225	128453 \\
};
\addplot [black, forget plot]
table [row sep=\\]{%
3.775	108952 \\
4.225	108952 \\
};
\nextgroupplot[
height=\figureheight,
tick align=outside,
title={Taxi},
title style={font=\small, yshift=-1.5ex},
xlabel style={font=\scriptsize},
yticklabel style={font=\scriptsize},
xticklabel style={font=\scriptsize},
width=\figurewidth,
x grid style={white!69.01960784313725!black},
xmajorticks=false,
xmin=0.5, xmax=4.5,
y grid style={white!69.01960784313725!black},
ymin=-13276.65, ymax=629203.65,
ytick pos=left
]
\addplot [black, forget plot]
table [row sep=\\]{%
1	103632 \\
1	40169 \\
};
\addplot [black, forget plot]
table [row sep=\\]{%
1	218317.25 \\
1	375256 \\
};
\addplot [black, forget plot]
table [row sep=\\]{%
0.8875	40169 \\
1.1125	40169 \\
};
\addplot [black, forget plot]
table [row sep=\\]{%
0.8875	375256 \\
1.1125	375256 \\
};
\addplot [black, mark=*, mark size=1, mark options={solid,fill opacity=0}, only marks, forget plot]
table [row sep=\\]{%
1	439381 \\
1	433196 \\
1	427389 \\
1	419946 \\
1	424997 \\
};
\addplot [black, forget plot]
table [row sep=\\]{%
2	71559.5 \\
2	29767 \\
};
\addplot [black, forget plot]
table [row sep=\\]{%
2	142922.25 \\
2	224589 \\
};
\addplot [black, forget plot]
table [row sep=\\]{%
1.8875	29767 \\
2.1125	29767 \\
};
\addplot [black, forget plot]
table [row sep=\\]{%
1.8875	224589 \\
2.1125	224589 \\
};
\addplot [black, mark=*, mark size=1, mark options={solid,fill opacity=0}, only marks, forget plot]
table [row sep=\\]{%
2	253951 \\
2	600000 \\
};
\addplot [black, forget plot]
table [row sep=\\]{%
3	140879.75 \\
3	50669 \\
};
\addplot [black, forget plot]
table [row sep=\\]{%
3	272254.25 \\
3	463977 \\
};
\addplot [black, forget plot]
table [row sep=\\]{%
2.8875	50669 \\
3.1125	50669 \\
};
\addplot [black, forget plot]
table [row sep=\\]{%
2.8875	463977 \\
3.1125	463977 \\
};
\addplot [black, mark=*, mark size=1, mark options={solid,fill opacity=0}, only marks, forget plot]
table [row sep=\\]{%
3	520112 \\
3	600000 \\
3	600000 \\
};
\addplot [black, forget plot]
table [row sep=\\]{%
4	48830 \\
4	15927 \\
};
\addplot [black, forget plot]
table [row sep=\\]{%
4	106537 \\
4	148933 \\
};
\addplot [black, forget plot]
table [row sep=\\]{%
3.8875	15927 \\
4.1125	15927 \\
};
\addplot [black, forget plot]
table [row sep=\\]{%
3.8875	148933 \\
4.1125	148933 \\
};
\path [draw=black, fill=color0] (axis cs:0.775,103632)
--(axis cs:1.225,103632)
--(axis cs:1.225,218317.25)
--(axis cs:0.775,218317.25)
--(axis cs:0.775,103632)
--cycle;

\path [draw=black, fill=color1] (axis cs:1.775,71559.5)
--(axis cs:2.225,71559.5)
--(axis cs:2.225,142922.25)
--(axis cs:1.775,142922.25)
--(axis cs:1.775,71559.5)
--cycle;

\path [draw=black, fill=color2] (axis cs:2.775,140879.75)
--(axis cs:3.225,140879.75)
--(axis cs:3.225,272254.25)
--(axis cs:2.775,272254.25)
--(axis cs:2.775,140879.75)
--cycle;

\path [draw=black, fill=color3] (axis cs:3.775,48830)
--(axis cs:4.225,48830)
--(axis cs:4.225,106537)
--(axis cs:3.775,106537)
--(axis cs:3.775,48830)
--cycle;

\addplot [black, forget plot]
table [row sep=\\]{%
0.775	158435 \\
1.225	158435 \\
};
\addplot [black, forget plot]
table [row sep=\\]{%
1.775	102426.5 \\
2.225	102426.5 \\
};
\addplot [black, forget plot]
table [row sep=\\]{%
2.775	204180 \\
3.225	204180 \\
};
\addplot [black, forget plot]
table [row sep=\\]{%
3.775	74478.5 \\
4.225	74478.5 \\
};
\nextgroupplot[
height=\figureheight,
tick align=outside,
title={Frozen Lake},
title style={font=\small, yshift=-1.5ex, xshift=0.75ex},
xlabel style={font=\scriptsize},
yticklabel style={font=\scriptsize},
xticklabel style={font=\scriptsize},
width=\figurewidth,
x grid style={white!69.01960784313725!black},
xmajorticks=false,
xmin=0.5, xmax=4.5,
y grid style={white!69.01960784313725!black},
ymin=-2054.3, ymax=45098.3,
ytick pos=left
]
\addplot [black, forget plot]
table [row sep=\\]{%
1	5076.75 \\
1	272 \\
};
\addplot [black, forget plot]
table [row sep=\\]{%
1	17430.75 \\
1	27486 \\
};
\addplot [black, forget plot]
table [row sep=\\]{%
0.8875	272 \\
1.1125	272 \\
};
\addplot [black, forget plot]
table [row sep=\\]{%
0.8875	27486 \\
1.1125	27486 \\
};
\addplot [black, forget plot]
table [row sep=\\]{%
2	5476.25 \\
2	1258 \\
};
\addplot [black, forget plot]
table [row sep=\\]{%
2	18038.75 \\
2	32570 \\
};
\addplot [black, forget plot]
table [row sep=\\]{%
1.8875	1258 \\
2.1125	1258 \\
};
\addplot [black, forget plot]
table [row sep=\\]{%
1.8875	32570 \\
2.1125	32570 \\
};
\addplot [black, forget plot]
table [row sep=\\]{%
3	4920.25 \\
3	89 \\
};
\addplot [black, forget plot]
table [row sep=\\]{%
3	14963.25 \\
3	28574 \\
};
\addplot [black, forget plot]
table [row sep=\\]{%
2.8875	89 \\
3.1125	89 \\
};
\addplot [black, forget plot]
table [row sep=\\]{%
2.8875	28574 \\
3.1125	28574 \\
};
\addplot [black, mark=*, mark size=1, mark options={solid,fill opacity=0}, only marks, forget plot]
table [row sep=\\]{%
3	42955 \\
};
\addplot [black, forget plot]
table [row sep=\\]{%
4	4855.5 \\
4	149 \\
};
\addplot [black, forget plot]
table [row sep=\\]{%
4	14076.25 \\
4	27098 \\
};
\addplot [black, forget plot]
table [row sep=\\]{%
3.8875	149 \\
4.1125	149 \\
};
\addplot [black, forget plot]
table [row sep=\\]{%
3.8875	27098 \\
4.1125	27098 \\
};
\addplot [black, mark=*, mark size=1, mark options={solid,fill opacity=0}, only marks, forget plot]
table [row sep=\\]{%
4	28196 \\
4	33646 \\
4	28958 \\
};
\path [draw=black, fill=color0] (axis cs:0.775,5076.75)
--(axis cs:1.225,5076.75)
--(axis cs:1.225,17430.75)
--(axis cs:0.775,17430.75)
--(axis cs:0.775,5076.75)
--cycle;

\path [draw=black, fill=color1] (axis cs:1.775,5476.25)
--(axis cs:2.225,5476.25)
--(axis cs:2.225,18038.75)
--(axis cs:1.775,18038.75)
--(axis cs:1.775,5476.25)
--cycle;

\path [draw=black, fill=color2] (axis cs:2.775,4920.25)
--(axis cs:3.225,4920.25)
--(axis cs:3.225,14963.25)
--(axis cs:2.775,14963.25)
--(axis cs:2.775,4920.25)
--cycle;

\path [draw=black, fill=color3] (axis cs:3.775,4855.5)
--(axis cs:4.225,4855.5)
--(axis cs:4.225,14076.25)
--(axis cs:3.775,14076.25)
--(axis cs:3.775,4855.5)
--cycle;

\addplot [black, forget plot]
table [row sep=\\]{%
0.775	10634.5 \\
1.225	10634.5 \\
};
\addplot [black, forget plot]
table [row sep=\\]{%
1.775	10947.5 \\
2.225	10947.5 \\
};
\addplot [black, forget plot]
table [row sep=\\]{%
2.775	8346.5 \\
3.225	8346.5 \\
};
\addplot [black, forget plot]
table [row sep=\\]{%
3.775	8672 \\
4.225	8672 \\
};
\nextgroupplot[
height=\figureheight,
legend cell align={left},
legend columns=6,
legend entries={{BQL},{OQL},{QL},{OIQL},{BDQN},{ODQN}},
legend style={at={(0.3,-0.3)}, anchor=south east, draw=white!80.0!black, font=\small},
tick align=outside,
title={Taxi},
title style={font=\small, yshift=-1.5ex},
xlabel style={font=\scriptsize},
yticklabel style={font=\scriptsize},
xticklabel style={font=\scriptsize},
width=\figurewidth,
x grid style={white!69.01960784313725!black},
xmajorticks=false,
xmin=0.5, xmax=2.5,
y grid style={white!69.01960784313725!black},
ymin=-14049.6, ymax=338513.6,
ytick pos=left
]
\addlegendimage{thick, dashdotted, color0, line width= 1.5 pt}
\addlegendimage{solid, color1, line width= 1.5 pt}
\addlegendimage{dashed, color2, line width= 1.5 pt}
\addlegendimage{densely dotted, color3, line width= 1.5 pt}
\addlegendimage{dashdotted, color4, line width= 1.5 pt}
\addlegendimage{solid, color5, line width= 1.5 pt}
\addplot [black, forget plot]
table [row sep=\\]{%
	1	14858.75 \\
	1	1976 \\
};
\addplot [black, forget plot]
table [row sep=\\]{%
	1	49412.75 \\
	1	93014 \\
};
\addplot [black, forget plot]
table [row sep=\\]{%
	0.9625	1976 \\
	1.0375	1976 \\
};
\addplot [black, forget plot]
table [row sep=\\]{%
	0.9625	93014 \\
	1.0375	93014 \\
};
\addplot [black, mark=*, mark size=1, mark options={solid,fill opacity=0}, only marks, forget plot]
table [row sep=\\]{%
	1	105976 \\
	1	180065 \\
	1	209635 \\
	1	226813 \\
	1	101789 \\
	1	322488 \\
	1	235295 \\
};
\addplot [black, forget plot]
table [row sep=\\]{%
	2	17524 \\
	2	3678 \\
};
\addplot [black, forget plot]
table [row sep=\\]{%
	2	49574.75 \\
	2	89689 \\
};
\addplot [black, forget plot]
table [row sep=\\]{%
	1.9625	3678 \\
	2.0375	3678 \\
};
\addplot [black, forget plot]
table [row sep=\\]{%
	1.9625	89689 \\
	2.0375	89689 \\
};
\addplot [black, mark=*, mark size=1, mark options={solid,fill opacity=0}, only marks, forget plot]
table [row sep=\\]{%
	2	213731 \\
	2	165236 \\
	2	199472 \\
};
\path [draw=black, fill=color4] (axis cs:0.925,14858.75)
--(axis cs:1.075,14858.75)
--(axis cs:1.075,49412.75)
--(axis cs:0.925,49412.75)
--(axis cs:0.925,14858.75)
--cycle;

\path [draw=black, fill=color5] (axis cs:1.925,17524)
--(axis cs:2.075,17524)
--(axis cs:2.075,49574.75)
--(axis cs:1.925,49574.75)
--(axis cs:1.925,17524)
--cycle;

\addplot [black, forget plot]
table [row sep=\\]{%
	0.925	27773.5 \\
	1.075	27773.5 \\
};
\addplot [black, forget plot]
table [row sep=\\]{%
	1.925	36097.5 \\
	2.075	36097.5 \\
};
\end{groupplot}

\end{tikzpicture}
	\caption{The first row shows the average return for each tabular algorithm in the $N$-Chain, Taxi, and
	 Frozen Lake environments and for each neural network based algorithm in the Taxi and Acrobot environments
	 together with $95\%$ confidence intervals. The
     second row shows the distribution over the number of time steps before observing the maximum reward in the \gls{mdp}.}
	\label{results1}
\end{figure*}
\subsection{Results}
Figure~\ref{results1} summarizes the results obtained by averaging over $64$ different seed the tabular algorithms, and $100$ seeds \gls{dqn} and BDEN (more details, e.g. results for different hyper-parameter 
settings, can be found in the supplement).
\gls{oql} learns faster than the other tabular algorithms. In the Chain environment, and in Taxi,
our algorithm \gls{oql} finds the highest reward faster than \gls{bql} or \gls{ql}. On the other hand, also OIQL seems to find
high rewards fast, as shown in the box-plots. However, OIQL requires more training epochs to escape the high initial
optimistic values of the value function. In contrast, \gls{oql} finds high values fast in all the problems
($50$-Chain, Taxi, Frozen Lake) using the optimistic Bellman equation
while converging to a near-optimal solution as suggested by our convergence proofs.

With neural network approximation of the value function, \gls{odqn} outperforms \gls{bdqn} in the Taxi environment
demonstrating that the principles of \gls{obe} work even with function approximation. In the Cartpole environment,
the learning curves of \gls{odqn} and \gls{odqn} are nearly identical, possibly, due to the simplicity of the environment.
